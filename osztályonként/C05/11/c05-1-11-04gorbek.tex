\documentclass{article}
\usepackage[utf8]{inputenc}
\usepackage{t1enc}
\usepackage{geometry}
 \geometry{
 a4paper,
 total={210mm,297mm},
 left=20mm,
 right=20mm,
 top=20mm,
 bottom=20mm,
 }
\usepackage{amsmath}
\usepackage{amssymb}
\frenchspacing
\usepackage{fancyhdr}
\pagestyle{fancy}
\lhead{Urbán János tanár úr feladatsorai}
\chead{C05/11/4.}
\rhead{Görbék}
\lfoot{}
\cfoot{\thepage}
\rfoot{}

\usepackage{enumitem}
\usepackage{multicol}
\usepackage{calc}
\newenvironment{abc}{\begin{enumerate}[label=\textit{\alph*})]}{\end{enumerate}}
\newenvironment{abc2}{\begin{enumerate}[label=\textit{\alph*})]\begin{multicols}{2}}{\end{multicols}\end{enumerate}}
\newenvironment{abc3}{\begin{enumerate}[label=\textit{\alph*})]\begin{multicols}{3}}{\end{multicols}\end{enumerate}}
\newenvironment{abc4}{\begin{enumerate}[label=\textit{\alph*})]\begin{multicols}{4}}{\end{multicols}\end{enumerate}}
\newenvironment{abcn}[1]{\begin{enumerate}[label=\textit{\alph*})]\begin{multicols}{#1}}{\end{multicols}\end{enumerate}}
\setlist[enumerate,1]{listparindent=\labelwidth+\labelsep}

\newcommand{\degre}{\ensuremath{^\circ}}
\newcommand{\tg}{\mathop{\mathrm{tg}}\nolimits}
\newcommand{\ctg}{\mathop{\mathrm{ctg}}\nolimits}
\newcommand{\arc}{\mathop{\mathrm{arc}}\nolimits}
\renewcommand{\arcsin}{\arc\sin}
\renewcommand{\arccos}{\arc\cos}
\newcommand{\arctg}{\arc\tg}
\newcommand{\arcctg}{\arc\ctg}

\parskip 8pt
\begin{document}

\section*{Görbék}

\subsection*{2009.12.14.}
\begin{enumerate}
\item Az $a>0$ valós paraméter, a Descartes-féle levél egyenlete: $x^3+y^3-3axy=0$.
\\ Az $\frac{y}{x}=t$ paramétert bezetve írjuk fel a görbe paraméteres egyenletrendszerét és jellemezzük, ábrázoljuk a görbét.
\item A lemniszkáta egyenlete derékszögű koordinátákkal: $(x^2+y^2)^2=a^2(x^2-y^2)$, ahol $a>0$ paraméter. Vezessük le az $x=r\cos\phi$, $y=r\sin\phi$ polár koordinátákat, írjuk át a görbe egyenletét és vázoljuk a görbe alakját.
\end{enumerate}

\subsection*{2010.01.04.}
\begin{enumerate}
\item Egy $a>0$ sugarú kör csúszás nélkül gördül egy egyenesen. Milyen pályát ír le a gördülő kör kerületének egy rögzített pontja?
\item Ábrázoljuk és jellemezzük polárkoordináták segítségével a következő egyenlettel megadott görbét: $x^4-y^4+xy=0$.
\item Vázoljuk a következő egyenletekkel megadott görbéket:
\begin{enumerate}
\item $y^2=(x-1)(x-2)(x-3)$;
\item $y^2=x^4(x+1)$;
\item $y^2=\displaystyle \frac{x-1}{x+1}$.
\end{enumerate}
\end{enumerate}

\subsection*{2010.01.05.}
\begin{enumerate}
\item Ábrázoljuk és jellemezzük a következő egyenletekkel adott görbéket:
\begin{enumerate}
\item $x^4+2y^3=4x^2y$, $(y=tx)$;
\item $x^3+y^3=3x^2$, $(y=tx)$;
\item $x^4-y^4+xy=0$, polárkoordinátákra érdemes áttérni;
\item $x^{\frac{2}{3}}+y^{\frac{2}{3}}=a^{\frac{2}{3}}$, $(a>0)$; válasszunk alkalmas paraméterezést.
\end{enumerate}
\item Ábrázoljuk és jellemezzük a következő, polárkoordinátákkal adott görbéket:
\begin{enumerate}
\item $r=a\sin^3\phi$ $(a>0)$;
\item $r^2=2a^2\cos^2\phi$ $(a>0)$.
\end{enumerate}
\end{enumerate}

\subsection*{2010.01.11.}
Ábrázoljuk a következő egyenletekkel megadott görbéket:
\begin{enumerate}
\item $y^2=8x^2-x^4$;
\item $x^3+y^3=3x^2$, $(y=tx)$.
\end{enumerate}

\end{document}
