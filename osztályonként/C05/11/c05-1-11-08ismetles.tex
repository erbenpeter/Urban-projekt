\documentclass{article}
\usepackage[utf8]{inputenc}
\usepackage{t1enc}
\usepackage{geometry}
 \geometry{
 a4paper,
 total={210mm,297mm},
 left=20mm,
 right=20mm,
 top=20mm,
 bottom=20mm,
 }
\usepackage{amsmath}
\usepackage{amssymb}
\frenchspacing
\usepackage{fancyhdr}
\pagestyle{fancy}
\lhead{Urbán János tanár úr feladatsorai}
\chead{C05/11/1.}
\rhead{Ismétlés}
\lfoot{}
\cfoot{\thepage}
\rfoot{}

\usepackage{enumitem}
\usepackage{multicol}
\usepackage{calc}
\newenvironment{abc}{\begin{enumerate}[label=\textit{\alph*})]}{\end{enumerate}}
\newenvironment{abc2}{\begin{enumerate}[label=\textit{\alph*})]\begin{multicols}{2}}{\end{multicols}\end{enumerate}}
\newenvironment{abc3}{\begin{enumerate}[label=\textit{\alph*})]\begin{multicols}{3}}{\end{multicols}\end{enumerate}}
\newenvironment{abc4}{\begin{enumerate}[label=\textit{\alph*})]\begin{multicols}{4}}{\end{multicols}\end{enumerate}}
\newenvironment{abcn}[1]{\begin{enumerate}[label=\textit{\alph*})]\begin{multicols}{#1}}{\end{multicols}\end{enumerate}}
\setlist[enumerate,1]{listparindent=\labelwidth+\labelsep}

\newcommand{\degre}{\ensuremath{^\circ}}
\newcommand{\tg}{\mathop{\mathrm{tg}}\nolimits}
\newcommand{\ctg}{\mathop{\mathrm{ctg}}\nolimits}
\newcommand{\arc}{\mathop{\mathrm{arc}}\nolimits}
\renewcommand{\arcsin}{\arc\sin}
\renewcommand{\arccos}{\arc\cos}
\newcommand{\arctg}{\arc\tg}
\newcommand{\arcctg}{\arc\ctg}

\parskip 8pt
\begin{document}

\section*{Ismétlés}

\subsection*{2010. 05. 11.}
\begin{enumerate}

\item Ábrázoljuk a koordinátasíkon azokat a P(x;y) pontokat, amelyekre
$\log_{x^2+y^2}(x+y) \geq 1$.

\item Oldjuk meg a valós számok halmazán:
\begin{abc2}
\item $\dfrac{xy}{x^2+y^2} = \dfrac{2}{5}$
\item $x^2-xy+2y^2 = 4$
\end{abc2}

\item Oldjuk meg a valós számok halmazán: 
$|x^2-9|+|x^2-4| = 5$.

\item Az x valós számra $x+\dfrac{1}{x} = 3$ teljesül.

\centerline{$x^5+x^3+x^2+\dfrac{1}{x^2}+\dfrac{1}{x^3}+\dfrac{1}{x^5} = $ ?}

\item Oldjuk meg a valós számok halmazán:
$\sin^42x+\cos^42x = \sin^2x \cdot \cos^2x$.

\end{enumerate}

\subsection*{2010. 05. 17.}
\begin{enumerate}

\item
\begin{abc}
\item Oldjuk meg az egész számok halmazán: 
$\left(5x + 60\right)^2 = \left(-x -192\right)^2$

\item Oldjuk meg a valós számok halmazán:
$\sqrt{5x + 60} = \sqrt{-x - 192}$

\item Oldjuk meg a $[ -\pi; \pi ]$ halmazon:
$\sin\left( 5\alpha + \dfrac{\pi}{3} \right) = -\sin\left( -\alpha - \left( \pi + \dfrac{\pi}{15}\right) \right)$
\end{abc}

\item Oldjuk meg a valós számok halmazán: 
\begin{abc2}
\item $x + y = \dfrac{1}{3}$
\item $\cos{(\pi \cdot x)} \cdot \cos{(\pi \cdot y)} = \dfrac{1}{2}$
\end{abc2}

\item Hány 1-nél nagyobb, de 2-nél kisebb tagja van az $a_n = -1 + \lg(n+3)$ sorozatnak?

\item Igazoljuk, hogy ha $n > 3$ egész, akkor $\dfrac{1}{n+1} + \dfrac{1}{n+2} + \cdots + \dfrac{1}{3n} > 1$.

\end{enumerate}

\subsection*{2010. 05. 18.}
\begin{enumerate}

\item Oldjuk meg a valós számok halmazán:
\begin{abc2}
\item * $4\sqrt{x} + \sqrt{4 - x} = 4 + x$
\item $25x^2 - 11x - 6\sqrt{x} - 8 = 0$
\item $x^2 + x\sqrt{x + 1} - 2(x + 1) = 0$
\item 
$y^3 + z^3 = 7x^3$;

$y - z - 3x = 0$;

$z - x = y - 2$

\end{abc2}

\item Oldjuk meg a valós számok halmazán:
\begin{abc2}
\item $x^2 + (x + 1)^2 < \dfrac{15}{x^2 + x + 1}$
\item $|x^2 - 2x - 3| < 3x - 3$
\item $x^4 - x^3 - x^2 - x - 2 \leq 0$ 
\item $2x(x - 1) + 1 > \sqrt{x^2-x+1}$
\end{abc2}

\item Oldjuk meg a valós számok halmazán: 
\begin{abc3}
\item $\sqrt{x+1} - \sqrt{\dfrac{x-1}{x}} = 1$ 
\item $\root4\of{x-2} + \root4\of{3-x} = 1$
\item $\root3\of{24+x} + \sqrt{12-x} = 6$
\end{abc3}

\item Oldjuk meg a valós számok halmazán:
\begin{abc3}
\item 
$|x|^{\lg{|y|}} = 4$;

$xy = 40$

\item $2^x - 3^x = \sqrt{6^x - 9^x}$
\item $\log_{\sqrt{2}\sin{x}}(1 + \cos{x}) = 2$
\item 
$(x^2+x+1)^{y-\frac12} < 1$;

$\log_y(yx - x) = 1$
\end{abc3}
\end{enumerate}

\subsection*{2010. 05. 19.}
\subsubsection*{Paraméteres feladatok}
\begin{enumerate}

\item Oldjuk meg a valós számok halmazán, ha \it a \rm valós paraméter:
\begin{abc}
\item $x-3\root3\of{x} = a^3 + \dfrac{1}{a^3}$
\item $\root3\of{(x+a)^2} + \root3\of{(x-a)^2} + \root3\of{x^2-a^2} = \root3\of{a^2}$
\item $2\sqrt{a+x} - \sqrt{a-x} = \sqrt{a-x+\sqrt{x(a+x)}}$
\end{abc}

\item Az $a > 0$ valós paraméter mely értékére van az $|x+2|-|2x+8| = a^x$ egyenletnek pontosan egy megoldása?

\item Az \it a \rm valós paraméter mely értékére van megoldása a $\cos^4{x} + 2\sin^4{x} = a$ egyenletnek?

\end{enumerate}
\subsection*{2010. 05. 26.}

\begin{enumerate}

\item Oldjuk meg a valós számok halmazán a következő egyenletrendszereket:
\begin{abc}
\item $x^2+xy+y^2 = 4$;

$x + xy + y = 2$
\item $\dfrac{x^2}{y} + \dfrac{y^2}{x} = 12$;

$\dfrac{1}{x} + \dfrac{1}{y} = \dfrac{1}{3}$ 
\item $2(x+y) = 5xy$;

$8(x^3+y^3) = 65$
\item $(x^2+y^2)\dfrac{x}{y} = 6$;

$(x^2-y^2)\dfrac{y}{x} = 1$
\item $(x^2+1)(y^2+1) = 10$;

$(x+y)(xy-1) = 3$

\end{abc}

\item Oldjuk meg a valós számok halmazán: 
\begin{abc2}
\item $\dfrac{x^2}{3}+ \dfrac{48}{x^2} = 10\left(\dfrac{x}{3}-\dfrac{4}{x}\right)$
\item 
$x+y+z = 2$;

$2xy-z^2 = 4$
\end{abc2}

\end{enumerate}
\subsection*{2010. 05. 31.}

\begin{enumerate}

\item Oldjuk meg a valós számok halmazán:
\begin{abc}
\item 
$\dfrac{x^2}{y}+\dfrac{y^2}{x} = 12$;

$\dfrac{1}{x}+\dfrac{1}{y} =\dfrac{1}{3}$
\item 
$x^3-y^3=19(x+y)$;

$x^3+y^3=7(x+y)$
\item $\dfrac{xyz}{x+y}=2$;

$\dfrac{xyz}{y+x}=\dfrac{6}{5}$;

$\dfrac{xyz}{z+x}=\dfrac{3}{2}$
\end{abc}

\item Oldjuk meg a valós számok halmazán, ha \it a \rm $a>0$ valós paraméter:
\begin{abc2}
\item $\sqrt{\dfrac{y+1}{x-y}}+2\sqrt{\dfrac{x-y}{y+1}}=3$;

$x+xy+y=7$
\item $\sqrt{x+y}-\sqrt{x-y}=a$;

$\sqrt{x^2+y^2}-\sqrt{x^2-y^2}=a^2$
\end{abc2}
\end{enumerate}
\subsection*{2010. 06. 01.}

\begin{enumerate}

\item Oldjuk meg a valós számok halmazán:
\begin{abc2}
\item $x^3+y^3=1$;

$x^2y+2xy^2+y^3=2$
\item $\sqrt{\dfrac{x}{y}}-\sqrt{\dfrac{y}{x}}=\dfrac{3}{2}$;

$x+yx+y=9$
\end{abc2}

\item Határozzuk meg a következő egyenlet valós megoldásait:

$\dfrac{\sqrt{x^2+8x}}{\sqrt{x+1}}+\sqrt{x+7}=\dfrac{7}{\sqrt{x+1}}$

\item Oldjuk meg a valós számok halmazán: 
$\root3\of{x-1}+\root3\of{x+1}=x\root3\of{2}$

\item Az \it a \rm valós paraméter; oldjuk meg az egyenletet a valós számok halmazán:

$\sqrt{x-4a+16}=2\sqrt{x-2a+4}-\sqrt{x}$
\end{enumerate}
\subsection*{2016. 06. 02.}
\begin{enumerate}

\item Oldjuk meg a valós számok halmazán: 
\begin{abc2}
\item $2+\cos{x}=2\tg{\dfrac{x}{2}}$
\item $\ctg{x}-2\sin{2x}=1$
\item $\cos^3{x}+\sin^3{x}=1-\dfrac{1}{2}\sin{2x}$
\item $\ctg^2{x}=\dfrac{1+\sin{x}}{1+\cos{x}}$
\end{abc2}

\item Oldjuk meg a valós számok halmazán:
\begin{abc2}
\item $\tg{x}+\tg{y}=1$;

$\cos{x}\cos{y}=\dfrac{1}{\sqrt{2}}$
\item $\sin{x}\sin{y}=\dfrac{1}{4\sqrt{2}}$;

$\tg{x}\tg{y}=\dfrac{1}{3}$
\end{abc2}

\item Igazoljuk, hogy ha $0 < \varphi < \dfrac{\pi}{2}$, akkor $\ctg{\dfrac{\varphi}{2}} > 1+\ctg{\varphi}$.

\end{enumerate}
\subsection*{2010. 06. 07.}
\begin{enumerate}

\item Oldjuk meg a valós számok halmazán:
\begin{abc3}
\item $2+\cos{x}=2\tg{\dfrac{x}{2}}$
\item $\ctg{x}-2\sin{2x}=1$
\item $\ctg^2{x}= \dfrac{1+\sin{x}}{1+\cos{x}}$
\end{abc3}

\item Oldjuk meg a valós számok halmazán:
\begin{abc}
\item $\tg{x}+\tg{y}=1$;

$\cos{x}+\cos{y}=\dfrac{1}{\sqrt{2}}$
\item $\sin{x}\sin{y}=\dfrac{1}{4\sqrt{2}}$;

$\cos{x}\cos{y}= \dfrac{3}{4} \cdot \dfrac{1}{\sqrt{2}}$
\item $x^{\tg{y}}=\dfrac{1}{8}$;

$x^{\dfrac{1+\tg{y}}{1-\tg{y}}}=4$
\end{abc}

\end{enumerate}
\subsection*{2010. 06. 08.}
\begin{enumerate}

\item Oldjuk meg a valós számok halmazán a $\sqrt{a-\sqrt{a+x}}=x$ egyenletet ($a > 0$ valós szám).

\item Hány gyöke van a valós számok halmazán a $\sin{x}= \dfrac{x}{100}$ egyenletnek?

\item Az \it a \rm valós paraméter mely értékeire teljesül, hogy az 

$x^2+ax+1=0$;

$x^2+x+a=0$ \\ egyenleteknek legalább egy közös gyökük van a valós számok halmazán?

\item Oldjuk meg a valós számok halmazán: 
\begin{abc3}
\item $\log_{3-x}x < -1$
\item $\log_{|x|}(x-1)^2 < 2$
\item $\log_{x^2-3}(4x+2)\geq 1$
\end{abc3}

\end{enumerate}
\subsection*{2010. 06. 09.}
\begin{enumerate}

\item Egy négyzetes oszlop alakú, felül nyitott medencét kell készíteni, amelynek térfogata 32 $m^3$. Hogyan kell méretezni ahhoz, hogy a felszíne minimális legyen?

\item Adott $r$ sugarú gömbbe írjunk olyan hengert, amelynek a palást felszíne maximális.

\item A derékszögű koordináta-rendszer $(1; 2)$ pontján áthaladó egyenesek közül melyik zár be legkisebb területű háromszöget az $x$ és $y$ tengely pozitív felével? 

\item Az $y^2=2x$ egyenletű parabola tengelyén van a $P(a; 0)$ pont ($a > 0$). A parabolán melyik pont van legközelebb $P$-hez?
\end{enumerate}
\end{document}
