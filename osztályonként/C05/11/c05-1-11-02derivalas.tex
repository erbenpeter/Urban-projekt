\documentclass{article}
\usepackage[utf8]{inputenc}
\usepackage{t1enc}
\usepackage{geometry}
 \geometry{
 a4paper,
 total={210mm,297mm},
 left=20mm,
 right=20mm,
 top=20mm,
 bottom=20mm,
 }
\usepackage{amsmath}
\usepackage{amssymb}
%\usepackage{ dsfont }
\frenchspacing
\usepackage{fancyhdr}
\pagestyle{fancy}
\lhead{Urbán János tanár úr feladatsorai}
\chead{C05/11/1.}
\rhead{Deriválás}
\lfoot{}
\cfoot{\thepage}
\rfoot{}

\usepackage{enumitem}
\usepackage{multicol}
\usepackage{calc}
\newenvironment{abc}{\begin{enumerate}[label=\textit{\alph*})]}{\end{enumerate}}
\newenvironment{abc2}{\begin{enumerate}[label=\textit{\alph*})]\begin{multicols}{2}}{\end{multicols}\end{enumerate}}
\newenvironment{abc3}{\begin{enumerate}[label=\textit{\alph*})]\begin{multicols}{3}}{\end{multicols}\end{enumerate}}
\newenvironment{abc4}{\begin{enumerate}[label=\textit{\alph*})]\begin{multicols}{4}}{\end{multicols}\end{enumerate}}
\newenvironment{abcn}[1]{\begin{enumerate}[label=\textit{\alph*})]\begin{multicols}{#1}}{\end{multicols}\end{enumerate}}
\setlist[enumerate,1]{listparindent=\labelwidth+\labelsep}

\newcommand{\degre}{\ensuremath{^\circ}}
\newcommand{\tg}{\mathop{\mathrm{tg}}\nolimits}
\newcommand{\ctg}{\mathop{\mathrm{ctg}}\nolimits}
\newcommand{\arc}{\mathop{\mathrm{arc}}\nolimits}
\renewcommand{\arcsin}{\arc\sin}
\renewcommand{\arccos}{\arc\cos}
\newcommand{\arctg}{\arc\tg}
\newcommand{\arcctg}{\arc\ctg}
\newcommand{\sh}{\mathop{\mathrm{sh}}\nolimits}
\newcommand{\ch}{\mathop{\mathrm{ch}}\nolimits}
%\newcommand{\th}{\mathop{\mathrm{th}}\nolimits}
\newcommand{\cth}{\mathop{\mathrm{cth}}\nolimits}


\parskip 8pt
\begin{document}

\section*{Deriválási szabályok}

\subsection*{2009.10.20. -- Deriválási szabályok}
\underline{Definíció}: Ha az $f$ függvény értelmezve van az \underline{$a$} hely egy 								környezetén, és létezik a \underline{véges} 
						\[ \lim_{x \to 0} \frac{f(x)-f(a)}{x-a} = A \]
						határérték, akkor $f$ differenciálható az \underline{$a$} helyen 							és $f$ differenciálhányadosa, vagy deriváltja $A$. Jelölés: 								$f'(a)=A$.

\begin{enumerate}
\item Igazoljuk:
	\begin{abc}
	\item $(x^n)'=nx^{n-1} $;
	\item $(\sin x)'=\cos x $;
	\item $(\cos x)'=-\sin x $;
	\item $(e^x)'=e^x $;
	\item $(\ln x)'=\frac{1}{x}$, ha $x>0 $;
	\item $(a^x)'=a^x \ln a$, ha $a>0$;
	\item $(\tg x)'=\frac{1}{\cos^2 x}$, ha $x \neq \frac{\pi}{2}+k\pi, k\in \mathbb{Z} $;
	\end{abc}

\end{enumerate}

\subsection*{2009.10.21.}
\begin{enumerate}
\item Igazoljuk, hogy ha $f$ és $g$ differenciálható az \underline{$a$} helyen, akkor $f+g		$ is differenciálható az \underline{$a$} helyen, és $(f+g)'(a)=f'(a)+g'(a)$.
\item Határozzuk meg a sorozat differenciálási szabályát.
\item Igazoljuk, hogy ha $g$ differenciálható az \underline{$a$} helyen, és $g(a)\neq 0$, 		akkor $\frac{1}{g}$ is differenciálható az \underline{$a$} helyen, és 
	\[ \bigg(\frac{1}{g}\bigg)'(a)=-\frac{g'(a)}{g^2(a)} \].
\item Határozzuk meg a hányados differenciálási szabályát.
\item Adjuk meg a következő függvények differenciálási szabályát:
	\begin{abc}
	\item $\sqrt[n]{x},~~~ x>0;$
	\item $\ctg x;~~~x\neq k\pi, k\in \mathbb{Z};$
	\item $\sh~x,~~~ \ch~x;$
	\item $\th~x,~~~ \cth~x;$
	\item $e^x$.
	\end{abc}
\end{enumerate}

\subsection*{2009.11.02.}
\begin{enumerate}
\item Igazoljuk, hogy $f$ akkor és csak akkor differenciálható az \underline{$a$} helyen. 		ha $f(x)$ az \underline{$a$} hely egy környezetében így írható:
	\[f(x)=f(a)+A(x-a)+\varepsilon (x)(x-a), \]
	ahol $A$ konstans és $\lim_{x \to a} \varepsilon (x)=0$.
\item Igazoljuk a közvetett függvény differenciálási szabályát, az u.n. "lánc-szabályt": 		Ha $f$ differenciálható a $g(a)$ helyen és $g$ differenciálható az $a$ helyen, akkor 		$f\circ g$ is differenciálható az $a$ helyen és \[(f \circ g)'(a)=f'(g(a))\cdot g'(a).		\] 
\item Számítsuk ki a következő függvények deriváltját, ahol létezik:
	\begin{abc}
	\item $f(x)=\sqrt{x^2+3x+2}$;
	\item $g(x)=\sin(x^2+2x)$;
	\item $h(x)=\ln(e^x+2)$;
	\item $k(x)=x^x,~~~x>0$;
	\item $l(x)=x^\alpha,~~~x>0,~~\alpha \in \mathbb{R}$;
	\item $t(x)=\sqrt{x+\sqrt{x+\sqrt{x}}},~~~x>0$
	\end{abc}
\end{enumerate}

\subsection*{2009.11.03.}
\begin{enumerate}
\item Számítsuk ki a következő függvények deriváltját, ahol létezik:
	\begin{abc}
	\item $f(x)=\arctg x$;
	\item $g(x)=\arcsin x$;
	\item $h(x)=\ln(\ln(\ln x))$;
	\item $k(x)=\ln \big(x+\sqrt{x^2+1}\big)$;
	\item $l(x)=x^2\cdot \sin x \cdot \ln x$;
	\item $n(x)=x^{x^x}$.
	\end{abc}
\item Igazoljuk, hogy az $f(x)=x\cdot \sin x$ függvény kielégíti a következő egyenletet:
	\[ \frac{f'}{\cos x}-x=\tg x. \]
\item Igazoljuk, hogy az 
	\[  f(x) = 
	  \begin{cases} 
	   x^2 \sin \frac{1}{x},~~~x\neq 0, \\
	   0~~\text{ha}~~x =0
	  \end{cases} \]
	  függvény minden $x$-re differenciálható.
\item A következő függvények minden valós $x$-re értelmezve vannak. Hol nem 					differenciálhatók:
	\begin{abc}
	\item $f(x)=\sqrt{1-\cos^2 x}$;
	\item $g(x)=\arcsin \frac{2x}{1-x^2}$;
	\item $h(x)=\sqrt{16-8x^2+x^4}$;
	\item $k(x)=\arcsin (\sin x)$.
	\end{abc}
\end{enumerate}

\subsection*{2009.11.04.}
\begin{enumerate}
\item Deriváljuk a következő függvényeket:
	\begin{abc}
	\item $f(x)=\sin x^{\cos x}~~~(\sin x>0)$;
	\item $g(x)=\lg(\cos x) ~~~ (\cos x>0)$;
	\item $h(x)=\ctg x,~~~x\neq k\pi,~~k\in \mathbb{Z}$;
	\item $k(x)=\sin(\cos x)+ \cos (\sin x)$.
	\end{abc}
\item Számítsuk ki a deriválási szabályok alkalmazásával a következő értékeket:
	\begin{abc}
	\item $1\cdot 2+ 2\cdot 3x+\ldots +n(n+1)x^{n-1}$;
	\item $1+ 3x^2+ 5x^4+\ldots +(2n-1)x^{2n-2}$;
	\item $1^2+ 3^2x^2+ 5^2x^4+\ldots +(2n-1)^2x^{2n-2}$
	\end{abc}
\item Igazoljuk, hogy az $f(x)=\sqrt{2x-x^2}$ függvény kielégíti a következő egyenletet: 
	\[f^3 \cdot f'' +1=0 .\]
\item
	\begin{abc}
	\item $f(x)=\frac{x^2}{1-x}$, $f^{(8)}=?$
	\item $g(x)=\frac{1+x}{\sqrt{1-x}}$, $g^{(100)}=?$
	\end{abc}
\end{enumerate}

\subsection*{2009.11.10.}
\begin{enumerate}
\item Deriváljuk a következő függvényeket:
	\begin{abc}
	\item $f(x)=\arctg \frac{1+x^3}{1+x^2}$;
	\item $g(x)=\sqrt[3]{x^2}$;
	\item $h(x)=\arcsin \frac{1-x^2}{1+x^2}$;
	\item $k(x)=\arccos \sqrt{1-x^2}$;
	\item $l(x)=x \ln \big(x+\sqrt{1+x^2}\big)-\sqrt{1+x^2}$.
	\end{abc}
\item Paraméteresen megadott görbe érintőjének irányvektora: Igazoljuk, hogy ha $x=x(t)$, 		$y=y(t)$, $\alpha \leq t \leq \beta$ és $\alpha <t_0 < \beta$, akkor az adott görbe 		érintőjének irányvektora a $t_0$ pontban $\Big(x'(t_0),y'(t_0)\Big)$.
\item Számítsuk ki a következő paraméteresen adott görbék érintőjének irányvektorát, ahol 		létezik: 
	\begin{equation}
 	\left.\begin{aligned}
	        a) ~x&= a \cos t,~~y=a \sin t; \\
	        b)~x&=a \cos^3 t,~~y=a \sin^3 t;\\
	        c)~x&=a(t-\sin t),~~y= a(1-\cos t).
	       \end{aligned}
	 \right\}
	 \qquad a>0~~ \text{állandó}
\end{equation}
\item Adjunk meg egyszerű eljárást az $y=x^2$, $y=x^3$ egyenletű görbék érintőjének szerkesztésére.
\end{enumerate}

\subsection*{2009.11.10 -- Deriválási szabályok}
\begin{enumerate}
\item $c'=0$;
\item $(cu)'=c\cdot u'$;
\item $(u+v)'=u'+v'$;
\item $(u\cdot v)= u'\cdot v + u\cdot v'$;
\item $\big(\frac{u}{v}\big)'=\frac{u'v-uv'}{v^2}$
\item $u^\alpha = \alpha u^{\alpha-1}\cdot u' ~~~(\alpha~\text{állandó,}~u>0) $;
\item $f(g(x))'=f'(g(x))\cdot g'(x)$.
\end{enumerate}
Az elemi függvények deriváltja:\\
$(x^n)'=nx^{n-1}$;\\
$(\sin x)'=\cos x $;\\
$(\cos x)'=-\sin x $;\\
$(\tg x)'=\frac{1}{\cos^2 x} $;\\
$(\ctg x)'=-\frac{1}{\sin^2 x} $;\\
$(e^x)'=e^x $;\\
$(a^x)'=a^x \ln a~~~(a>0) $;\\
$(\log_a x)'=\frac{1}{x \ln a},~~~a>0 $;\\
$(\arcsin x)'=\frac{1}{\sqrt{1-x^2}} $;\\
$(\arccos x)'=-\frac{1}{\sqrt{1-x^2}}$;\\
$(\arctg x)'=\frac{1}{1+x^2} $;\\
$(\arcctg x)'=-\frac{1}{1+x^2} $;\\
$(\sh x)'=\ch x $;\\
$(\ch x)'=\sh x $;\\



\subsection*{2009.11.11 -- Deriválás}
\begin{enumerate}
\item Hol differenciálhatók a következő függvények: 
	\begin{abc}
	\item $f(x)=x |x|$;
	\item $g(x)=\arcsin (\sin x)$.
	\end{abc}
	Ahol differenciálhatók, ott határozzuk meg a deriváltat.
\item Írjuk fel az $x=2t-t^2$, $y=3t-t^3$ paraméteres megadású görbe érintőjének egyenletét a $t=0$ és $t=1$ paraméterű pontokon.
\item Számítsuk ki az $f(x)=x \ln x~~~x>0$  függvény 5-ödik deriváltját.
\item $g(x)=\frac{1}{x(1-x)}$, $0<x<1$, $n>0$ egész. $g^{(n)}(x)=?$
\item Számítsuk ki a következő függvények deriváltját, ahol létezik: 
	\begin{abc}
	\item $f(x)=x \ln (x+ \sqrt{1+x^2})-\sqrt{1+x^2}$;
	\item $g(x)=\frac{1}{2}\ctg^2 x+\ln \sin x$;
	\item $h(x)=x^{a^a}+a^{x^a}+a^{a^x},~~~x>0,~~a>0~\text{rögzített.}$
	\end{abc}
\end{enumerate}

\end{document}
