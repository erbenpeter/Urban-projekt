\documentclass{article}
\usepackage[magyar]{babel}
\usepackage[utf8]{inputenc}
\usepackage{t1enc}
\usepackage{graphicx}
\usepackage{geometry}
 \geometry{
 a4paper,
 total={210mm,297mm},
 left=20mm,
 right=20mm,
 top=20mm,
 bottom=20mm,
 }
\usepackage{amsmath}
\usepackage{amssymb}
\frenchspacing
\usepackage{fancyhdr}
\pagestyle{fancy}
\lhead{Urbán János tanár úr feladatsorai}
\chead{C05/07/1. csoport}
\rhead{Halmazok}
\lfoot{}
\cfoot{\thepage}
\rfoot{}

\usepackage{pgf,tikz}
\usepackage{enumitem}
\usepackage{multicol}
\usepackage{calc}
\newenvironment{abc}{\begin{enumerate}[label=\textit{\alph*})]}{\end{enumerate}}
\newenvironment{abc2}{\begin{enumerate}[label=\textit{\alph*})]\begin{multicols}{2}}{\end{multicols}\end{enumerate}}
\newenvironment{abc3}{\begin{enumerate}[label=\textit{\alph*})]\begin{multicols}{3}}{\end{multicols}\end{enumerate}}
\newenvironment{abc4}{\begin{enumerate}[label=\textit{\alph*})]\begin{multicols}{4}}{\end{multicols}\end{enumerate}}
\newenvironment{abcn}[1]{\begin{enumerate}[label=\textit{\alph*})]\begin{multicols}{#1}}{\end{multicols}\end{enumerate}}
\setlist[enumerate,1]{listparindent=\labelwidth+\labelsep}

\newcommand{\degre}{\ensuremath{^\circ}}
\newcommand{\tg}{\mathop{\mathrm{tg}}\nolimits}
\newcommand{\ctg}{\mathop{\mathrm{ctg}}\nolimits}
\newcommand{\arc}{\mathop{\mathrm{arc}}\nolimits}
\renewcommand{\arcsin}{\arc\sin}
\renewcommand{\arccos}{\arc\cos}
\newcommand{\arctg}{\arc\tg}
\newcommand{\arcctg}{\arc\ctg}

\parskip 8pt
\begin{document}

\section*{Halmazok}

\subsection*{2006.05.23.}
\begin{enumerate}
\item Sorold fel a következő halmaz összes részhalmazát:
(a) $\{1, 2, 3\}$;  (b) $\{1, 2, 3, 4\}$.
\item Adjuk meg az $A\cap B$, $A\cup B$, $A\setminus B$, $B\setminus A$ halmazokat, ha
\begin{abc}
\item $A = \{-1, 0, 3, 4\}$, $B=\{0, 4, 6\}$;
\item $A = \{0,1,2,3\}$, $B=\{-1,0,1,2,3\}$;
\item $A = [0; 2]$, $B = [1;3]$.
\end{abc}
\item Igazoljuk, hogy tetszőleges $A, B, C$ halmazokra teljesülnek a következő azonosságok:
\begin{abc2}
\item $A\cup(A\cap B)=A$;
\item $A\cap(A\cup B)=A$;
\item $((A\setminus B)\setminus C)=(A\setminus C)\setminus B$;
\item $A\setminus(A\setminus B) = A\cap B$;
\item $A\setminus(B\cap C)=(A\setminus B)\cup(A\setminus C)$.
\end{abc2}
\end{enumerate}
\subsection*{2006.05.25.}
\begin{enumerate}
\item $A=\{2,4,6,8,10\}$, $B=\{3,6,9,12,15\}$ és $C=\{5,10,15,20\}$.
\begin{abcn}{5}
\item $A\cup B\cup C=?$
\item $A\cap B\cap C=?$
\item $A\setminus B=?$
\item $C\setminus A=?$
\item $B\setminus C=?$
\end{abcn}
\item Melyek azok az $A$ és $B$ halmazok, amelyekre teljesül a következő három egyenlőség:
\begin{abc}
\item $A\cap B=\{3,5,7\}$, $A\setminus B=\{2,6\}$, $A\cup B=\{1,2,3,4,5,6,7\}$;
\item $A\setminus B=\{1,2,5\}$, $B\setminus A=\{7,8,9\}$, $A\cap B=\{3,4,6\}$.
\end{abc}
\item Az $A$ halmaznak 8, a $B$-nek 9 eleme van, $A\cup B$-nek 15 eleme van. Hány eleme van $A\cap B$-nek?
\item Írjuk fel a következő halmazokat elemeik felsorolásával:
\begin{abc2}
\item $A=\{x\in \mathbb{N} ~|~ \frac{24}{x}\in \mathbb{Z}\}$
\item $B=\{x\in \mathbb{Z} ~|~ -5\le x \le 4\}$
\item $C=\{x\in \mathbb{Z} ~|~ |x-3|=5\}$
\item $D=\{x\in \mathbb{Q} ~|~ |x-3|+|x-1|=3\}$
\end{abc2}
\item Az $\mathbb{N}$ halmazt részhalmazokra bontottuk:
$\{0\}, \{1,2\}, \{3,4,5\}, \{6,7,8,9\}, \ldots$
Melyik számmal kezdődik a századik részhalmaz?
\end{enumerate}

\subsection*{2006.05.30.}
\begin{enumerate}
\item $A$: a páros kétjegyű számok halmaza; $B$: a 100-nál kisebb, 3-mal osztható számok halmaza; $C$: a 30-cal osztható egész számok halmaza.
$
A\cap B=?\qquad
A\cap C=?\qquad
B\cap C=?
$
\item Adjunk példát olyan $A$, $B$ és $C$ halmazra, hogy teljesüljenek:
$|A\cap B \cap C|=1$, $|A|=|B|=|C|=2$ és $A\ne B$, $B\ne C$.
\item $A=B=\{0,1,2\}$, $A\times B=?$
\item Tudjuk, hogy $A\cup B=\{1,2,3,4,5,6\}$, $A\setminus B=\{2,4,6\}$, 
$A\cap B=\{1,3\}$. $A=?$, $B=?$
\item ($*$) Tudjuk, hogy $|A \times B|=100$. Mekkora lehet $|A\cup B|$ legalább és legfeljebb?
\item $A=\{1,2,3\}$, $B=\{2,3,4\}$. Hány eleme van a következő halmazoknak?
\begin{abc4}
\item $(A\setminus B)\times(B\setminus A)$;
\item $(A\cup B)\times(A\cap B)$;
\item $(A\setminus B)\times(A\cap B)$;
\item $(B\setminus A)\times(B\cup A)$.
\end{abc4}
\item Tudjuk, hogy $|A|=5$, $|B|=8$, $|A\setminus B|=3$. $|A\cap B|=?$, $|A\cup B|=?$
\end{enumerate}

\subsection*{2006.05.31.}
\begin{enumerate}
\item Legyen $A=\{\text{a 7-tel osztható kétjegyű számok}\}$,
$B=\{\text{a 3-mal osztható kétjegyű számok}\}$. $|A\cup B|=?$, $|A\cap B|=?$
\item Hány eleme van legalább annak a halmaznak, amelynek legalább 1000-rel több részhalmaza van, mint eleme?
\item Egy 15 elemű halmaznak 9 elemű részhalmazából vagy a 6 elemű részhalmazából van több?
\item Hány háromelemű részhalmaza van az $\{1,2,3,4,5\}$ halmaznak?
\item Legyen $A$ az 1000-nél kisebb pozitív egész számok halmaza. Hány elemű $A$ összes részhalmazának egyesítése?
\item $A=\{1,2\}$, $B=\{1,2,3,4\}$. $(A\times A)\cap(B\times B)=?$
\item Hány olyan részhalmaza van az egyjegyű pozitív egész számok halmazának, amelynek
\begin{abc3}
\item a 4 és az 5 is eleme;
\item a 4 eleme, de az 5 nem;
\item sem az 5, sem a 4 nem eleme?
\end{abc3}
\item Hány olyan részhalmazát lehet megadni az $A$ halmaznak, hogy semelyik kettő közös része se legyen üres, ha
(a) $A=\{1,2,3,4\}$;\quad (b) $A=\{1,2,3,4,5,6,7,8\}$?
\end{enumerate}

\subsection*{2006.06.01.}

\begin{enumerate}
\item $A\cup B=\{1,2,3,4,5\}$, $A\cap B=\{3,5\}$, $A\setminus B=\{1\}$,
$B\setminus A=\{2,4\}$. $A=?$, $B=?$
\item $A\cup B=\{1,2,3,4,5\}$, $A\setminus B=\{1,4\}$, $A\cap B\not\subset\{3,4,5\}$, $|A|=|B|$. $A=?$, $B=?$
\item Hány páros számú elemet tartalmazó részhalmaza van az $A=\{1,2,3,4,5\}$ halmaznak? És hány páratlan számú elemet tartalmazó részhalmaza van?
\item ($*$) Legyen $M=\{\frac{2}{1},\frac{4}{3},\frac{6}{5},\frac{8}{7},\ldots,\frac{1002}{1001}\}$. Hány olyan $x$ eleme van az $M$ halmaznak, amelyekre $|x-1|<0{,}1$?
\item Legyenek $A$, $B$ és $C$ olyan halmazok, amelyekre $A\cap C = B\cap C$ és
$A\setminus C=B\setminus C$. Igazoljuk, hogy ekkor $A=B$.
\item ($*$) Mutass példát három olyan halmazra, hogy bármely kettőnek végtelen sok közös eleme van, de a három halmaz közös része üres.
\item Legyen $A=\{x\in \mathbb{R}~|~x|\le 1\}$, $B=\{y\in \mathbb{R}~|~|y|\le 1\}$.
Milyen ponthalmazt határoznak meg azok az $(x;y)$ koordinátapárok, amelyek az $A\times B$ halmaz elemeit alkotják?
\end{enumerate}

\subsection*{Témazáró -- 2006.06.06.}

\begin{enumerate}
\item Írjuk fel külön-külön sorba az $A=\{1,3,5,7\}$ halmaz 0; 1; 2; 3; 4; elemű részhalmazait!
\item Adjunk meg három olyan kételemű halmazt, melyeknek páronként vett közös része nem üres halmaz, de mindhárom halmaznak nincs közös eleme.
\item Az $A$, $B$, $C$ halmazok mindegyikének 5 eleme van, $|A\cap B \cap C|=1$.
Mekkora lehet $A\cup B \cup C$ maximális és minimális elemszáma?
\item Legyen $A=\{x\in\mathbb{N} ~|~ 2x\le 4x-6\}$ és $B=\{x\in\mathbb{N}~|~4x-11\le 2x+11\}$. Mik az elemei az $A\cap B$ halmaznak?
\item Hány eleme van annak a $P(x;y)$ pontokból halmaznak, amelyre teljesül, hogy $|x|+|y|\le 5$ és $x,y$ egész számok? Ábrázold a halmazt!
\item A $H$ halmaz a következő:
$$H=\left\{\frac{2}{3},\frac{4}{5},\frac{6}{7},\ldots,\frac{400}{401}\right\}.$$
Hány olyan $x$ eleme van $H$-nak, amelyre igaz, hogy $|x-1|<\frac{1}{100}$?
\end{enumerate}

\end{document}