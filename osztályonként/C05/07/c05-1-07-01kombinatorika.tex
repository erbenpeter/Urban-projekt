\documentclass{article}
\usepackage[utf8]{inputenc}
\usepackage{t1enc}
\usepackage{geometry}
 \geometry{
 a4paper,
 total={210mm,297mm},
 left=20mm,
 right=20mm,
 top=20mm,
 bottom=20mm,
 }
\usepackage{amsmath}
\usepackage{amssymb}
\frenchspacing
\usepackage{fancyhdr}
\pagestyle{fancy}
\lhead{Urbán János tanár úr feladatsorai}
\chead{C05/07/1.}
\rhead{Kombinatorika}
\lfoot{}
\cfoot{\thepage}
\rfoot{}

\usepackage{pgf,tikz}

\usepackage{enumitem}
\usepackage{multicol}
\usepackage{calc}
\newenvironment{abc}{\begin{enumerate}[label=\textit{\alph*})]}{\end{enumerate}}
\newenvironment{abc2}{\begin{enumerate}[label=\textit{\alph*})]\begin{multicols}{2}}{\end{multicols}\end{enumerate}}
\newenvironment{abc3}{\begin{enumerate}[label=\textit{\alph*})]\begin{multicols}{3}}{\end{multicols}\end{enumerate}}
\newenvironment{abc4}{\begin{enumerate}[label=\textit{\alph*})]\begin{multicols}{4}}{\end{multicols}\end{enumerate}}
\newenvironment{abcn}[1]{\begin{enumerate}[label=\textit{\alph*})]\begin{multicols}{#1}}{\end{multicols}\end{enumerate}}
\setlist[enumerate,1]{listparindent=\labelwidth+\labelsep}

\newcommand{\degre}{\ensuremath{^\circ}}
\newcommand{\tg}{\mathop{\mathrm{tg}}\nolimits}
\newcommand{\ctg}{\mathop{\mathrm{ctg}}\nolimits}
\newcommand{\arc}{\mathop{\mathrm{arc}}\nolimits}
\renewcommand{\arcsin}{\arc\sin}
\renewcommand{\arccos}{\arc\cos}
\newcommand{\arctg}{\arc\tg}
\newcommand{\arcctg}{\arc\ctg}

\parskip 8pt
\begin{document}

\section*{Kombinatorika}

\subsection*{2005.09.13.}
\begin{enumerate}
\item Állíts össze 15 dominóból egy $5\times 6$-os
téglalapot úgy, hogy a téglalap bármelyik oldalával
párhuzamos egyenes minden esetben legalább egy dominót kettévágjon!
\item Az előző feladat eredménye alapján 20 dominóból állíts össze $5\times 8$-as téglalapot ugyanilyen módon!
\item 24 dominóból állíts össze $6\times 8$-as téglalapot az előző két feladatban megismert tulajdonsággal!
\item Összeállítható-e téglalap az ötféle \textit{tetraminóból} (mindegyikből csak egy példányunk van)?
\item Állíts össze egy $6\times 10$-es téglalapot
a 12 \textit{pentaminóból} (mindegyikből csak egy példányunk van)?
\item A $8\times 8$-as sakktábla egyik sarokmezőjét kihagyjuk. Lefedhető-e a megmaradt rész 21 darab ilyen \textit{tri\-minóval}: 
\tikz[x=0.2cm,y=0.2cm]{
\draw (0.,4.)-- (3.,4.);
\draw (3.,4.)-- (3.,3.);
\draw (3.,3.)-- (0.,3.);
\draw (0.,3.)-- (0.,4.);
\draw (1.,4.)-- (1.,3.);
\draw (2.,4.)-- (2.,3.);
}
?
\end{enumerate}

\subsection*{2005.09.15.}
\begin{enumerate}
\item Lefedhető-e a szokásos $8\times 8$-as sakktábla 15 T alakú és egy
\tikz[x=0.2cm,y=0.2cm]{
\draw (0.,0.)-- (2.,0.);
\draw (2.,0.)-- (2.,2.);
\draw (2.,2.)-- (0.,2.);
\draw (0.,2.)-- (0.,0.);
\draw (1.,0.)-- (1.,2.);
\draw (0.,1.)-- (2.,1.);
}
alakú tetraminóval?
\item Állíts össze a 12 pentaminóból $5\times 12$-es
téglalapot!
\item A sakktábla 4 sarokmezőjét kivágtuk.
A megmaradt részt fedjétek le a 12 pentaminóval!
\item A 12 pentaminóból állíts össze 3 darab $7\times 3$-as téglalapot úgy, hogy mindegyik téglalapból 1 mező üresen marad! 
\item A $8\times 8$-as sakktáblán jelölj ki 16 mezőt úgy, hogy a kimaradt részre ne lehessen letenni a következő pentaminókat: U, Y, I, L.
\end{enumerate}


\subsection*{2005.09.20.}
\begin{enumerate}
\item  Legfeljebb hány bástya helyezhető el a sakktáblán úgy, hogy egyik se üsse a másikat?
\item Legfeljebb hány futó helyezhető el a sakktáblán úgy, hogy egyik se üsse a másikat?
\item Legfeljebb hány király helyezhető el a sakktáblán úgy, hogy egyik se üsse a másikat?
\item Legfeljebb hány ló helyezhető el a sakktáblán úgy, hogy egyik se üsse a másikat?
\item Legfeljebb hány királynő helyezhető el a sakktáblán úgy, hogy egyik se üsse a másikat?
\item Legfeljebb hány
\begin{abcn}{5}
\item királyt;
\item bástyát;
\item futót;
\item lovat;
\item királynőt
\end{abcn}
kell elhelyezni a sakktáblán úgy, hogy minden
szabad mezőt ütés alatt tartsanak?
\end{enumerate}


\subsection*{2005.09.21.}
\begin{enumerate}
\item Egy kocka lapjait pirosra és kékre festjük. Hányféleképpen lehetséges ez, ha az elmozgatással egymásba vihető színezéseket nem tekintjük különbözőeknek?

\item Legfeljebb hány részre osztja a síkot 3 egyenes, 4 egyenes, 5 egyenes?

\item Egy négyzetet fel lehet-e darabolni 4, 6, 7, 8 négyzetre?

\item Egy $8 \times 8$-as sakktábla bal alsó sarkából el lehet-e jutni a jobb felső sarokba úgy, hogy közben minden sorba pontosan egyszer lépünk?

\item Egy $8 \times 8$-as sakktábla egyik mezőjét letakarjuk. Mutassuk meg, hogy a maradék rész lefedhető 21 darab
\tikz[x=0.2cm,y=0.2cm]{
\draw (-3.,5.)-- (-3.,3.);
\draw (-3.,3.)-- (-1.,3.);
\draw (-1.,3.)-- (-1.,4.);
\draw (-1.,4.)-- (-3.,4.);
\draw (-2.,5.)-- (-2.,3.);
\draw (-3.,5.)-- (-2.,5.);}
alakzattal.

\item Egy kockát fel lehet-e darabolni 27, 34, 90 kisebb kockára?

\item Adott a síkon 8 egyenes. Legfeljebb hány metszéspontjuk lehet?

\end{enumerate}


\subsection*{2005.09.27.}
\begin{enumerate}
\item Hány olyan háromjegyű szám van, amelynek jegyei között csak az 1, 2, 3 szerepel?

\item Hány olyan háromjegyű szám van, amelynek jegyei között csak a 0, 1, 2, 3, szerepel?

\item Hány olyan ötjegyű páros szám van, amelynek jegyei között nincs 5-nél nagyobb számjegy?

\item Hányféleképpen választhatunk ki az 1 és 10 közti egész számokból három számot?

\item Hány olyan téglatest van, amelynek élei egész hosszúságúak és legfeljebb 10 egység a hosszuk?

\item Egy piros és egy zöld kockával dobunk, leírjuk egymás mellé a kapott eredményt. Hányféle kétjegyű számot kaphatunk?

\item Hányféleképpen választhatunk ki az első 30 pozitív egész számból hármat úgy, hogy az összegük osztható legyen 3-mal?
\end{enumerate}


\subsection*{2005.09.28.}
\begin{enumerate}
\item Hány olyan 0001 és 9999 közötti egész szám van, amelyben két jegy összege megegyezik a harmadik és a negyedik jegy összegével?

\item Mennyi az 1, 2, 3, 4, 5, 6 számjegyekkel felírható hatjegyű számok összege?

\item Legfeljebb hány metszéspontot határoznak meg a konvex 7-szög, konvex 8-szög átlói?

\item Hány átlója van egy konvex 7, 8, 9-szögnek, 
$n$-szögnek?

\item Hány olyan téglatest van, amelynek az élei egész hosszúságúak és 1 és 20 közé esnek?

\item Hány olyan 4 jegyű szám van, amelynek a számjegyei 1,2,3 közül kerülnek ki, és osztható 
3-mal?
\end{enumerate}


\subsection*{2005.09.29.}
\begin{enumerate}
\item Mennyi az 1, 2, 3, 4 számjegyekkel felírható négyjegyű számok összege?

\item Hány 4-gyel osztható, 4-jegyű szám készíthető az 1, 2, 3, 4, 5 számjegyekből, ha ezek mindegyike többször is felhasználható?

\item Hány különböző forgalmi rendszám készíthető, ha egy rendszám 3 betűből és 3 számjegyből áll, és 26 betűből lehet választani, más megkötés nincs.

\item Hányféleképpen lehet a 10-et pozitív egész számok összegére bontani, ha a sorrendben különböző felbontásokat különbözőnek tekintjük? És a 20-at?

\item Hányféleképpen lehet a 10-et nemnegatív egész számok összegére bontani, ha a sorrendben különböző felbontásokat is különbözőnek tekintjük? És a 20-at?
\end{enumerate}


\subsection*{2005.10.05.}
\begin{enumerate}
\item Az 5-nél nem nagyobb számjegyekkel elkészítjük az összes 4 jegyű számot, amelyekben nincs ismétlődő számjegy. Mennyi ezek összege?

\item Egy konvex tízszög átlói legfeljebb hány metszéspontot határoznak meg a sokszög belsejében?

\item Egy piros és egy zöld dobókockával dobunk, a dobott számokat összeadjuk. Milyen számokat kaphatunk és melyiket hányféleképpen?

\item Egy 10 lépcsőfokból álló lépcsőn úgy mehetünk fel, hogy egyszerre 1 vagy 2 lépcsőfokot lépük. Hányféleképpen mehetünk fel?

\item Hány részre osztják a teret a kocka lapsíkjai?

\item Hány olyan ötjegyű szám van, amelyekben az 5-nél nem nagyobb számjegyek szerepelnek és osztható 4-gyel?
\end{enumerate}


\subsection*{2005.10.06. -- Kombinatorika dolgozat}
\begin{enumerate}
\item Mennyi az 1, 2, 3, 4 számjegyekből készíthető
négyjegyű számok összege? Minden számjegy csak egyszer szerepelhet a számban.
\item Hány olyan 4-jegyű szám van, amelyben a számjegyek összege páros?
\item Legfeljebb hány részre osztja a síkot 6 egyenes?
\item Hányféleképpen lehet a 12-t három pozitív egész szám összegére bontani, ha a csak sorrendben különböző felbontásokat is különbözőnek tekintjük?
\item Egy kocka lapjait három színnel, pirossal, kékkel és zölddel színezzük. A színezéshez mind a három színt felhasználjuk.
Hányféle kockát kaphatunk, ha az elforgatással egymásba vihetőket nem tekintjük különbözőnek?
\end{enumerate}


\end{document}
