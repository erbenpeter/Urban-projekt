\documentclass{article}
\usepackage[utf8]{inputenc}
\usepackage{t1enc}
\usepackage{geometry}
 \geometry{
 a4paper,
 total={210mm,297mm},
 left=20mm,
 right=20mm,
 top=20mm,
 bottom=20mm,
 }
\usepackage{amsmath}
\usepackage{amssymb}
\frenchspacing
\usepackage{fancyhdr}
\pagestyle{fancy}
\lhead{Urbán János tanár úr feladatsorai}
\chead{C05/09/1.}
\rhead{Trigonometria}
\lfoot{}
\cfoot{\thepage}
\rfoot{}

\usepackage{enumitem}
\usepackage{multicol}
\usepackage{calc}
\newenvironment{abc}{\begin{enumerate}[label=\textit{\alph*})]}{\end{enumerate}}
\newenvironment{abc2}{\begin{enumerate}[label=\textit{\alph*})]\begin{multicols}{2}}{\end{multicols}\end{enumerate}}
\newenvironment{abc3}{\begin{enumerate}[label=\textit{\alph*})]\begin{multicols}{3}}{\end{multicols}\end{enumerate}}
\newenvironment{abc4}{\begin{enumerate}[label=\textit{\alph*})]\begin{multicols}{4}}{\end{multicols}\end{enumerate}}
\newenvironment{abcn}[1]{\begin{enumerate}[label=\textit{\alph*})]\begin{multicols}{#1}}{\end{multicols}\end{enumerate}}
\setlist[enumerate,1]{listparindent=\labelwidth+\labelsep}

\newcommand{\degre}{\ensuremath{^\circ}}
\newcommand{\tg}{\mathop{\mathrm{tg}}\nolimits}
\newcommand{\ctg}{\mathop{\mathrm{ctg}}\nolimits}
\newcommand{\arc}{\mathop{\mathrm{arc}}\nolimits}
\renewcommand{\arcsin}{\arc\sin}
\renewcommand{\arccos}{\arc\cos}
\newcommand{\arctg}{\arc\tg}
\newcommand{\arcctg}{\arc\ctg}

\parskip 8pt
\begin{document}

\section*{Trigonometria}

\subsection*{2007. 11. 19. -- Trigonometria}
%ru3022
%KL/AM
\begin{enumerate}
\item Ábrázoljuk a következő függvényeket:
\begin{abc4}
\item $x \mapsto \sin 2x$;
\item $x \mapsto \sin \frac{1}{2}x$;
\item $x \mapsto |\sin x|$;
\item $x \mapsto \sin |x|$.
\end{abc4}
\item Igazoljuk, hogy
\begin{abc}
\item $\displaystyle{\frac{1}{\cos^2\alpha}=\tg^2\alpha+1}$, ha $\displaystyle{\alpha \ne \frac{\pi}{2}+n\pi}$, $\alpha \in \mathbb{Z}$;
\item $\displaystyle{\frac{1}{\sin^2\alpha}=\ctg^2\alpha+1}$, ha $\alpha \ne n\pi$, $n \in \mathbb{Z}$.
\end{abc}
\item Ábrázoljuk:
\begin{abc}
\item $x \mapsto \tg 2x$; \qquad $\displaystyle{x \ne \frac{\pi}{4}+k\frac{\pi}{2}}$, $k \in \mathbb{Z}$;
\item $x \mapsto \tg |x|$;  \qquad $\displaystyle{x \ne \frac{\pi}{2}+n\pi}$, $n \in \mathbb{Z}$;
\item $x \mapsto \ctg \frac{1}{2}x$; \qquad $\displaystyle{x \ne 2n\pi}$, $n \in \mathbb{Z}$.
\end{abc}
\item Oldjuk meg a kövekező egyenleteket:
\begin{abc4}
\item $\displaystyle{\sin x=\frac{1}{2}}$;
\item $\displaystyle{\cos x=\frac{\sqrt{3}}{2}}$;
\item $\displaystyle{\tg x=1}$;
\item $\displaystyle{\ctg x=\sqrt{3}}$.
\end{abc4}
\end{enumerate}

\subsection*{2007. 11. 20. -- Trigonometriai azonosságok}
%ru3023
%KL
\begin{enumerate}
\item Igazoljuk a következő azonosságokat:
\begin{abc}
\item $\sin(\alpha \pm \beta)=\sin\alpha \cos\beta \pm \cos\alpha \sin\beta$;
\item $\cos(\alpha \pm \beta)=\cos\alpha\cos\beta \mp \sin\alpha\sin\beta$;
\item $\sin 2\alpha=2\sin\alpha\cos\alpha$;
\item $\cos 2\alpha=\cos^2\alpha-\sin^2\alpha$;
\item $\displaystyle{\tg(\alpha \pm \beta)=\frac{\tg\alpha \pm \tg\beta}{1 \mp \tg\alpha\tg\beta}}$;
\item $\displaystyle{\tg 2\alpha=\frac{2\tg\alpha}{1-\tg^2\alpha}}$;
\item $\sin 3\alpha=3\sin\alpha-4\sin^3\alpha$;
\item $\cos 3\alpha=4\cos^3\alpha-3\cos\alpha$;
\item $\displaystyle{\sin^2\alpha=\frac{1-\cos2\alpha}{2}}$;
\item $\displaystyle{\cos^2\alpha=\frac{1+\cos2\alpha}{2}}$.
\end{abc}
\end{enumerate}

\subsection*{2007. 11. 26.}
%ru3024
%SB
\begin{enumerate}
\item Igazoljuk a következő azonosságokat:
\begin{abc}
\item $\sin^6x+\cos^6x=1-\displaystyle{\frac{3}{4}}\sin^22x$;
\item $\tg3x=\tg x\tg\left(\dfrac{\pi}{3}-x\right)\tg\left(\dfrac{\pi}{3}+x\right)$, ha minden tényezőnek van értelme;
\item $\tg\dfrac{\alpha}{2}\tg\dfrac{\beta}{2}+\tg\dfrac{\beta}{2}\tg\dfrac{\gamma}{2}\tg\dfrac{\gamma}{2}\tg\dfrac{\alpha}{2}=1$, ha $\alpha$, $\beta$, $\gamma$ egy háromszög szögei.
\end{abc}
\item Igazoljuk, hogy $\tg20^{\circ}\cdot\tg40^{\circ}\cdot\tg80^{\circ}=\sqrt3$.
\item Oldjuk meg a következő egyenleteket:
\begin{abc}
\item $\sin^3x\cos x-\sin x\cos^3x=\displaystyle{\frac{1}{4}}$;
\item $\displaystyle{\frac{1-\tg x}{1+\tg x}}=1+\sin^2x$;
\item $1+\sin x+\cos x+\sin2x+\cos2x=0$;
\item $\sin^3x+\cos^3x=1-\displaystyle{\frac{1}{2}}\sin 2x$;
\item $\ctg^2x=\displaystyle{\frac{1+\sin x}{1+\cos x}}$.
\end{abc}
\end{enumerate}

\subsection*{2007. 11. 28. -- Egyenletek}
%ru3025
%SB
\begin{enumerate}
\item Oldjuk meg a következő egyenleteket:
\begin{abc}
\item $\displaystyle{\frac{1+\tg x}{1-\tg x}}=1+\sin 2x$;
\item $\displaystyle{\frac{1}{2}}\left(\sin^4x+\cos^4x\right)=\sin^2x\cos^2x+\sin x\cos x$;
\item $\ctg^2x=\displaystyle{\frac{1+\sin x}{1+\cos x}}$;
\item ($*$) $\displaystyle{\sin^5x-\cos^5x=\frac{1}{\cos x}-\frac{1}{\sin x}}$;
\item $(\sin x\cdot\cos x)\cdot\sqrt2=\tg x + \ctg x$;
\item ($*$) $\left(\sin x+\sqrt3\cos x\right)\sin 4x = 2$;
\item $2\ctg2x-3\ctg3x=\tg2x$;
\item $2+\cos x=2\tg\displaystyle{\frac{x}{2}}$.
\end{abc}
\end{enumerate}

\subsection*{2007. 11. 28. -- Trigonometrikus azonosságok II.}
%ru3026
%SB
\begin{enumerate}
\item Igazoljuk a következő azonosságokat:
\begin{abc}
\item $\sin x+\sin y=2\sin\displaystyle{\frac{x+y}{2}}\cos\displaystyle{\frac{x-y}{2}}$;
\item $\sin x-\sin y=2\cos\displaystyle{\frac{x+y}{2}}\sin\displaystyle{\frac{x-y}{2}}$;
\item $\cos x+\cos y=2\cos\displaystyle{\frac{x+y}{2}}\cos\displaystyle{\frac{x-y}{2}}$;
\item $\cos x-\cos y=-2\sin\displaystyle{\frac{x+y}{2}}\sin\displaystyle{\frac{x-y}{2}}$;
\item $\sin x\sin y=\displaystyle{\frac{1}{2}}(\cos(x-y)-\cos(x+y))$;
\item $\cos x\cos y=\displaystyle{\frac{1}{2}}(\cos(x-y)+\cos(x+y))$;
\item $\sin x\cos y=\displaystyle{\frac{1}{2}}(\sin(x-y)+\sin(x+y))$;
\item $\sin x=\displaystyle{\frac{2\tg\displaystyle{\frac{x}{2}}}{1+\tg^2\displaystyle{\frac{x}{2}}}}$;
\item $\cos x=\displaystyle{\frac{1-\tg\displaystyle{\frac{x}{2}}}{1+\tg^2\displaystyle{\frac{x}{2}}}}$;
\item $\tg x=\displaystyle{\frac{2\tg\displaystyle{\frac{x}{2}}}{1-\tg^2\displaystyle{\frac{x}{2}}}}$.
\end{abc}
\end{enumerate}

\subsection*{2007. 12. 04. -- Azonosságok és egyenletek}
%ru3027
%SB
\begin{enumerate}
\item Igazoljuk, hogy $\cos\displaystyle{\frac{\pi}{7}}\cdot\cos\displaystyle{\frac{4\pi}{7}}\cdot\cos\displaystyle{\frac{5\pi}{7}}=\displaystyle{\frac{1}{8}}$.
\item Oldjuk meg:
\begin{abc}
\item $\tg3x=\tg x$;
\item $\cos3x+\sin5x=0$;
\item $\sin2x\sin6x=\cos x\cos3x$;
\item $2\cos^2x-1=\sin3x$;
\item $\sin x+\sin2x+\sin3x+\sin4x=0$.
\end{abc}
\item Oldjuk meg a következő egyenlőtlenségeket:
\begin{abc}
\item $\sin3x<\sin x$;
\item $\tg^2x-\left(1+\sqrt3\right)\tg x+\sqrt3<0$;
\item $\sin x+\sqrt3\cos x>0$;
\item $2\cos^2x+5\cos x+2\ge0$.
\end{abc}
\end{enumerate}

\subsection*{2007. 12. 05. -- Újabb feladatok}
%ru3028
%KL
\begin{enumerate}
\item Oldjuk meg a következő egyenleteket:
\begin{abc}
\item $\tg x+\tg 2x+\tg 3x+\tg 4x=0$;
\item $\sin^8x+\cos^8x=\frac{17}{32}$;
\item $\left(\sin x+\sqrt{3}\cos x\right)\sin 4x=2$;
\item ($*$) $\sin^{2n}x+\cos^{2n}x=1$, $n>0$ egész;
\end{abc}
\item Oldjuk meg az egyenletrendszert!

$\left.
\begin{aligned}
\tg x+\tg y&=1\\
\cos x\cos y&=\frac{1}{\sqrt{2}}
\end{aligned}
\right\}$
\item Oldjuk meg az egyenlőtlenségeket:
\begin{abc}
\item $\displaystyle{\tg \frac{x}{2}>\frac{\tg x-2}{\tg x+2}}$;
\item $\displaystyle{\cos^3x\cos 3x-\sin^3x\sin 3x>\frac{5}{8}}$.
\end{abc}
\item Igazoljuk, hogy ha $0<x<\frac{\pi}{4}$, akkor

$\displaystyle{\frac{\cos x}{\sin^2x(\cos x-\sin x)}>8}$.
\end{enumerate}

\subsection*{2007. 12. 12. -- Ismétlő feladatok}
%ru3029
%KL
\begin{enumerate}
\item Oldjuk meg a következő egyenleteket:
\begin{abc}
\item $\sin 2x=\sqrt{2}\cos x$;
\item $\sin x\cos x\sin 2x=\frac{1}{8}$;
\item $\tg 2x=3\tg x$;
\item $\sin^6x+\cos^6x=\frac{7}{16}$.
\end{abc}
\item Oldjuk meg a következő egyenlőtlenségeket:
\begin{abc}
\item $\displaystyle{\sin x\cdot\left(\cos x+\frac{1}{2}\right)\le 0}$;
\item $\sin x < \cos x$;
\item $\sqrt{5-2\sin x}\ge 6\sin x-1$.
\end{abc}
\item Oldjuk meg a következő egyenletrendszereket!
\begin{abc}
\item $\left.
\begin{aligned}
x+y&=\frac{\pi}{3}\\
\sin x+\cos y&=1
\end{aligned}
\right\}$
\item $\left.
\begin{aligned}
\sin x\sin y&=\frac{1}{4}\\
\cos x\cos y&=\frac{3}{4}
\end{aligned}
\right\}$
\item $\left.
\begin{aligned}
\sin x+\cos y&=0\\
\sin^2x+\cos^2y&=\frac{1}{2}
\end{aligned}
\right\}$, $0<x, y<\pi$.
\end{abc}
\end{enumerate}

\subsection*{2007. 12. 17. -- Trigonometria I. (dolgozat)}
%ru3030
%KL
\begin{enumerate}
\item Ábrázoljuk a következő függvényeket:
\begin{abc}
\item $x \mapsto |\sin x|$, $\quad$ $x \in \mathbb{R}$;
\item $x \mapsto \tg |x|$, $\quad$ $x \in \mathbb{R}$, $x \ne \frac{\pi}{2}+k\pi$, $k \in \mathbb{Z}$.
\end{abc}
\item Igazoljuk a következő azonosságokat:
\begin{abc}
\item $\cos 5\alpha=\cos^5\alpha-10\cos^3\alpha \sin^2\alpha+5\cos\alpha \sin^4\alpha$;
\item $\displaystyle{\cos\frac{2\pi}{5}+\cos\frac{4\pi}{5}=-\frac{1}{2}}$.
\end{abc}
\item Oldjuk meg a következő egyenleteket:
\begin{abc}
\item $\cos 3x+\sin 2x-\sin 4x=0$;
\item $\sin^4x+\cos^4x=\sin x\cos x$.
\end{abc}
\end{enumerate}

\subsection*{2007. 12. 19. -- További trigonometria feladatok}
%ru3031
%KL
\begin{enumerate}
\item ($*$) Igazoljuk, hogy ha $n \ge 1$, egész, akkor

$\sin^{2n}x+\cos^{2n}x \ge \displaystyle{\frac{1}{2^{n-1}}}$.
\item Oldjuk meg:

$\sin^{2007}x+\cos^{2007}x=1$.
\item Igazoljuk:

$\sin^6x+\cos^6x \ge \frac{1}{4}$.
\item Igazoljuk:

$\displaystyle{\frac{1}{2\sin 10^{\circ}}-2\sin 70^{\circ}=1}$.
\item Hány gyöke van az

$\displaystyle{|\sin x|=\frac{2x}{201\pi}}$ egyenletnek?
\item Oldjuk meg:

$\left.
\begin{aligned}
\sin^3x&=\frac{1}{2}\sin y\\
\cos^3x&=\frac{1}{2}\cos y
\end{aligned}
\right\}$, $0<x,y<\frac{\pi}{2}$.
\item Írjuk egyszerűbb alakba:

$\cos\alpha \cdot \cos2\alpha \cdot \cos4\alpha \cdot\ldots\cdot \cos\left(2^{n-1}\alpha\right)$.
\end{enumerate}

\subsection*{2008. 01. 07.}
%ru3032
%AM
\begin{enumerate}
\item Oldjuk meg:
\begin{abc4}
\item $\ctg^2x+\ctg x\geq0$;
\item $\sin x+\cos x<\sqrt{2}$;
\item $\displaystyle\cos x\cdot|\cos x|\leq\frac{1}{2}$;
\item $\sin x+\sqrt{3}\cos x>0$.
\end{abc4}
\item Igazoljuk, hogy ha $0<x<\dfrac{\pi}{2}$, akkor

$\left(1+\dfrac{1}{\sin x}\right)\left(1+\dfrac{1}{\cos x}\right)>5$.
\item Igazoljuk, hogy ha $x,y,z>0$, $x+y+z=\pi$, akkor

$\displaystyle\sin\frac{x}{2}\cdot\displaystyle\sin\frac{y}{2}\cdot\displaystyle\sin\frac{z}{2}\displaystyle<\frac{1}{4}.$

\item Oldjuk meg az egyenletrendszert!

$\left.
\begin{aligned}
\tg x+\tg y&=2\\
\tg 2x+\tg 2y&=1
\end{aligned}
\right\}$
\end{enumerate}

\subsection*{2008. 01. 09.}
%ru3033
%AM
\begin{enumerate}
\item Oldjuk meg a következő egyenleteket:
\begin{abc2}
\item $\sin 4x+\cos2x=0$;
\item $\cos2x+3\sin x=1$;
\item $\sin^6x+\cos^6x=\displaystyle\frac{7}{16}$;
\item $\tg3x=\sin6x$;
\item $\sqrt{3}\sin x+\cos x=\sqrt{3}$.
\end{abc2}
\item Igazoljuk táblázat és zsebszámológép használata nélkül:

$\sin45^\circ=\sin75^\circ-\sin15^\circ.$
\item Igazoljuk:

$\displaystyle\frac{\cos\alpha+\sin\alpha}{\cos\alpha-\sin\alpha}=\tg2\alpha+\frac{1}{\cos2\alpha}$.

\item ($*$) Igazoljuk, hogy ha $\alpha$, $\beta$ és $\gamma$ egy háromszög szögei, akkor

$\displaystyle\frac{1}{\sin\alpha}+\frac{1}{\sin\beta}+\frac{1}{\sin\gamma}\geq2\sqrt{3}.$
 
\end{enumerate}

\subsection*{2008. 01. 14. -- Trigonometria II. (dolgozat)}
%ru3034
%AM
\begin{enumerate}
\item Oldjuk meg a következő egyenletet:
\begin{abc3}
\item $\sin x+\cos x=1+\sin2x$;
\item $\ctg x+\displaystyle\frac{\sin x}{1+\cos x}=2$;
\item $3\cos^2x=\sin^2x+\sin2x$.
\end{abc3}
\item Igazoljuk, hogy minden olyan $\alpha$-ra, amire értelmezve van mindkét oldal, igaz, hogy
$\displaystyle\frac{\tg2\alpha}{\tg\alpha}=1+\frac{1}{\cos2\alpha}$.
\item Oldjuk meg a következő egyenletrendszert:

$\displaystyle\sin^3x=\frac{1}{2}\sin y;$
$\displaystyle\cos^3x=\frac{1}{2}\cos y.$
\end{enumerate}

\subsection*{2008. 01. 16. -- Ismétlő feladatok}
%ru3035
%AM
\begin{enumerate}
\item Oldjuk meg az egyenleteket!
\begin{abc}
\item $5(1+\cos x)=2+\sin^4x-\cos^4x$
\item $\cos2x=2-5\cos x$
\item $\cos x=\sin|x|$
\item $\sqrt{3}|\tg x+\ctg x|=4$
\item $\sqrt{1-\cos^2x}=1-\sin|x|$
\end{abc}
\item Oldjuk meg az egyenlőtlenségeket!
\begin{abc}
\item $\tg^2x+(2-\sqrt{3})\tg x-2\sqrt{3}<0$
\item $3\sin x<\sqrt{3}\cos x$
\item $\displaystyle\frac{1}{\sqrt{6}}\leq\sin x<\dfrac{1}{2}$
\end{abc}
\end{enumerate}

\subsection*{2008. 01. 21.}
%ru3036
%SB
\begin{enumerate}
\item Ábrázoljuk a következő függvényeket:
\begin{abc}
\item $x \mapsto \sin(\arcsin x)$, $|x| \le 1$;
\item $x \mapsto \arcsin(\sin x)$, $x \in \mathbb{R}$;
\item $x \mapsto \tg(\arc \tg x)$, $x \in \mathbb{R}$;
\item $x \mapsto \arc \tg(\tg x)$, $x \in \mathbb{R} \setminus \left\{ \displaystyle{\frac{\pi}{2}}+k\pi^2\right\}$, $k\in\mathbb{Z}$.
\end{abc}
\item Számítsuk ki táblázat és zsebszámológép használata nélkül:
\begin{abc3}
\item $\arcsin\displaystyle{\frac{1}{2}}$;
\item $\arc \tg\sqrt3$;
\item $\arc \tg\displaystyle{\frac{2}{3}}+\arc \tg\displaystyle{\frac{1}{3}}$.
\end{abc3}
\item Értelmezzük és ábrázoljuk, jellemezzük az
\begin{abc}
\item $x \mapsto \arccos x$ függvényt;
\item $x \mapsto \arc\ctg x$ függvényt.
\end{abc}
\item Igazoljuk, hogy ha $k>0$ és egész, akkor $\arc\tg \displaystyle{\frac{1}{1+k+k^2}}=\arc\tg(k+1)-\arc\tg k$.
\item Oldjuk meg a valós számok körében: $\arc \tg x+ \arc \tg(1-x)=2\arc \tg\sqrt{x-x^2}$.
\end{enumerate}

\subsection*{2008. 01. 22.}
%ru3037
%AM
\begin{enumerate}
\item Számítsuk ki:
\begin{abc2}
\item $\arccos\left(\sin\left(\displaystyle-\frac{\pi}{7}\right)\right);$
\item $\arcsin\left(\cos\left(\displaystyle\frac{33}{5}\pi\right)\right).$ 
\end{abc2}
\item Igazoljuk:
\begin{abc}
\item $\displaystyle\arctan\frac{1}{3}+\arctan\frac{1}{5}+\arctan\frac{1}{7}+\arctan\frac{1}{8}=\frac{\pi}{8}$;
\item $\displaystyle\arcsin x+\arccos x=\frac{\pi}{2}$.
\end{abc}
\item Igazoljuk:
\begin{abc}
\item $\arccos x=\arcsin\sqrt{1-x^2 },$ ha $0\leq x \leq1$;
\item $\pi-\arcsin\sqrt{1-x^2},$ ha $-1\leq x <0$.
\end{abc}
\item Számítsuk ki:

$\displaystyle\sin\left(2\arctan\frac{1}{5}-\arctan\frac{5}{12}\right).$
\item Igazoljuk, hogy ha $\alpha$, $\beta$, $\gamma$ egy háromszög szögei, akkor

$\displaystyle\sin\frac{\alpha}{2}\sin\frac{\beta}{2}\sin\frac{\gamma}{2}\leq\frac{1}{8}.$
 \item Igazoljuk, hogy ha $|x|<1$, akkor
 
$\displaystyle\tg(\arcsin x)=\frac{x}{\sqrt{1-x^2}}.$
\end{enumerate}

\subsection*{2008. 01. 23.}
%ru3038
%KL
\begin{enumerate}
\item Ábrázoljuk a következő függvényeket:
\begin{abc}
\item $x \mapsto \arcsin(\cos x)$;
\item $x \mapsto \arctan x+\arctan \frac{1}{x}$, $x \ne 0$.
\end{abc}
\item Igazoljuk:
\begin{abc}
\item $3 \arccos x-\arccos\left(3x-4x^3\right)=\pi$, ha $|x| \le \dfrac{1}{2}$;
\item $2 \arctan x+\arcsin\displaystyle{\frac{2x}{1+x^2}}=\pi\sin x$, ha $|x| \ge 1$.
\end{abc}
\item Igazoljuk, hogy ha $\alpha$, $\beta$, $\gamma$ egy háromszög szögei, akkor

$\displaystyle{\cos\alpha+\cos\beta+\cos\gamma\le\frac{3}{2}}$.
\end{enumerate}

\subsection*{2008. 01. 29.}
%ru3039
%KL
\begin{enumerate}
\item Igazoljuk:
\begin{abc}
\item  $\displaystyle{\arctan\left(3+2\sqrt{2}\right)-\arctan\frac{\sqrt{2}}{2}=\frac{\pi}{4}}$;
\item $\displaystyle{\arcsin\frac{4}{5}+\arcsin\frac{5}{13}+\arcsin\frac{16}{65}=\frac{\pi}{2}}$.
\end{abc}
\item Oldjuk meg a valós számok halmazán:
\begin{abc}
\item $\displaystyle{\arctan(x+2)-\arctan(x+1)=\frac{\pi}{4}}$;
\item $\displaystyle{\arcsin 3x=\arccos 4x}$;
\item $\displaystyle{\arctan(x^2-3x-3)=\frac{\pi}{4}}$;
\item $\displaystyle{2\arcsin x=\arcsin \frac{10x}{13}}$.
\end{abc}
\item Oldjuk meg a következő egyenletrendszert:

$\left.
\begin{aligned}
x+y&=\arctan\dfrac{2a}{1-a^2}\\
\tg x \tg y&=a^2
\end{aligned}
\right\}$, $\displaystyle{|a|<1}$.
\end{enumerate}

\subsection*{2008. 02. 04. -- Ismétlő feladatok}
%ru3040
%AM
\begin{enumerate}
\item Igazoljuk, hogy ha $\alpha$, $\beta$, $\gamma$ egy háromszög szögei, akkor
\begin{abc2}
\item $\displaystyle\tg^2\frac{\alpha}{2}+\tg^2\frac{\beta}{2}+\tg^2\frac{\gamma}{2}\geq1$;
\item $\displaystyle\frac{1}{\sin\alpha}+\frac{1}{\sin\beta}+\frac{1}{\sin\gamma}\geq2\sqrt{3}$.
\end{abc2}
\item Oldjuk meg a valós számok halmazán:

$\displaystyle\left(\sin^4x+1\right)\left(\cos^4x+1\right)=\frac{25}{16}\cdot\sin^4 2x.$
\item Oldjuk meg a valós számok halmazán:

$\sqrt{3}\sin x+\cos x=\sqrt{3}.$
\item Igazoljuk, hogy minden $x$-re, amire értelme van,

$\displaystyle\tg\dfrac{x}{2}=\frac{1-\cos x}{\sin x}.$
\item Igazoljuk, hogy 

$\displaystyle\tg(\arcsin x)=\frac{x}{\sqrt{1-x^2}}$, $|x|<1$.
\end{enumerate}

\subsection*{2008. 02. 06. -- Trigonometria III. (dolgozat)}
%ru3041
%KL
\begin{enumerate}
\item Oldjuk meg a következő egyenleteket:
\begin{abc}
\item $\displaystyle{2\sin x+3\cos x=\frac{2}{\sin x}}$;
\item $\displaystyle{\sqrt{\frac{\sqrt{3}\tg x}{4}+1}+\cos x=0}$.
\item $\displaystyle{\tg(\arccos x)=\frac{\sqrt{1-x^2}}{x}}$.
\end{abc}
\item Igazoljuk, hogy ha $|x| \le 1$, akkor

$\displaystyle{\tg(\arccos x)=\frac{\sqrt{1-x^2}}{x}}$.

\item Igazoljuk a következő azonosságot:

$\displaystyle{\frac{1+\sin 2x}{\cos 2x}=\tg\left(\frac{\pi}{4}+\alpha\right)}$, ha mindkét oldalnak van értelme.
\item Bizonyítsuk be, hogy

$\displaystyle{\cos\frac{\pi}{5}+\cos\frac{3\pi}{5}=\frac{1}{2}}$.
\end{enumerate}

\subsection*{2008. 02. 11. -- Ismétlő feladatok}
%ru3042
%KL
\begin{enumerate}
\item Oldjuk meg a valós számok körében:
\begin{abc2}
\item $\displaystyle{\ctg\left(2\pi\cos^2 2\pi x\right)=0}$;
\item $\displaystyle{\sqrt{5-\cos^2x-4\sin x}=2-\sin x}$.
\end{abc2}
\item Igazoljuk, hogy

$\displaystyle{\ctg 70^{\circ}+4\cos 70^{\circ}=\sqrt{3}}$.

\item Oldjuk meg a valós számok halmazán:
\begin{abc4}
\item $\displaystyle{\cos 2x \ge \sin 2x}$;
\item $\displaystyle{\cos x \cdot |\cos x| \le \frac{1}{2}}$;
\item $\displaystyle{\sin x+\cos x<\sqrt{2}}$;
\item $\displaystyle{3 \sin x<\sqrt{3}\cos x}$.
\end{abc4}
\item Igazoljuk, hogy ha $\alpha$, $\beta$, $\gamma$ egy háromszög szögei, akkor

$\displaystyle{\sin\frac{\alpha}{2}\sin\frac{\beta}{2}\sin\frac{\gamma}{2}<\frac{1}{4}}$.
\end{enumerate}

\subsection*{2008. 02. 13. -- Trigonometria pótdolgozat}
%ru3043
%AM
\begin{enumerate}
\item Oldjuk meg a valós számok halmazán:
\begin{abc2}
\item $\displaystyle\ctg^2x=\frac{1+\sin x}{1+\cos x}$;
\item $\ctg x=1+2\sin2x$.
\end{abc2}
\item Melyek azok a $0\leq x<2\pi$ valós számok, amelyekre teljesül, hogy
$3\sin x>2\cos^2x$?
\item Igazoljuk, hogy

$\displaystyle\arctan\frac{1}{7}+2\arcsin\frac{1}{\sqrt{10}}=\frac{\pi}{4}.$
\item Oldjuk meg a valós számok körében:

$\left.
\begin{aligned}
\sin(x+y)&=0\\
\sin(x-y)&=0
\end{aligned}
\right\}$, $0\leq x;y\leq\pi$.
\end{enumerate}

\end{document}