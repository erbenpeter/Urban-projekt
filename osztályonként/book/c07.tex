\chapter{C07, 7. évfolyam}
\section{Számelmélet}





\subsection*{2007. 09. 05. -- Feladatlap}
\begin{enumerate}
\item Milyen számjegyre végződhet két egymás után következő természetes szám szorzata?
\item Milyen számjegyre végződhet egy természetes szám négyzete?
\item Hány olyan legfeljebb 3-jegyű pozitív egész szám van, amelyik nem osztható sem 2-vel, sem 5-tel?
\item Melyik az a legkisebb pozitív egész szám, amely 1-gyel kezdődik és ha az elejéről az 1-est a végére írjuk, akkor az így kapott szám az eredeti szám 3-szorosa lesz?
\item Hány 0-ra végződik az első 100 pozitív egész szám szorzata?
\item Melyik az a legnagyobb, tízes számrendszerben felírt háromjegyű páros szám, amely nem változik, ha az egyesek és a százasok helyén álló számjegyet felcseréljük?
\item Keressük meg a következő szorzások eredményeiben a közös tulajdonságot:

$12\cdot99$; $12\cdot999$; $12\cdot9999$; \ldots
\end{enumerate}


\subsection*{2007. 09. 06. -- Feladatlap}
\begin{enumerate}
\item Mi az utolsó számjegye:
\begin{abc}
\item $2^{20}$-nak;
\item $7^{20}$-nak.
\end{abc}
\item Mi az utolsó két számjegye $2^{100}$-nak?
\item Az 1-től 1000-ig terjedő egész számok szorzata 7-nek melyik legfeljebb mekkora kitevőjű hatványával osztható?
\item Hány 0-ra végződik az 1-nél nagyobb vagy egyenlő, legfeljebb háromjegyű egész számok szorzata?
\item Tudjuk, hogy $11^2=121$. Igaz-e, hogy $10201$, $1002001$, $10002001$ is négyzetszám?
\item Van-e olyan négyzetszám, amelynek a számjegyei valamilyen sorrendben $0$; $2$; $3$; $5$?
\item Mutassuk meg, hogy minden pozitív egész szám ötödik hatvány ugyanarra a számjegyre végződik, mint maga a szám!
\end{enumerate}


\subsection*{2007. 09. 10. -- Feladatlap}
\begin{enumerate}
\item Adjuk meg $p$ értékét úgy, hogy $p$, $p+4$, $p+14$ is prímszám legyen! 
\item Melyek azok a $p$ prímek, amelyekre $p+10$ és $p+14$ is prím?
\item Melyek azok a $p$ prímek, amelyekre igaz, hogy $8p^2+1$ is prímszám?
\item A következő számok közül melyek prímek és melyek összetett számok:

$12$, $17$, $121$, $203$, $217$, $251$, $348$, $757$, $991$.
\item Mennyi a következő számok 9-cel vett osztási maradéka:

$234$, $512$, $106$, $113$, $4132$, $9503$, $3246$.
\item Igazoljuk, hogy $37^{37}-23^{23}$ osztharó 10-zel.
\item Igazoljuk, hogy 3 egymást követő egész szám szorzata osztható 6-tal!
\item Igazoljuk, hogy $15 \mid 2^{16}-1$ és $17 \mid 2^{16}-1$.
\end{enumerate}


\subsection*{2007. 09. 11.}
\begin{enumerate}
\item Igazoljuk, hogy
\begin{abc3}
\item $24 \mid 5^{20}-1$;
\item $181 \mid 3^{105}+4^{105}$;
\item $13 \mid 2^{60}+7^{30}$.
\end{abc3}
\item Melyik az a legnagyobb szám, amellyen bármely három egymást követő páros szám szorzata osztható?
\item Igazoljuk, hogy
\begin{abc2}
\item $11 \mid 100^{100}-1$;
\item $15 \mid 1000^5-1000$.
\end{abc2}
\item Adott 11 darab, egyenként $1$, $2$, $3$, $4$, $\ldots$ , $11$ kg súlyú csomag. Osszuk ezeket 3 csoportra úgy, hogy az egyes csoportokba tartozó csomagok súlyának összege megegyezzen.
\item Bizonyítsuk be, hogy ha $n > 0$, egész szám, akkor $10^n+98$ osztható 18-cal!
\item Adjuk meg az összes $\overline{2ab}$ alakú tízes számrendszerbeli, 6-tal osztható háromjegyű számot!
\end{enumerate}


\subsection*{2007. 09. 13.}
\begin{enumerate}
\item Igazoljuk, hogy $11^{10}-1$ osztható 100-zal!
\item Mi a két utolsó számjegye:
\begin{abc2}
\item $2^{999}$-nek;
\item $3^{999}$-nek?
\end{abc2}
\item Bizonyos számú egymást követő pozitív egész szám összege 1000. Melyek ezek a számok?
\item Az első 100 pozitív egész számból válasszunk ki 51-et tetszőlegesen. Igazoljuk, hogy a kiválasztott számok között van kettő, hogy egyik osztója a másiknak.
\item Igazoljuk, hogy ha $p$ és $8p-1$ prímszámok, akkor $8p+1$ összetett szám.
\item Számítsuk ki azoknak a háromjegyű számoknak az összegét, amelyek nem nagyobbak 200-nál és 4-gyel osztva 1 maradékot adnak.
\item Melyik az a kétjegyű, tízes számrendszerbeli szám, amelyet megszorozva a számjegyei megfordításával kapott kétjegyű számmal eredményül 2430-at kapunk?
\end{enumerate}


\subsection*{2007. 09. 17. -- Oszthatóság}
\begin{enumerate}
\item Igazoljuk, hogy
\begin{abc3}
\item $3 \mid 2\cdot7^{100}+1$;
\item $7 \mid 3^{101}+2^{52}$;
\item $99 \mid 3^{53}\cdot2^{102}-108$.
\end{abc3}
\item Bizonyítsuk be, hogy
\begin{abc2}
\item $11 \mid 3^{102}+2^{301}$; 
\item $19 \mid 5^{99}\cdot2^{51}+3^{51}\cdot2^{99}$.
\end{abc2}
\item Igazoljuk, hogy
\begin{abc}
\item $7 \mid 1+2^5+3^5+4^5+5^5+6^5$;
\item $13 \mid 1+2^9+3^9+4^9+5^9+6^9+7^9+8^9+9^9+10^9+11^9+12^9$.
\end{abc}
\item Igazoljuk, hogy ha $p>3$ prímszám, akkor $p^2-1$ osztható 24-gyel!
\item Igazoljuk, hogy $11^{10}-1$ osztható 120-szal!
\item Igazoljuk, hogy $2222^{5555}+5555^{2222}$ osztható 7-tel!
\end{enumerate}


\subsection*{2007. 09. 26. -- Oszthatóság, prímszámok dolgozat}
\begin{enumerate}
\item A 43 elé és után írjatok egy-egy számjegyet úgy, hogy a kapott négyjegyű szám osztható legyen 45-tel!
\item Igazoljuk, hogy $72 \mid 10^{10}+8$.
\item Igazoljuk, hogy $7 \mid 2^{24}-1$.
\item Lehet-e két prímszám összege 2007?
\item A következő törtet egyszerűsítjük addig, amíg lehet:

\begin{center}
$\displaystyle{\frac{1\cdot2\cdot3\cdot4\cdot5\cdot\ldots\cdot99\cdot100}{2^{100}}}$
\end{center}

Mi lesz a kapott tört nevezője?
\item Igazoljuk, hogy ha egy háromjegyű számot kétszer egymás után írunk, akkor a kapott hatjegyű szám osztható 7-tel, 11-gyel és 13-mal!
\end{enumerate}


\subsection*{2007. 09. 30. -- Versenyfeladatok}
\begin{enumerate}
\item Keressünk három olyan számot, amelyek 9-jegyűek, minden 0-tól különböző számjegy pontosan egyszer szerepel mindegyikben és két szám összege a harmadik.
\item Hány olyan csupa különböző számjegyből álló ötjegyű szám van, amelyben a számjegyek csökkenő sorrendben állnak?
\item Adott öt szám. Ezekből képeztük az összes lehetséges háromtagú összeget. A következő összegeket kaptuk: 3, 4, 6, 7, 9, 10, 11, 14, 15 és 17. Mi volt a kiindulásul választott öt szám?
\item Egy háromszög egyik belső szöge $60\degre$-os, a szöget közrefogó oldalai pedig 2 és 3 egység hosszúak. Daraboljuk föl a háromszöget szakaszokkal három részre úgy, hogy a részekből egy szabályos hatszöget lehessen összerakni.
\item Adjunk meg 2009 darab (nem feltétlenül különböző) pozitív egész számot úgy, hogy az összegük egyenlő legyen a szorzatukkal!
\end{enumerate}


\subsection*{2007. 10. 03.}
\begin{enumerate}
\item Igazoljuk kongruenciával:
\begin{abc3}
\item $14 \mid 15^{231}-1$;
\item $16 \mid 15^{321}+1$;
\item $13 \mid 12^{1233}+14{1324}$;
\item $3 \mid 10{15}+17$;
\item $11 \mid 26^{30}-1$.
\end{abc3}
\item Mi az utolsó számjegye a követkető számoknak:
\begin{abc3}
\item $289^{289}$;
\item $2^{524}$;
\item $203^{203}$?
\end{abc3}
\item Igazoljuk, hogy ha $n>0$ egész szám, akkor
\begin{abc2}
\item $3 \mid 10^n+17$;
\item $10 \mid 3^{4n+3}-17$.
\end{abc2}
\end{enumerate}


\subsection*{2007. 10. 04. -- Gyakorló feladatok kongruenciával}
\begin{enumerate}
\item Igazoljuk, hogy
\begin{abc3}
\item $24 \mid 5^{20}-1$;
\item $13 \mid 2^{60}+7^{30}$;
\item $181 \mid 3^{105}+4^{105}$.
\end{abc3}
\item Igazoljuk, hogy ha $n>0$ tetszőleges egész szám, akkor
\begin{abc4}
\item $6 \mid 17^n-11^n$;
\item $15 \mid 2^{4n}-1$;
\item $5 \mid 4\cdot6^n+5^n-4$;
\item $7 \mid 3^{2n+1}+2^{n+2}$.
\end{abc4}
\item Mi az utolsó két számjegye a következő számnak:
\begin{abc}
\item $2^{2007}$;
\item $7+7^2+7^3+7^4+\ldots+7^{2007}$?
\end{abc}
\end{enumerate}


\subsection*{2007. 10. 10. -- Kongruenciák}
\begin{enumerate}
\item Igazoljuk kongruenciával:
\begin{abc}
\item $13 \mid 40^{10}-1$;
\item $10^{100}-7$ összetett szám.
\end{abc}
\item Mi lesz a $81^{89}+19^{89}$ szám utolsó két jegye?
\item Igazoljuk, hogy
\begin{abc}
\item $7 \mid 19\cdot8^n+23$, $n>0$, egész;
\item $18 \mid 17^{19}+19^{17}$.
\end{abc}
\item Kongruenciával bizonyítsuk be, hogy

$2008 \mid 2009^{2009}-2007^{2007}-2010$.
\end{enumerate}


\subsection*{2007. 10. 17.}
\begin{enumerate}
\item Számítsuk ki a következő számpárok legnagyobb közös osztóját:
\begin{abc3}
\item $(101;211)$;
\item $(567;1053)$;
\item $(875;2625)$.
\end{abc3}
\item Két pozitív egész szám összege 1323, legnagyobb közös osztójuk: 147. Melyek ezek a számok?
\item Igazoljuk, hogy $(2^n+1;2^n-1)=1$, ha $n>0$, egész szám
\item A következő számok közül keressük ki azokat a párokat, amelyek legnagyobb közös osztója 1:

3, 4, 6, 10, 15, 21, 28, 35, 42, 63.
\item Határozzuk meg a következő számpárok legkisebb közös többszörösét:
\begin{abc4}
\item $[16;28]$;
\item $[105;180]$;
\item $[475;570]$;
\item $[348;476]$.
\end{abc4}
\end{enumerate}


\subsection*{2007. 10. 18.}
\begin{enumerate}
\item Hány pozitív egész osztója van a következő számoknak: 2, 6, 8, 12, 100, 625?
\item Jelölje $d(n)$ az $n>0$ egész szám pozitív osztóinak számát. Keressünk képletet $d(n)$-re!
\item Melyik az a legkisebb szám, amelyre igaz, hogy a pozitív osztóinak száma: 
\begin{abc3}
\item 10;
\item 12;
\item 18.
\end{abc3}
\item Mik azok az $a,b>0$ egész számok, amelyekre igaz, hogy $(a;b)=10$ és $a\cdot b=2400$?
\item Hány olyan számpár van, amelyre $(a;b)=4$ és $a+b=100$?
\item Milyen $a,b>0$ egész számokra igaz, hogy $[a;b]=720$ és $a+b=98$?
\end{enumerate}


\subsection*{2007. 10. 24.}
\begin{enumerate}
\item Határozzuk meg a hiányzó számjegyeket:
\begin{abc3}
\item $36 \mid \overline{52x2y}$;
\item $45 \mid \overline{24x68y}$;
\item $99 \mid \overline{62xy427}$.
\end{abc3}
\item Igazoljuk, hogy a következő számok összetett számok:
\begin{abc2}
\item $4^{20}-1$;
\item $10^{100}-7$.
\end{abc2}
\item Négy egymást követő pozitív egész szám szorzata 3024. Melyek ezek a számok?
\item Lehet-e egyszerre prímszám a következő három szám: $n+5$, $n+7$, $n+15$ ($n>0$, egész)?
\item Hány pozitív egész osztója van:
\begin{abc3}
\item 100-nak;
\item 289-nek;
\item 2007-nek?
\end{abc3}
\item Melyek azok a háromjegyű számok, amelyek osztóinak száma 5?
\end{enumerate}


\subsection*{2007. 11. 05.}
\begin{enumerate}
\item Az 1, 2, 3, 4, 5, 6 számjegyeket milyen sorrendben kell leírni ahhoz, hogy a legnagyobb 12-vel osztható hatjegyű számot kapjuk?
\item Melyek azok a kétjegyű számok, amelyeknek a legtöbb osztójuk van?
\item Mely egész $n$-re lehet az $\displaystyle{\frac{n+11}{n-9}}$ törtet egyszerűsíteni? Mikor kaphatunk egész számot? 
\item Van-e olyan $n>0$ egész, amelyre a $\displaystyle{\frac{12n+1}{30n+2}}$ tört egyszerűsíthető?
\item Melyek azok a háromjegyű számok, amelyeknek 5 osztójuk van?
\item Hány osztója van a következő számoknak:
\begin{abc3}
\item $777777$;
\item $10^{100}$;
\item $5^{5^5}$.
\end{abc3}
\item Mutassuk meg, hogy hét négyzetszám között van kettő, amelyek különbsége osztható 10-zel!
\end{enumerate}


\subsection*{2007. 11. 07.}
\begin{enumerate}
\item Melyek azok a 4-jegyű számok, amelyeknek páratlan számú osztójuk van?
\item Igazoljuk, hogy

$7 \mid 333^{444}+444^{333}$.
\item Osztható-e $\displaystyle{\frac{1000!}{(500!)^2}}$ 7-tel?
\item Igazoljuk, hogy ha $a$-t és $1000a$-t 111-gyel osztjuk, akkor a kapott maradékok egyenlők.
\item Igazoljuk, hogy ha $n>0$ egész, akkor a $\displaystyle{\frac{21n+4}{14n+3}}$ tört nem egyszerűsíthető.
\item Mely $n>0$ egészekre igaz, hogy $2^n-1$ osztható 7-tel?
\item Igazoljuk, hogy ha $n>0$ egész, akkor

$40 \mid 11^{2n}+31^{2n}+38\cdot11^n\cdot31^n$.
\end{enumerate}


\subsection*{2007. 11. 08. -- Ismétlő feladatok}
\begin{enumerate}
\item Igazoljuk, hogy ha $n>0$ egész, akkor $19\cdot8^{2n}+17$ összetett szám.
\item Bizonyítsuk be:
\begin{abc2}
\item $40 \mid 1+3+3^2+3^3+3^4+\ldots+3^{99}$;
\item $35 \mid 3^{6n}-2^{6n}$.
\end{abc2}
\item A hatjegyű $\overline{523ABC}$ szám osztható 7-tel, 8-cal és 9-cel. Mik lehetnek az A, B és C számjegyek?
\item Igazoljuk, hogy ha $p$ és $8p-1$ prímek, akkor $8p+1$ összetett szám.
\item Az $a$, $b$, $c$ különböző számjegyek. Igazoljuk, hogy az ezekből készíthető összes háromjegyű szám összege osztható $a+b+c$-vel!
\item Mennyi az első 10 pozitív egész szám legnagyobb közös osztója és legkisebb közös többszöröse?
\item Két pozitív egész szorzata 7875, legnagyobb közös osztója 15. Mi lehet a két szám?
\end{enumerate}


\subsection*{2007. 11. 12. -- Számelmélet dolgozat}
\begin{enumerate}
\item Határozzuk meg az $a$ és $b$ számokat, ha
\begin{abc}
\item $(a;b)=15$ és $[a;b]=420$;
\item $(a;b)=37$ és $a+b=666$.
\end{abc}
\item Milyen $n>0$ egész számokra egyszerűsíthető a következő tört:

$\displaystyle{\frac{5n+4}{8n+7}}$?
\item Melyik az a pozitív egész szám, amely osztható 12-vel és 14 osztója van?
\item Hány 0-ra végződik $123!$, azaz az $1\cdot2\cdot3\cdot4\cdot\ldots\cdot122\cdot123$ szorzás eredménye?
\item Igazoljuk, hogy $4^{90}+1$ osztható 17-tel!
\item Igazoljuk, hogy $2^n-1$ és $2^n+1$ legnagyobb közös osztója 1, ha $n>0$ egész szám!
\end{enumerate}


\subsection*{2007. 11. 15.}
\begin{enumerate}
\item $(a;b)=13$; $[a;b]=1989$; $a=?$ $b=?$
\item $(x;60)=15$, mi lehet $x$ értéke?
\item $(x;20)=1$, mi lehet $x$ értéke?
\item Adjuk meg a következő számok legnagyobb közös osztóját:
\begin{abc2}
\item $(17;34;263)$;
\item $(187;323;391)$;
\end{abc2}
\item Mennyi a következő számok pozitív egész osztóinak szorzata:
\begin{abc3}
\item $72$;
\item $180$;
\item $625$.
\end{abc3}
\item Melyik az a pozitív egész szám, amelynek 6 osztója van és a pozitív osztói szorzata 91125?
\item A $2^4\cdot3^2\cdot5$ számnak hány olyan osztója van, ami \underline{nem osztható} 15-tel?
\end{enumerate}


\subsection*{2007. 11. 19. -- Pótdolgozat}
\begin{enumerate}
\item Melyek azok az $a$, $b$ pozitív egész számok, amelyekre $(a;b)=10$ és $a\cdot b=2400$?
\item Két pozitív egész szám összege 1323, a két szám legnagyobb közös osztója 147. Melyik ez a két szám?
\item Van-e olyan $n>0$ egész szám, amelyre a $\displaystyle{\frac{2n+5}{3n+7}}$ tört egyszerűsíthető?
\item Hány olyan pozitív osztója van 2160-nak, ami \underline{nem osztható} 15-tel?
\item Határozzuk meg azt a pozitív egész számot, amelynek 15 pozitív osztója van, és ezek szorzata $2^{60}\cdot5^{15}$.
\item Melyek azok a négyjegyű, öttel osztható tízes számrendszerbeli számok, amelyeknek 15 osztójuk van?
\end{enumerate}


\subsection*{2007. 11. 22. -- Számrendszerek}
\begin{enumerate}
\item Írjuk fel tízes számrendszerben:
\begin{abc3}
\item $543_{\underline{6}}$;
\item $1212_{\underline{3}}$;
\item $8888_{\underline{9}}$.
\end{abc3}
\item Írjuk fel 8-as számrendszerben a következő, tízes számrendszerben felírt számokat:
\begin{abc3}
\item $63$;
\item $79$;
\item $2007$.
\end{abc3}
\item Fogalmazzuk meg a 8-as számrendszerben a 7-tel való oszthatóság szabályát!
\item Bizonyítsuk be, hogy $144_{\underline{a}}$ bármilyen $a>4$ számrendszerben négyzetszám.
\item Végezzük el a következő összeadást:

$23334_{\underline{6}}+444_{\underline{6}}+12341_{\underline{6}}$.
\end{enumerate}


\subsection*{2007. 11. 26.}
\begin{enumerate}
\item Írjuk át tízes számrendszerbe a következő számokat:
\begin{abc4}
\item $26014_{\underline{7}}$;
\item $11001101_{\underline{2}}$;
\item $42123_{\underline{5}}$;
\item $53041_{\underline{6}}$.
\end{abc4}
\item Végezzük el a következő műveleteket:
\begin{abc}
\item $10111_{\underline{2}}+1100_{\underline{2}}+1010101_{\underline{2}}=$
\item $1234_{\underline{5}}\cdot322_{\underline{5}}=$
\item $351_{\underline{6}}\cdot14_{\underline{6}}=$
\end{abc}
\item Határozzuk meg $x$ értékét:
\begin{abc3}
\item $106_{\underline{x}}=153_{\underline{7}}$;
\item $324_{\underline{x}}=10022_{\underline{3}}$;
\item $541_{\underline{x}}=2014_{6}$.
\end{abc3}
\item Milyen számrendszerben igaz:
\begin{abc3}
\item $35+40=115$;
\item $342+63=425$;
\item $216\cdot3=654$.
\end{abc3}
\end{enumerate}


\subsection*{2007. 11. 28.}
\begin{enumerate}
\item Mi lehet $x$ értéke:
\begin{abc2}
\item $41_{\underline{8}}=201_{\underline{x}}$;
\item $10110001_{\underline{2}}=342_{\underline{x}}$?
\end{abc2}
\item Egy számrendszerben $4^2=20_{\underline{x}}$. Mennyi ebben a számrendszerben $5^2$?
\item Mennyi x értéke, ha $13_{\underline{x}}$ és $31_{\underline{x}}$ 2-nek két egymás után következő hatványa?
\item Számítsuk ki:
\begin{abc}
\item $2334_{\underline{6}}+33020_{\underline{6}}+444_{\underline{6}}+12341_{\underline{6}}$;
\item $\overline{43ab5}_{\underline{12}}+\overline{3a6}_{\underline{12}}+\overline{4b25}_{\underline{12}}$, ahol $a$ értéke 10, $b$ értéke 11;
\item $234134_{\underline{7}}\cdot123_{\underline{7}}$.
\end{abc}
\item Milyen alapú számrendszerben igazak a következő egyenlőségek: 
\begin{abc}
\item $12_{\underline{x}}+13_{\underline{x}}=30_{\underline{x}}$;
\item $17_{\underline{x}}+38_{\underline{x}}=54_{\underline{x}}$;
\item $6_{\underline{x}}\cdot 6_{\underline{x}}=51_{\underline{x}}$;
\item $100_{\underline{x}}-1_{\underline{x}}=11_{\underline{x}}$;
\item $12_{\underline{x}}\cdot 7_{\underline{x}}=80_{\underline{x}}$.
\end{abc}
\end{enumerate}


\subsection*{2007. 11. 29.}
\begin{enumerate}
\item Az $111_{\underline{x}}$ szám $x=2, 3, \ldots , 9$ esetén mikor lesz prímszám?
\item Igaz-e, hogy $34780160_{\underline{9}}$ szám osztható 27-tel?
\item Mi a feltétele annak, hogy egy 6-os számrendszerben felírt szám osztható legyen 3-mal?
\item Mi a feltétele annak, hogy
\begin{abc}
\item egy 5-ös számrendszerben felírt szám osztható legyen 4-gyel;
\item egy 9-es számrendszerben felírt szám osztható legyen 8-cal?
\end{abc}
\item Milyen számjegyet írhatunk az üres helyre, hogy
\begin{abc}
\item $1827\Box_{\underline{9}}$ osztható legyen 8-cal és 3-mal;
\item $2465\Box_{\underline{8}}$ osztható legyen 7-tal és 4-gyel;
\item $3925\Box_{\underline{12}}$ osztható legyen 11-gyel és 3-mal?
\end{abc}
\end{enumerate}


\subsection*{2007. 12. 03. -- Számrendszerek dolgozat}
\begin{enumerate}
\item Írjuk fel 7-es alapú számrendszerbe a következő számokat:
\begin{abc2}
\item $2007$;
\item $12345$.
\end{abc2}
\item Végezzük el a következő műveleteket:
\begin{abc2}
\item $12127_{\underline{11}}+16_{\underline{11}}+999_{\underline{11}}$;
\item $452_{\underline{7}}\cdot231_{\underline{7}}$.
\end{abc2}
\item Határozzuk meg x értékét:
\begin{abc3}
\item $106_{\underline{x}}=153_{\underline{7}}$;
\item $236_{\underline{x}}=1240_{\underline{5}}$;
\item $216_{\underline{x}}\cdot3_{\underline{x}}=654_{\underline{x}}$.
\end{abc3}
\item Egy tízes számrendszerbeli háromjegyű szám $\overline{aaa}$ alakú. Ugyanaz a szám egy más alapú számrendszerben szintén háromjegyű, de $\overline{(4a)(2a)a}_{\underline{x}}$ alakú, ahol $4a$ és $2a$ is egy egy számjegy. Határozzuk meg $x$ és $a$ értékét!
\item Milyen számjegyet írhatunk az üres helyre, ha tudjuk, hogy $1223\Box_{\underline{5}}$ osztható 20-szal?
\end{enumerate}

\chapter{C07, 8. évfolyam}

\section{Számelmélet}

\subsection*{2008.09.04. - Feladatok}
\begin{enumerate}
\item Alakítsuk át tizedes törtté a következő törteket:

\begin{abcn}{6}
\item $\dfrac{1}{2}$; 
\item $\dfrac{2}{5}$;
\item $\dfrac{1}{20}$;
\item $\dfrac{4}{25}$;
\item $\dfrac{3}{40}$;
\item $\dfrac{7}{80}$.
\end{abcn}

\item Írjuk át közönséges törtté a következő tizedes törteket:

\begin{abc3}
\item $0,12$; 
\item $0,23$; 
\item $0,65$.
\end{abc3}

\item Írjuk át tizedes törtté a következő törteket:

\begin{abcn}{6}
\item $\dfrac{1}{9}$;
\item $\dfrac{2}{3}$; 
\item $\dfrac{1}{11}$;
\item $\dfrac{1}{7}$;
\item $\dfrac{3}{7}$;
\item $\dfrac{5}{7}$.
\end{abcn}

\item ($*$) Igazoljuk ,,szemléletesen'', hogy $\dfrac{1}{9}=0,1111\ldots$
\end{enumerate}

\subsection*{2008.09.09. - Számfogalom}
\begin{enumerate}

\item Igazoljuk, hogy a $\sqrt{6}$ irracionális szám.
\item Igazoljuk, hogy a $\sqrt{2}-1$ szám irracionális.
\item Lehet-e két irracionális szám összege; szorzata racionális?
\item Igazoljuk a következő összefüggéseket:

\begin{abc2}
\item $\sqrt{8}+\sqrt{18}=\sqrt{50}$;
\item $\sqrt{10+4\sqrt{6}}-\sqrt{10-4\sqrt{6}}=\sqrt{4}$;
\item $\sqrt{\dfrac{2+\sqrt{3}}{2-\sqrt{3}}}+\sqrt{\dfrac{2-\sqrt{3}}{2+\sqrt{3}}}=4$.
\end{abc2} 
 
\item Igazoljuk, hogy $\root3\of{2}$ irracionális.
\item Számítsuk ki:

\begin{abc2}
\item $\sqrt{7+2\sqrt{6}}-\sqrt{7-2\sqrt{6}}$;
\item $\sqrt{2+\sqrt{3}}+\sqrt{2-\sqrt{3}}$.
\end{abc2}
\end{enumerate}
\subsection*{2008.09.10.}
\begin{enumerate}
\item Számítsuk ki a következő összeget:

$\dfrac{1}{\sqrt{2}+1}+\dfrac{1}{\sqrt{3}+\sqrt{2}}+\dfrac{1}{\sqrt{4}+\sqrt{3}}+\dfrac{1}{\sqrt{5}+\sqrt{4}}+\ldots+\dfrac{1}{\sqrt{100}+\sqrt{99}}$.
\item Számítsuk ki a következő összeget:

$\dfrac{1}{2\sqrt{1}+1\sqrt{2}}+\dfrac{1}{3\sqrt{2}+2\sqrt{3}}+\ldots+\dfrac{1}{11\sqrt{10}+10\sqrt{11}}$.
\item Számítsuk ki:

\begin{abc3}
\item $\sqrt{4+\sqrt{7}}-\sqrt{4-\sqrt{7}}$;
\item $\sqrt{7+4\sqrt{3}}+\sqrt{7-4\sqrt{3}}$;
\item $\sqrt{12}+\sqrt{75}-\sqrt{147}$.
\end{abc3}

\item ($*$) Igazoljuk, hogyha $n\geq1$ egész szám, akkor $[\sqrt{n}+\sqrt{n+1}]=[\sqrt{4n+2}].$
\item Számítsuk ki:  $(\root3\of{24}+\root3\of{81}-\root3\of{192})\cdot(\root3\of{375}-\root3\of{192})$.
\end{enumerate}

\subsection*{2008.09.15.}
\begin{enumerate}
\item Számítsuk ki a következő kifejezések pontos értékét:

\begin{abc}
\item  $\left(\dfrac{8}{\sqrt{5}+3}-\dfrac{20}{\sqrt{5}-3}-\dfrac{4}{\sqrt{5}-2}\right)\cdot(29+\sqrt{5})$;

\item   $\dfrac{6+4\sqrt{2}}{4}+\dfrac{3-6\sqrt{2}}{6}$;

\item   $(3\sqrt{2}-2\sqrt{3})\cdot(3\sqrt{2}+2\sqrt{3})$;

\item   $(3\sqrt{5}+2\sqrt{20})\cdot (\sqrt{45}+2\sqrt{5}-\sqrt{125})$;

\item   $\sqrt{2+\sqrt{3}}\cdot \sqrt{2+\sqrt{2+\sqrt{3}}}\cdot \sqrt{2+\sqrt{2+\sqrt{2+\sqrt{3}}}}\cdot \sqrt{2-\sqrt{2+\sqrt{2+\sqrt{3}}}}$ ;

\item   $\left(\dfrac{5}{\sqrt{3}-\sqrt{2}})-\dfrac{3}{\sqrt{3}+\sqrt{2}}\right)\cdot(\sqrt{2}+4\sqrt{2})$;

\item   $\left(\dfrac{1}{\sqrt{7}-2}+\dfrac{3\sqrt{7}}{\sqrt{7}+2}\right)\cdot(23+5\sqrt{7})$.
\end{abc}
\end{enumerate}

\subsection*{2008.09.16.}
\begin{enumerate}
\item Igazoljuk, hogy $\sqrt{2}+\sqrt{3}$ irracionális szám.
\item Számítsuk ki:

\begin{abc2}
\item $\left(\sqrt{4+\sqrt{7}}+\sqrt{4-\sqrt{7}}\right)^2$; 
\item $\left(\sqrt{2\sqrt{3}+2\sqrt{2}}-\sqrt{2\sqrt{3}-2\sqrt{2}}\right)^2$;
\item $\left(\sqrt{15+\sqrt{100}}-\sqrt{15-10\sqrt{2}}\right)^2$;
\item $\left(\sqrt{12}+\sqrt{3}+1\right)^2$.
\end{abc2}
\item Melyik az a legkisebb pozitív egész $n$ szám, amelyre teljesül, hogy $\sqrt{n+1}-\sqrt{n}<10^{-2}$?
\item Számítsuk ki $[(\sqrt{2}+1)^n]$ értékét, ha $1\leq n\leq 10$.
\item ($*$) Mutassuk meg, hogy $(7+\sqrt{50})^3$ tizedestört alakjában a tizedes vessző után három $0$ áll!
\item Számítsuk ki:
\begin{abc2}
\item $(3-\sqrt{5})\cdot \root3\of{9+4\sqrt{5}}$;
\item $\sqrt{4\sqrt{5}+3}\cdot \sqrt{4\sqrt{5}-3}$.  
\end{abc2}

\end{enumerate}

\subsection*{2008.09.17.}
\begin{enumerate}
\item  Igazoljuk, hogy a következő számok racionálisak:

\begin{abc3}
\item $\left(\root6\of{27}-\sqrt{6\frac{3}{4}}\right)^2$;
\item $\sqrt{3+2\sqrt{2}}+\sqrt{3-2\sqrt{2}}$;
\item $\root3\of{5\sqrt{2}+7}-\root3\of{5\sqrt{2}-7}$.
\end{abc3}
\item Adjunk meg két szomszédos egész számot úgy, hogy a megadott számok ezek közé essenek:
\begin{abc3}
\item $\sqrt{15}-3$; 
\item $14-2\sqrt{6}$; 
\item $11-\sqrt{110}$.
\end{abc3}
\item ($*$) Az $A$ és $B$ egész számok, tudjuk még, hogy $A>B>0$. Igaz-e, hogy $\sqrt{A+B+\sqrt{4AB}}\cdot\sqrt{A+B-\sqrt{4AB}}$ is egész szám?
\item Tudjuk, hogy $\sqrt{18-4\sqrt{15}+2\sqrt{5}-4\sqrt{3}}=a+b\sqrt{3}+c\sqrt{5}$, ahol $a,b,c$ egész számok. Adjuk meg $a,b,c$ értékét!

\end{enumerate}

\subsection*{2008.09.22.}
\begin{enumerate}
\item Számítsuk ki:
\begin{abc2}
\item $\sqrt{7-4\sqrt{3}}+\sqrt{7+4\sqrt{3}}$;
\item $(\root3\of{24}+\root3\of{81}-\root9\of{192})\cdot \root3\of{3}$;
\item $\root3\of{10+6\sqrt{3}}+\root3\of{10-6\sqrt{3}}$;
\item $\sqrt{28+16\sqrt{3}}-\sqrt{28-16\sqrt{3}}$.
\end{abc2}
\item Számítsuk ki: $\sqrt{2+\sqrt{3}}\cdot \root3\of{\dfrac{\sqrt{2}(3\sqrt{3}-5)}{2}}$.
\item Melyik nagyobb $\sqrt{3}$ vagy $\root3\of{5}$. (Zsebszámológép használata nélkül!)
\item Állítsuk növekvő sorrendbe: $\sqrt{2},\root3\of{3},\root4\of{4},\root5\of{5}$.
\item Igazoljuk $\left(\dfrac{1}{\sqrt{5}-2}\right)^3-\left(\dfrac{1}{\sqrt{5}+2}\right)^3=76$.
\end{enumerate}

\subsection*{2008.09.23.}
\begin{enumerate}
\item Számítsuk ki:
\begin{abc2}
\item $\left(2\sqrt{3}-\sqrt{5}+\sqrt{12}\right)\left(\sqrt{48+\sqrt{5}}\right)$; 
\item $\sqrt{5\sqrt{3}+\sqrt{59}}\cdot \sqrt{5\sqrt{3}-\sqrt{59}}$;
\item $\left(\dfrac{6}{4+\sqrt{10}}+\dfrac{24}{8-\sqrt{40}}\right)\cdot\left(12-\sqrt{10}\right)$;
\item $\left(\dfrac{1}{\sqrt{7}-2}+\dfrac{3\sqrt{7}}{\sqrt{7}+2}\right)\cdot \left((23+5\sqrt{7}\right)$.
\end{abc2}
\item $\sqrt{7+2\sqrt{10}}=\sqrt{a}+\sqrt{b}$, ahol $a,b$ pozitív egészek; $a=?$, $b=?$
\item Igaz-e hogy $\left(3-\sqrt{5}\right)\cdot \left(\root3\of{9+4\sqrt{5}}\right)$ egész szám?
\item Milyen előjelű a következő szám: $\sqrt{3}-\sqrt{2-\sqrt{3}}-\sqrt{2+\sqrt{3}}$?
\end{enumerate}

\subsection*{2008.09.24. - Irracionális számok}
\begin{enumerate}
\item Igazoljuk, hogy $\sqrt{3}+\sqrt{5}$ irracionális szám.
\item Számítsuk ki:
\begin{abc3}
\item $\dfrac{(\sqrt{5}+\sqrt{3})(4-\sqrt{15})}{\sqrt{5}-\sqrt{3}}$;
\item $\sqrt{2\sqrt{2}+\sqrt{7}}+\sqrt{2\sqrt{2}-\sqrt{7}}$;
\item $(\sqrt{5+\sqrt{24}}-\sqrt{5-\sqrt{24}})$.
\end{abc3}
\item Igazoljuk, hogy ha $x\geq y\geq 0$, akkor $\sqrt{\sqrt{x}+\sqrt{y}}=\sqrt{\dfrac{\sqrt{x}+\sqrt{x-y}}{2}}+\sqrt{\dfrac{\sqrt{x}-\sqrt{x-y}}{2}}$.
\item ($*$) Igazoljuk, hogy bármely két racionális szám között van irracionális szám.
\end{enumerate}

\subsection*{2008.09.29.}
\begin{enumerate}
\item Számítsuk ki:
\begin{abc2}
\item $(\sqrt{5}-2)^2$;
\item $(2\sqrt{3}+3\sqrt{2})^2$;
\item $(2\sqrt{3}-\sqrt{5}+\sqrt{12})\cdot (\sqrt{48}+\sqrt{5})$;
\item $\sqrt{5\sqrt{3}+\sqrt{59}}\cdot \sqrt{5\sqrt{3}-\sqrt{59}}$;
\item $\left(\dfrac{6}{4+\sqrt{10}}+\dfrac{24}{8-\sqrt{40}}\right)\cdot (12-\sqrt{10})$.
\end{abc2}
\item Végezzük el:
\begin{abc2}
\item $(\sqrt{2}-\root4\of{3})\cdot (\sqrt{3}+\root4\of{2})$;
\item $\sqrt{2+\sqrt{3}}\cdot \root3\of{\dfrac{\sqrt{2}(3\sqrt{3}-5)}{2}}$.
\end{abc2}
\item Igazoljuk: $\root3\of{1-12\cdot \root3\of{7}+6\cdot \root3\of{49}}+\root3\of{7}=2-\root3\of{7}$
\end{enumerate}

\subsection*{2008.09.30. - Pótdolgozat}
\begin{enumerate}
\item Számítsuk ki:
\begin{abc}
\item  $\dfrac{\sqrt{2+\sqrt{2}}-\sqrt{2-\sqrt{2}}}{\sqrt{2+\sqrt{2}}+\sqrt{2-\sqrt{2}}}$;
\item $\left(\sqrt{8+2\sqrt{10+2\sqrt{5}}}+\sqrt{8-2\sqrt{10+2\sqrt{5}}}\right)^2$;
\item $\sqrt{6+\sqrt{11}}\cdot \sqrt{3+\sqrt{3+\sqrt{11}}}\cdot \sqrt{3-\sqrt{3+\sqrt{11}}}$.
 \end{abc}
 \item Melyik nagyobb: $3\sqrt{2}-2\sqrt{3}$ vagy $\sqrt{30-12\sqrt{6}}$?
 \item Igazoljuk: $\dfrac{2-\sqrt{3}}{\sqrt{2}-\sqrt{2-\sqrt{3}}}+\dfrac{2+\sqrt{3}}{\sqrt{2}+\sqrt{2+\sqrt{3}}}=\sqrt{2}$.
 \item ($*$) Igazoljuk, hogy $\root3\of{1-27\cdot\root3\of{26}+9\cdot \root3\of{26^2}}+\root3\of{26}$ egész szám!
\end{enumerate}

\subsection*{2008.10.07. - Versenyfeladatok}
\begin{enumerate}
\item A $0$-tól különböző számjegyek $S$ halmazát fel lehet-e bontani két részhalmazra úgy, hogy egyikre se legyen igaz a következő tulajdonság: a részhalmaz két elemmel együtt tartalmazza azok különbségét is?
\item Egy kockát mind a $6$ lapjára tükrözzük. Az eredeti kockával együtt így egy új testet kapunk. Hányszorosa a kapott test felszíne az eredeti kocka felszínének?
\item $14$ különböző pozitív egész szám összege $110$. Melyek ezek a számok?
\item A pozitív egész számokat felírtuk egy nagy papírra $1$-től $10000$-ig, ezután kihúztuk azokat, amelyekben szerepel a $0$ vagy a $9$ számjegy. Hány szám maradt?
\item Egy szabályos hatszög minden oldalát öt egyenlő részre osztjuk. Hány olyan háromszög van, amelynek csúcsai az így kapott osztópontok közül kerülnek ki?
\end{enumerate}

\subsection*{2008.10.13.}
\begin{enumerate}
\item Ábrázoljuk a következő függvényeket:
\begin{abc3}
\item $x\mapsto\sqrt{x}$, $x\geq0$;
\item $x\mapsto\sqrt{x-3}$, $x\geq3$;
\item $x\mapsto\sqrt{x+2}$, $x\geq-2$; 
\item $x\mapsto\sqrt{-x}$, $x\leq0$;
\item $x\mapsto\sqrt{3-x}$, $x\leq3$;
\item $x\mapsto\sqrt{|x|}$;
\item $x\mapsto\sqrt{x-[x]}$;
\item $x\mapsto[x]+\sqrt{x-[x]}$;
\item $x\mapsto\sqrt{x^2-2x+1}$;
\item $x\mapsto\sqrt{x^2-6x+9}$.
\end{abc3}
\end{enumerate}

\subsection*{2008.10.14.}
\begin{enumerate}
\item Ábrázoljuk és jellemezzük a következő függvényeket (növekedés, fogyás, szélsőérték): 
\begin{abc3}
\item $x\mapsto-\sqrt{x}$, $x\geq0 $;
\item $x\mapsto\sqrt{2x+1}$, $x\geq-\dfrac{1}{2}$;
\item $x\mapsto\sqrt{x^4+2x^2+1}$;
\item $x\mapsto\sqrt{|x-3|}$;
\item $x\mapsto\sqrt{4x-2}$, $x\geq\dfrac{1}{2}$.
\end{abc3}
\item Igazoljuk a következő állítást: ha $a,b\geq0$, akkor $\dfrac{a+b}{2}\geq\sqrt{ab}$, és itt az "$=$" csak akkor igaz, ha $a=b$.
\item Igazoljuk, hog yha $a,b\geq0$, akkor $\dfrac{a+b}{2}\leq\sqrt{\dfrac{a^2+b^2}{2}}$, és egyenlőség csak $a=b$ esetén igaz.
\end{enumerate}

\subsection*{2008.10.14. - Versenyfeladatok}
\begin{enumerate}
\item Igazoljuk, hogy az $1+2+3+4+\ldots+n$ összeg értékének tízes számrendszerbeli alakjában az egyesek helyén álló számjegy periodikusan ismétlődik.
\item Igazoljuk, hogy $11$ pozitív egész szám között mindig van kettő, amelyek különbsége osztható $10$-zel!
\item Adott a síkon húsz pont. Ezek közül bizonyos pontpárokat összekötöttünk szakaszokkal. Igazoljuk, hogy mindig van van a $20$ pont között kettő olyan, amelyekből azonos számú szakasz indul ki.
\item Egy ötemeletes házat hányféleképpen tudunk kifesteni, ha minden emeletet vagy fehérre, vagy zöldre festhetünk, de két fehér emelet nem kerülhet egymás fölé? Oldjuk meg a feladatot $6,7,8$ emeletes házra is!
\item Egy paralelogramma egyik átlóján kiválasztunk egy pontot és ezen át párhuzamosokat húzunk az oldalakkal. Igazoljuk, hogy a bevonalazott két kis paralelogramma területe egyenlő! 
\end{enumerate}
\subsection*{2008.10.15.}
\begin{enumerate}
\item Igazoljuk az u.n. ,,rendezési tételt'' $3$ valós szám esetére: T.f.h. $a_1\geq a_2\geq a_3$ és $b_1\geq b_2\geq b_3$, ekkor $a_1b_1+a_2b_2+a_3b_3$ a maximális, és $a_1b_3+a_2b_2+a_3b_1$ a minimális az összes lehetséges hasonló szorzatok összege közül. 
\item  Igazoljuk a következő egyenlőtlenségeket: 
\begin{abc}
\item Ha $a,b,c>0$ valós számok, akkor $a^2+b^2+c^2\geq ab+bc+ac$;

\item  ha $a,b,c$ tetszőleges valós számok, akkor $a^4+b^4+c^4\geq a^2b^2+b^2c^2+a^2c^2$;

\item ($*$) ha $a\geq b\geq c>0$, akkor $\dfrac{a}{b+c}+\dfrac{b}{a+c}+\dfrac{c}{a+b}\geq \dfrac{3}{2}$;

\item ($*$) ha $a,b,c>0$ valós számok, akkor $\dfrac{1}{a}+\dfrac{1}{b}+\dfrac{1}{c}\leq \dfrac{a^8+b^8+c^8}{a^3b^3c^3}$. 
\end{abc}
\end{enumerate}
\subsection*{2008.10.16.}
\begin{enumerate}
\item Vázoljuk fel a következő függvények grafikonját:
\begin{abc3}
\item $x\mapsto x+\dfrac{1}{x}$, $x\neq0$;
\item $x\mapsto 2x+\dfrac{1}{x}$, $x\neq 0$;
\item $x\mapsto \dfrac{1}{x^2+1}$. 
\end{abc3}
\item Bizonyítsuk be, hogy az azonos kerületű téglalapok közül legnagyobb területű a négyzet.
\item Igazoljuk, hogyha $a,b>0$ és $a+b=1$, akkor $\left(a+\dfrac{1}{a}\right)^2+\left(b+\dfrac{1}{b}\right)^2\geq\dfrac{25}{2}$.
\item Határozzuk meg a következő függvény legkisebb értékét: $x\mapsto \dfrac{9+4x^2}{x^2}$, $x\neq0$.
\item Igazoljuk, hogy ha $a,b,c>0$, akkor $\dfrac{ab}{c}+\dfrac{bc}{a}+\dfrac{ca}{b}\geq a+b+c$.
\end{enumerate}
\subsection*{2008.10.20.}
\begin{enumerate}
\item Igazoljuk, hogy ha $a,b,c>0$, akkor $\dfrac{1}{ab}+\dfrac{1}{bc}+\dfrac{1}{ca}\leq \dfrac{1}{a}+\dfrac{1}{b}+\dfrac{1}{c}$.
\item Igazoljuk, hogy ha $a,b,c>0$, akkor $a\sqrt{bc}+b\sqrt{ac}+c\sqrt{ab}\leq ab+ac+bc$.
\item ($*$) Tudjuk, hogy $x,y,z>0$, és $\dfrac{1}{2x}+\dfrac{1}{3y}+\dfrac{1}{6z}=\dfrac{1}{\dfrac{x}{2}+\dfrac{y}{3}+\dfrac{z}{6}}$. Igazoljuk, hogy $x=y=z$.
\item Igazoljuk, hogy ha $a\geq b\geq c>0$, akkor $a+b+c\leq \dfrac{a^2+b^2}{2c}+\dfrac{b^2+c^2}{2a}+\dfrac{c^2+a^2}{2b}$.
\item Mennyi az $f(x)=\dfrac{x^2}{x^4+16}$, $x>0$ függvény legnagyobb értéke?
\end{enumerate}
\subsection*{2008.10.21.}
\begin{enumerate}
\item Tegyük fel, hogy $a,b,c,d\geq 0$ számok és igazoljuk, hogy $\dfrac{a+b+c+d}{4}\geq \root4\of{abcd}$, ahol az "$=$" csak akkor igaz, ha $a=b=c=d$.
\item Az előző feladat felhasználásával igazoljuk, hogy ha $a,b,c\geq 0$, akkor $\dfrac{a+b+c}{3}\geq \root3\of{abc}$, és "$=$" csak akkor igaz, ha $a=b=c$.
\item Egy $12$ egység oldalú négyzet alakú papírlapból fedél nélküli dobozt akarunk készíteni (négyzetes oszlop alakút). Milyen méretek esetén lesz a doboz térfogata maximális?
\item Igazoljuk, hogy ha $a,b,c>0$, akkor $\dfrac{a}{b}+\dfrac{b}{c}+\dfrac{c}{a}\geq 3$.
\item Bizonyítsuk be, hogy ha a tetszőleges valós szám, akkor $a^2+2\geq 2\sqrt{a^2+1}$.
\item Mutassuk meg, hogy ha $a,b,c>0$, akkor $\dfrac{1}{a}+\dfrac{1}{b}+\dfrac{1}{c}\geq\dfrac{9}{a+b+c}$.
\end{enumerate}
\subsection*{2008.10.21. - Versenyfeladatok}
\begin{enumerate}
\item Kati $2009$-ben éppen annyi éves lesz, mint születése évszámának számjegyei összege. Melyik évben született Kati?
\item Egy tömör kockát az egyik lapjával párhuzamos síkokkal rétegekre szeletelünk. Hány síkkal kell szétvágni a kockát ahhoz, hogy a keletkezett testek együttes felszíne $4$-szerese legyen a kocka felszínének?
\item A pozitív egész számokat $1$-től $100$-ig összeszorozzuk. Hány $0$-ra végződik a kapott szorzat?
\item Melyek azok az egymást követő pozitív egész számok, amelyeknek összege 45? 
\item Melyek azok az ötjegyű számok, amelyekre igaz, hogy mindegyik számjegye nagyobb, mint a tőle jobbra álló számjegyek összege?
\item Egy évben legfeljebb hány olyan hónap lehet, amelyben öt vasárnap van?
\item Egy kocka csúcsait pirosra és kékre festhetjük. Hány különböző kifestés lehetséges, ha két kifestést akkor tekintünk azonosnak, ha egymásba forgathatók?
\end{enumerate}
\subsection*{2008.10.22.}
\begin{enumerate}
\item Igazoljuk, hogy ha $a,b>0$, akkor $\left(\dfrac{a+b}{2}\right)^3\leq \dfrac{a^3+b^3}{2}$.  
\item Bizonyítsuk be, hogy ha $a+b=1$, akkor $a,b\geq 0$ esetén $a^4+b^4\geq \dfrac{1}{8}$.
\item Mutassuk meg, hogy ha $a\geq b\geq c>0$, akkor $a^4+b^4+c^4\geq abc\cdot(a+b+c)$.
\item Igazoljuk, hogy ha $a\geq b\geq c>0$, akkor $\dfrac{1}{a}+\dfrac{1}{b}+\dfrac{1}{c}\geq \dfrac{1}{\sqrt{bc}}+\dfrac{1}{\sqrt{ca}}+\dfrac{1}{\sqrt{ab}}$.
\item Azok közül a téglatestek közül, amelyeknek a felszíne $54$ területegység, melyiknek a térfogata a lehető legnagyobb?
\end{enumerate}
\subsection*{2008.11.03.}
\begin{enumerate}
\item Igazoljuk, hogy ha $a,b>0$, akkor $a^4+b^4\geq ab\cdot(a^2+b^2)$. 
\item Igazoljuk, hogy ha $x>0$, akkor $\dfrac{4}{x^2}+x\geq 3$. Mikor igaz az $=$ jel?
\item Azok közül a téglatestek közül, amelyeknek a térfogata $8$ egység, melyiknek a felszíne a legkisebb?
\item Igazoljuk, hogy bármely $a$ számra igaz, hogy $2a^2\leq 1+a^4$.
\item Bizonyítsuk be, hogy bármely $a,b,c$ pozitív számra igaz, hogy $a^3+b^3+c^3\geq 3abc$.
\item Bizonyítsuk be, hogy ha $a\neq 0$, akkor $\left|a+\dfrac{1}{a}\right|\geq 2$.
\end{enumerate}
\subsection*{2008.11.04.}
\begin{enumerate}
\item Igazoljuk, hogy $a^2+3c^2+b^2+1\geq 2ac+2bc+2c$ bármely $a,b,c$ számra.
\item Igazoljuk, hogy ha $a\geq b\geq c>0$, akkor $\dfrac{ab}{c}+\dfrac{bc}{a}+\dfrac{ca}{b}\geq a+b+c$.
\item Igazoljuk, hogy bármely $a,b,c>0$ számra igaz, hogy $\dfrac{a^2+b^2+c^2+3}{2}\geq a+b+c$.
\item Bizonyítsuk be, hogy ha $x,y\neq 0$ valós számok, és $\dfrac{1}{x^2}+\dfrac{1}{y^2}=2$, akkor $x^2+y^2\geq 2$.
\item Bizonyítsuk be, hogy ha $a,b,c\geq 0$ és $\sqrt{a}+\sqrt{b}+\sqrt{c}=1$, akkor $a+b+c\geq \dfrac{1}{3}$.
\item Igazoljuk, hogy ha $a\geq 0$, akkor: 
\begin{abc3}
\item $(a^3+a^2+a+1)^2\geq 16a^3$;
\item $a^4+a^3+4a+4\geq 8a^2$;
\item $2a^3+11\geq 9a$.
\end{abc3}
\end{enumerate}
\subsection*{2008.11.04. - Versenyfeladatok}
\begin{enumerate}
\item Írjuk fel a legnagyobb olyan tízes számrendszerbeli egész számot, melyben a harmadik jegytől kezdve (balról jobbra) minden számjegy az előző két számjegy összege.  
\item Egy szabályos ötszögnek meghúztuk mindegyik átlóját. Hány egyenlőszárú háromszög látható így az ábrán?
\item Egy négyzet oldalait $3-3$ egyenlő részre osztottuk, majd az osztópontokat az ábrán látható módon összekötöttük. Hányad része a bevonalazott négyszög területe a négyzet területének?

\centerline{
\definecolor{uuuuuu}{rgb}{0.26666666666666666,0.26666666666666666,0.26666666666666666}
\begin{tikzpicture}[line cap=round,line join=round,>=triangle 45,x=1.0cm,y=1.0cm]
\clip(3.86,0.8599999999999969) rectangle (7.2,4.179999999999998);
\fill[fill=black,fill opacity=0.35] (5.5,3.5) -- (4.5,2.5) -- (5.5,1.5) -- (6.5,2.5) -- cycle;
\draw (4.0,4.0)-- (4.0,1.0);
\draw (4.0,1.0)-- (7.0,1.0);
\draw (7.0,1.0)-- (7.0,4.0);
\draw (7.0,4.0)-- (4.0,4.0);
\draw (4.0,2.0)-- (6.0,4.0);
\draw (5.0,4.0)-- (7.0,2.0);
\draw (7.0,3.0)-- (5.0,1.0);
\draw (6.0,1.0)-- (4.0,3.0);
\draw (5.5,3.5)-- (4.5,2.5);
\draw (4.5,2.5)-- (5.5,1.5);
\draw (5.5,1.5)-- (6.5,2.5);
\draw (6.5,2.5)-- (5.5,3.5);
\begin{scriptsize}
\draw [fill=uuuuuu] (5.5,3.5) circle (1.5pt);
\draw [fill=uuuuuu] (4.5,2.5) circle (1.5pt);
\draw [fill=uuuuuu] (5.5,1.5) circle (1.5pt);
\draw [fill=uuuuuu] (6.5,2.5) circle (1.5pt);
\end{scriptsize}
\end{tikzpicture}
}
\item Hány olyan háromjegyű szám van, amelyben legalább két számjegy egyforma?
\item Számítsuk ki azoknak a négyjegyű számoknak az összegét, amelyekben a $2,3,4,5$ és $6$ számjegyek szerepelnek csak.
\item Melyik nagyobb $2^{2008}+2^{2009}$, vagy $2^{2007}+2^{2010}$?
\item Igazoljuk, hogy a következő szám összetett szám: $10^{18}-2\cdot 10^4+1$.
\end{enumerate}

