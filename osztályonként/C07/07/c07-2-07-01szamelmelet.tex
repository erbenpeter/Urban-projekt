\documentclass{article}
\usepackage[utf8]{inputenc}
\usepackage{t1enc}
\usepackage{geometry}
 \geometry{
 a4paper,
 total={210mm,297mm},
 left=20mm,
 right=20mm,
 top=20mm,
 bottom=20mm,
 }
\usepackage{amsmath}
\usepackage{amssymb}
\frenchspacing
\usepackage{fancyhdr}
\pagestyle{fancy}
\lhead{Urbán János tanár úr feladatsorai}
\chead{C07/7./2. csoport}
\rhead{Számelmélet}
\lfoot{}
\cfoot{\thepage}
\rfoot{}

\usepackage{enumitem}
\usepackage{multicol}
\usepackage{calc}
\newenvironment{abc}{\begin{enumerate}[label=\textit{\alph*})]}{\end{enumerate}}
\newenvironment{abc2}{\begin{enumerate}[label=\textit{\alph*})]\begin{multicols}{2}}{\end{multicols}\end{enumerate}}
\newenvironment{abc3}{\begin{enumerate}[label=\textit{\alph*})]\begin{multicols}{3}}{\end{multicols}\end{enumerate}}
\newenvironment{abc4}{\begin{enumerate}[label=\textit{\alph*})]\begin{multicols}{4}}{\end{multicols}\end{enumerate}}

\newcommand{\degre}{\ensuremath{^\circ}}
\newcommand{\tg}{\mathop{\mathrm{tg}}\nolimits}
\newcommand{\ctg}{\mathop{\mathrm{ctg}}\nolimits}
\newcommand{\arc}{\mathop{\mathrm{arc}}\nolimits}
\renewcommand{\arcsin}{\arc\sin}
\renewcommand{\arccos}{\arc\cos}
\newcommand{\arctg}{\arc\tg}
\newcommand{\arcctg}{\arc\ctg}
 \newcommand{\sgn}{\operatorname{sgn}}


\parskip 8pt
\begin{document}

\section*{Számelmélet}





\subsection*{2007. 09. 05. -- Feladatlap}
\begin{enumerate}
\item Milyen számjegyre végződhet két egymás után következő természetes szám szorzata?
\item Milyen számjegyre végződhet egy természetes szám négyzete?
\item Hány olyan legfeljebb 3-jegyű pozitív egész szám van, amelyik nem osztható sem 2-vel, sem 5-tel?
\item Melyik az a legkisebb pozitív egész szám, amely 1-gyel kezdődik és ha az elejéről az 1-est a végére írjuk, akkor az így kapott szám az eredeti szám 3-szorosa lesz?
\item Hány 0-ra végződik az első 100 pozitív egész szám szorzata?
\item Melyik az a legnagyobb, tízes számrendszerben felírt háromjegyű páros szám, amely nem változik, ha az egyesek és a százasok helyén álló számjegyet felcseréljük?
\item Keressük meg a következő szorzások eredményeiben a közös tulajdonságot:

$12\cdot99$; $12\cdot999$; $12\cdot9999$; \ldots
\end{enumerate}


\subsection*{2007. 09. 06. -- Feladatlap}
\begin{enumerate}
\item Mi az utolsó számjegye:
\begin{abc}
\item $2^{20}$-nak;
\item $7^{20}$-nak.
\end{abc}
\item Mi az utolsó két számjegye $2^{100}$-nak?
\item Az 1-től 1000-ig terjedő egész számok szorzata 7-nek melyik legfeljebb mekkora kitevőjű hatványával osztható?
\item Hány 0-ra végződik az 1-nél nagyobb vagy egyenlő, legfeljebb háromjegyű egész számok szorzata?
\item Tudjuk, hogy $11^2=121$. Igaz-e, hogy $10201$, $1002001$, $10002001$ is négyzetszám?
\item Van-e olyan négyzetszám, amelynek a számjegyei valamilyen sorrendben $0$; $2$; $3$; $5$?
\item Mutassuk meg, hogy minden pozitív egész szám ötödik hatvány ugyanarra a számjegyre végződik, mint maga a szám!
\end{enumerate}


\subsection*{2007. 09. 10. -- Feladatlap}
\begin{enumerate}
\item Adjuk meg $p$ értékét úgy, hogy $p$, $p+4$, $p+14$ is prímszám legyen! 
\item Melyek azok a $p$ prímek, amelyekre $p+10$ és $p+14$ is prím?
\item Melyek azok a $p$ prímek, amelyekre igaz, hogy $8p^2+1$ is prímszám?
\item A következő számok közül melyek prímek és melyek összetett számok:

$12$, $17$, $121$, $203$, $217$, $251$, $348$, $757$, $991$.
\item Mennyi a következő számok 9-cel vett osztási maradéka:

$234$, $512$, $106$, $113$, $4132$, $9503$, $3246$.
\item Igazoljuk, hogy $37^{37}-23^{23}$ osztharó 10-zel.
\item Igazoljuk, hogy 3 egymást követő egész szám szorzata osztható 6-tal!
\item Igazoljuk, hogy $15 \mid 2^{16}-1$ és $17 \mid 2^{16}-1$.
\end{enumerate}


\subsection*{2007. 09. 11.}
\begin{enumerate}
\item Igazoljuk, hogy
\begin{abc3}
\item $24 \mid 5^{20}-1$;
\item $181 \mid 3^{105}+4^{105}$;
\item $13 \mid 2^{60}+7^{30}$.
\end{abc3}
\item Melyik az a legnagyobb szám, amellyen bármely három egymást követő páros szám szorzata osztható?
\item Igazoljuk, hogy
\begin{abc2}
\item $11 \mid 100^{100}-1$;
\item $15 \mid 1000^5-1000$.
\end{abc2}
\item Adott 11 darab, egyenként $1$, $2$, $3$, $4$, $\ldots$ , $11$ kg súlyú csomag. Osszuk ezeket 3 csoportra úgy, hogy az egyes csoportokba tartozó csomagok súlyának összege megegyezzen.
\item Bizonyítsuk be, hogy ha $n > 0$, egész szám, akkor $10^n+98$ osztható 18-cal!
\item Adjuk meg az összes $\overline{2ab}$ alakú tízes számrendszerbeli, 6-tal osztható háromjegyű számot!
\end{enumerate}


\subsection*{2007. 09. 13.}
\begin{enumerate}
\item Igazoljuk, hogy $11^{10}-1$ osztható 100-zal!
\item Mi a két utolsó számjegye:
\begin{abc2}
\item $2^{999}$-nek;
\item $3^{999}$-nek?
\end{abc2}
\item Bizonyos számú egymást követő pozitív egész szám összege 1000. Melyek ezek a számok?
\item Az első 100 pozitív egész számból válasszunk ki 51-et tetszőlegesen. Igazoljuk, hogy a kiválasztott számok között van kettő, hogy egyik osztója a másiknak.
\item Igazoljuk, hogy ha $p$ és $8p-1$ prímszámok, akkor $8p+1$ összetett szám.
\item Számítsuk ki azoknak a háromjegyű számoknak az összegét, amelyek nem nagyobbak 200-nál és 4-gyel osztva 1 maradékot adnak.
\item Melyik az a kétjegyű, tízes számrendszerbeli szám, amelyet megszorozva a számjegyei megfordításával kapott kétjegyű számmal eredményül 2430-at kapunk?
\end{enumerate}


\subsection*{2007. 09. 17. -- Oszthatóság}
\begin{enumerate}
\item Igazoljuk, hogy
\begin{abc3}
\item $3 \mid 2\cdot7^{100}+1$;
\item $7 \mid 3^{101}+2^{52}$;
\item $99 \mid 3^{53}\cdot2^{102}-108$.
\end{abc3}
\item Bizonyítsuk be, hogy
\begin{abc2}
\item $11 \mid 3^{102}+2^{301}$; 
\item $19 \mid 5^{99}\cdot2^{51}+3^{51}\cdot2^{99}$.
\end{abc2}
\item Igazoljuk, hogy
\begin{abc}
\item $7 \mid 1+2^5+3^5+4^5+5^5+6^5$;
\item $13 \mid 1+2^9+3^9+4^9+5^9+6^9+7^9+8^9+9^9+10^9+11^9+12^9$.
\end{abc}
\item Igazoljuk, hogy ha $p>3$ prímszám, akkor $p^2-1$ osztható 24-gyel!
\item Igazoljuk, hogy $11^{10}-1$ osztható 120-szal!
\item Igazoljuk, hogy $2222^{5555}+5555^{2222}$ osztható 7-tel!
\end{enumerate}


\subsection*{2007. 09. 26. -- Oszthatóság, prímszámok dolgozat}
\begin{enumerate}
\item A 43 elé és után írjatok egy-egy számjegyet úgy, hogy a kapott négyjegyű szám osztható legyen 45-tel!
\item Igazoljuk, hogy $72 \mid 10^{10}+8$.
\item Igazoljuk, hogy $7 \mid 2^{24}-1$.
\item Lehet-e két prímszám összege 2007?
\item A következő törtet egyszerűsítjük addig, amíg lehet:

\begin{center}
$\displaystyle{\frac{1\cdot2\cdot3\cdot4\cdot5\cdot\ldots\cdot99\cdot100}{2^{100}}}$
\end{center}

Mi lesz a kapott tört nevezője?
\item Igazoljuk, hogy ha egy háromjegyű számot kétszer egymás után írunk, akkor a kapott hatjegyű szám osztható 7-tel, 11-gyel és 13-mal!
\end{enumerate}


\subsection*{2007. 09. 30. -- Versenyfeladatok}
\begin{enumerate}
\item Keressünk három olyan számot, amelyek 9-jegyűek, minden 0-tól különböző számjegy pontosan egyszer szerepel mindegyikben és két szám összege a harmadik.
\item Hány olyan csupa különböző számjegyből álló ötjegyű szám van, amelyben a számjegyek csökkenő sorrendben állnak?
\item Adott öt szám. Ezekből képeztük az összes lehetséges háromtagú összeget. A következő összegeket kaptuk: 3, 4, 6, 7, 9, 10, 11, 14, 15 és 17. Mi volt a kiindulásul választott öt szám?
\item Egy háromszög egyik belső szöge $60\degre$-os, a szöget közrefogó oldalai pedig 2 és 3 egység hosszúak. Daraboljuk föl a háromszöget szakaszokkal három részre úgy, hogy a részekből egy szabályos hatszöget lehessen összerakni.
\item Adjunk meg 2009 darab (nem feltétlenül különböző) pozitív egész számot úgy, hogy az összegük egyenlő legyen a szorzatukkal!
\end{enumerate}


\subsection*{2007. 10. 03.}
\begin{enumerate}
\item Igazoljuk kongruenciával:
\begin{abc3}
\item $14 \mid 15^{231}-1$;
\item $16 \mid 15^{321}+1$;
\item $13 \mid 12^{1233}+14{1324}$;
\item $3 \mid 10{15}+17$;
\item $11 \mid 26^{30}-1$.
\end{abc3}
\item Mi az utolsó számjegye a követkető számoknak:
\begin{abc3}
\item $289^{289}$;
\item $2^{524}$;
\item $203^{203}$?
\end{abc3}
\item Igazoljuk, hogy ha $n>0$ egész szám, akkor
\begin{abc2}
\item $3 \mid 10^n+17$;
\item $10 \mid 3^{4n+3}-17$.
\end{abc2}
\end{enumerate}


\subsection*{2007. 10. 04. -- Gyakorló feladatok kongruenciával}
\begin{enumerate}
\item Igazoljuk, hogy
\begin{abc3}
\item $24 \mid 5^{20}-1$;
\item $13 \mid 2^{60}+7^{30}$;
\item $181 \mid 3^{105}+4^{105}$.
\end{abc3}
\item Igazoljuk, hogy ha $n>0$ tetszőleges egész szám, akkor
\begin{abc4}
\item $6 \mid 17^n-11^n$;
\item $15 \mid 2^{4n}-1$;
\item $5 \mid 4\cdot6^n+5^n-4$;
\item $7 \mid 3^{2n+1}+2^{n+2}$.
\end{abc4}
\item Mi az utolsó két számjegye a következő számnak:
\begin{abc}
\item $2^{2007}$;
\item $7+7^2+7^3+7^4+\ldots+7^{2007}$?
\end{abc}
\end{enumerate}


\subsection*{2007. 10. 10. -- Kongruenciák}
\begin{enumerate}
\item Igazoljuk kongruenciával:
\begin{abc}
\item $13 \mid 40^{10}-1$;
\item $10^{100}-7$ összetett szám.
\end{abc}
\item Mi lesz a $81^{89}+19^{89}$ szám utolsó két jegye?
\item Igazoljuk, hogy
\begin{abc}
\item $7 \mid 19\cdot8^n+23$, $n>0$, egész;
\item $18 \mid 17^{19}+19^{17}$.
\end{abc}
\item Kongruenciával bizonyítsuk be, hogy

$2008 \mid 2009^{2009}-2007^{2007}-2010$.
\end{enumerate}


\subsection*{2007. 10. 17.}
\begin{enumerate}
\item Számítsuk ki a következő számpárok legnagyobb közös osztóját:
\begin{abc3}
\item $(101;211)$;
\item $(567;1053)$;
\item $(875;2625)$.
\end{abc3}
\item Két pozitív egész szám összege 1323, legnagyobb közös osztójuk: 147. Melyek ezek a számok?
\item Igazoljuk, hogy $(2^n+1;2^n-1)=1$, ha $n>0$, egész szám
\item A következő számok közül keressük ki azokat a párokat, amelyek legnagyobb közös osztója 1:

3, 4, 6, 10, 15, 21, 28, 35, 42, 63.
\item Határozzuk meg a következő számpárok legkisebb közös többszörösét:
\begin{abc4}
\item $[16;28]$;
\item $[105;180]$;
\item $[475;570]$;
\item $[348;476]$.
\end{abc4}
\end{enumerate}


\subsection*{2007. 10. 18.}
\begin{enumerate}
\item Hány pozitív egész osztója van a következő számoknak: 2, 6, 8, 12, 100, 625?
\item Jelölje $d(n)$ az $n>0$ egész szám pozitív osztóinak számát. Keressünk képletet $d(n)$-re!
\item Melyik az a legkisebb szám, amelyre igaz, hogy a pozitív osztóinak száma: 
\begin{abc3}
\item 10;
\item 12;
\item 18.
\end{abc3}
\item Mik azok az $a,b>0$ egész számok, amelyekre igaz, hogy $(a;b)=10$ és $a\cdot b=2400$?
\item Hány olyan számpár van, amelyre $(a;b)=4$ és $a+b=100$?
\item Milyen $a,b>0$ egész számokra igaz, hogy $[a;b]=720$ és $a+b=98$?
\end{enumerate}


\subsection*{2007. 10. 24.}
\begin{enumerate}
\item Határozzuk meg a hiányzó számjegyeket:
\begin{abc3}
\item $36 \mid \overline{52x2y}$;
\item $45 \mid \overline{24x68y}$;
\item $99 \mid \overline{62xy427}$.
\end{abc3}
\item Igazoljuk, hogy a következő számok összetett számok:
\begin{abc2}
\item $4^{20}-1$;
\item $10^{100}-7$.
\end{abc2}
\item Négy egymást követő pozitív egész szám szorzata 3024. Melyek ezek a számok?
\item Lehet-e egyszerre prímszám a következő három szám: $n+5$, $n+7$, $n+15$ ($n>0$, egész)?
\item Hány pozitív egész osztója van:
\begin{abc3}
\item 100-nak;
\item 289-nek;
\item 2007-nek?
\end{abc3}
\item Melyek azok a háromjegyű számok, amelyek osztóinak száma 5?
\end{enumerate}


\subsection*{2007. 11. 05.}
\begin{enumerate}
\item Az 1, 2, 3, 4, 5, 6 számjegyeket milyen sorrendben kell leírni ahhoz, hogy a legnagyobb 12-vel osztható hatjegyű számot kapjuk?
\item Melyek azok a kétjegyű számok, amelyeknek a legtöbb osztójuk van?
\item Mely egész $n$-re lehet az $\displaystyle{\frac{n+11}{n-9}}$ törtet egyszerűsíteni? Mikor kaphatunk egész számot? 
\item Van-e olyan $n>0$ egész, amelyre a $\displaystyle{\frac{12n+1}{30n+2}}$ tört egyszerűsíthető?
\item Melyek azok a háromjegyű számok, amelyeknek 5 osztójuk van?
\item Hány osztója van a következő számoknak:
\begin{abc3}
\item $777777$;
\item $10^{100}$;
\item $5^{5^5}$.
\end{abc3}
\item Mutassuk meg, hogy hét négyzetszám között van kettő, amelyek különbsége osztható 10-zel!
\end{enumerate}


\subsection*{2007. 11. 07.}
\begin{enumerate}
\item Melyek azok a 4-jegyű számok, amelyeknek páratlan számú osztójuk van?
\item Igazoljuk, hogy

$7 \mid 333^{444}+444^{333}$.
\item Osztható-e $\displaystyle{\frac{1000!}{(500!)^2}}$ 7-tel?
\item Igazoljuk, hogy ha $a$-t és $1000a$-t 111-gyel osztjuk, akkor a kapott maradékok egyenlők.
\item Igazoljuk, hogy ha $n>0$ egész, akkor a $\displaystyle{\frac{21n+4}{14n+3}}$ tört nem egyszerűsíthető.
\item Mely $n>0$ egészekre igaz, hogy $2^n-1$ osztható 7-tel?
\item Igazoljuk, hogy ha $n>0$ egész, akkor

$40 \mid 11^{2n}+31^{2n}+38\cdot11^n\cdot31^n$.
\end{enumerate}


\subsection*{2007. 11. 08. -- Ismétlő feladatok}
\begin{enumerate}
\item Igazoljuk, hogy ha $n>0$ egész, akkor $19\cdot8^{2n}+17$ összetett szám.
\item Bizonyítsuk be:
\begin{abc2}
\item $40 \mid 1+3+3^2+3^3+3^4+\ldots+3^{99}$;
\item $35 \mid 3^{6n}-2^{6n}$.
\end{abc2}
\item A hatjegyű $\overline{523ABC}$ szám osztható 7-tel, 8-cal és 9-cel. Mik lehetnek az A, B és C számjegyek?
\item Igazoljuk, hogy ha $p$ és $8p-1$ prímek, akkor $8p+1$ összetett szám.
\item Az $a$, $b$, $c$ különböző számjegyek. Igazoljuk, hogy az ezekből készíthető összes háromjegyű szám összege osztható $a+b+c$-vel!
\item Mennyi az első 10 pozitív egész szám legnagyobb közös osztója és legkisebb közös többszöröse?
\item Két pozitív egész szorzata 7875, legnagyobb közös osztója 15. Mi lehet a két szám?
\end{enumerate}


\subsection*{2007. 11. 12. -- Számelmélet dolgozat}
\begin{enumerate}
\item Határozzuk meg az $a$ és $b$ számokat, ha
\begin{abc}
\item $(a;b)=15$ és $[a;b]=420$;
\item $(a;b)=37$ és $a+b=666$.
\end{abc}
\item Milyen $n>0$ egész számokra egyszerűsíthető a következő tört:

$\displaystyle{\frac{5n+4}{8n+7}}$?
\item Melyik az a pozitív egész szám, amely osztható 12-vel és 14 osztója van?
\item Hány 0-ra végződik $123!$, azaz az $1\cdot2\cdot3\cdot4\cdot\ldots\cdot122\cdot123$ szorzás eredménye?
\item Igazoljuk, hogy $4^{90}+1$ osztható 17-tel!
\item Igazoljuk, hogy $2^n-1$ és $2^n+1$ legnagyobb közös osztója 1, ha $n>0$ egész szám!
\end{enumerate}


\subsection*{2007. 11. 15.}
\begin{enumerate}
\item $(a;b)=13$; $[a;b]=1989$; $a=?$ $b=?$
\item $(x;60)=15$, mi lehet $x$ értéke?
\item $(x;20)=1$, mi lehet $x$ értéke?
\item Adjuk meg a következő számok legnagyobb közös osztóját:
\begin{abc2}
\item $(17;34;263)$;
\item $(187;323;391)$;
\end{abc2}
\item Mennyi a következő számok pozitív egész osztóinak szorzata:
\begin{abc3}
\item $72$;
\item $180$;
\item $625$.
\end{abc3}
\item Melyik az a pozitív egész szám, amelynek 6 osztója van és a pozitív osztói szorzata 91125?
\item A $2^4\cdot3^2\cdot5$ számnak hány olyan osztója van, ami \underline{nem osztható} 15-tel?
\end{enumerate}


\subsection*{2007. 11. 19. -- Pótdolgozat}
\begin{enumerate}
\item Melyek azok az $a$, $b$ pozitív egész számok, amelyekre $(a;b)=10$ és $a\cdot b=2400$?
\item Két pozitív egész szám összege 1323, a két szám legnagyobb közös osztója 147. Melyik ez a két szám?
\item Van-e olyan $n>0$ egész szám, amelyre a $\displaystyle{\frac{2n+5}{3n+7}}$ tört egyszerűsíthető?
\item Hány olyan pozitív osztója van 2160-nak, ami \underline{nem osztható} 15-tel?
\item Határozzuk meg azt a pozitív egész számot, amelynek 15 pozitív osztója van, és ezek szorzata $2^{60}\cdot5^{15}$.
\item Melyek azok a négyjegyű, öttel osztható tízes számrendszerbeli számok, amelyeknek 15 osztójuk van?
\end{enumerate}


\subsection*{2007. 11. 22. -- Számrendszerek}
\begin{enumerate}
\item Írjuk fel tízes számrendszerben:
\begin{abc3}
\item $543_{\underline{6}}$;
\item $1212_{\underline{3}}$;
\item $8888_{\underline{9}}$.
\end{abc3}
\item Írjuk fel 8-as számrendszerben a következő, tízes számrendszerben felírt számokat:
\begin{abc3}
\item $63$;
\item $79$;
\item $2007$.
\end{abc3}
\item Fogalmazzuk meg a 8-as számrendszerben a 7-tel való oszthatóság szabályát!
\item Bizonyítsuk be, hogy $144_{\underline{a}}$ bármilyen $a>4$ számrendszerben négyzetszám.
\item Végezzük el a következő összeadást:

$23334_{\underline{6}}+444_{\underline{6}}+12341_{\underline{6}}$.
\end{enumerate}


\subsection*{2007. 11. 26.}
\begin{enumerate}
\item Írjuk át tízes számrendszerbe a következő számokat:
\begin{abc4}
\item $26014_{\underline{7}}$;
\item $11001101_{\underline{2}}$;
\item $42123_{\underline{5}}$;
\item $53041_{\underline{6}}$.
\end{abc4}
\item Végezzük el a következő műveleteket:
\begin{abc}
\item $10111_{\underline{2}}+1100_{\underline{2}}+1010101_{\underline{2}}=$
\item $1234_{\underline{5}}\cdot322_{\underline{5}}=$
\item $351_{\underline{6}}\cdot14_{\underline{6}}=$
\end{abc}
\item Határozzuk meg $x$ értékét:
\begin{abc3}
\item $106_{\underline{x}}=153_{\underline{7}}$;
\item $324_{\underline{x}}=10022_{\underline{3}}$;
\item $541_{\underline{x}}=2014_{6}$.
\end{abc3}
\item Milyen számrendszerben igaz:
\begin{abc3}
\item $35+40=115$;
\item $342+63=425$;
\item $216\cdot3=654$.
\end{abc3}
\end{enumerate}


\subsection*{2007. 11. 28.}
\begin{enumerate}
\item Mi lehet $x$ értéke:
\begin{abc2}
\item $41_{\underline{8}}=201_{\underline{x}}$;
\item $10110001_{\underline{2}}=342_{\underline{x}}$?
\end{abc2}
\item Egy számrendszerben $4^2=20_{\underline{x}}$. Mennyi ebben a számrendszerben $5^2$?
\item Mennyi x értéke, ha $13_{\underline{x}}$ és $31_{\underline{x}}$ 2-nek két egymás után következő hatványa?
\item Számítsuk ki:
\begin{abc}
\item $2334_{\underline{6}}+33020_{\underline{6}}+444_{\underline{6}}+12341_{\underline{6}}$;
\item $\overline{43ab5}_{\underline{12}}+\overline{3a6}_{\underline{12}}+\overline{4b25}_{\underline{12}}$, ahol $a$ értéke 10, $b$ értéke 11;
\item $234134_{\underline{7}}\cdot123_{\underline{7}}$.
\end{abc}
\item Milyen alapú számrendszerben igazak a következő egyenlőségek: 
\begin{abc}
\item $12_{\underline{x}}+13_{\underline{x}}=30_{\underline{x}}$;
\item $17_{\underline{x}}+38_{\underline{x}}=54_{\underline{x}}$;
\item $6_{\underline{x}}\cdot 6_{\underline{x}}=51_{\underline{x}}$;
\item $100_{\underline{x}}-1_{\underline{x}}=11_{\underline{x}}$;
\item $12_{\underline{x}}\cdot 7_{\underline{x}}=80_{\underline{x}}$.
\end{abc}
\end{enumerate}


\subsection*{2007. 11. 29.}
\begin{enumerate}
\item Az $111_{\underline{x}}$ szám $x=2, 3, \ldots , 9$ esetén mikor lesz prímszám?
\item Igaz-e, hogy $34780160_{\underline{9}}$ szám osztható 27-tel?
\item Mi a feltétele annak, hogy egy 6-os számrendszerben felírt szám osztható legyen 3-mal?
\item Mi a feltétele annak, hogy
\begin{abc}
\item egy 5-ös számrendszerben felírt szám osztható legyen 4-gyel;
\item egy 9-es számrendszerben felírt szám osztható legyen 8-cal?
\end{abc}
\item Milyen számjegyet írhatunk az üres helyre, hogy
\begin{abc}
\item $1827\Box_{\underline{9}}$ osztható legyen 8-cal és 3-mal;
\item $2465\Box_{\underline{8}}$ osztható legyen 7-tal és 4-gyel;
\item $3925\Box_{\underline{12}}$ osztható legyen 11-gyel és 3-mal?
\end{abc}
\end{enumerate}


\subsection*{2007. 12. 03. -- Számrendszerek dolgozat}
\begin{enumerate}
\item Írjuk fel 7-es alapú számrendszerbe a következő számokat:
\begin{abc2}
\item $2007$;
\item $12345$.
\end{abc2}
\item Végezzük el a következő műveleteket:
\begin{abc2}
\item $12127_{\underline{11}}+16_{\underline{11}}+999_{\underline{11}}$;
\item $452_{\underline{7}}\cdot231_{\underline{7}}$.
\end{abc2}
\item Határozzuk meg x értékét:
\begin{abc3}
\item $106_{\underline{x}}=153_{\underline{7}}$;
\item $236_{\underline{x}}=1240_{\underline{5}}$;
\item $216_{\underline{x}}\cdot3_{\underline{x}}=654_{\underline{x}}$.
\end{abc3}
\item Egy tízes számrendszerbeli háromjegyű szám $\overline{aaa}$ alakú. Ugyanaz a szám egy más alapú számrendszerben szintén háromjegyű, de $\overline{(4a)(2a)a}_{\underline{x}}$ alakú, ahol $4a$ és $2a$ is egy egy számjegy. Határozzuk meg $x$ és $a$ értékét!
\item Milyen számjegyet írhatunk az üres helyre, ha tudjuk, hogy $1223\Box_{\underline{5}}$ osztható 20-szal?
\end{enumerate}



\end{document}