\documentclass{article}
\usepackage[utf8]{inputenc}
\usepackage{t1enc}
\usepackage{geometry}
 \geometry{
 a4paper,
 total={210mm,297mm},
 left=20mm,
 right=20mm,
 top=20mm,
 bottom=20mm,
 }
\usepackage{amsmath}
\usepackage{amssymb}
\frenchspacing
\usepackage{fancyhdr}
\pagestyle{fancy}
\lhead{Urbán János tanár úr feladatsorai}
\chead{C07/08/1.}
\rhead{Számelmélet}
\lfoot{}
\cfoot{\thepage}
\rfoot{}

\usepackage{pgf,tikz}
\usetikzlibrary{arrows}

\usepackage{enumitem}
\usepackage{multicol}
\usepackage{calc}
\newenvironment{abc}{\begin{enumerate}[label=\textit{\alph*})]}{\end{enumerate}}
\newenvironment{abc2}{\begin{enumerate}[label=\textit{\alph*})]\begin{multicols}{2}}{\end{multicols}\end{enumerate}}
\newenvironment{abc3}{\begin{enumerate}[label=\textit{\alph*})]\begin{multicols}{3}}{\end{multicols}\end{enumerate}}
\newenvironment{abc4}{\begin{enumerate}[label=\textit{\alph*})]\begin{multicols}{4}}{\end{multicols}\end{enumerate}}
\newenvironment{abcn}[1]{\begin{enumerate}[label=\textit{\alph*})]\begin{multicols}{#1}}{\end{multicols}\end{enumerate}}
\setlist[enumerate,1]{listparindent=\labelwidth+\labelsep}

\newcommand{\degre}{\ensuremath{^\circ}}
\newcommand{\tg}{\mathop{\mathrm{tg}}\nolimits}
\newcommand{\ctg}{\mathop{\mathrm{ctg}}\nolimits}
\newcommand{\arc}{\mathop{\mathrm{arc}}\nolimits}
\renewcommand{\arcsin}{\arc\sin}
\renewcommand{\arccos}{\arc\cos}
\newcommand{\arctg}{\arc\tg}
\newcommand{\arcctg}{\arc\ctg}

\parskip 8pt
\begin{document}

\section*{Számelmélet}

\subsection*{2008.09.04. - Feladatok}
\begin{enumerate}
\item Alakítsuk át tizedes törtté a következő törteket:

\begin{abcn}{6}
\item $\dfrac{1}{2}$; 
\item $\dfrac{2}{5}$;
\item $\dfrac{1}{20}$;
\item $\dfrac{4}{25}$;
\item $\dfrac{3}{40}$;
\item $\dfrac{7}{80}$.
\end{abcn}

\item Írjuk át közönséges törtté a következő tizedes törteket:

\begin{abc3}
\item $0,12$; 
\item $0,23$; 
\item $0,65$.
\end{abc3}

\item Írjuk át tizedes törtté a következő törteket:

\begin{abcn}{6}
\item $\dfrac{1}{9}$;
\item $\dfrac{2}{3}$; 
\item $\dfrac{1}{11}$;
\item $\dfrac{1}{7}$;
\item $\dfrac{3}{7}$;
\item $\dfrac{5}{7}$.
\end{abcn}

\item ($*$) Igazoljuk ,,szemléletesen'', hogy $\dfrac{1}{9}=0,1111\ldots$
\end{enumerate}

\subsection*{2008.09.09. - Számfogalom}
\begin{enumerate}

\item Igazoljuk, hogy a $\sqrt{6}$ irracionális szám.
\item Igazoljuk, hogy a $\sqrt{2}-1$ szám irracionális.
\item Lehet-e két irracionális szám összege; szorzata racionális?
\item Igazoljuk a következő összefüggéseket:

\begin{abc2}
\item $\sqrt{8}+\sqrt{18}=\sqrt{50}$;
\item $\sqrt{10+4\sqrt{6}}-\sqrt{10-4\sqrt{6}}=\sqrt{4}$;
\item $\sqrt{\dfrac{2+\sqrt{3}}{2-\sqrt{3}}}+\sqrt{\dfrac{2-\sqrt{3}}{2+\sqrt{3}}}=4$.
\end{abc2} 
 
\item Igazoljuk, hogy $\root3\of{2}$ irracionális.
\item Számítsuk ki:

\begin{abc2}
\item $\sqrt{7+2\sqrt{6}}-\sqrt{7-2\sqrt{6}}$;
\item $\sqrt{2+\sqrt{3}}+\sqrt{2-\sqrt{3}}$.
\end{abc2}
\end{enumerate}
\subsection*{2008.09.10.}
\begin{enumerate}
\item Számítsuk ki a következő összeget:

$\dfrac{1}{\sqrt{2}+1}+\dfrac{1}{\sqrt{3}+\sqrt{2}}+\dfrac{1}{\sqrt{4}+\sqrt{3}}+\dfrac{1}{\sqrt{5}+\sqrt{4}}+\ldots+\dfrac{1}{\sqrt{100}+\sqrt{99}}$.
\item Számítsuk ki a következő összeget:

$\dfrac{1}{2\sqrt{1}+1\sqrt{2}}+\dfrac{1}{3\sqrt{2}+2\sqrt{3}}+\ldots+\dfrac{1}{11\sqrt{10}+10\sqrt{11}}$.
\item Számítsuk ki:

\begin{abc3}
\item $\sqrt{4+\sqrt{7}}-\sqrt{4-\sqrt{7}}$;
\item $\sqrt{7+4\sqrt{3}}+\sqrt{7-4\sqrt{3}}$;
\item $\sqrt{12}+\sqrt{75}-\sqrt{147}$.
\end{abc3}

\item ($*$) Igazoljuk, hogyha $n\geq1$ egész szám, akkor $[\sqrt{n}+\sqrt{n+1}]=[\sqrt{4n+2}].$
\item Számítsuk ki:  $(\root3\of{24}+\root3\of{81}-\root3\of{192})\cdot(\root3\of{375}-\root3\of{192})$.
\end{enumerate}

\subsection*{2008.09.15.}
\begin{enumerate}
\item Számítsuk ki a következő kifejezések pontos értékét:

\begin{abc}
\item  $\left(\dfrac{8}{\sqrt{5}+3}-\dfrac{20}{\sqrt{5}-3}-\dfrac{4}{\sqrt{5}-2}\right)\cdot(29+\sqrt{5})$;

\item   $\dfrac{6+4\sqrt{2}}{4}+\dfrac{3-6\sqrt{2}}{6}$;

\item   $(3\sqrt{2}-2\sqrt{3})\cdot(3\sqrt{2}+2\sqrt{3})$;

\item   $(3\sqrt{5}+2\sqrt{20})\cdot (\sqrt{45}+2\sqrt{5}-\sqrt{125})$;

\item   $\sqrt{2+\sqrt{3}}\cdot \sqrt{2+\sqrt{2+\sqrt{3}}}\cdot \sqrt{2+\sqrt{2+\sqrt{2+\sqrt{3}}}}\cdot \sqrt{2-\sqrt{2+\sqrt{2+\sqrt{3}}}}$ ;

\item   $\left(\dfrac{5}{\sqrt{3}-\sqrt{2}})-\dfrac{3}{\sqrt{3}+\sqrt{2}}\right)\cdot(\sqrt{2}+4\sqrt{2})$;

\item   $\left(\dfrac{1}{\sqrt{7}-2}+\dfrac{3\sqrt{7}}{\sqrt{7}+2}\right)\cdot(23+5\sqrt{7})$.
\end{abc}
\end{enumerate}

\subsection*{2008.09.16.}
\begin{enumerate}
\item Igazoljuk, hogy $\sqrt{2}+\sqrt{3}$ irracionális szám.
\item Számítsuk ki:

\begin{abc2}
\item $\left(\sqrt{4+\sqrt{7}}+\sqrt{4-\sqrt{7}}\right)^2$; 
\item $\left(\sqrt{2\sqrt{3}+2\sqrt{2}}-\sqrt{2\sqrt{3}-2\sqrt{2}}\right)^2$;
\item $\left(\sqrt{15+\sqrt{100}}-\sqrt{15-10\sqrt{2}}\right)^2$;
\item $\left(\sqrt{12}+\sqrt{3}+1\right)^2$.
\end{abc2}
\item Melyik az a legkisebb pozitív egész $n$ szám, amelyre teljesül, hogy $\sqrt{n+1}-\sqrt{n}<10^{-2}$?
\item Számítsuk ki $[(\sqrt{2}+1)^n]$ értékét, ha $1\leq n\leq 10$.
\item ($*$) Mutassuk meg, hogy $(7+\sqrt{50})^3$ tizedestört alakjában a tizedes vessző után három $0$ áll!
\item Számítsuk ki:
\begin{abc2}
\item $(3-\sqrt{5})\cdot \root3\of{9+4\sqrt{5}}$;
\item $\sqrt{4\sqrt{5}+3}\cdot \sqrt{4\sqrt{5}-3}$.  
\end{abc2}

\end{enumerate}

\subsection*{2008.09.17.}
\begin{enumerate}
\item  Igazoljuk, hogy a következő számok racionálisak:

\begin{abc3}
\item $\left(\root6\of{27}-\sqrt{6\frac{3}{4}}\right)^2$;
\item $\sqrt{3+2\sqrt{2}}+\sqrt{3-2\sqrt{2}}$;
\item $\root3\of{5\sqrt{2}+7}-\root3\of{5\sqrt{2}-7}$.
\end{abc3}
\item Adjunk meg két szomszédos egész számot úgy, hogy a megadott számok ezek közé essenek:
\begin{abc3}
\item $\sqrt{15}-3$; 
\item $14-2\sqrt{6}$; 
\item $11-\sqrt{110}$.
\end{abc3}
\item ($*$) Az $A$ és $B$ egész számok, tudjuk még, hogy $A>B>0$. Igaz-e, hogy $\sqrt{A+B+\sqrt{4AB}}\cdot\sqrt{A+B-\sqrt{4AB}}$ is egész szám?
\item Tudjuk, hogy $\sqrt{18-4\sqrt{15}+2\sqrt{5}-4\sqrt{3}}=a+b\sqrt{3}+c\sqrt{5}$, ahol $a,b,c$ egész számok. Adjuk meg $a,b,c$ értékét!

\end{enumerate}

\subsection*{2008.09.22.}
\begin{enumerate}
\item Számítsuk ki:
\begin{abc2}
\item $\sqrt{7-4\sqrt{3}}+\sqrt{7+4\sqrt{3}}$;
\item $(\root3\of{24}+\root3\of{81}-\root9\of{192})\cdot \root3\of{3}$;
\item $\root3\of{10+6\sqrt{3}}+\root3\of{10-6\sqrt{3}}$;
\item $\sqrt{28+16\sqrt{3}}-\sqrt{28-16\sqrt{3}}$.
\end{abc2}
\item Számítsuk ki: $\sqrt{2+\sqrt{3}}\cdot \root3\of{\dfrac{\sqrt{2}(3\sqrt{3}-5)}{2}}$.
\item Melyik nagyobb $\sqrt{3}$ vagy $\root3\of{5}$. (Zsebszámológép használata nélkül!)
\item Állítsuk növekvő sorrendbe: $\sqrt{2},\root3\of{3},\root4\of{4},\root5\of{5}$.
\item Igazoljuk $\left(\dfrac{1}{\sqrt{5}-2}\right)^3-\left(\dfrac{1}{\sqrt{5}+2}\right)^3=76$.
\end{enumerate}

\subsection*{2008.09.23.}
\begin{enumerate}
\item Számítsuk ki:
\begin{abc2}
\item $\left(2\sqrt{3}-\sqrt{5}+\sqrt{12}\right)\left(\sqrt{48+\sqrt{5}}\right)$; 
\item $\sqrt{5\sqrt{3}+\sqrt{59}}\cdot \sqrt{5\sqrt{3}-\sqrt{59}}$;
\item $\left(\dfrac{6}{4+\sqrt{10}}+\dfrac{24}{8-\sqrt{40}}\right)\cdot\left(12-\sqrt{10}\right)$;
\item $\left(\dfrac{1}{\sqrt{7}-2}+\dfrac{3\sqrt{7}}{\sqrt{7}+2}\right)\cdot \left((23+5\sqrt{7}\right)$.
\end{abc2}
\item $\sqrt{7+2\sqrt{10}}=\sqrt{a}+\sqrt{b}$, ahol $a,b$ pozitív egészek; $a=?$, $b=?$
\item Igaz-e hogy $\left(3-\sqrt{5}\right)\cdot \left(\root3\of{9+4\sqrt{5}}\right)$ egész szám?
\item Milyen előjelű a következő szám: $\sqrt{3}-\sqrt{2-\sqrt{3}}-\sqrt{2+\sqrt{3}}$?
\end{enumerate}

\subsection*{2008.09.24. - Irracionális számok}
\begin{enumerate}
\item Igazoljuk, hogy $\sqrt{3}+\sqrt{5}$ irracionális szám.
\item Számítsuk ki:
\begin{abc3}
\item $\dfrac{(\sqrt{5}+\sqrt{3})(4-\sqrt{15})}{\sqrt{5}-\sqrt{3}}$;
\item $\sqrt{2\sqrt{2}+\sqrt{7}}+\sqrt{2\sqrt{2}-\sqrt{7}}$;
\item $(\sqrt{5+\sqrt{24}}-\sqrt{5-\sqrt{24}})$.
\end{abc3}
\item Igazoljuk, hogy ha $x\geq y\geq 0$, akkor $\sqrt{\sqrt{x}+\sqrt{y}}=\sqrt{\dfrac{\sqrt{x}+\sqrt{x-y}}{2}}+\sqrt{\dfrac{\sqrt{x}-\sqrt{x-y}}{2}}$.
\item ($*$) Igazoljuk, hogy bármely két racionális szám között van irracionális szám.
\end{enumerate}

\subsection*{2008.09.29.}
\begin{enumerate}
\item Számítsuk ki:
\begin{abc2}
\item $(\sqrt{5}-2)^2$;
\item $(2\sqrt{3}+3\sqrt{2})^2$;
\item $(2\sqrt{3}-\sqrt{5}+\sqrt{12})\cdot (\sqrt{48}+\sqrt{5})$;
\item $\sqrt{5\sqrt{3}+\sqrt{59}}\cdot \sqrt{5\sqrt{3}-\sqrt{59}}$;
\item $\left(\dfrac{6}{4+\sqrt{10}}+\dfrac{24}{8-\sqrt{40}}\right)\cdot (12-\sqrt{10})$.
\end{abc2}
\item Végezzük el:
\begin{abc2}
\item $(\sqrt{2}-\root4\of{3})\cdot (\sqrt{3}+\root4\of{2})$;
\item $\sqrt{2+\sqrt{3}}\cdot \root3\of{\dfrac{\sqrt{2}(3\sqrt{3}-5)}{2}}$.
\end{abc2}
\item Igazoljuk: $\root3\of{1-12\cdot \root3\of{7}+6\cdot \root3\of{49}}+\root3\of{7}=2-\root3\of{7}$
\end{enumerate}

\subsection*{2008.09.30. - Pótdolgozat}
\begin{enumerate}
\item Számítsuk ki:
\begin{abc}
\item  $\dfrac{\sqrt{2+\sqrt{2}}-\sqrt{2-\sqrt{2}}}{\sqrt{2+\sqrt{2}}+\sqrt{2-\sqrt{2}}}$;
\item $\left(\sqrt{8+2\sqrt{10+2\sqrt{5}}}+\sqrt{8-2\sqrt{10+2\sqrt{5}}}\right)^2$;
\item $\sqrt{6+\sqrt{11}}\cdot \sqrt{3+\sqrt{3+\sqrt{11}}}\cdot \sqrt{3-\sqrt{3+\sqrt{11}}}$.
 \end{abc}
 \item Melyik nagyobb: $3\sqrt{2}-2\sqrt{3}$ vagy $\sqrt{30-12\sqrt{6}}$?
 \item Igazoljuk: $\dfrac{2-\sqrt{3}}{\sqrt{2}-\sqrt{2-\sqrt{3}}}+\dfrac{2+\sqrt{3}}{\sqrt{2}+\sqrt{2+\sqrt{3}}}=\sqrt{2}$.
 \item ($*$) Igazoljuk, hogy $\root3\of{1-27\cdot\root3\of{26}+9\cdot \root3\of{26^2}}+\root3\of{26}$ egész szám!
\end{enumerate}

\subsection*{2008.10.07. - Versenyfeladatok}
\begin{enumerate}
\item A $0$-tól különböző számjegyek $S$ halmazát fel lehet-e bontani két részhalmazra úgy, hogy egyikre se legyen igaz a következő tulajdonság: a részhalmaz két elemmel együtt tartalmazza azok különbségét is?
\item Egy kockát mind a $6$ lapjára tükrözzük. Az eredeti kockával együtt így egy új testet kapunk. Hányszorosa a kapott test felszíne az eredeti kocka felszínének?
\item $14$ különböző pozitív egész szám összege $110$. Melyek ezek a számok?
\item A pozitív egész számokat felírtuk egy nagy papírra $1$-től $10000$-ig, ezután kihúztuk azokat, amelyekben szerepel a $0$ vagy a $9$ számjegy. Hány szám maradt?
\item Egy szabályos hatszög minden oldalát öt egyenlő részre osztjuk. Hány olyan háromszög van, amelynek csúcsai az így kapott osztópontok közül kerülnek ki?
\end{enumerate}

\subsection*{2008.10.13.}
\begin{enumerate}
\item Ábrázoljuk a következő függvényeket:
\begin{abc3}
\item $x\mapsto\sqrt{x}$, $x\geq0$;
\item $x\mapsto\sqrt{x-3}$, $x\geq3$;
\item $x\mapsto\sqrt{x+2}$, $x\geq-2$; 
\item $x\mapsto\sqrt{-x}$, $x\leq0$;
\item $x\mapsto\sqrt{3-x}$, $x\leq3$;
\item $x\mapsto\sqrt{|x|}$;
\item $x\mapsto\sqrt{x-[x]}$;
\item $x\mapsto[x]+\sqrt{x-[x]}$;
\item $x\mapsto\sqrt{x^2-2x+1}$;
\item $x\mapsto\sqrt{x^2-6x+9}$.
\end{abc3}
\end{enumerate}

\subsection*{2008.10.14.}
\begin{enumerate}
\item Ábrázoljuk és jellemezzük a következő függvényeket (növekedés, fogyás, szélsőérték): 
\begin{abc3}
\item $x\mapsto-\sqrt{x}$, $x\geq0 $;
\item $x\mapsto\sqrt{2x+1}$, $x\geq-\dfrac{1}{2}$;
\item $x\mapsto\sqrt{x^4+2x^2+1}$;
\item $x\mapsto\sqrt{|x-3|}$;
\item $x\mapsto\sqrt{4x-2}$, $x\geq\dfrac{1}{2}$.
\end{abc3}
\item Igazoljuk a következő állítást: ha $a,b\geq0$, akkor $\dfrac{a+b}{2}\geq\sqrt{ab}$, és itt az "$=$" csak akkor igaz, ha $a=b$.
\item Igazoljuk, hog yha $a,b\geq0$, akkor $\dfrac{a+b}{2}\leq\sqrt{\dfrac{a^2+b^2}{2}}$, és egyenlőség csak $a=b$ esetén igaz.
\end{enumerate}

\subsection*{2008.10.14. - Versenyfeladatok}
\begin{enumerate}
\item Igazoljuk, hogy az $1+2+3+4+\ldots+n$ összeg értékének tízes számrendszerbeli alakjában az egyesek helyén álló számjegy periodikusan ismétlődik.
\item Igazoljuk, hogy $11$ pozitív egész szám között mindig van kettő, amelyek különbsége osztható $10$-zel!
\item Adott a síkon húsz pont. Ezek közül bizonyos pontpárokat összekötöttünk szakaszokkal. Igazoljuk, hogy mindig van van a $20$ pont között kettő olyan, amelyekből azonos számú szakasz indul ki.
\item Egy ötemeletes házat hányféleképpen tudunk kifesteni, ha minden emeletet vagy fehérre, vagy zöldre festhetünk, de két fehér emelet nem kerülhet egymás fölé? Oldjuk meg a feladatot $6,7,8$ emeletes házra is!
\item Egy paralelogramma egyik átlóján kiválasztunk egy pontot és ezen át párhuzamosokat húzunk az oldalakkal. Igazoljuk, hogy a bevonalazott két kis paralelogramma területe egyenlő! 
\end{enumerate}
\subsection*{2008.10.15.}
\begin{enumerate}
\item Igazoljuk az u.n. ,,rendezési tételt'' $3$ valós szám esetére: T.f.h. $a_1\geq a_2\geq a_3$ és $b_1\geq b_2\geq b_3$, ekkor $a_1b_1+a_2b_2+a_3b_3$ a maximális, és $a_1b_3+a_2b_2+a_3b_1$ a minimális az összes lehetséges hasonló szorzatok összege közül. 
\item  Igazoljuk a következő egyenlőtlenségeket: 
\begin{abc}
\item Ha $a,b,c>0$ valós számok, akkor $a^2+b^2+c^2\geq ab+bc+ac$;

\item  ha $a,b,c$ tetszőleges valós számok, akkor $a^4+b^4+c^4\geq a^2b^2+b^2c^2+a^2c^2$;

\item ($*$) ha $a\geq b\geq c>0$, akkor $\dfrac{a}{b+c}+\dfrac{b}{a+c}+\dfrac{c}{a+b}\geq \dfrac{3}{2}$;

\item ($*$) ha $a,b,c>0$ valós számok, akkor $\dfrac{1}{a}+\dfrac{1}{b}+\dfrac{1}{c}\leq \dfrac{a^8+b^8+c^8}{a^3b^3c^3}$. 
\end{abc}
\end{enumerate}
\subsection*{2008.10.16.}
\begin{enumerate}
\item Vázoljuk fel a következő függvények grafikonját:
\begin{abc3}
\item $x\mapsto x+\dfrac{1}{x}$, $x\neq0$;
\item $x\mapsto 2x+\dfrac{1}{x}$, $x\neq 0$;
\item $x\mapsto \dfrac{1}{x^2+1}$. 
\end{abc3}
\item Bizonyítsuk be, hogy az azonos kerületű téglalapok közül legnagyobb területű a négyzet.
\item Igazoljuk, hogyha $a,b>0$ és $a+b=1$, akkor $\left(a+\dfrac{1}{a}\right)^2+\left(b+\dfrac{1}{b}\right)^2\geq\dfrac{25}{2}$.
\item Határozzuk meg a következő függvény legkisebb értékét: $x\mapsto \dfrac{9+4x^2}{x^2}$, $x\neq0$.
\item Igazoljuk, hogy ha $a,b,c>0$, akkor $\dfrac{ab}{c}+\dfrac{bc}{a}+\dfrac{ca}{b}\geq a+b+c$.
\end{enumerate}
\subsection*{2008.10.20.}
\begin{enumerate}
\item Igazoljuk, hogy ha $a,b,c>0$, akkor $\dfrac{1}{ab}+\dfrac{1}{bc}+\dfrac{1}{ca}\leq \dfrac{1}{a}+\dfrac{1}{b}+\dfrac{1}{c}$.
\item Igazoljuk, hogy ha $a,b,c>0$, akkor $a\sqrt{bc}+b\sqrt{ac}+c\sqrt{ab}\leq ab+ac+bc$.
\item ($*$) Tudjuk, hogy $x,y,z>0$, és $\dfrac{1}{2x}+\dfrac{1}{3y}+\dfrac{1}{6z}=\dfrac{1}{\dfrac{x}{2}+\dfrac{y}{3}+\dfrac{z}{6}}$. Igazoljuk, hogy $x=y=z$.
\item Igazoljuk, hogy ha $a\geq b\geq c>0$, akkor $a+b+c\leq \dfrac{a^2+b^2}{2c}+\dfrac{b^2+c^2}{2a}+\dfrac{c^2+a^2}{2b}$.
\item Mennyi az $f(x)=\dfrac{x^2}{x^4+16}$, $x>0$ függvény legnagyobb értéke?
\end{enumerate}
\subsection*{2008.10.21.}
\begin{enumerate}
\item Tegyük fel, hogy $a,b,c,d\geq 0$ számok és igazoljuk, hogy $\dfrac{a+b+c+d}{4}\geq \root4\of{abcd}$, ahol az "$=$" csak akkor igaz, ha $a=b=c=d$.
\item Az előző feladat felhasználásával igazoljuk, hogy ha $a,b,c\geq 0$, akkor $\dfrac{a+b+c}{3}\geq \root3\of{abc}$, és "$=$" csak akkor igaz, ha $a=b=c$.
\item Egy $12$ egység oldalú négyzet alakú papírlapból fedél nélküli dobozt akarunk készíteni (négyzetes oszlop alakút). Milyen méretek esetén lesz a doboz térfogata maximális?
\item Igazoljuk, hogy ha $a,b,c>0$, akkor $\dfrac{a}{b}+\dfrac{b}{c}+\dfrac{c}{a}\geq 3$.
\item Bizonyítsuk be, hogy ha a tetszőleges valós szám, akkor $a^2+2\geq 2\sqrt{a^2+1}$.
\item Mutassuk meg, hogy ha $a,b,c>0$, akkor $\dfrac{1}{a}+\dfrac{1}{b}+\dfrac{1}{c}\geq\dfrac{9}{a+b+c}$.
\end{enumerate}
\subsection*{2008.10.21. - Versenyfeladatok}
\begin{enumerate}
\item Kati $2009$-ben éppen annyi éves lesz, mint születése évszámának számjegyei összege. Melyik évben született Kati?
\item Egy tömör kockát az egyik lapjával párhuzamos síkokkal rétegekre szeletelünk. Hány síkkal kell szétvágni a kockát ahhoz, hogy a keletkezett testek együttes felszíne $4$-szerese legyen a kocka felszínének?
\item A pozitív egész számokat $1$-től $100$-ig összeszorozzuk. Hány $0$-ra végződik a kapott szorzat?
\item Melyek azok az egymást követő pozitív egész számok, amelyeknek összege 45? 
\item Melyek azok az ötjegyű számok, amelyekre igaz, hogy mindegyik számjegye nagyobb, mint a tőle jobbra álló számjegyek összege?
\item Egy évben legfeljebb hány olyan hónap lehet, amelyben öt vasárnap van?
\item Egy kocka csúcsait pirosra és kékre festhetjük. Hány különböző kifestés lehetséges, ha két kifestést akkor tekintünk azonosnak, ha egymásba forgathatók?
\end{enumerate}
\subsection*{2008.10.22.}
\begin{enumerate}
\item Igazoljuk, hogy ha $a,b>0$, akkor $\left(\dfrac{a+b}{2}\right)^3\leq \dfrac{a^3+b^3}{2}$.  
\item Bizonyítsuk be, hogy ha $a+b=1$, akkor $a,b\geq 0$ esetén $a^4+b^4\geq \dfrac{1}{8}$.
\item Mutassuk meg, hogy ha $a\geq b\geq c>0$, akkor $a^4+b^4+c^4\geq abc\cdot(a+b+c)$.
\item Igazoljuk, hogy ha $a\geq b\geq c>0$, akkor $\dfrac{1}{a}+\dfrac{1}{b}+\dfrac{1}{c}\geq \dfrac{1}{\sqrt{bc}}+\dfrac{1}{\sqrt{ca}}+\dfrac{1}{\sqrt{ab}}$.
\item Azok közül a téglatestek közül, amelyeknek a felszíne $54$ területegység, melyiknek a térfogata a lehető legnagyobb?
\end{enumerate}
\subsection*{2008.11.03.}
\begin{enumerate}
\item Igazoljuk, hogy ha $a,b>0$, akkor $a^4+b^4\geq ab\cdot(a^2+b^2)$. 
\item Igazoljuk, hogy ha $x>0$, akkor $\dfrac{4}{x^2}+x\geq 3$. Mikor igaz az $=$ jel?
\item Azok közül a téglatestek közül, amelyeknek a térfogata $8$ egység, melyiknek a felszíne a legkisebb?
\item Igazoljuk, hogy bármely $a$ számra igaz, hogy $2a^2\leq 1+a^4$.
\item Bizonyítsuk be, hogy bármely $a,b,c$ pozitív számra igaz, hogy $a^3+b^3+c^3\geq 3abc$.
\item Bizonyítsuk be, hogy ha $a\neq 0$, akkor $\left|a+\dfrac{1}{a}\right|\geq 2$.
\end{enumerate}
\subsection*{2008.11.04.}
\begin{enumerate}
\item Igazoljuk, hogy $a^2+3c^2+b^2+1\geq 2ac+2bc+2c$ bármely $a,b,c$ számra.
\item Igazoljuk, hogy ha $a\geq b\geq c>0$, akkor $\dfrac{ab}{c}+\dfrac{bc}{a}+\dfrac{ca}{b}\geq a+b+c$.
\item Igazoljuk, hogy bármely $a,b,c>0$ számra igaz, hogy $\dfrac{a^2+b^2+c^2+3}{2}\geq a+b+c$.
\item Bizonyítsuk be, hogy ha $x,y\neq 0$ valós számok, és $\dfrac{1}{x^2}+\dfrac{1}{y^2}=2$, akkor $x^2+y^2\geq 2$.
\item Bizonyítsuk be, hogy ha $a,b,c\geq 0$ és $\sqrt{a}+\sqrt{b}+\sqrt{c}=1$, akkor $a+b+c\geq \dfrac{1}{3}$.
\item Igazoljuk, hogy ha $a\geq 0$, akkor: 
\begin{abc3}
\item $(a^3+a^2+a+1)^2\geq 16a^3$;
\item $a^4+a^3+4a+4\geq 8a^2$;
\item $2a^3+11\geq 9a$.
\end{abc3}
\end{enumerate}
\subsection*{2008.11.04. - Versenyfeladatok}
\begin{enumerate}
\item Írjuk fel a legnagyobb olyan tízes számrendszerbeli egész számot, melyben a harmadik jegytől kezdve (balról jobbra) minden számjegy az előző két számjegy összege.  
\item Egy szabályos ötszögnek meghúztuk mindegyik átlóját. Hány egyenlőszárú háromszög látható így az ábrán?
\item Egy négyzet oldalait $3-3$ egyenlő részre osztottuk, majd az osztópontokat az ábrán látható módon összekötöttük. Hányad része a bevonalazott négyszög területe a négyzet területének?

\centerline{
\definecolor{uuuuuu}{rgb}{0.26666666666666666,0.26666666666666666,0.26666666666666666}
\begin{tikzpicture}[line cap=round,line join=round,>=triangle 45,x=1.0cm,y=1.0cm]
\clip(3.86,0.8599999999999969) rectangle (7.2,4.179999999999998);
\fill[fill=black,fill opacity=0.35] (5.5,3.5) -- (4.5,2.5) -- (5.5,1.5) -- (6.5,2.5) -- cycle;
\draw (4.0,4.0)-- (4.0,1.0);
\draw (4.0,1.0)-- (7.0,1.0);
\draw (7.0,1.0)-- (7.0,4.0);
\draw (7.0,4.0)-- (4.0,4.0);
\draw (4.0,2.0)-- (6.0,4.0);
\draw (5.0,4.0)-- (7.0,2.0);
\draw (7.0,3.0)-- (5.0,1.0);
\draw (6.0,1.0)-- (4.0,3.0);
\draw (5.5,3.5)-- (4.5,2.5);
\draw (4.5,2.5)-- (5.5,1.5);
\draw (5.5,1.5)-- (6.5,2.5);
\draw (6.5,2.5)-- (5.5,3.5);
\begin{scriptsize}
\draw [fill=uuuuuu] (5.5,3.5) circle (1.5pt);
\draw [fill=uuuuuu] (4.5,2.5) circle (1.5pt);
\draw [fill=uuuuuu] (5.5,1.5) circle (1.5pt);
\draw [fill=uuuuuu] (6.5,2.5) circle (1.5pt);
\end{scriptsize}
\end{tikzpicture}
}
\item Hány olyan háromjegyű szám van, amelyben legalább két számjegy egyforma?
\item Számítsuk ki azoknak a négyjegyű számoknak az összegét, amelyekben a $2,3,4,5$ és $6$ számjegyek szerepelnek csak.
\item Melyik nagyobb $2^{2008}+2^{2009}$, vagy $2^{2007}+2^{2010}$?
\item Igazoljuk, hogy a következő szám összetett szám: $10^{18}-2\cdot 10^4+1$.
\end{enumerate}
\end{document}   
  