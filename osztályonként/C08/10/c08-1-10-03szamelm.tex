\documentclass{article}
\usepackage[utf8]{inputenc}
\usepackage{t1enc}
\usepackage{geometry}
\geometry{
	a4paper,
	total={210mm,297mm},
	left=20mm,
	right=20mm,
	top=20mm,
	bottom=20mm,
}
\usepackage{amsmath}
\usepackage{amssymb}
\frenchspacing
\usepackage{fancyhdr}
\pagestyle{fancy}
\lhead{Urbán János tanár úr feladatsorai}
\chead{C08/10./1. csoport}
\rhead{Számelmélet}
\lfoot{}
\cfoot{\thepage}
\rfoot{}

\usepackage{enumitem}
\usepackage{multicol}
\usepackage{calc}
\newenvironment{abc}{\begin{enumerate}[label=\textit{\alph*})]}{\end{enumerate}}
\newenvironment{abc2}{\begin{enumerate}[label=\textit{\alph*})]\begin{multicols}{2}}{\end{multicols}\end{enumerate}}
\newenvironment{abc3}{\begin{enumerate}[label=\textit{\alph*})]\begin{multicols}{3}}{\end{multicols}\end{enumerate}}
\newenvironment{abc4}{\begin{enumerate}[label=\textit{\alph*})]\begin{multicols}{4}}{\end{multicols}\end{enumerate}}

\newcommand{\degre}{\ensuremath{^\circ}}
\newcommand{\tg}{\mathop{\mathrm{tg}}\nolimits}
\newcommand{\ctg}{\mathop{\mathrm{ctg}}\nolimits}
\newcommand{\arc}{\mathop{\mathrm{arc}}\nolimits}
\renewcommand{\arcsin}{\arc\sin}
\renewcommand{\arccos}{\arc\cos}
\newcommand{\arctg}{\arc\tg}
\newcommand{\arcctg}{\arc\ctg}
\newcommand{\sgn}{\operatorname{sgn}}


\parskip 8pt
\begin{document}
	
	\section*{Számelmélet}
	
	
	
	\subsection*{2011. 11. 09. -- Kongruenciák}
	\underline{Definíció:} $a$, $b$ egészek, $m>0$ egész, $a\equiv b\ \pmod{m}$, ha $m\mid  a-b$.
	\begin{enumerate}
		\item Igazoljuk a következő alaptulajdonságokat:
		\begin{abc}
			\item $a\equiv a\ (m)$;
			\item $a\equiv b\ (m) \Rightarrow b\equiv a\ (m)$;
			\item $a\equiv b\ (m)$ és $b\equiv c\ (m) \Rightarrow a\equiv c\ (m)$.
		\end{abc}
		\item Igazoljuk, hogy (azonos modulus esetén) kongruenciák mindkét oldalához hozzáadhatunk egy egész számot; szabad kongruenciákat összevonni; szabad mindkét oldalt egy egész számmal szorozni; szabad kongruenciákat összeszorozni; szabad hatványozni.
		\item Kongruenciák felhasználásával igazoljuk:
		\begin{abc2}
			\item $17\mid 4^{80}+1$;
			\item $7\mid 333^{444}+444^{333}$;
			\item $13\mid 3^{3n+2}+4$, ha $n\ge0$, egész;
			\item $15\mid 2^{4n}-1$, ha $n\ge0$, egész;
			\item $a-b\mid  a^n-b^n$, ha $n\ge$, egész, $a$, $b$ egészek.
			\item $13\mid 2^{60}+7^{30}$;
			\item $24\mid 5^{20}-1$.
		\end{abc2}
	\end{enumerate}
	
	
	\subsection*{2011. 11. 10.}
	\begin{enumerate}
		\item Igazoljuk, hogy $7\mid 2222^{5555}+5555^{2222}$.
		\item Határozzuk meg a következő számok tízes számrendszerbeli alakjának utolsó számjegyét:
		\begin{abc2}
			\item $9^{(9^9)}$;
			\item $2^{(3^4)}$.
		\end{abc2}
		\item Igazoljuk, hogy bármely $n\ge0$ egész esetén
		
		$35\mid 3^{6n}-2^{6n}$
		\item Igazoljuk, hogy
		\begin{abc2}
			\item $3^6\equiv1\ (7)$;
			\item $5^{10}\equiv1\ (11)$.
		\end{abc2}
		\item * Igazoljuk Fermat tételét: ha $(p;a)=1$ és $p$ prím, akkor
		\begin{center}
			{$\underline{a^{p-1}\equiv1\ \pmod{p}}$}
		\end{center}
		\item Igazoljuk, hogy ha $(11;n)=1$, $n>0$ egész, akkor $11\mid  n^5-1$ vagy $11\mid  n^5+1$.
		\item Határozzuk meg, hogy $247^{244}$ mennyi maradékot ad 23-mal osztva.
	\end{enumerate}
	
	
	\subsection*{2011. 11. 14.}
	\begin{enumerate}
		\item \underline{Definíció}: $\varphi(n)$ jelölje az $n$-nél nem nagyobb pozitív egészek közt az $n$-hez relatív prímek számát.\\
		Számítsuk ki a következőket:
		\begin{abc3}
			\item $\varphi(11)$;
			\item $\varphi(13)$;
			\item $\varphi(10)$;
			\item $\varphi(49)$;
			\item $\varphi(100)$;
			\item $\varphi(1000)$.
		\end{abc3}
		\item * Igazoljuk \underline{Euler tételét:} ha $(m;a)=1$, akkor $a^{\varphi(n)}\equiv1\ \pmod{n}$.
		\item Számítsuk ki, hogy
		\begin{abc2}
			\item $2^{999}$,
			\item $3^{999}$
		\end{abc2}
		-nek mi az utolsó két számjegye.
		\item Igazoljuk Wilson tételét: ha $p$ prím, akkor $(p-1)!\equiv-1\ \pmod{p}$.
		\item Oldjuk meg az egész számok körében:
		\begin{abc2}
			\item $20x\equiv4\ \pmod{30}$;
			\item $15x\equiv24\ \pmod{35}$.
		\end{abc2}
	\end{enumerate}
	
	
	\subsection*{2011. 11. 16.}
	Használjunk kongruenciákat, igazoljuk:
	\begin{enumerate}
		\item $7\mid 37^{n+2}+16^{n+1}+23^n$;
		\item $25\mid 72^{2n+2}-47^{2n}+28^{2n-1}$;
		\item $7\mid 3^{2n+1}+25^{2n+1}$;
		\item $19\mid 5^{2n+1}\cdot2^{n+1}+3^{n+1}\cdot2^{2n-1}$;
		\item $181\mid 3^{105}+4^{105}$;
		\item $7\mid 16^n(2^n+1)+9^n(9^{n+1}-1)+5^n(5^{n+2}-5^n)$;
		\item $11\mid 30^n+4^n(3^n-2^n)-1$;
		\item $117\mid 3^{2(n+1)}\cdot5^{2n}-3^{3n+2}\cdot2^{2n}$.
	\end{enumerate}
	
	
	\subsection*{2011. 11. 21.}
	\begin{enumerate}
		\item Oldjuk meg a következő kongruenciákat:
		\begin{abc3}
			\item $2x\equiv5\ (21)$;
			\item $15x\equiv1\ (31)$;
			\item * $3x\equiv18\ (48)$;
			\item $2^9x\equiv71\ (283)$;
			\item $45x\equiv72\ (84)$.
		\end{abc3}
		\item Melyek azok az egész számok, amelyek 3-mal osztva 1, 4-gyel osztva 2, 5-tel osztva 3 maradékot adnak?
		\item Igazoljuk, hogy ha $a$, $b$ egészek és $p$ prím, akkor $(a+b)^p\equiv a^p+b^p\ (p)$.
		\item Mi az utolsó két számjegye a $3^{400}$ szám tízes számrendszerbeli alakjának?
		\item Melyik az a legkisebb $x$ pozitív egész szám, amelyre igaz, hogy $13\mid  x^2+1$?
	\end{enumerate}
	
	
	\subsection*{2011. 11. 23.}
	\begin{enumerate}
		\item Igazoljuk, hogy $84\mid 4^{2n}-3^{2n}-7$ ha $n\ge1$, egész.
		\item Igazoljuk, hogy $44\mid 19^{19}+69^{69}$.
		\item Oldjuk meg az egész számok halmazán: $21x-34y=2$.
		\item * Igazoljuk, hoghy ha $n\ge1$, egész, akkor $676\mid 3^{3n+3}-26n-27$.
		\item Határozzuk meg a 3-nak azokat a többszöröseit, amelyek 7-tel osztva 2-t adnak maradékul.
		\item Egy négyjegyű tízes számrendszerbeli szám 131-gyel osztva maradékul 112-t, 132-vel osztva 88-at ad. Melyik ez a szám?
	\end{enumerate}
	
	
	\subsection*{2011. 11. 28. -- Dolgozat}
	\begin{enumerate}
		\item Igazoljuk, hogy $11\cdot31\cdot61\mid 20^{15}-1$.
		\item Igazoljuk, hogy ha $n\ge1$, egész, akkor $528\mid 7^{2n}-4^{2n}-33$.
		\item Igazoljuk, hogy ha $n\ge0$ egész szám, akkor $8\mid 3^{6n}+3^{5n+1}+3^{4n+1}+3^{3n}$.
		\item Oldjuk meg (az egész számok halmazán) a következő kongruenciákat:
		\begin{abc2}
			\item $9x \equiv 14\ (17)$;
			\item $3^8 x \equiv 23\ (100)$.
		\end{abc2}
		\item Oldjuk meg az egész számok halmazán a következő egyenletet: $60x-77y=1$.
	\end{enumerate}
	
	
	\subsection*{2011. 11. 30.}
	\begin{enumerate}
		\item Jelölje $d(n)$ az $n$ pozitív egész szám pozitív osztóinak számát. Hogyan lehet $d(n)$ értékét kiszámítani?
		\item Számítsuk ki a $d(n)$ értékét, ha $n = 72$; $100$; $121$; $1000$; $96$; $3600$.
		\item * Igazoljuk, hogy ha $(n;k)=1$, $n$, $k>0$ egészek, akkor $\varphi(n\cdot k)=\varphi(n)\cdot\varphi(k)$.
		\item Adjunk meg $\varphi(n)$ értékeinek kiszámítására képletet, ha $n=p^{\alpha}$, ahol $p$ prím, $\alpha\ge1$ egész.
		\item Igazoljuk, hogy $\sum\limits_{d|k}\varphi(d)=n$, ha $n=p^{\alpha}$, ahol $p$ prím, $\alpha\ge1$, egész.
		\item Adjunk meg képletet $\varphi(n)$ kiszámítására, ha $n\ge1$, egész.
		\item Az $n$ melyik értékeire lesz $\varphi(n)$ páratlan?
	\end{enumerate}
	
	
	\subsection*{2011. 12. 01. -- Beadható feladatok}
	\begin{enumerate}
		\item Mi az utolsó számjegye a $403^{402}$ szám tízes számrendszebeli alakjának?
		\item Mi az utolsó két számjegye: $29^{(39^{40})}$?
		\item Mi az utolsó két számjegye: $7^{(7^{1000})}$?
		\item Mi az utolsó két számjegye: $14^{(14^{14})}$?
		\item Mennyi maradékot ad
		\begin{center}
			$10^{10}+10^{(10^2)}+10^{(10^3)}+\ldots+10^{(10^{10})}$,
		\end{center}
		ha 7-tel osztjuk?
	\end{enumerate}
	
	
	\subsection*{2011. 12. 07.}
	\begin{enumerate}
		\item Mely $x>0$ egészekre igaz, hogy $\varphi(2^x)=128$?
		\item Igazoljuk, hogy ha $n=2^k$, ahol $k>0$, egész, akkor $\varphi(n)=\dfrac{n}{2}$
		\item Melyek azok a $p$, $q>0$ különböző prímek, amelyekre $\varphi(p^2q^2)=2200$?
		\item Jelölje $\varsigma(n)$ az $n>0$ pozitív egész szám pozitív osztóinak összegét. Számítsuk ki a következőket:
		\begin{abc3}
			\item $\varsigma(6)$;
			\item $\varsigma(8)$;
			\item $\varsigma(18)$;
			\item $\varsigma(19)$;
			\item $\varsigma(31)$;
			\item $\varsigma(243)$;
			\item $\varsigma(1000)$.
		\end{abc3}
		\item Igazoljuk, hogy ha $n=p_1^{\alpha_1}p_2^{\alpha_2}$, $p_1\ne p_2$ prímek, akkor $\varsigma(n)=\varsigma(p_1^{\alpha_1})\cdot\varsigma(p_2^{\alpha_2})$.
		\item Számítsuk ki:
		\begin{abc4}
			\item $\varsigma(2^{31})$,
			\item $\varsigma(p^4)$,
			\item $\varsigma(p^{11})$,
			\item $\varsigma(p^k)$
		\end{abc4}
		ha $p>0$ prím.
	\end{enumerate}
	
	
	\subsection*{2011. 12. 08.}
	\begin{enumerate}
		\item Igazoljuk, hogy ha az $n>0$ egész szám pozitív osztói $d_1$, $d_2$, $\ldots$, $d_k$, akkor 
		\begin{center}
			$\dfrac{1}{d_1}+\dfrac{1}{d_2}+\ldots+\dfrac{1}{d_k}=\dfrac{\varsigma(n)}{n}$.
		\end{center}
		\underline{Definíció:} $n>0$ tökéletes szám, ha $\varsigma(n)=2n$. Pl. 6, 28 tökéletes számok.
		\item * Igazoljuk, hogy ha $n=2^{p-1}(2^p-1)$ ahol $p$ prím és $2^p-1$ is prím, páros tökéletes szám (Euklidész).
		\item Igazoljuk, hogy minden páros tökéletes szám 6-ra, vagy 8-ra végződik.
		\item Oldjuk meg a $\varsigma(p^x)=2801$, ahol $p$ prím, $x>0$, egész.\\\\
		\underline{Definíció:} Az $f(n)$ függvény számelméleti függvény, ha $n>0$ egész esetén $f(n)>0$ egész is teljesül. Az $f(n)$ számelméleti függvény összegzési függvénye $g(n)$, ha $g(n)=\sum\limits_{d|n}f(d)$ minden $n>0$ egészre igaz.
		\item Határozzuk meg az $f(n)\equiv1$ és $f(n)=n$ függvények öszegzési függvényét.
	\end{enumerate}
	
	
	\subsection*{2011. 12. 12. -- Beadható dolgozat}
	\begin{enumerate}
		\item Tudjuk, hogy $p$ és $q$ ikerprímek, $n=p\cdot q$ és $\varphi(n)=120$. Mi lehet $p$ és $q$ értéke?
		\item Oldjuk meg a pozitív egészek körében:
		\begin{abc2}
			\item $\varphi(5^x)=100$;
			\item $\varphi(7^x)=294$.
		\end{abc2}
		\item Mi lehet $p$ és $x>0$, egész értéke, ha $p$ prím, és $\varsigma(p^x)=781$.
		\item Oldjuk meg a pozitív egészek körében:\\
		$\varphi(x)=40$, $\varsigma(x)=x+1$.
		\item Legyen $s_m$ az $1$, $2$, $3$, $\ldots$, $m-1$ számok kötül az $m$-hez relatív prímek összege. Igazoljuk, hogy
		\begin{center}
			$s_m=\dfrac{m\cdot\varphi(m)}{2}$.
		\end{center}
	\end{enumerate}
	
	
	
\end{document}