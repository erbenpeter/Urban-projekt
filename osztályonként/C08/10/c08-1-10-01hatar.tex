\documentclass{article}
\usepackage[utf8]{inputenc}
\usepackage{t1enc}
\usepackage{geometry}
 \geometry{
 a4paper,
 total={210mm,297mm},
 left=20mm,
 right=20mm,
 top=20mm,
 bottom=20mm,
 }
\usepackage{amsmath}
\usepackage{amssymb}
\frenchspacing
\usepackage{fancyhdr}
\pagestyle{fancy}
\lhead{Urbán János tanár úr feladatsorai}
\chead{C08/10/1.}
\rhead{Sorozatok határértéke}
\lfoot{}
\cfoot{\thepage}
\rfoot{}

\usepackage{enumitem}
\usepackage{multicol}
\usepackage{calc}
\newenvironment{abc}{\begin{enumerate}[label=\textit{\alph*})]}{\end{enumerate}}
\newenvironment{abc2}{\begin{enumerate}[label=\textit{\alph*})]\begin{multicols}{2}}{\end{multicols}\end{enumerate}}
\newenvironment{abc3}{\begin{enumerate}[label=\textit{\alph*})]\begin{multicols}{3}}{\end{multicols}\end{enumerate}}
\newenvironment{abc4}{\begin{enumerate}[label=\textit{\alph*})]\begin{multicols}{4}}{\end{multicols}\end{enumerate}}
\newenvironment{abcn}[1]{\begin{enumerate}[label=\textit{\alph*})]\begin{multicols}{#1}}{\end{multicols}\end{enumerate}}
\setlist[enumerate,1]{listparindent=\labelwidth+\labelsep}

\newcommand{\degre}{\ensuremath{^\circ}}
\newcommand{\tg}{\mathop{\mathrm{tg}}\noits}
\newcommand{\ctg}{\mathop{\mathrm{ctg}}\noits}
\newcommand{\arc}{\mathop{\mathrm{arc}}\nolimits}
\renewcommand{\arcsin}{\arc\sin}
\renewcommand{\arccos}{\arc\cos}
\newcommand{\arctg}{\arc\tg}
\newcommand{\arcctg}{\arc\ctg}

\parskip 8pt
\begin{document}

\section*{Sorozatok határértéke}

\subsection*{2011. 09. 01.-- Ismétlő feladatok}
\begin{enumerate}
\item Számítsuk ki a következő összegeket:
	\begin{abc}
	\item $1+\dfrac{1}{2}+\dfrac{1}{2^2}+\ldots+\dfrac{1}{2^{n-2}}$;
	\item $\dfrac{1}{1\cdot2}+\dfrac{1}{2\cdot3}+\dfrac{1}{3\cdot4}+\ldots+\dfrac{1}{(n-1)\cdot n}$;
    \item $\dfrac{1}{2}+2\cdot{\dfrac{1}{2^2}}+3\cdot{\dfrac{1}{2^3}}+\ldots+n{\cdot\dfrac{1}{(n-1)^n}}$;
     \item $\dfrac{7}{2\cdot9}+2\cdot\dfrac{7}{9\cdot16}+\ldots+\dfrac{7}{(7_n-5)(7_n+2)}$.
    \end{abc}
\item Számítsuk ki a következő sorozatok első $n$ tagjának összegét:
	\begin{abc}
    \item $1, 11, 111, \ldots, \underbrace{11\ldots 1}_{\text{$n$ db $1$}},\ldots$;
    \item $2\cdot1^2, 3\cdot2^2, \ldots, (n+1)\cdot{n^2}, \ldots$;
    \item $1\cdot2, 2\cdot3, \ldots, n\cdot(n+1), \ldots$;
    \item (*) $2\cdot1^3, 3\cdot2^3, \ldots, (n+1)\cdot{n^3}, \ldots$;
    \item $1^2, 3^2, 5^2, \ldots, (2n-1)^2, \ldots$.
    \end{abc}
    
\end{enumerate}

\subsection*{2011. 09. 05.}
\begin{enumerate}
\item Szemléltessük az
	$\dfrac{1}{4}+\dfrac{1}{4^2}+\ldots+\dfrac{1}{4^n}+\ldots$
	végtelen összeget. Mennyi lehet ennek az értéke?

\item Melyik az a szám, amit az
   $\dfrac{1}{2}+\dfrac{1}{2^2}+\dfrac{1}{2^3}+\ldots+\dfrac{1}{2^n}$
   sorozat tetszőleges pontossággal megközelít, ha $n$ elég nagy?

\item (*) Mutassuk meg, hogy az
	$1+\dfrac{1}{2}+\dfrac{1}{3}+\ldots+\dfrac{1}{n}$
	összeg akármilyen nagy lehet, ha $n$ elég nagy.

\item Melyik számot közelíti meg az
	$\dfrac{1}{1\cdot2}+\dfrac{1}{2\cdot3}+\ldots+\dfrac{1}{(n-1)n}$
   összeg tetszőleges pontossággal, ha $n$ elég nagy?

\item Vizsgáljuk meg a következő sorozatokat, melyik számot közelítik meg tetszőleges pontossággal elég nagy $n$-re:
	\begin{abc}
    \item $1-\dfrac{1}{2}+\dfrac{1}{4}-\dfrac{1}{8}+\ldots+(-1)^n\dfrac{1}{2^n}$;
    \item $1$; $1,1$; $1,11$; $1,111$;\ldots $1,\underbrace{1\ldots1}_{\text{=$n$ db $1$}}$;
    \item $1, 1+\dfrac{1}{1}, 1+\dfrac{1}{1+\dfrac{1}{1}}, 1+\dfrac{1}{1+\dfrac{1}{1+\dfrac{1}{1}}}\ldots$.
	\end{abc}
\end{enumerate}

\subsection*{2011. 09. 07.}
\begin{enumerate}
\item \underline{Definíció:} Azt mondjuk, hogy az $a_n$ sorozat \underline{konvergens} és \underline{határértéke} $A$, ha $A$ bármely környezetét adjuk meg, van olyan $n_0$, hogy ha $n>n_0$, akkor $a_n$ már $A$-nak ebben a környezetében van.
\\
\underline{jelölés:} $\displaystyle\lim_{n \to \infty} a_n=A$, vagy röviden $a_n \to A$ ha $n \to \infty$.

\item Határozzuk meg a következő sorozatok határértékét:
	\begin{abc}
	\item $\left(\dfrac{1}{n}\right)$;
    \item $\left(\dfrac{n-1}{n}\right)$;
    \item $\left(\dfrac{1}{2^n}\right)$;
    \item (*) $a_1=\underbrace{\sqrt{2+\sqrt{2+\ldots+\sqrt{2}}}}_{\text{$n$ db gyökjel}}$;
    \item (*) $\left(\dfrac{2^n+3^n}{5^n+7^n}\right)$;
    \item (*) $(q^n)$, $|q|<1$.
	\end{abc}

Szemléltessük az
	$\dfrac{1}{4}+\dfrac{1}{4^2}+\ldots+\dfrac{1}{4^n}+\ldots$
	végtelen összeget. Mennyi lehet ennek az értéke?

\item Melyik az a szám, amit az
   $\dfrac{1}{2}+\dfrac{1}{2^2}+\dfrac{1}{2^3}+\ldots+\dfrac{1}{2^n}$
   sorozat tetszőleges pontossággal megközelít, ha $n$ elég nagy?

\end{enumerate}

\subsection*{2011. 09. 12.}
\begin{enumerate}
\item Igazoljuk, hogy ha egy sorozat monoton növő és felülről korlátos, akkor van határértéke.
\item (*) Igazoljuk, hogy $n$ darab nemnegatív számra is igaz a számtani és mértani közép közötti egyenlőtlenség.
\item Igazoljuk, hogy
	\begin{abc}
	\item az $a_n= \left(1+\dfrac{1}{n}\right)^n$ sorozat monoton nő;
    \item a $b_n= \left(1+\dfrac{1}{n}\right)^{n+1}1$ sorozat monoton fogy;
    \item (*) $\displaystyle\lim_{n \to \infty} a_n= \displaystyle\lim_{n \to \infty} b_n$.
	\end{abc}

\item Számítsuk ki a következő sorozatok határértékét:
	\begin{abc}
	\item $a_n\left(1-\dfrac{1}{n^2}\right)^n$;
    \item $a_n\left(1-\dfrac{1}{n}\right)^n$;
    \item $a_n\left(1-\dfrac{2}{n}\right)^n$.
	\end{abc}
\end{enumerate}

\subsection*{2011. 09. 14.}
\begin{enumerate}
	\item Számítsuk ki a következő sorozatok határértékét:
	\begin{abc}
	\item $a_n=\dfrac{3n^2+5n+4}{2+n^2}$;
    \item $a_n=\dfrac{1^2+2^2+\ldots+n^2}{5n^3+n+1}$;
	\item $a_n=\sqrt[n]{5}$;
    \item (*) $a_n=\sqrt[n]{n}$;
    \item $a_n=\sqrt[n]{6n+3}$;
    \item $a_n=\sqrt{2n+3}-\sqrt{n-1}$;
    \item $a_n=\sqrt{n^2+n+1}-\sqrt{n^2-n+1}$;
    \item $a_n=\dfrac{\sqrt{n}}{\sqrt{n+1}+\sqrt{n}}$;
    \item $a>1, a_1>0, a_{n+1}=\dfrac{a_n}{a+a_n}$;
    \item (*) $a_n=\dfrac{10^n}{n!}$.
\end{abc}
\end{enumerate}

\subsection*{2011. 09. 15.}
\begin{enumerate}
\item Határozzuk meg a következő sorozatok határértékét:
	\begin{abc}
	\item $a_n=\dfrac{\sqrt[n]{e}-1}{\dfrac{1}{n}}$;
    \item $a_n=\dfrac{1\cdot3\cdot5\cdot\ldots\cdot(2n-1)}{2\cdot4\cdot6\cdot\ldots\cdot2n}$;
    \item $a_n=\sqrt[n]{5}$;
    \item $a_n=\sqrt[n]{\dfrac{1}{5}}$;
    \item $a_n=\sqrt[n]{n}$;
    \item $a_n=\sqrt[n]{n^2+1}$.
	\end{abc}
\item Igazoljuk, hogy az $a_n=1-\dfrac{1}{2}+\dfrac{1}{3}-\dfrac{1}{4}+\ldots+\dfrac{1}{2n-1}-\dfrac{1}{2n}$ sorozat konvergens.
\end{enumerate}


\subsection*{2011. 09. 19.}
\begin{enumerate}
\item Számítsuk ki a következő sorozatok határértékét:
	\item $a_n=\dfrac{1^2+2^2+\ldots+n^2}{n^2}-\dfrac{n}{4}$;     
    \item $a_n=\sqrt[n]{1+x^n}$, $x\ge0$;
    \item $a_n=\sqrt[n]{1+x^n+\left(\dfrac{x^2}{2}\right)^n}$, $x\ge0$;
    \item $a_n=\dfrac{a^n-a^{-n}}{a^n+a^{-n}}$, $x\not=0$;
    \item (*) $a_n=x\mathrm{sgn}(\sin^(n!)\pi x)$;   
    \item $a_1=1,$ $a_n=\dfrac{1}{1+a_n}$;
    \item $a_n=3,$ $a_{n+1}=\dfrac{1}{2}\left(a_n+\dfrac{5}{a_n}\right)$.
\end{enumerate}


\subsection*{2011. 09. 22.}
\begin{enumerate}
\item Számítsuk ki a következő sorozatok határértékét:
	\item $a_n=\dfrac{1^2+3^2+\ldots+(2n-1)^2}{2^2+4^2+\ldots+(2n)^2}$;
    \item $a_n=\dfrac{n^k}{a^m}$, $a>1$, $k>0$, egész;
    \item (*) $a_n=1+\dfrac{1}{1!}+\dfrac{1}{2!}+\dfrac{1}{3!}+\ldots+\dfrac{1}{n!}$;
    \item $a_n=\left(1+\dfrac{1}{2}\right)\left(1+\dfrac{1}{4}\right)\cdot\ldots\cdot\left(1+\dfrac{1}{2^{2^2}}\right)$;
    \item $a_1=a$, $a_2=b$, $(a<b)$,\\ $a_{n+2}=\dfrac{a_n+a_{n+1}}{2}$;
    \item $a_n=\dfrac{x^{n+2}}{\sqrt{2^{2n}+x^{2n}}}$, $x\ge 0$.
\end{enumerate}

\subsection*{2011. 09. 26. -- Témazáró dolgozat}
\begin{enumerate}
	\item $\displaystyle\lim_{n \to \infty}$ $\dfrac{100n}{n^2+1}$;
	\item $\displaystyle\lim_{n \to \infty}$ $\dfrac{(-2)^n+3}{(-2)^{n+1}+3^{n+1}}$;
    \item $\displaystyle\lim_{n \to \infty}$ $n\cdot q^n$, $0<q<1$;
	\item $\displaystyle\lim_{n \to \infty}$ $\sin^{2n}x$;
    \item $a_1=0$, $a_2=1$, \\ $a_{n+2}=\dfrac{a_n+a_{n+1}}{2}$, \\ $\displaystyle\lim_{n \to \infty} a_n=?$
    \item $a_1=\sqrt{a}$, $a>0$\\ $a_{n+1}=\sqrt{a+a_n}$, \\ $\displaystyle\lim_{n \to \infty} a_n=?$





\end{enumerate}


\end{document}