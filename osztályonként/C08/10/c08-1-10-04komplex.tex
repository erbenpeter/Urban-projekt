\documentclass{article}
\usepackage[utf8]{inputenc}
\usepackage{t1enc}
\usepackage{geometry}
 \geometry{
 a4paper,
 total={210mm,297mm},
 left=20mm,
 right=20mm,
 top=20mm,
 bottom=20mm,
 }
\usepackage{amsmath}
\usepackage{amssymb}
\frenchspacing
\usepackage{fancyhdr}
\pagestyle{fancy}
\lhead{Urbán János tanár úr feladatsorai}
\chead{C08/10/1.}
\rhead{Komplex számok}
\lfoot{}
\cfoot{\thepage}
\rfoot{}

\usepackage{enumitem}
\usepackage{multicol}
\usepackage{calc}
\newenvironment{abc}{\begin{enumerate}[label=\textit{\alph*})]}{\end{enumerate}}
\newenvironment{abc2}{\begin{enumerate}[label=\textit{\alph*})]\begin{multicols}{2}}{\end{multicols}\end{enumerate}}
\newenvironment{abc3}{\begin{enumerate}[label=\textit{\alph*})]\begin{multicols}{3}}{\end{multicols}\end{enumerate}}
\newenvironment{abc4}{\begin{enumerate}[label=\textit{\alph*})]\begin{multicols}{4}}{\end{multicols}\end{enumerate}}
\newenvironment{abcn}[1]{\begin{enumerate}[label=\textit{\alph*})]\begin{multicols}{#1}}{\end{multicols}\end{enumerate}}
\setlist[enumerate,1]{listparindent=\labelwidth+\labelsep}

\newcommand{\degre}{\ensuremath{^\circ}}
\newcommand{\tg}{\mathop{\mathrm{tg}}\nolimits}
\newcommand{\ctg}{\mathop{\mathrm{ctg}}\nolimits}
\newcommand{\arc}{\mathop{\mathrm{arc}}\nolimits}
\renewcommand{\arcsin}{\arc\sin}
\renewcommand{\arccos}{\arc\cos}
\newcommand{\arctg}{\arc\tg}
\newcommand{\arcctg}{\arc\ctg}

\parskip 8pt
\begin{document}

\section*{Komplex számok}

\subsection*{2011.12.14.}

\underline{Definíció:} Az $x^2+1=0$ egyenletről megköveteljük, hogy legyen gyöke, ezek közül az egyiket így jelöljük: $i$, tehát $i^2=-1$. Az $a+ib$ alakú számokat, ahol $a$ és $b$ valós számok \underline{komplex számoknak} nevezzük.

Az összeadás és szorzás definíciója:
\begin{align*}
(a+ib)+(c+id)&=(a+c)+i(b+d)\cr
(a+ib)\cdot(c+id)&=(ac-bd)+i(ad+bc)
\end{align*}

\begin{enumerate}
\item Végezzük el a következő műveleteket:
\begin{abc4}
\item $(1+i)(1-i)$;
\item $(3-5i)(4-2i)$;
\item $(1+i)^4$;
\item $(3+2i)^3$.
\end{abc4}
\item Megállapodás: az $a+ib$ komplex számnak a képe a $P(a;b)$ pont vagy az $\overrightarrow{OP}$ vektor. Mi felel meg az összeadásnak? Mi a geometriai jelentése az $i$-vel való szorzásnak?
\item Megállapodás: $|a+ib|=\sqrt{a^2+b^2}$. Mi a geometriai jelentése az $1+i$-vel való szorzásnak?
\end{enumerate}

\subsection*{2011.12.15.}
\begin{enumerate}
\item Számítsuk ki:
\begin{abc4}
\item $\dfrac{(1+3i)(-2+2i)}{1+2i}$;
\item $\dfrac{1+2i}{3-4i}+\dfrac{2-i}{5i}$;
\item $(1+i)^5$;
\item $\dfrac{1-i^2+i^4-i^6+\ldots-i^{18}}{1+i+i^2+i^3+\ldots+i^9}$.
\end{abc4}
\item Hol vannak a komplex számsíkon azok a pontok, amelyekre teljesül:
\begin{abc3}
\item $\operatorname{Re}z>2$;
\item $1<\operatorname{Im}z<2$;
\item $|z|<3$.
\end{abc3}
\item Írjuk fel trigonometrikus alakban:
\begin{abc3}
\item $1-i$;
\item $-2-2i$;
\item $\sqrt{12}-2i$.
\end{abc3}
\item Végezzük el:
\begin{abc3}
\item $(1-i)^6$;
\item $(2+i\sqrt{12})^5$;
\item $\dfrac{(1+i)^{100}}{(1-i)^{36}-i(1+i)^{98}}$.
\end{abc3}
\end{enumerate}

\subsection*{2011.12.19.}
\begin{enumerate}
\item Oldjuk meg a komplex számok halmazán:
\begin{abc3}
\item $|z|+z=1+2i$;
\item $\overline{z}=2-z$;
\item $|(3-i)z|-2(1-2i)z=-5i$.
\end{abc3}
\item Számítsuk ki:
\begin{abc3}
\item $\sqrt{5+12i}$;
\item $\sqrt{24+10i}$;
\item $\sqrt{1+i\sqrt 3}+\sqrt{1-i\sqrt 3}$.
\end{abc3}

\item Oldjuk meg a komplex számok halmazán:
\begin{abc2}
\item $z^2+(6+i)z+5+5i=0$;
\item $z^2-(5+5i)z+2+11i=0$.
\end{abc2}

\item Számítsuk ki:
\begin{abc3}
\item $(1+2i)^3$;
\item $(2+i)^5$;
\item $(1-i)^6+(1+i)^6$;
\item $\left(\frac{4}{\sqrt 3 + i}\right)^{12}$;
\item $\dfrac{(1+i)^{100}}{(1-i)^{96}+(1+i)^{96}}$.
\end{abc3}

\end{enumerate}

\subsection*{2011.12.21.}
\begin{enumerate}
\item Számítsuk ki
\begin{abc3}
\item $(\cos \varphi-i\sin \varphi)^n$;
\item $\dfrac{(1+i\ctg \varphi)^5}{(1-i\ctg \varphi)^5}$;
\item $(1+\cos \varphi+i\sin \varphi)^n$.
\end{abc3}
\item Fejezzük ki $\sin x$ és $\cos x$ segítségével:
\begin{abc2}
\item $\sin 4x$-et és $\cos 4x$-et;
\item $\sin 6x$-et és $\cos 6x$-et.
\end{abc2}
\item Fejezzük ki $\tg 4x$-et $\tg x$ segítségével.
\item Igazoljuk komplex számok felhasználásával:
\begin{abc3}
\item $\cos \frac{\pi}{5}+\cos \frac{3\pi}{5}=\frac{1}{2}$;
\item $\cos \frac{2\pi}{5}+\cos \frac{4\pi}{5}=-\frac{1}{2}$;
\item $\cos \frac{\pi}{7}+\cos \frac{3\pi}{7}+\cos \frac{5\pi}{7}=\frac{1}{2}$.
\end{abc3}
\end{enumerate}

\subsection*{2012.01.05.}
\begin{enumerate}
\item Igazoljuk komplex számok felhasználásával és általánosítsuk:
\begin{abc2}
\item $\binom{9}{0}-\binom{9}{2}+\binom{9}{4}-\binom{9}{6}+\binom{9}{8}=16$;
\item $\binom{9}{1}-\binom{9}{3}+\binom{9}{5}-\binom{9}{7}+\binom{9}{9}=16$.
\end{abc2}
\item Számítsuk ki komplex számok felhasználásával a következő összegeket:
\begin{abc2}
\item $\binom{2k}{0}-\binom{2k}{2}+\binom{2k}{4}-\ldots+(-1)^k\binom{2k}{2k}$;
\item $\binom{2k}{1}-\binom{2k}{3}+\binom{2k}{5}-\ldots+(-1)^{k-1}\binom{2k}{2k-1}$.
\end{abc2}

\item A $z^n=1$ ($n>0$, egész) egyenlet komplex megoldásait \underline{$n$-edik egységgyököknek} nevezzük. Számítsuk ki a 3., 4., 6. egységgyököket.
\item Oldjuk meg a komplex számok körében a következő egyenleteket:
\begin{abc3}
\item $z^6=1+i\sqrt 3$;
\item $z^4=-1-i$;
\item $z^6=\sqrt 3-i$.
\end{abc3}
\end{enumerate}

\subsection*{2012.01.09.}
\begin{enumerate}
\item Számítsuk ki:
\begin{abc2}
\item $(1-i)(2-i)(1+2i)$;
\item $\dfrac{4+3i}{1+2i}$.
\end{abc2}
\item Végezzük el a kijelölt műveleteket:
\begin{abc2}
\item $(-1+i)^{12}$;
\item $\dfrac{(1+i\sqrt 3)^4}{(1+i)^8}$.
\end{abc2}

\item Oldjuk meg a komplex számok halmazán: 
\begin{abc2}
\item $z^4=-4$;
\item $z^2+(6+i)z+5+5i=0$.
\end{abc2}

\item Számítsuk ki a következő összegeket, ha $k>0$, egész:
\begin{abc2}
\item $\binom{4k}{0}-\binom{4k}{2}+\binom{4k}{4}-\ldots+\binom{4k}{4k}$;
\item $\binom{4k}{1}-\binom{4k}{3}+\binom{4k}{5}-\binom{4k}{4k-1}$.
\end{abc2}

\end{enumerate}

\subsection*{2012.01.11.}
A CARDANO-képlet a harmadfokú egyenlet megoldására:
$$x^3+px+q=0$$
alakú harmadfokú egyenletet oldunk meg, ahol $p,q$ adott valós (komplex) számok.\\
Keressük $x$-et $u+v$ alakban:
$x=u+v\Rightarrow x^3=u^3+v^3+3uv(u+v)$,
ebből:
$x^3-3uvx-(u^3+v^3)=0$.

\begin{tabular}{c}
$u^3+v^3=-q$\cr
\hline
$3uv=-p$\cr
$u^3v^3=-\dfrac{p^3}{27}$\cr
\hline
\end{tabular}

\noindent
Tehát $u^3$ és $v^3$ a következő másodfokú egyenlet gyökei:
$$z^2+qz-\frac{p^3}{27}=0,$$
innen
$$z_{1,2}=-\frac{q}{2}\pm\sqrt{\left(\frac{q}{2}\right)^2+\left(\frac{p}{3}\right)^3},$$
tehát
$$x=\sqrt[3]{-\frac{q}{2}+\sqrt{\left(\frac{q}{2}\right)^2+\left(\frac{p}{3}\right)^3}}+
\sqrt[3]{-\frac{q}{2}-\sqrt{\left(\frac{q}{2}\right)^2+\left(\frac{p}{3}\right)^3}}.$$

\subsection*{2012.01.12.}
\begin{enumerate}
\item Oldjuk meg a Cardano-képlet felhasználásával a következő harmadfokú egyenletet:
$$x^3-9x^2+20x-12=0.$$
\item Oldjuk meg (tetszőleges módszerrel) a következő harmadfokú egyenleteket:
\begin{abc3}
\item $x^3+3x^2-x-3=0$;
\item $x^3-6x^2+11x-6=0$;
\item $x^3-9x^2+23x-15=0$.
\end{abc3}
\item ($*$) Oldjuk meg a komplex számok halmazán a következő \underline{negyedfokú} egyenletet:
$$x^4-8x^3+18x^2-27=0.$$
\end{enumerate}

\end{document}
