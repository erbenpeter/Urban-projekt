\documentclass{article}
\usepackage[utf8]{inputenc}
\usepackage{t1enc}
\usepackage{geometry}
 \geometry{
 a4paper,
 total={210mm,297mm},
 left=20mm,
 right=20mm,
 top=20mm,
 bottom=20mm,
 }
 
\usepackage{pgf,tikz}
\usepackage{mathrsfs}
\usetikzlibrary{arrows}
\usepackage{amsmath}
\usepackage{amssymb}
\usepackage{pgfplots}
\frenchspacing
\usepackage{fancyhdr}
\pagestyle{fancy}
\lhead{Urbán János tanár úr feladatsorai}
\chead{C08/10/6.}
\rhead{Vegyes feladatok}
\lfoot{}
\cfoot{\thepage}
\rfoot{}

\usepackage{pgf,tikz}

\usepackage{enumitem}
\usepackage{multicol}
\usepackage{calc}
\newenvironment{abc}{\begin{enumerate}[label=\textit{\alph*})]}{\end{enumerate}}
\newenvironment{abc2}{\begin{enumerate}[label=\textit{\alph*})]\begin{multicols}{2}}{\end{multicols}\end{enumerate}}
\newenvironment{abc3}{\begin{enumerate}[label=\textit{\alph*})]\begin{multicols}{3}}{\end{multicols}\end{enumerate}}
\newenvironment{abc4}{\begin{enumerate}[label=\textit{\alph*})]\begin{multicols}{4}}{\end{multicols}\end{enumerate}}
\newenvironment{abcn}[1]{\begin{enumerate}[label=\textit{\alph*})]\begin{multicols}{#1}}{\end{multicols}\end{enumerate}}
\setlist[enumerate,1]{listparindent=\labelwidth+\labelsep}

\newcommand{\degre}{\ensuremath{^\circ}}
\newcommand{\tg}{\mathop{\mathrm{tg}}\nolimits}
\newcommand{\ctg}{\mathop{\mathrm{ctg}}\nolimits}
\newcommand{\arc}{\mathop{\mathrm{arc}}\nolimits}
\renewcommand{\arcsin}{\arc\sin}
\renewcommand{\arccos}{\arc\cos}
\newcommand{\arctg}{\arc\tg}
\newcommand{\arcctg}{\arc\ctg}

\parskip 8pt
\begin{document}

\section*{Vegyes feladatok}

\subsection*{2012.02.20 -- Versenyfeladatok}
\begin{enumerate}
\item Igazoljuk, hogy ha $a$, $b$, $c$ egész számok és $a+b+c$ osztható 6-tal, akkor $a^3+b^3+c^3$ is osztható 6-tal.
\item Szerkesszük meg a háromszöget, ha adott az $a$ és $b$ oldala $(a<b)$ és a közbezárt szög $f$ felezője.
\item Tegyük fel, hogy $x,y>0$. Legfeljebb mekkora lehet $x,y+\frac{1}{x}$ és $\frac{1}{y}$ legkisebbik?
\item Oldjuk meg és diszkutáljuk az $$\frac{x+a}{x-a}+\frac{x+b}{x-b}=2$$ egyenletet, ahol $a,b$ valós paraméterek.
\item Három prímszám szorzata az összegük ötszörösével egyenlő. Mik lehetnek ezek a prímek?

\end{enumerate}

\subsection*{2012.02.22}
\begin{enumerate}
\item Az $1,2,3,4,9,10,12,13,...$ sorozatot úgy képezzük,hogy a $3$ nemnegatív egész kitevős hatványait, illetve különböző ilyen hatványok összegét nagyság szerint rendezzük. Mi lesz a sorozat 100. tagja?
\item Tudjuk,hogy $2^8+2^{12}+2^n$ ($n>0$, egész) teljes négyzet. Mi lehet n értéke?
\item A $p_1,p_2,...,p_n$ prímeket ($p_i$ az $i$-edik prím) két csoportra osztjuk. Az egyik csoportbeli prímek szorzata $A$, a másik csoportbelieké $B$. Tegyük fel, hogy $A+B< p_{n^2+1}$ és $|A-B| < p_{n^2+1}$. Igazoljuk, hogy $A+B$ és $|A-B|$ is prímszám. 
\item Határozzuk meg a $$(\sqrt{2}+\sqrt{3})^{2012}$$ számban a tizedesvessző előtt és utána álló számjegyet.
\item Igazoljuk, hogy ha $x,y,z>0$, akkor $$\frac{x}{y+z}+\frac{y}{x+z}+\frac{z}{x+y}>\frac{3}{2}$$ Nesbitt-egyenlőtlenség
\item Igazoljuk, hogy ha $a,b,c,d>0$ és $a+b+c+d=1$, akkor $$\sqrt{4a+1}+\sqrt{4b+1}+\sqrt{4c+1}+\sqrt{4d+1}<6.$$ 
\end{enumerate}


\subsection*{2012.02.27 -- Gráfelmélet }
\underline{Definíciók}:

\begin{description}
\item[\underline{$G$ gráf},] ha $G$ pontok és ezekből álló párok véges halmaza, a pontpárok az élek.

\item[\underline{Hurokél}:] a pár két eleme azonos;

\item[\underline{Többszörös él}:]
két pontot több él köt össze.

\item[\underline{Egyszerű gráf}:] nincs benne hurokél és többszörös él.

\item[\underline{Séta}:] $G$-ben élek egymáshoz csatlakozó sorozata; (véges)

\item[\underline{Vonal}:] olyan séta, amelyben minden él különböző;

\item[\underline{Út}:] olyan vonal, amely minden ponton legfeljebb egyszer halad át.

\item[\underline{Kör}:] olyan egymáshoz csatlakozó élsorozat, amelyben minden él legfeljebb egyszer szerepel és a kezdő és végpont azonos, ezeken kívül minden pont is legfeljebb egyszer szerepel.

\item[\underline{Pont foka}:] a pontból induló élek száma.

\end{description}

\subsection*{2012.02.27 -- Gráfelmélet }
\begin{enumerate}

\item Igazoljuk, hogy ha egy gráfban minden pont foka legalább 2, akkor van benne kör. 

\item Hány éle van egy $n$ pontból álló útnak és egy $n$ pontból álló körnek?

\item Egy $G$ gráf összefüggő,ha bármely pontjából bármely másik pontjába úton eljuthatunk.Igazoljuk, hogy ha egy összefüggő gráfban van kör, akkor a kör bármely élét elhagyva a gráf összefüggő marad.

\item Igazoljuk,hogy minden egyszerű gráfban a páratlan fokú csúcsok száma páros.

\item Igazoljuk, hogy ha $n$ csapat ($n\geq 4$) körmérkőzéses versenyén legalább $n+1$ mérkőzést lejátszottak, akkor van olyan csapat, amely legalább háromszor játszott.

\item Hány olyan 5 csúcsú egyszerű gráf van, amelynek van két harmadfokú és két negyedfokú csúcsa?

\item Egy 5 csúcsú egyszerű gráfnak 8 éle van. Mekkorák lehetnek a csúcsok fokszámai?

\item Igazoljuk, hogy ha egy egyszerű gráf minden csúcsának foka legalább 2, akkor van benne kör.

\item Igazoljuk, hogy ha egy $n$ csúcsú gráfnak legalább $n$ éle van, akkor van benne kör ($n\geq 3$).

\item Milyen szerkezetűek azok a gráfok, amelyekben minden csúcs fokszáma legfeljebb 2?
\end{enumerate}


\subsection*{2012.02.29 -- Gráfelmélet }

\underline{Definíciók}:

\begin{description}
\item[\underline{Összefüggő gráf}:] bármely pontjából bármely másik pontjába élek mentén el lehet jutni.

\item[\underline{Fa}:] összefüggő, körmentes egyszerű gráf.

\item[\underline{Teljes gráf}:] olyan egyszerű gráf, amelynek ha $n$ csúcsa van, akkor $\binom{n}{2}$\ éle van (az összes lehetséges él a gráfhoz tartozik).

\end{description}


\subsection*{2012.02.29 -- Gráfelmélet}
\begin{enumerate}
\item Adjunk meg olyan $6$ csúcsú, nem összefüggő gráfot, amelynek minden csúcsa másodfokú.

\item Adjunk meg  7 csúcsú, nem összefüggő, 15 élű gráfot.

\item Igazoljuk, hogy bármely egyszerű gráf vagy maga vagy komplementere összefüggő.

\item Tegyük fel, hogy $n\geq 4$, egész szám. Igazoljuk, hogy van olyan $n$ csúcsú gráf, ami összefüggő és a komplementere is összefüggő.

\item Igazoljuk,hogy bármely fagráfban van elsőfokú pont. Igaz-e, hogy van legalább $2$ elsőfokú pont?

\item Igazoljuk, hogy minden összefüggő gráfnak van olyan részgráfja, amely fa és az eredeti gráf összes pontját tartalmazza (a gráf faváza).

\item Tudjuk, hogy egy egyszerű gráfnak minden csúcsa legfeljebb másodfokú. Jellemezzük a gráfot.

\item Hány különböző  ötcsúcsú fa van?

\item Igazoljuk, hogy egy $n$ csúcsú fának $n-1$ éle van.

\item Rajzoljuk meg a $6$, illetve $7$ csúcsú fákat.

\item Hány olyan $8$ csúcsú egyszerű gráf van, amelynek minden csúcsa másodfokú?

\end{enumerate}


%\newpage
\subsection*{2012.03.05 -- A trigonometrikus függvények inverzei}

\begin{abc2}
\item $f: x \mapsto \arcsin x$\quad
$D_f=[-1;1]$\quad
$R_f=[-\frac{\pi}{2};\frac{\pi}{2}]$


\begin{tikzpicture}
    \begin{axis}[
        domain=-1:1,
        xmin=-2.1, xmax=2.1,
        ymin=-2, ymax=2,
        samples=400,
        axis lines=middle,
        ytick={-1.57,1.57}, yticklabels={$-\pi$/2,$\pi$/2}
    ]
        \addplot+[mark=none] {asin(x)/180*pi};
    \end{axis}
\end{tikzpicture}


\item $g: x \mapsto \arccos x$\quad
$D_g=[-1;1]$\quad
$R_g=[0;\pi]$

\begin{tikzpicture}
    \begin{axis}[
        domain=-1:1,
        xmin=-2.1, xmax=2.1,
        ymin=-0.1, ymax=3.5,
        samples=400,
        axis lines=middle,
        ytick={1.57,3.14}, yticklabels={$\pi$/2,$\pi$}
    ]
        \addplot+[mark=none] {acos(x)/180*pi};
    \end{axis}
\end{tikzpicture}

\item $h: x \mapsto \arc\tg x$\quad
$D_h=]-\infty ; \infty[ \quad R_h=]-\frac{\pi}{2};\frac{\pi}{2}[$

\begin{tikzpicture}
    \begin{axis}[
        domain=-3:3,
        xmin=-3.1, xmax=3.1,
        ymin=-2, ymax=2,
        samples=400,
        axis lines=middle,
        ytick={-1.57,1.57}, yticklabels={$-\pi$/2,$\pi$/2}
    ]
        \addplot+[mark=none] {atan(x)/180*pi};
    \end{axis}
\end{tikzpicture}

\item $k: x \mapsto \arc\ctg x$\quad 
$D_k=]-\infty;\infty[ \quad R_k:]0;\pi[$

\begin{tikzpicture}
    \begin{axis}[
        domain=-3:3,
        xmin=-3.1, xmax=3.1,
        ymin=-0.1, ymax=3.5,
        samples=400,
        axis lines=middle,
        ytick={1.57,3.14}, yticklabels={$\pi$/2,$\pi$}
    ]
        \addplot+[mark=none] {rad(90-atan(x))};
    \end{axis}
\end{tikzpicture}

\end{abc2}

\begin{enumerate}
\item Oldjuk meg valós számok halmazán
\begin{abc}
\item $\sin^6x+\cos^6x=1$;
\item $2\sin x+3\cos^2x\geq5$;
\item $(\sin x+\sqrt{3}\cos x)\cdot \sin 4x =2$.
\end{abc}
\item Határozzuk meg a következő függvények legnagyobb és legkisebb értékét:
\begin{abc}
\item $f(x)=\sin^6x+\cos^6x;\qquad x\in \mathbb{R}$
\item $h(x)=\frac{7}{2}\sqrt{(\sin x+\cos x)^2+2};\qquad x\in \mathbb{R}$

\end{abc}
\end{enumerate}

%\newpage
\subsection*{2012.03.07 -- Komplex számok}
\underline{Definíciók}: $i^2=-1,\quad \mathbb{C}=\{a+ib~|~ a,b \in \mathbb{R}\}$,

$(a+ib)+(c+id)=(a+c)+(b+d)$,

$(a+ib)(c+id)=(ac-bd)+i(ad+bc)$;

$|a+ib|=\sqrt{a^2+b^2}$;\quad $Re(a+ib)=e$,\quad $Im(a+ib)=b,$\quad$\overline{a+ib}=a-ib.$

Trigonometrikus alak: $z=a+ib$, $r=\sqrt{a^2+b^2}$, $z=r(\cos\varphi+i\sin\varphi)$

\definecolor{qqwuqq}{rgb}{0.,0.39215686274509803,0.}
\definecolor{uuuuuu}{rgb}{0.26666666666666666,0.26666666666666666,0.26666666666666666}
\begin{tikzpicture}[line cap=round,line join=round,>=triangle 45,x=1.0cm,y=1.0cm]
\draw[->,color=black] (-1,0.) -- (7,0.);
\foreach \x in {2.,4.,6.}
\draw[shift={(\x,0)},color=black] (0pt,2pt) -- (0pt,-2pt);
\draw[->,color=black] (0.,-1) -- (0.,4.1);
\foreach \y in {2.,4.}
\draw[shift={(0,\y)},color=black] (2pt,0pt) -- (-2pt,0pt);
%\clip(-3.924300151749257,-2.9589479655429174) rectangle (9.607446517450182,7.207888341866245);
\draw [shift={(0.,0.)},color=qqwuqq,fill=qqwuqq,fill opacity=0.1] (0,0) -- (0.:0.721052220383629) arc (0.:30.963756532073532:0.721052220383629) -- cycle;
\draw [->] (0.,0.) -- (5.,3.);
\draw (5.,3.)-- (5.,0.);
\draw (4.992712306994956,3.2901712777818637) node[anchor=north west] {$a+ib$};
\begin{scriptsize}
\draw [fill=uuuuuu] (0.,0.) circle (1.5pt);
\draw[color=uuuuuu] (0.9307847988338456,0.1896467301322608) node {$\varphi$};
\draw[color=black] (2.5170996836778294,1.7759616149762436) node {$r$};
\end{scriptsize}
\end{tikzpicture}


\begin{enumerate}
\item Igazoljuk a Moivre-tételt: $z_1=r_1(\cos\varphi_1+ i \sin\varphi_1),\quad z_2=r_2(\cos\varphi_2+\sin\varphi_2),$ akkor,

$z_1z_2=r_1r_2(\cos(\varphi_1+\varphi_2)+i \sin (\varphi_1+\varphi_2))$
\item Számítsuk ki:
\begin{abc}
\item $\dfrac{(i+1)^{2n}}{(i-1)^{2n-2}}$, $n>0$ egész;
\item $(1+i)^{52}$,
\item $\left(\dfrac{1+i\sqrt{3}}{1-i}\right)^{20}$.
\end{abc}
\item Legyen $\omega=-\frac{1}{2}+i\frac{\sqrt{3}}{2}$, $a,b$ tetszőleges valós számok. Számítsuk ki:
\begin{abc2}
\item $(a+b)(a+b\omega)(a+b\omega^2)$;
\item $(a\omega^2+b\omega)(b\omega^2+a\omega)$.
\end{abc2}
\end{enumerate}

%\newpage
\subsection*{2012.05.08 -- Valószínűségszámítás}
\underline{Definíciók}:

\underline{Esemény:} az elemi események halmazának részhalmaza, jelölése $A,B, ...\quad;$

$A+B$ az az esemény, amiben vagy $A$ vagy $B$ bekövetkezik;

$A\cdot B$ az az esemény, amiben $A$ is és $B$ is bekövetkezik;

$\overline{A}$ az az esemény, amiben $A$ nem következik be.

$I$ vagy $E$ a biztos esemény (mindig bekövetkezik)

$O$ a lehetetlen esemény, soha nem következik be;

\underline{$A$ valószínűsége:}
\begin{itemize}
\item $P(A)$ egy függvény, ami az eseményeken van értelmezve, 
\item $0\leq P(A) \leq 1$, 
\item ha $A\cdot B=0$, akkor $P(A+B)=P(A)+P(B)$, 
\item $A$ és $B$ függetlenek, ha $P(A\cdot B)=P(A)\cdot P(B)$. 
\end{itemize}
\underline{Binomiális elosztás:} $0\leq p \leq 1$,$\quad$ $q=1-p$,\qquad Várható értéke: $M(X)=np$

$$P(X=k)=\binom{n}{k}\cdot p^k \cdot q^{n-k}$$

\noindent
\underline{Hipergeometrikus eloszlás:} $N,M,n$  \qquad Várható értéke: $M(X)=n\frac{M}{N}$
$$P(x=k)=\frac{\binom{M}{k}\binom{N-M}{n-k}}{\binom{N}{n}}$$


\noindent
\underline{Geometrikus eloszlás:} $0\leq p\leq  1,\quad q=1-p$\qquad Várható értéke : $M(X)=\frac{1}{p}$
$$P(x=k)=q^{k-1}p$$ 

%\newpage
\subsection*{2012.03.14 -- Kombinatorika}

\underline{Definíciók}:

\begin{description}
\item[\underline{Permutációk}:] adott $n$ elemű összes lehetséges sorrendje egy-egy permutáció.

Az $n$ elem permutációinak száma: $n!$ (teljes indukció).

Ha az $n$ elem közül $k_1,k_2,...,k_r$ azonos $(k_1+k_2+...k_r=n)$ akkor az ilyen ismétléses permutációk száma: $$\frac{n!}{k_1!\cdot k_2!\cdots k_r}$$

\item[\underline{Variációk}:] adott $n$ elem közül $k$-t hányféleképpen lehet sorbarakni $(k\leq n)$ : $V_n^k=n(n-1)...(n-k+1)$.

Ismétléses variáció, ha ugyanaz az elem az elem akárhányszor szerepelhet, ezek száma: $n^k$,

\item[\underline{Kombinációk}:] $n$ elemből hányféleképpen lehet $k$-t kiválasztani (a sorrend nem számít): $$\binom{n}{k}\qquad (0\geq k\geq n).$$

Ismétléses kombinációk, ha ugyanazt az elemet többször is választhatjuk: $$\binom{n+k-1}{k-1}=\binom{n+k-1}{n}.$$

Egy $n$ elemű halmaz összes $$\binom{n}{0}+\binom{n}{1}+\binom{n}{2}+...+\binom{n}{n}=2^n.$$

\item[\underline{Logikai szita formula}:] $A,B,C$ véges halmazok $|A|$ az $A$ elemeinek száma, $$|A\cup B|=|A|+|B|-|A\cap B|;$$
$$|A\cup B\cup C|=|A|+|B|+|C|-|A\cap B|-|A\cap C|-|B\cap C|+|A\cap B\cap C|$$
Általánosítható $n$ halmazra is.

\item[\underline{A Pascal háromszög képzési szabálya}:]
$$\binom{n}{k}+\binom{n}{k+1}=\binom{n+1}{k+1}\quad\binom{n}{0}=\binom{n}{n}=1.$$

A Pascal háromszög $n$-edik sora váltakozó előjellel: $n>0,\quad n\in$ $N$
$$\binom{n}{0}-\binom{n}{1}+\binom{n}{2}-\binom{n}{3}+...+(-1)^n\binom{n}{n}=0;$$
További összegek :
$${\binom{n}{0}}^2+{\binom{n}{1}}^2+{\binom{n}{2}}^2+...+{\binom{n}{n}}^2=\binom{2n}{n};$$
$$\binom{2n}{0}-\binom{2n}{2}+\binom{2n}{4}-\binom{2n}{6}+...+(-1)^n\binom{2n}{2n}=2^n\cos n\frac{\pi}{2};$$
$$\binom{2n}{1}-\binom{2n}{3}+\binom{2n}{5}-...+(-1)^{n-1}\binom{2n}{2n-1}=2^n\sin n\frac{\pi}{2}.$$

\end{description}
\subsection*{2012.03.22 -- Számelméleti függvények}

\underline{Definíciók}:

$d$(n) az $n>0$ egész szám pozitív osztóinak száma;

$\delta (n)$ az $n>0$ egész szám pozitív osztóinak összege;

$\varphi(n)$ az $n>0$ egész, akkor az $1,2,...,n$ számok között az $n$-hez relatív prímek száma .

$n$ \underline{tökéletes szám}, ha $\delta (n)=2n$.

\noindent\underline{Tételek}:

ha $n=p_1^{\alpha_1}p_2^{\alpha_2}\ldots p_k^{\alpha_k}$ az $n>0$ prímhatványok szorzataként való előállítása, akkor
$$d(n)=(\alpha_1 +1)(\alpha_2 +1)\ldots(\alpha_k +1);$$
$$\delta(n)=\frac{p_1^{\alpha_1 +1}-1}{p_1 -1}\cdot\frac{p_2^{\alpha_2 +1}-1}{p_2 -1}\cdot\ldots\cdot\frac{p_k^
{\alpha_2 +1}-1}{p_k -1};$$
$$\varphi(n)=n\left(1-\frac{1}{p_1}\right)\left(1-\frac{1}{p_2}\right)\ldots\left(1-\frac{1}{p_k}\right);$$

$n$ akkor és csak akkor páros tökéletes szám, ha $n=2^{p-1}(2^p-1)$ alakú, ahol $p$ prím és $2^{p-1}$ is prím.

\begin{enumerate}
\item Melyik az az $n>0$ egész, szám, amely osztható $12$-vel és $14$ pozitív osztója van?
\item Igaz-e, hogy 
\begin{abc}
\item $7\mid \binom{1000}{500}$;
\item $7\mid \binom{10000}{5000}$?
\end{abc}
\item Határozzuk meg azokat a négyjegyű számokat, amelyeknek $15$ különböző pozitív osztója van. 
\end{enumerate}

%\newpage
\subsection*{2012.03.26 -- Sorozat határértéke, nevezetes sorozatok }
\underline{Definíció}: az $(a_n)$ sorozat konvergens és határértéke $A$, ha $A$ bármely környezetén kívül $a_n$-nek csak véges sok tagja van; azaz bármely $\varepsilon>0$ számhoz van olyan $n_0$ index, hogy ha $n>n_0$, akkor $|a_n-A|<\varepsilon$.

\noindent\underline{Jelölés}: $\lim\limits_{n\to\infty} a_n=A$, vagy rövidebben $a_n\to A$ ha $n\to\infty$.

\noindent\underline{Tételek}: 
Ha $a_n\to A$, $b_n\to B$, akkor $a_n\pm b_n\to A\pm B, a_n\cdot b_n\to A\cdot B$ és ha $B\not=0$, akkor $\frac{a_n}{b_n}\to\frac{A}{B}$.


Ha $(a_n)$ monoton növő $(a_n\leq a_{n+1})$ és felülről korlátos $(a_n\leq K)$, akkor $(a_n)$ konvergens.

Ha $a_n\leq c_n\geq b_n$ és $a_n\to A, b_n\to A$, akkor $c_n\to A$.

Az $\left(\left(1+\frac{1}{n}\right)^n\right)$ sorozat monoton nő és felülről korlátos, tehát konvergens, határértéke $e=2{,}71...$

Az $\left(\left(1+\frac{1}{n}\right)^{n+1}\right)$ sorozat monoton fogy, alulról korlátos, határértéke $e$.

A $(q^n)$ sorozat ha $|q|<1$ konvergens és határértéke $0$, (ha $q=1$, határértéke $1$).

Az $(1+q+...+q^{n-1})$ sorozat konvergens, ha $|q|<1$ és határértéke $\frac{1}{1-q}$.

\noindent További példák:

$\lim\limits_{n\to\infty} \left(\frac{1}{1\cdot 2}+\frac{1}{2\cdot 3}+...+\frac{1}{(n-1)\cdot n}\right)=1$;

$\lim\limits_{n\to\infty} \frac{a^n}{n!}=0$ ha $a>1$;

$\lim\limits_{n\to\infty} \frac{n^k}{a^n}=0$ ha $a>1$, $k$ adott szám.

\noindent\underline{Definíciók}:

ha $a_{n+1}-a_n=d$ állandó, akkor $(a_n)$ számtani sorozat;

ha $a_{n+1}=qa_n$, $q$ állandó, akkor $(a_n)$ mértani sorozat;

a Fibonacci sorozat:
$f_1=f_2=1,\quad f_{n+2}=f_n+f_{n+1}$.

\noindent\underline{Tételek}:

Ha $(a_n)$ számtani sorozat, akkor $a_n=a_1+(n-1)d$;

ha $(a_n)$ mértani sorozat, akkor $a_n=a_1\cdot q^{n-1}$;

Összegképletek:

számtani sorozat: $a_1+a_2+...+a_n=\frac{a_1+a_n}{2}n$;

mértani sorozat: $a_1+a_2+...+a_n=
\left\{
\begin{array}{lr} 
a_1\frac{q^n-1}{q-1} &\text{~ha~} q\ne 1\cr 
na_1 &\text{~ha~} q=1
\end{array}
\right.$. 

A Fibonacci-sorozat tulajdonságai:

$f_1+f_3+f_5+...+f_{2n-1}=f_{2n};$

$f_2+f_4+f_6+...+f_{2n}=f_{2n+1}-1;$

$f_1^2+f_2^2+...+f_n^2=f_n\cdot f_{n+1};$

$f_1\cdot f_2+f_2\cdot +...+f_{2n-1}\cdot f_{2n}=f_{2n}^2.$

%\newpage
\subsection*{2012.03.29 -- Kongruenciák, számrendszerek}
\underline{Definíció}: $a\equiv b \pmod{m},$ ha $m|a-b$.

\noindent\underline{Tételek}:

$a\equiv a\pmod{m}$ :reflexív

$a\equiv b\pmod{m} \longrightarrow b\equiv a\pmod{m}$ :szimmetrikus

$a\equiv b\pmod{m}$ és $b\equiv c\pmod{m}\longrightarrow a\equiv c\pmod{m};\qquad$ :tranzitív

$a\equiv b\pmod{m}\longrightarrow a+c=b+c\pmod{m};$

$a\equiv b$ és $c\equiv d\pmod{m} \longrightarrow a+c\equiv b+d\pmod{m};$

$a\equiv b\longrightarrow ac\equiv bd\pmod{m};$

$a\equiv b$ és $c\equiv d\pmod{m}\longrightarrow ac\equiv bd\pmod{m};$

$a\equiv b\pmod{m}\longrightarrow a^n\equiv b^n\pmod{m};$

\noindent{Kis-Fermat tétel}: ha $p$ prím, $(a,p)=1$, akkor 
$a^{p-1}\equiv 1\pmod{p}$;

\noindent{Euler tétel}: ha $(a,m)=1$, akkor 
$a^{\varphi(m)}\equiv 1\pmod{m}$;

\noindent{Wilson tétel}: ha $p$ prím, akkor 
$(p-1)!\equiv -1\pmod{p}$.

\noindent{A maradékos osztás tétele}: Tetszőleges $a$ és $b\not=0$ egész számokhoz egyértelműen léteznek olyan $q$ és $r$ pozitív egészek, hogy
$$a=qb+r\quad ;\quad 0\leq r<|b|.$$

\noindent\underline{Definíció}: Ha $a$ és $b$ adott egész számok, akkor a  $D$ egész szám az $a$ és $b$ legnagyobb közös osztója, ha $D\mid a$ és $D\mid b$, továbbá ha $d$ egészre $d\mid a$ és $d\mid b$ teljesül, akkor $d\mid D$.

Jelölése: $(a,b)=D$.

\noindent\underline{Euklédeszi algoritmus}: adott  $a,b$ egészek,

\noindent{ha} $b=0$, akkor $(a,b)=a$, ha $b\not=0$

$a=q_1b+r_1$ (ha $r_1=0$, akkor  $(a,b)=b),$

\noindent{ha}  $r_1\not=0$ $b=q_2r_1+r_2,$ ha $r_2\not=0$

$r_1=q_3r_2+r_3$, és így tovább

\ldots

\noindent{ha} az utolsó nem nulla maradék $r_n$, akkor

$r_{n-1}=q_{n+1}r_n$ és $(a,b)=r_n$.

\noindent{Az} $a>1$ alapú számrendszer: ha  tetszőleges egész szám,akkor 
$$n=b_0a^k+b_1a^{k-1}+\ldots +b_{k-1}a+b_k,$$

ahol $b_0,b_1,\ldots ,b_2$ egész számok és
$$0\leq b_i\leq a-1\qquad i=0,1,2, \ldots,k$$

az adott felírás egyértelmű.

\noindent{Igazoljuk}, hogy egy $n$ szám akkor és csak akkor osztható $a-1$-gyel, ha $b_0+b_1+\ldots+b_k$ osztható $a-1$-gyel.













\end{document}

