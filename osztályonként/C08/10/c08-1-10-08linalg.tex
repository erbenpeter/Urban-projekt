\documentclass{article}
\usepackage[utf8]{inputenc}
\usepackage{t1enc}
\usepackage{geometry}
 \geometry{
 a4paper,
 total={210mm,297mm},
 left=20mm,
 right=20mm,
 top=20mm,
 bottom=20mm,
 }
\usepackage{amsmath}
\usepackage{amssymb}
\frenchspacing
\usepackage{fancyhdr}
\pagestyle{fancy}
\lhead{Urbán János tanár úr feladatsorai}
\chead{C08/10/1.}
\rhead{Lineáris algebra}
\lfoot{}
\cfoot{\thepage}
\rfoot{}

\usepackage{enumitem}
\usepackage{multicol}
\usepackage{calc}
\newenvironment{abc}{\begin{enumerate}[label=\textit{\alph*})]}{\end{enumerate}}
\newenvironment{abc2}{\begin{enumerate}[label=\textit{\alph*})]\begin{multicols}{2}}{\end{multicols}\end{enumerate}}
\newenvironment{abc3}{\begin{enumerate}[label=\textit{\alph*})]\begin{multicols}{3}}{\end{multicols}\end{enumerate}}
\newenvironment{abc4}{\begin{enumerate}[label=\textit{\alph*})]\begin{multicols}{4}}{\end{multicols}\end{enumerate}}
\newenvironment{abcn}[1]{\begin{enumerate}[label=\textit{\alph*})]\begin{multicols}{#1}}{\end{multicols}\end{enumerate}}
\setlist[enumerate,1]{listparindent=\labelwidth+\labelsep}

\newcommand{\degre}{\ensuremath{^\circ}}
\newcommand{\tg}{\mathop{\mathrm{tg}}\nolimits}
\newcommand{\ctg}{\mathop{\mathrm{ctg}}\nolimits}
\newcommand{\arc}{\mathop{\mathrm{arc}}\nolimits}
\renewcommand{\arcsin}{\arc\sin}
\renewcommand{\arccos}{\arc\cos}
\newcommand{\arctg}{\arc\tg}
\newcommand{\arcctg}{\arc\ctg}

%függőleges vonalhoz
\makeatletter
\renewcommand*\env@matrix[1][*\c@MaxMatrixCols c]{%
  \hskip -\arraycolsep
  \let\@ifnextchar\new@ifnextchar
  \array{#1}}
\makeatother

\newcommand{\deta}[4]{\ensuremath{\begin{vmatrix} #1 & #2 \\ #3 & #4\end{vmatrix}}}
\newcommand{\detb}[9]{\ensuremath{\begin{vmatrix} #1 & #2 & #3\\ #4 & #5 & #6\\ #7 & #8 & #9\end{vmatrix}}}
\newcommand{\mata}[4]{\ensuremath{\begin{pmatrix} #1 & #2 \\ #3 & #4\end{pmatrix}}}
\newcommand{\matb}[9]{\ensuremath{\begin{pmatrix} #1 & #2 & #3\\ #4 & #5 & #6\\ #7 & #8 & #9\end{pmatrix}}}


\parskip 8pt
\begin{document}

\section*{Lineáris algebra}

\subsection*{2012.04.23.}
\begin{enumerate}
\item Oldjuk meg a következő lineáris egyenletrendszert:
\begin{align*}
a_{11}x_1+a_{12}x_2&=b_1,\cr
a_{21}x_1+a_{22}x_2&=b_2,
\end{align*}
ahol $a_{11},a_{12},a_{21},a_{22}$ adott számok és
$a_{11}a_{22}-a_{21}a_{12}\ne 0$.

\medskip
\noindent
\underline{Definíció:} Az $\deta{a_{11}}{a_{12}}{a_{21}}{a_{22}}$ szimbólumot $2\times 2$-es determinánsnak nevezzük és értéke $a_{11}a_{22}-a_{21}a_{12}$.\\ Az
$\detb{a_{11}}{a_{12}}{a_{13}}{a_{21}}{a_{22}}{a_{23}}{a_{31}}{a_{32}}{a_{33}}$ egy $3\times 3$-as determináns, értéke:
$$
a_{11}a_{22}+a_{33}+
a_{12}a_{23}+a_{31}+
a_{13}a_{21}+a_{32}-
a_{13}a_{22}+a_{31}-
a_{12}a_{21}+a_{33}-
a_{11}a_{23}+a_{32}.
$$
\item Számítsuk ki a következő determinánsok értékét:
\begin{abc4}
\item $\detb{1}{-3}{2}{4}{3}{-5}{2}{1}{0}$;
\item $\deta{\cos x}{-\sin x}{\sin x}{\cos x}$;
\item $\detb{0}{1}{0}{0}{0}{1}{1}{0}{0}$;
\item $\detb{\sin 2x}{-\cos 2x}{1}{\sin x}{-\cos x}{\cos x}{\cos x}{\sin x}{\sin x}$.
\end{abc4}

\item Oldjuk meg:
\begin{abc2}
\item $\detb{2}{-1}{3}{1}{x}{-1}{-5}{2}{7}=0$;
\item $\detb{x^2}{4}{8}{x}{2}{3}{1}{1}{1}=0$.
\end{abc2}
\end{enumerate}

\subsection*{2012.04.25.}
\begin{enumerate}
\item Igazoljuk, hogy ha egy determináns egy sorában vagy egy oszlopában minden elem 0, akkor a determináns értéke 0.
\item Ha egy determináns egy sorának vagy egy oszlopának minden elemét egy $\lambda$ számmal szorozzuk, akkor a determináns értéke is $\lambda$-val szorzódik.
\item Ha egy determináns egy sorának vagy egy oszlopának minden eleme egy kéttagú összeg, akkor a determináns két determináns összege, egyikben az illető sort az összegek egyik tagjai adják, a másikban a megfelelő sort a másik tagok adják.
\item Ha a determináns főátlója felett mindenütt 0 áll, akkor a determináns értéke a főátlóban álló elemek szorzata.
\item Ha egy determináns két sorát felcseréljük, akkor értéke $-1$-gyel szorzódik.
\item Számoljuk ki a következő determinánsokat:
\begin{abc3}
\item $\detb{1}{2}{2}{2}{2}{2}{2}{2}{3}$
\item $\detb{x}{y}{x+y}{y}{x+y}{x}{x+y}{x}{y}$
\item $\begin{vmatrix}
1&a&a&a\\
0&2&a&a\\
0&0&3&a\\
0&0&0&4
\end{vmatrix}
$;
\item $
\begin{vmatrix}
1&0&0&0\\
2&3&4&5\\
3&4&5&2\\
4&5&2&3
\end{vmatrix}
$;
\item $
\begin{vmatrix}
0&1&0&0\\
2&3&4&5\\
3&4&5&2\\
4&5&2&3
\end{vmatrix}
$.
\end{abc3}
\end{enumerate}

\subsection*{2012.04.26.}
\begin{enumerate}
\item Ha egy determináns két sora azonos, akkor a determináns értéke 0.
\item Ha egy determináns egyik sorához hozzáadjuk egy másik sor $\lambda$-szorosát, akkor a determináns értéke nem változik.
\item A determináns értéke nem változik, ha sorait és oszlopait felcseréljük.
\item Ha egy determinánsban az első sor első eleme nem 0, akkor értéke az első sor első eleme szorozva azzal a determinánssal, amit úgy kapunk, hogy az első sort és az első oszlopot elhagyjuk. A tétel általánosítható.
\item Számítsuk ki a következő determinánsokat:
\begin{abc4}
\item $\detb{1}{0}{0}{2}{2}{1}{3}{3}{2}$;
\item $\begin{vmatrix}
1&0&0&2 \\
3&0&0&4 \\
0&5&6&0 \\
0&7&8&0
\end{vmatrix}$;
\item  $\begin{vmatrix}
1&2&3&4 \\
2&1&2&3 \\
3&2&1&2 \\
4&3&2&1
\end{vmatrix}$;
\item  $\begin{vmatrix}
3&1&5&8 \\
4&-2&-1&7 \\
6&3&2&1 \\
7&4&4&5
\end{vmatrix}$.
\end{abc4}
\item Oldjuk meg a következő egyenleteket:
\begin{abc2}
\item $\begin{vmatrix}
1&1&4&4 \\
-1&3-x^2&3&3 \\
7&7&5&5 \\
-7&-7&6&x^2-3
\end{vmatrix}=0$;
\item $\begin{vmatrix}
1&2&3&4 \\
-2&2-x&1&7 \\
3&6&4+x&12 \\
-4&x-14&2&3
\end{vmatrix}=0$;
\end{abc2}
\end{enumerate}


\subsection*{2012.05.02.}
\begin{enumerate}
\item \underline{Definíció:} az $e$-edrendű determináns $a_{ik}$ eleméhez tartozó előjelezett aldetermináns az az ($n-1$)-edrendű determináns a $(-1)^{i+k}$ előjellel ellátva, amely az eredetiből úgy következik, hogy az $i$-edik sort és a $k$-adik oszlopot elhagyjuk. Jele: $A_{ik}$. Igazoljuk, hogy a determináns értékét megkapjuk, ha egy sorának elemeit megszorozzuk a hozzájuk tartozó előjelezett aldeterminánsokkal és a szorzatokat összeadjuk (kifejtési tétel):$$D=a_{i1}A_{i1}+a_{i2}A_{i2}+\ldots+a_{in}A_{in}.$$
\item Számítsuk ki a következő determinánsok értékét:
\begin{abc3}
\item $\begin{vmatrix}
1&2&3&4\\
2&3&4&1\\
3&4&1&2\\
4&1&2&3
\end{vmatrix}$;
\item $\begin{vmatrix}
5&6&0&0&0\\
1&5&6&0&0\\
0&1&5&6&0\\
0&0&1&5&6\\
0&0&0&1&5
\end{vmatrix}$;
\item $\begin{vmatrix}
1&2&2&\ldots&2\\
2&2&2&\ldots&2\\
2&2&3&\ldots&2\\
\vdots&&&&\vdots\\
2&2&2&\ldots&n
\end{vmatrix}$;
\item $\begin{vmatrix}
1&1&0&0\\[0.4em]
1&\binom21&\binom22&0\\[0.4em]
1&\binom31&\binom32&\binom33\\[0.4em]
1&\binom41&\binom42&\binom43
\end{vmatrix}$;
\item $\begin{vmatrix}
1&1&0&0&\ldots&0\\[0.4em]
1&\binom21&\binom22&0&\ldots&0\\[0.4em]
1&\binom31&\binom32&\binom33&\ldots&0\\[0.4em]
\vdots&&&\ddots&&\vdots\\[0.4em]
1&\binom n1&\binom n2&\binom n3&\ldots&\binom{n}{n-1}
\end{vmatrix}$;
\item $\begin{vmatrix}
1&a&a&\ldots&a&a\\
a&2&a&\ldots&a&a\\
a&a&3&\ldots&a&a\\
\vdots&&&&&\vdots\\
a&a&a&\ldots&n&a\\
a&a&a&\ldots&a&a
\end{vmatrix}$.
\end{abc3}
\end{enumerate}


\subsection*{2012.05.03.}
\begin{enumerate}
\item Igazoljuk az u.n. Cramer-szabályt: ha az
\begin{align*}
a_{11}x_1+a_{12}x_2+\ldots+a_{1n}x_n&=b_1\\
a_{21}x_1+a_{22}x_2+\ldots+a_{2n}x_n&=b_2\\
\ldots\\
a_{n1}x_1+a_{n2}x_2+\ldots+a_{nn}x_n&=b_n
\end{align*}
egyenletrendszer determinánsa $\ne 0$, azaz
$$D=\begin{vmatrix}
a_{11}&a_{12}&\ldots&a_{1n}\\
a_{21}&a_{22}&\ldots&a_{2n}\\
&\ldots&&\\
a_{n1}&a_{n2}&\ldots&a_{nn}\\
\end{vmatrix}\ne 0,$$
akkor
$$
x_1=\frac{D_1}{D},\quad
x_2=\frac{D_2}{D},\quad
\ldots,\quad
x_n=\frac{D_n}{D},
$$
ahol $D_i$ úgy keletkezik $D$-ből, hogy az $i$-dik oszlopát az egyenletrendszer jobb oldalán álló számokra cseréljük ki.
\item Oldjuk meg a következő egyenletrendszereket:
\begin{abc3}
\item \begin{align*}
x_1-3x_2-4x_3&=4,\\
2x_1+x_2-3x_3&=-1,\\
3x_1-2x_2+x_3&=11.
\end{align*}
\item
\begin{align*}
x+y+z&=1,\\
ax+by+cz&=d,\\
a^2x+b^2y+c^2z&=d^2.
\end{align*}
\item
\begin{align*}
x_1+2x_2-x_3+3x_4&=0,\\
3x_1-x_2+x_3-5x_4&=-12,\\
2x_1+2x_2+x_3-4x_4&=-13,\\
x_1-3x_2-6x_3+x_4&=1.
\end{align*}

\end{abc3}
\end{enumerate}


\subsection*{2012.05.10.}
\begin{enumerate}
\item Oldjuk meg a Cramer-szabály felhasználásával:
\begin{abc2}
\item \begin{align*}
\lambda x+y+z&=1,\\
x+\lambda y + z &= \lambda,\\
x+y+\lambda z&= \lambda^2;\\
(\lambda\ne 1, \lambda&\ne -2.)
\end{align*}
\item \begin{align*}
a^2x+ay+z&=a^3,\\
b^2x+by+z&=b^3,\\
c^2x+cy+z&=c^3.\\
(a\ne b, a\ne c&, b\ne c)
\end{align*}
\item \begin{align*}
(1+i)x-2iy&=-2,\\
(1-i)x+(2-i)y&=3-3i.
\end{align*}
\item \begin{align*}
2x-(9a^2-2)y&=3a,\\
x+y&=1.\\
(a&\ne 0)
\end{align*}
\end{abc2}
\end{enumerate}


\subsection*{2012.05.14.}
\underline{Definíció:} Egy $n$ sorból és $k$ oszlopból álló téglalap alakú számtáblázatot \underline{mátrixnak} nevezünk. \underline{Négyzetes mátrixot} kapunk, ha $n=k$.
\begin{enumerate}
\item Oldjuk meg a következő egyenletrendszert a mátrix jelölés felhasználásával:
\begin{align*}
x_1+3x_2+3x_3&=0,\\
2x_1-x_2+3x_3&=0,\\
3x_1-5x_2+4x_3&=0,\\
x_1+17x_2+4x_3&=0.
\end{align*}
\item A következő egyenletrendszert is oldjuk meg:
\begin{align*}
x_1+x_2+5x_3&=-7,\\
2x_1+x_2+x_3&=2,\\
x_1+3x_2+x_3&=5,\\
2x_1+3x_2-3x_3&=14.
\end{align*}

\item Oldjuk meg:
 $$\begin{pmatrix}[cccc|c]
 1&2&3&4&11\\
 2&3&4&1&12\\
 3&4&1&2&13\\
 4&1&2&3&14
 \end{pmatrix}$$
\end{enumerate}


\subsection*{2012.05.16.}
\underline{Definíciók:} Azonos típusú mátrixokat össze lehet adni, az összeadást komponensenként végezzük. Egy $n\times k$-as és egy $k\times l$-es mátrixot sor-oszlop szorzással lehet összeszorozni, az eredmény egy $n\times l$-es mátrix lesz. Például:
$$
\begin{pmatrix}
1&2&3\\
2&1&1
\end{pmatrix}\cdot
\begin{pmatrix}
2&0\\
1&2\\
1&1
\end{pmatrix}=
\begin{pmatrix}
1\cdot 2+2\cdot 1+3\cdot 1& 1\cdot 0+2\cdot 2+3\cdot 1\\
2\cdot 2+1\cdot 1+1\cdot 1& 2\cdot 0+1\cdot 2+1\cdot 1
\end{pmatrix}=
\begin{pmatrix}
7&7\\
6&3
\end{pmatrix}
$$
\begin{enumerate}
\item Végezzük el a következő szorzásokat:
\begin{abc2}
\item $\matb{2}{1}{1}{5}{2}{3}{6}{5}{2}
\matb{1}{3}{2}{-3}{-4}{-5}{2}{1}{3}$;
\item $
\matb{3}{4}{9}{2}{-1}{6}{5}{3}{5}
\matb{5}{6}{4}{8}{9}{7}{-4}{-5}{-3}
$;
\item $
\begin{pmatrix}
1&-3&3&-1\\
1&3&-5&1
\end{pmatrix}
\begin{pmatrix}
1&1\\
1&2\\
1&1\\
1&-2
\end{pmatrix}
$;
\item $
\mata{5}{-4}{6}{-5}^n
$, $n>0$ egész;
\item $
\matb{2}{1}{0}{0}{1}{0}{0}{0}{2}^n
$, $n>0$ egész;
\item $
\mata{\cos \varphi}{-\sin \varphi}{\sin \varphi}{\cos \varphi}^n
$, $n>0$ egész.
\end{abc2}
\end{enumerate}


\subsection*{2012.05.17.}
\begin{enumerate}
\item Végezzük el a következő műveleteket:
\begin{abc2}
\item $
\begin{pmatrix}
1&1&1\\
4&1&0
\end{pmatrix}\cdot
\begin{pmatrix}
-1&2\\
4&5\\
6&-1
\end{pmatrix}
$;
\item $
\matb{2}{-1}{3}{1}{3}{-1}{4}{5}{1}\cdot
\begin{pmatrix}
-8\\
5\\
7
\end{pmatrix}
$.
\end{abc2}

\item Számítsuk ki a következő mátrixok inverzét:
\begin{abc2}
\item $\matb{2}{-3}{-4}{-1}{7}{2}{3}{1}{-6}$;
\item $\matb{1}{2}{3}{1}{3}{4}{1}{4}{3}$.
\end{abc2}
\item Legyen $A=\mata{3}{2}{4}{3}$ és $B=\mata{-1}{7}{3}{5}$. Oldjuk meg a következő egyenleteket: $A\cdot X = B, \qquad Y\cdot A=B$.

\item Igazoljuk, hogy az $\matb{1}{2}{-2}{1}{0}{3}{1}{3}{0}$ mátrix gyöke az 
$f(x)=x^3-x^2-9x+9$ polinomnak.
\end{enumerate}


\subsection*{2012.05.21.}
\begin{enumerate}
\item Igazoljuk, hogy a következő mátrixok csoportot alkotnak a szorzás műveletére:
\begin{abc2}
\item $\left\{\mata{a}{-b}{b}{a}~|~a,b\in\mathbb{R},\quad a^2+b^2>0\right\}$;
\item $\left\{\mata{a}{3b}{b}{a}~|~a,b\in\mathbb{R},\quad a^2+b^2>0\right\}$.
\end{abc2}
\item \underline{Definíció:} egy mátrix rangja $r$, ha a mátrix elemeiből készíthető minden $r+1$-edrendű aldetermináns 0, de van olyan $r$-edrendű aldetermináns, ami nem 0. Számítsuk ki a következő mátrix rangját:
$$\begin{pmatrix}
6&9&7&9\\
4&6&-2&5\\
4&6&18&8
\end{pmatrix}$$
\item Igazoljuk, hogy a mátrix rangja nem változik, ha

(1) csupa 0-ból álló sorát vagy oszlopát elhagyjuk;

(2) valamelyik sorát vagy oszlopát egy $\lambda\ne 0$ számmal szorozzuk;

(3) egyik sorának (vagy oszlopának) többszörösét egy másik sorához (ill. oszlopához) adjuk.
\item Számítsuk ki a következő mátrix rangját:
$$
\begin{pmatrix}
4&9&0&7&2\\
-1&1&6&0&3\\
0&-1&2&1&-2\\
4&-3&-1&9&6
\end{pmatrix}
$$
\end{enumerate}


\subsection*{2012.05.23.}
\begin{enumerate}
\item Igazolható, hogy egy lineáris egyenletrendszernek akkor és csak akkor van megoldása, ha az egyenletrendszer mátrixának és a kibővített mátrixának rangja megegyezik.
\item Döntsük el, hogy a következő egyenletrendszereknek van-e megoldása:
\begin{abc2}
\item \begin{align*}
x_1-3x_2+4x_3+2x_4&=1,\\
2x_1+4x_2-3x_3+3x_4&=-1,\\
3x_1+x_2+2x_3-x_4&=0,\\
12x_1+4x_2+7x_3+2x_4&=0.
\end{align*}
\item \begin{align*}
x_1+x_2+x_3-3x_4&=1,\\
x_1+x_2-3x_3+x_4&=-1,\\
x_1-3x_2+x_3+x_4&=0,\\
-3x_1+x_2+x_3+x_4&=0.
\end{align*}
\end{abc2}
\item Számítsuk ki a következő mátrix rangját:
$$
\begin{pmatrix}
-1&-3&-2&1&-1\\
4&1&2&4&1\\
-6&9&-10&-2&6\\
4&6&1&12&-3
\end{pmatrix}
$$
\item Határozzuk meg a következő mátrix inverzét:
$$
\begin{pmatrix}
1&1&1&1\\
1&1&-1&-1\\
1&-1&1&-1\\
1&-1&-1&1
\end{pmatrix}
$$
\end{enumerate}


\subsection*{2012.05.31. -- Determinánsok, mátrixok}
\begin{enumerate}
\item Számítsuk ki a következő determinánsok értékét:
\begin{abc2}
\item $
\begin{vmatrix}
1&3&3&1\\
1&1&3&3\\
3&1&1&3\\
3&3&1&1
\end{vmatrix}
$;
\item $
\begin{vmatrix}
1&2&3&\ldots&n\\
1&a&3&\ldots&n\\
1&2&a&\ldots&n\\
&&\vdots&\ddots&\\
1&2&3&\ldots&a
\end{vmatrix}
$
\end{abc2}

\item Számítsuk ki a következő mátrix inverzét:

$$
\begin{pmatrix}
1&1&0&0\\
-1&0&1&1\\
0&0&1&1\\
1&1&1&0
\end{pmatrix}
$$

\item Oldjuk meg a Cramer-szabály felhasználásával:

\begin{align*}
x_1-x_2+2x_3&=11,\\
x_1+2x_2-x_3&=11,\\
4x_1-3x_2-3x_3&=24.
\end{align*}

\item Oldjuk meg a következő egyenletrendszert:

\begin{align*}
-2x_1-x_2-3x_3+4x_4&=5,\\
2x_1+x_2+2x_3-3x_4&=-3,\\
4x_1+2x_2+x_3-3x_4&=0,\\
2x_1+x_2-x_4&=1,\\
2x_1+x_2-x_3&=3.
\end{align*}
\end{enumerate}

\end{document}
