\documentclass{article}
\usepackage[utf8]{inputenc}
\usepackage{t1enc}
\usepackage{geometry}
 \geometry{
 a4paper,
 total={210mm,297mm},
 left=20mm,
 right=20mm,
 top=20mm,
 bottom=20mm,
 }
\usepackage{amsmath}
\usepackage{amssymb}
\frenchspacing
\usepackage{fancyhdr}
\pagestyle{fancy}
\lhead{Urbán János tanár úr feladatsorai}
\chead{C08/09/1.}
\rhead{Binomiális együtthatók}
\lfoot{}
\cfoot{\thepage}
\rfoot{}

\usepackage{enumitem}
\usepackage{multicol}
\usepackage{calc}
\newenvironment{abc}{\begin{enumerate}[label=\textit{\alph*})]}{\end{enumerate}}
\newenvironment{abc2}{\begin{enumerate}[label=\textit{\alph*})]\begin{multicols}{2}}{\end{multicols}\end{enumerate}}
\newenvironment{abc3}{\begin{enumerate}[label=\textit{\alph*})]\begin{multicols}{3}}{\end{multicols}\end{enumerate}}
\newenvironment{abc4}{\begin{enumerate}[label=\textit{\alph*})]\begin{multicols}{4}}{\end{multicols}\end{enumerate}}
\newenvironment{abcn}[1]{\begin{enumerate}[label=\textit{\alph*})]\begin{multicols}{#1}}{\end{multicols}\end{enumerate}}
\setlist[enumerate,1]{listparindent=\labelwidth+\labelsep}

\newcommand{\degre}{\ensuremath{^\circ}}
\newcommand{\tg}{\mathop{\mathrm{tg}}\nolimits}
\newcommand{\ctg}{\mathop{\mathrm{ctg}}\nolimits}
\newcommand{\arc}{\mathop{\mathrm{arc}}\nolimits}
\renewcommand{\arcsin}{\arc\sin}
\renewcommand{\arccos}{\arc\cos}
\newcommand{\arctg}{\arc\tg}
\newcommand{\arcctg}{\arc\ctg}

\parskip 8pt
\begin{document}

\section*{Binomiális együtthatók}

\subsection*{2010. 10. 19.-- Ismétlő feladatok (A Pascal háromszög tulajdonságai):}
\begin{enumerate}
\item Igazoljuk, hogy a Pascal háromszög $n$-edik sorban található számok összege $2^{n}$;
\\ $\binom{n}{0}+\binom{n}{1}+\binom{n}{2}+...+\binom{n}{n}=2^{n}$;
\item Számítsuk ki a következő összegeket:
	\begin{abc}
    	\item(*) $\binom{2n}{0}-\binom{2n}{2}+\binom{2n}{4}-\binom{2n}{6}+...+(-1)^{n}\binom{2n}{2n}$;
        \item $\binom{n}{0}-\binom{n}{1}+\binom{n}{2}-\binom{n}{3}+...+(-1)^{n}\binom{n}{n}$.
    \end{abc}
\item Igazoljuk teljes indukcióval: $(a+b)^{n}a^{n}=\binom{n}{1}a^{n-1}b+\binom{n}{2}a^{n-2}b^{2}+...+\binom{n}{n}b^{n}$.
\item Igazoljuk, hogy  $\binom{n}{0}^{2}+\binom{n}{1}^{2}+\binom{n}{2}^{2}+...+\binom{n}{n}^{2}=\binom{2n}{n}$.
\item Igazoljuk, hogy ha $0<k<n$, egészek, akkor $\binom{1}{k}+\binom{n-1}{k}+\binom{n-2}{k}+...+\binom{k}{k}=\binom{n+1}{k+1}$.
\item (*) Milyen $n$ és $k$ mellett lesz $\binom{n}{k-1}, \binom{n}{k}, \binom{n}{k+1}$ $(0<k<n)$ egy számtani sorozat három szomszédos tagja?
\end{enumerate}


\subsection*{2010. 10. 20.}
\begin{enumerate}
\item Tegyük fel, hogy egy konvex $n$-szög összes átlóját meghúztuk, és a sokszög belsejében egyetlen ponton sem halad át kettőnél több átló. Hány metszéspontot kapunk a sokszög belsejében?
\item Az előző feladatban szereplő átlók hány részre vágták szét a sokszöget?
\item Az első feladatbeli átlók hány olyan háromszöget határoznak meg, amelynek minden csúcsa a sokszög belsejében van?
\item Igazoljuk, hogy egy konvex poliédernek  mindig van két azonos oldalszámú lapja.
\end{enumerate}


\subsection*{2010. 10. 21.}
\begin{enumerate}
\item A binomiális tételt így is írhatjuk:
$(1+x)^{n}=\binom{n}{0}+\binom{n}{1}x+\binom{n}{2}x^{2}+...+\binom{n}{n}x^{n}$; \\ felhasználva, hogy $(1+x)^{n}\cdot(1+x)=(1+x)^{n+1}$, vezessük le az $\binom{n+1}{k+1}=\binom{n}{k}+\binom{n}{k+1}$ azonosságot.
\item Az előző feladathoz hasonlóan vezessük le az $\binom{n}{0}^{2}+\binom{n}{1}^{2}+...+\binom{n}{n}^{2}=\binom{2n}{n}$ azonosságot.
\item Számítsuk ki:
	\begin{abc}
    	\item $\binom{n}{1}+2\binom{n}{2}+3\binom{n}{3}+...+n\binom{n}{n}=$
        \item $\binom{n}{0}+2\binom{n}{1}+3\binom{n}{2}+...+(n+1)\binom{n}{n}=$
        \item $\binom{n}{2}+2\binom{n}{3}+3\binom{n}{4}+...+(n+1)\binom{n}{n}=$
        \item (*) $\binom{n}{0}^{2}-\binom{n}{1}^{2}+\binom{n}{2}^{2}-...+(-1)^{n}\binom{n}{n}^{2}=$
    \end{abc}
\end{enumerate}


\subsection*{2010. 10. 26.}
\begin{enumerate}
\item Határozzuk meg $(\sqrt{26}-5)^{100}$ tizedestört alakjában a tizedesvessző utáni első $100$ jegyet.
\item Az előző feladatot oldjuk meg $(\sqrt{26}+5)^{100}$ esetre is.
\item Igazoljuk, hogy ha $p$ prímszám és $p-1\ge{k}\ge{1}$, akkor $p$ osztója $\binom{p}{k}$-nak.
\item Igazoljuk, hogy ha $n>0$ egész szám, akkor $k^{n}\ge{\binom{2n}{n}}$.
\item Igazoljuk, hogy $\binom{2n}{n}\ge{\binom{2n}{k}}$ ha $2n\ge{k}\ge{0}$.
\item A $\binom{2n+1}{0}, \binom{2n+1}{1}, \binom{2n+1}{2},... ,\binom{2n+1}{n+1}$ számok közül melyik a legnagyobb?
\end{enumerate}


\subsection*{2010. 10. 28.}
\begin{enumerate}
\item Hány $0$-ra végződik $((3!)!)!$?
\item Hányféleképpen ülhet le egy kör alakú asztalhoz hét ember? (Két ülésrend nem különböző, ha mindenkinek ugyanaz a szomszédja.)
\item Hányféleképpen lehet $n$ embert körbeállítani, ha két körbeállás azonos, amennyiben mindenkinek ugyanaz a bal (illetve jobb) szomszédja?
\item Hányféleképpen lehet $n$ embert körökbe állítani? (Egy ember is lehet kör!)
\item \underline{Definíció:} Ha adott egy $S$ halmaz és egy $f: S\rightarrow{s}$ egy egyértelmű leképzés, más szóval permutáció, akkor $s{\in}S$ esetén az $s, f(s), f(f(s)),\dots$ sorozat periodikus. Ezt a sorozatot az $s$ \underline{ciklusának} nevezzük. Az $S$ halmaz két ciklusa az f szerint  vagy azonos, vagy nincsen közös elemük. Az {{1, 2, \dots, 6}} halmaznak hány olyan permutációja van, amelynek két ciklusa van?
\end{enumerate}


\subsection*{2010. 11. 09.}
\begin{enumerate}
\item Számítsuk ki az $(x+y+z)^{3}$ kifejezést, azaz írjuk fel összeg alakjában.
\item (*) Keressünk általános képletet az $(x+y+z)^{n}$, n>0, egész kifejezés összeg alakjában történő felírásához.
\item Írjuk fel az $(x+y+z)^{5}$-t összeg alakjában.
\item Igazoljuk, hogy $\binom{n-1}{k-1}=\frac{k}{n}\binom{n}{k} $, ahol $n\ge{k}\ge{0}$ egész számok.
\item Igazoljuk a következő azonosságot: $n(1+x)^{n-1} = \binom{n}{1}+2\binom{n}{2}x+\dots+k\binom{n}{k}x^{k-1}+\dots+n\binom{n}{n}x^{n-1}$.
\item Igazoljuk, hogy ${\binom{n}{1}}^{2}+2{\binom{n}{2}}^{2}+3{\binom{n}{3}}^{2}+\dots+n{\binom{n}{n}}^{2}=\dfrac{(2n-1)!}{((n-1)!)^{2}}$.
\item Mennyi a következő összeg értéke: $\binom{n}{0}+3\binom{n}{1}+5\binom{n}{2}+\dots+(2n+1)\binom{n}{n}$?
\end{enumerate}


\subsection*{2010. 11. 10.}
\begin{enumerate}
\item Az $(1+x^{2}-x^{3})^{9}$ összeg alakjában határozzuk meg $x^{8}$ együtthatóját.
\item Számítsuk ki:
	\begin{abc}
		\item $\binom{n}{1}+6\binom{n}{2}+6\binom{n}{3}$;
        \item $\binom{n}{0}+7\binom{n}{1}+12\binom{n}{2}+6\binom{n}{3}$.
	\end{abc}
\item Számísuk ki a következő összegeket:
	\begin{abc}
    	\item $\binom{n}{1}+2\binom{n}{2}+3\binom{n}{3}+\dots+n\binom{n}{n}$;
        \item $\binom{n}{1}-2\binom{n}{2}+3\binom{n}{3}-\dots+(-1)^{n-1}n\binom{n}{n}$;
        \item $\frac{\binom{n}{0}}{1}+\frac{\binom{n}{1}}{2}+\frac{\binom{n}{2}}{3}+\dots+\frac{\binom{n}{n}}{n+1}$   
    \end{abc}
\item Határozzuk meg $(a+b+c)^{10}$ összeg alakjában a legnagyobb együtthatót.
\item Hány különböző tagja lesz az $(x_{1}+x_{2}+\dots+x_{n})^{3}$ kifejezés összeg alakjának?
\end{enumerate}


\subsection*{2010. 11. 11. -- Ismétlő feladatok}
\begin{enumerate}
\item Számítsuk ki az $x^{17}$ illetve $x^{18}$ együtthatóját az $(1+x^{5}+x^{7})^{20}$ összeg alakjában.
\item Mennyi lesz $x^{17}$ együtthatója az $(1-x^{2}+x^{3})^{1000}$ összeg alakjában?
\item Számítsuk ki:
	\begin{abc}
    	\item  $\binom{n}{0}-2\binom{n}{1}+3\binom{n}{2}-\dots+(-1)^{n}(n+1)\binom{n}{n}^{2}$;
        \item  $\binom{n}{0}^{2}-\binom{n}{1}^{2}+\binom{n}{2}^{2}-\dots+(-1)^{n}\binom{n}{n}^{2}$
        \item  $3\binom{n}{1}+7\binom{n}{2}+11\binom{n}{3}+\dots+(4n-1)\binom{n}{n}$
    \end{abc}
\item A dominójátékban $(0;0)$-tól $(8;8)$-ig minden dominó egyszer szerepel. Hány dominó van egy szettben?
\item Hány olyan tízjegyű szám van, amelynek minden jegye különböző?
\end{enumerate}


\subsection*{2010. 11. 10.-- Kombinatorika}
\begin{enumerate}
\item Az $1000$-et hányféleképpen bonthatjuk fel három pozitív egész tényező szorzatára? (A sorrend is számít.)
\item Hányféleképpen oszthatunk el $3n$ különböző könyvet 3 ember között úgy, hogy mindegyiknek pontosan $n$ könyv jusson?
\item Számítsuk ki:
	\begin{abc}
		\item $\binom{n}{0}-2\binom{n}{1}+3\binom{n}{2}-\dots+(-1)^{n}(n+1)\binom{n}{n}=$
   		\item $\frac{\binom{n}{0}}{2}+\frac{\binom{n}{1}}{3}+\frac{\binom{n}{2}}{4}+\dots+\frac{\binom{n}{n}}{n+2}=$
    	\item $\binom{n}{1}+14\binom{n}{2}+36\binom{n}{3}+24\binom{n}{4}=$
	\end{abc}
\item Határozzuk meg az $(1+x+x^{2}+x^{3})^{10}$ összeg alakjában az $x^{10}$ együtthatóját.
\end{enumerate}


\end{document}