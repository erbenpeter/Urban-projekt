\documentclass{article}
\usepackage[utf8]{inputenc}
\usepackage{t1enc}
\usepackage{geometry}
 \geometry{
 a4paper,
 total={210mm,297mm},
 left=20mm,
 right=20mm,
 top=20mm,
 bottom=20mm,
 }
\usepackage{amsmath}
\usepackage{amssymb}
\frenchspacing
\usepackage{fancyhdr}
\pagestyle{fancy}
\lhead{Urbán János tanár úr feladatsorai}
\chead{C08/09/1.}
\rhead{Sorozatok}
\lfoot{}
\cfoot{\thepage}
\rfoot{}

\usepackage{enumitem}
\usepackage{multicol}
\usepackage{calc}
\newenvironment{abc}{\begin{enumerate}[label=\textit{\alph*})]}{\end{enumerate}}
\newenvironment{abc2}{\begin{enumerate}[label=\textit{\alph*})]\begin{multicols}{2}}{\end{multicols}\end{enumerate}}
\newenvironment{abc3}{\begin{enumerate}[label=\textit{\alph*})]\begin{multicols}{3}}{\end{multicols}\end{enumerate}}
\newenvironment{abc4}{\begin{enumerate}[label=\textit{\alph*})]\begin{multicols}{4}}{\end{multicols}\end{enumerate}}
\newenvironment{abcn}[1]{\begin{enumerate}[label=\textit{\alph*})]\begin{multicols}{#1}}{\end{multicols}\end{enumerate}}
\setlist[enumerate,1]{listparindent=\labelwidth+\labelsep}

\newcommand{\degre}{\ensuremath{^\circ}}
\newcommand{\tg}{\mathop{\mathrm{tg}}\nolimits}
\newcommand{\ctg}{\mathop{\mathrm{ctg}}\nolimits}
\newcommand{\arc}{\mathop{\mathrm{arc}}\nolimits}
\renewcommand{\arcsin}{\arc\sin}
\renewcommand{\arccos}{\arc\cos}
\newcommand{\arctg}{\arc\tg}
\newcommand{\arcctg}{\arc\ctg}

\parskip 8pt
\begin{document}

\section*{Sorozatok}

\subsection*{2010.09.02. -- Bemelegítő feladatok}
\begin{enumerate}
\item Oldjuk meg az egész számok halmazán: 
$\sqrt{n+\sqrt{n+\ldots+\sqrt{n+\sqrt{n}}}}=k$.
\item ($*$) Oldjuk meg a valós számok halmazán:
$2x^3=(3x^2-x-1)\sqrt{1+x}$.
\item ($*$) Határozzuk meg az
$f(x)=\sqrt{x^2-6x+13}+\sqrt{x^2-14x+58}, x\in\mathbb{R}$ függvény legkisebb értékét.
\item Igazoljuk, hogy
\begin{abc2}
\item $\sqrt{6+\sqrt{6+\sqrt{6+\sqrt{6+\sqrt{3}}}}}<3$;
\item $\frac{2-\sqrt{2+\sqrt{2+\sqrt{2+\sqrt{2}}}}}{2-\sqrt{2+\sqrt{2+\sqrt{2}}}}>\frac{1}{4}$.
\end{abc2}
\item ($*$) Tudjuk, hogy $x,y,z>0$ és $x+y+z=1$. Határozzuk meg 
$\sqrt{1+x^2}+\sqrt{1+y^2}+\sqrt{1+z^2}$ legkisebb értékét.
\end{enumerate}

\subsection*{2010.09.07. -- Sorozatok és tulajdonságaik}
\begin{enumerate}
\item Számítsuk ki a kétjegyű páros számok összegét.
\item Adott a következő sorozat: $a_n=2^n, (n=0,1,2,\ldots)$. Számítsuk ki a sorozat első $n$ tagjának összegét!
\item Az $a_n=\left(\frac{1}{2}\right)^n$ sorozat ($n=0,1,2,\ldots$) első $n$ tagjának számítsuk ki az összegét!
\item Egy sorozat első tagja 10, minden további tag 3-mal nagyobb az előző tagnál. Számítsuk ki a sorozat
\begin{abc2}
\item első 100;
\item első $n$
\end{abc2}
tagjának összegét.
\item Egy sorozat első tagja 3, minden további tag az előző tag kétszerese. Számítsuk ki a sorozat
\begin{abc2}
\item első 100;
\item első $n$
\end{abc2}
tagjának összegét.
\item Egy sorozat első tagja $a_1$, minden további tag $d$-vel nagyobb az előző tagnál. Számítsuk ki az első $n$ tag összegét és a sorozat $n$-edik tagját.
\end{enumerate}

\subsection*{2010.09.08.}
\begin{enumerate}
\item Egy sorozat első tagja $a_1$, minden további tag az előző tag $q$-szorosa. Számítsuk ki a sorozat $n$-edik tagját és első $n$ tagjának összegét (mértani sorozat).
\item Az $a_n$ sorozat definíciója:
$$a_1=1,\quad a_{n+1}=\frac{1}{1+a_n}.$$
Vizsgáljuk meg, hogyan viselkedik a sorozat a növekedés és fogyás szempontjából.
\item Keressünk olyan mértani sorozatot, amelyre teljesül, hogy $a_{n+2}=a_n+a_{n+1}$.
\item Az $1,4,10,19,\ldots$ sorozat szomszédos tagjainak különbségéből álló sorozat egy számtani sorozat. Adjuk meg a sorozat $n$-edik tagját és első $n$ tagjának összegét.
\item Számítsuk ki a következő összeget:
$$\frac{1}{2}+\frac{3}{2^2}+\frac{5}{2^3}+\ldots
+\frac{2n-1}{2^n}.$$
\item Számítsuk ki:
$$1+2x+3x^2+\ldots+(n+1)x^n.$$
\end{enumerate}


\subsection*{2010.09.09.}
\begin{enumerate}
\item Az $(a_n)$ sorozat definíciója: $a_1=1, a_2=2$ és ha $n\ge 1$, akkor $a_{n+2}=3a_{n+1}-2a_n$. Fejezzük ki $a_n$-et $n$-nel.
\item Igazoljuk, hogy az $a_1=2$, $a_{n+1}=\frac{1}{2}\left(a_n+\frac{1}{a_n}\right)$ sorozat monoton fogyó és alulról korlátos. Szemléltessük a sorozat elemeit az 
$$f(x)=\frac{1}{2}\left(x+\frac{1}{x}\right)$$
függvény grafikonjával.
\item Egy számtani sorozatról a következőket tudjuk:
\begin{align*}
a_2+a_4+a_6&=36;\cr
a_2a_3&=54.
\end{align*}
Számítsuk ki a sorozat első tagját és különbségét.
\item Egy számtani sorozat három egymást követő elemének összege 3, a három egymást követő elem köbének összege 15. Számítsuk ki a számtani sorozat differenciáját.
\item Számítsuk ki minél egyszerűbben a következő összeget:
$$100^2-99^2+98^2-97^2+\ldots+2^2-1^2.$$
\end{enumerate}


\subsection*{2010.09.15.}
\begin{enumerate}
\item Egy mértani sorozatnál tudjuk, hogy 
$$a_1+a_2=4\text{~és~}a_5a_2=6.$$
Számítsuk ki a sorozat első elemét és hányadosát.
\item Egy számtani sorozat első három elemének összege 9. Ha az első elemből 4-et elveszünk, a másodikhoz 1-et, a harmadikhoz 15-öt hozzáadunk, akkor egy mértani sorozat első három elemét kapjuk. Melyik a számtani sorozat első három eleme?
\item Számítsuk ki azoknak a kétjegyű pozitív egész számoknak az összegét, amelyek 3-mal osztva 2-t adnak maradékul.
\item Igazoljuk, az $x^2-yz$, $y^2-zx$, $z^2-xy$ számok egy számtani sorozat első három elemét adják, ha $x,y,z$ egy számtani sorozat első három eleme.
\item Egy mértani sorozatban $a_1-a_2=8$, $a_2+a_3=12$. Számítsuk ki a sorozat első tagját és hányadosát.
\item Egy mértani sorozatban $a_1=3$, $a_{10}=1536$. Számítsuk ki $q$-t és $s_{10}$-et.
\item Igazoljuk, hogy ha $x,y,z$ egy mértani sorozat első három eleme, akkor 
$x^3+2xy^2+y^2z$,
$x^2y+y^3+xyz+yz^2$ és 
$xy^2+2y^2z+z^3$ is egy mértani három egymást követő eleme.
\end{enumerate}


\subsection*{2010.09.16.}
\begin{enumerate}
\item Egy szabályos háromszög oldala 2 egység. A háromszögbe egyenlő sugarú köröket írunk úgy, hogy érintik egymást és a háromszög oldalait. Egy-egy háromszögoldalt $n$ kör érint. Számítsuk ki a körök sugarát és területének összegét.
\item Számítsuk ki ki:
\begin{abc}
\item $\frac{1^2+3^2+5^2+\ldots+(2n-1)^2}{2^2+4^2+6^2+\ldots+(2n)^2}$;
\item $\frac{1^3+^3+3^3+\ldots+n^3}{n^3}-\frac{n}{4}$;
\item $(1+x)(1+x^2)(1+x^4)\ldots(1+x^{2^n})$.
\end{abc}
\item Vizsgáljuk a következő sorozatot:
$a_1=\frac{x}{2}$, ahol $0<x<1$; $a_{n+1}=\frac{x}{2}+\frac{a_n^2}{2}$. Igazoljuk, hogy $a_n<1$ minden $n$-re és $a_{n+1}>a_n$ szintén minden $n$-re igaz.

\item Igazoljuk, hogy az $a_1=\sqrt 2$, $a_{n+1}=\sqrt{2+a_n}$ sorozat növekvő, de felülről korlátos, azaz van olyan szám, aminél a sorozat minden eleme kisebb.
\end{enumerate}


\subsection*{2010.09.21.}
\begin{enumerate}
\item Az $1,4,10,19,\ldots$ sorozat szomszédos tagjainak különbsége számtani sorozatot alkot. Számítsuk ki a sorozat 100. tagját és az első 100 tag összegét.
\item Számítsuk ki az
$$1,3,6,10,\ldots,\frac{n(n+1)}{2},\ldots$$
sorozat első 200 tagjának összegét.
\item Adott két számtani sorozat: az egyik
$17,21,25,\ldots$, a másik $16,21,26,\ldots$. A két sorozatban vannak közös elemek is. Számítsuk ki a két sorozat első 100 közös elemének összegét.
\item ($*$) Egy számtani sorozatról tudjuk, hogy a negyedik tagja 4. A sorozat $d$ különbségének mely értékére igaz, hogy a sorozat első három tagjának páronként vett szorzatát összeadva a legkisebb értéket kapjuk?
\item Számítsuk ki a következő összeg értékét:
$$\frac{1}{1\cdot 3}+\frac{1}{3\cdot 5}
+\frac{1}{5\cdot 7}+\ldots+\frac{1}{999\cdot 1001}.$$
\end{enumerate}


\subsection*{2010.09.23.}
\begin{enumerate}
\item Valaki január 1-én 100~000~Ft-ot tesz egy bankba évi 6\%-os kamatra. Kamatos kamattal 6 év alatt hány forintra nő a betét?
\item Évi 5\%-os kamat esetén 6 év alatt 80~000~Ft hány forintra növekszik?
\item Igazoljuk a számtani sorozat következő tulajdonságát:
$$a_n=\frac{a_{n-k}+a_{n+k}}{2},$$
ha $n>k>0$ egészek.
\item Igazoljuk a mértani sorozat következő tulajdonságát:
$$\sqrt{a_{n-k}\cdot a_{n+k}}=a_n,$$
ha $a_1>0$, $q>0$ és $n>k>0$ egészek.
\item Igazoljuk a Fibonacci-sorozat következő tulajdonságait:
\begin{abc}
\item $f_1^2+f_2^2+\ldots+f_n^2=f_n\cdot f_{n+1}$;
\item $f_n^2-f_{n-1}\cdot f_{n+1}=(-1)^{n-1}$;
\item $f_{n-1}^2+f_n^2=f_{2n-1}$.
\end{abc}
\end{enumerate}


\subsection*{2010.09.28.}
\begin{enumerate}
\item Egy vállalatnál azt tervezik, hogy 5 éven keresztül évente 10\%-kal növelik a termelést. Hány \%-kal kell évente növelni a termelést, ha már a 4. év végére el akarják érni a tervezett végeredményt?
\item Négy szám egy számtani sorozat négy szomszédos eleme. Ha sorra 5-tel, 6-tal, 9-cel, 15-tel növeljük ezeket a számokat, akkor egy mértani sorozat négy szomszédos elemét kapjuk. Melyik ez a négy szám?
\item Három szám egy mértani sorozat három szomszédos eleme. Ha a középső számot 10-zel növeljük, akkor egy számtani sorozat három szomszédos elemét kapjuk. Melyik ez a három szám?
\item Négy szám egy mértani sorozat négy szomszédos eleme. A sorozat két szélső tagjának összege 14, a két középső tag összege 6. Melyik ez a négy szám?
\item Egy mértani sorozatban $a_6+a_5=48$ és $a_7-a_5=48$. Az elejétől kezdve hány tagot kell összeadni, hogy 1023-at kapjunk?
\item Egy mértani sorozat második eleme 20, első három elemének összege 70. Határozzuk meg a sorozat hatodik elemét!
\end{enumerate}


\subsection*{2010.09.29.}
\begin{enumerate}
\item Három szám egy mértani sorozat három szomszédos eleme, összegük 26. Ha az elsőhöz 1-et, a másodikhoz 6-ot, a harmadikhoz 3-mat adunk, akkor egy számtani sorozat három szomszédos tagját kapjuk. Mik lesznek a számtani sorozat tagjai?
\item Három szám egy mértani sorozat három szomszédos eleme. Ha a harmadik számból 64-et kivonunk, akkor a kapott számok egy számtani sorozat szomszédos tagjai. Ha most a középső számból vonunk ki 8-cat, akkor ismét egy mértani sorozat három szomszédos tagját kapjuk. Melyik volt az eredeti három szám?
\item Adjuk meg azt a számtani sorozatot, amelyben
$s_n=3n^2$.
\item Melyik az a mértani sorozat, amelyben az első és a negyedik tag összege 35, a második és harmadik tag összege 30?
\item Egy mértani sorozatban
\begin{align*}
a_1+a_2+a_3+a_4+a_5&=31,\cr
a_2+a_3+a_4+a_5+a_6&=62.
\end{align*}
Mennyi $a_1$ és $q$ értéke?
\item Igazoljuk, hogy ha $a$, $x$ tetszőleges valós számok, akkor
$(a+x)^2$, $a^2+x^2$, $(a-x)^2$ egy számtani sorozat első három eleme. Határozzuk meg ennek a sorozatnak az első $n$ tagjának összegét.
\end{enumerate}


\subsection*{2010.09.30.}
\begin{enumerate}
\item Oldjuk meg a következő egyenleteket:
\begin{abc}
\item $1+4+7+\ldots+x=117$;
\item $(x+1)+(x+4)+(x+7)+\ldots+(x+28)=155$.
\end{abc}
A bal oldalon számtani sorozatok összege áll.
\item Egy számtani sorozat első $n$ elemének összege
\begin{abc2}
\item $3n^2$;
\item $5n^2+3n$.
\end{abc2}
Határozzuk meg a sorozat első három elemét.
\item Igazoljuk, hogy ha $a,b,c$ egy számtani sorozat első három eleme, akkor
$$3(a^2+b^2+c^2)=6(a-b)^2+(a+b+c)^2.$$
\item Készítsünk olyan képletet, amelynek segítségével kiszámíthatjuk egy mértani sorozat első $n$ elemének szorzatát.
\item Három szám egy mértani sorozat első három eleme. A három szám összege 26, négyzetük összege 364. Melyek ezek a számok?
\end{enumerate}


\subsection*{2010.10.05. -- Ismétlő feladatok}
\begin{enumerate}
\item Igazoljuk, hogy az $a_1=\sqrt{a}$, $a_{n+1}=\sqrt{a+a_n}\quad(a>0)$ sorozat növekszik és felülről korlátos.
\item Három egész szám, amelyek összege 60, egy számtani sorozat három egymást követő tagja.
Ha ezekhez a számokhoz sorra $2{,}2$-et, $4$-et és $7$-et adunk, akkor a kapott számok egy mértani sorozat szomszédos tagjai. Melyik ez a három szám?

\item Számítsuk ki azoknak a háromjegyű pozitív egész számoknak az összegét, amelyeknek minden számjegye páros.
\item Egy számtani sorozat első tagja $a$, különbsége $d$. A $d$ mely értékére igaz, hogy $a_1a_3+a_2a_4$ értéke a lehető legkisebb?
\item Számítsuk ki a 30-cal osztható négyjegyű számok összegét.
\end{enumerate}


\subsection*{2010.10.06.}
\begin{enumerate}
\item Az $1,2,4,7,11,\ldots$ sorozat szomszédos tagjainak különbségei számtani sorozatot alkotnak. Számítsuk ki a sorozat $n$-edik tagját és első $n$ tagjának összegét.
\item Egy mértani sorozat első két tagjának összege 3, az első két tag négyzetének összege 5. Számítsuk ki a sorozat első tagját és hányadosát.
\item Az $a_n$ sorozat definíciója: $a_1=2$ és 
$a_{n+1}=3a_n+1$. Számítsuk ki a sorozat első $n$ tagjának összegét.
\item Számítsuk ki a következő összegeket:
\begin{abc}
\item $\frac{1}{1\cdot 5}+\frac{1}{5\cdot 9}
+\frac{1}{9\cdot 13}+\ldots+\frac{1}{(4n-3)\cdot(4n+1)}$;
\item $1+2\cdot 2+3\cdot 2^2+4\cdot 2^3+\ldots+
n\cdot 2^{n-1}$.
\end{abc}
\item Végezzük el a négyzetre emeléseket és adjuk össze:
$$
\left(x+\frac{1}{x}\right)^2+
\left(x^2+\frac{1}{x^2}\right)^2+
\ldots
+\left(x^n+\frac{1}{x^n}\right)^2.
$$
\end{enumerate}


\subsection*{2010.10.12.}
\begin{enumerate}
\item Egy számtani sorozat minden tagja pozitív egész szám. A második tag 12 és az első kilenc tag összege 200-nál nagyobb, de 220-nál kisebb. Adjuk meg a sorozatot.
\item Egy mértani sorozat első három tagjának összege 31, az első és a harmadik tag összege 26. Számítsuk ki a sorozat első tagját és hányadosát.
\item Három szám egy mértani sorozat három egymást követő tagja, szorzatuk 64, számtani közepük $\frac{14}{3}$. Melyik ez a három szám?
\item Adott 4 szám, az első három egy mértani, az utolsó három egy számtani sorozat első három eleme. Az első és a negyedik szám összege 21, a két középső összege 18. Melyik ez a négy szám?
\item Egy mértani sorozat első három tagjának összege 13, az első három tag négyzetének összege 91. Határozzuk meg a sorozatot.
\end{enumerate}


\subsection*{2010.10.13.}
\begin{enumerate}
\item Adott az $a_1=\frac{1}{4}$, $a_{n+1}=a_n(1-a_n)$, $n\ge 1$ sorozat. Igazoljuk, hogy $a_n\ge a_{n+1}$ és $a_n>0$ minden $n$-re. Ábrázoljuk a sorozat tagjait az $f(x)=x(1-x)$ függvény segítségével.
\item Egy mértani sorozat első három tagjának összege 6, az első, a harmadik és az ötödik tag összege $10{,}5$. Határozzuk meg a sorozat első tagját és hányadosát.
\item Határozzuk meg azt a háromjegyű számot, amelynek számjegyei egy mértani sorozat szomszédos tagjai, a nála 400-zal kisebb szám számjegyei pedig egy számtani sorozat szomszédos tagjai.
\item Az $a,b,12$ számok egy mértani sorozat, az $a,b,9$ számok pedig egy számtani sorozat szomszédos elemei. $a=?$, $b=?$
\item Négy szám egy mértani sorozat első 4 tagja.
A két középső összege 48, a két szélső szám összege 112. Melyik ez a négy szám?
\end{enumerate}


\subsection*{2010.10.14. -- Pótdolgozat}
\begin{enumerate}
\item egy mértani sorozat első három tagjának összege 21, az első és a harmadik tag összege 15. Számítsuk ki a sorozat első tagját és hányadosát.
\item Melyik az a négy szám, amelyek közül az első három egy mértani sorozat, a második három pedig egy számtani sorozat első három tagja és az első és negyedik összege 14, a második és harmadik összege 12.
\item Egy mértani sorozat harmadik tagjának és második tagjának összege 6, a negyedik tag 24-gyel nagyobb a második tagnál. Adjuk meg a sorozat első tagját és hányadosát.
\item Az $5x-y$, $2x+3y$, $x+2y$ egy számtani sorozat, az $(y+1)^2$, $xy+1$, $(x-1)^2$ pedig egy mértani sorozat három szomszédos tagja. Számítsuk ki $x$ és $y$ értékét!
\item Adott az $a_1=\frac{1}{4}$, $a_{n+1}=2a_n(1-a_n)\quad (n\ge 1)$ sorozat. Igazoljuk, hogy minden $n$-re $a_n\le a_{n+1}$ és $a_n<\frac{1}{2}$.
\end{enumerate}

\end{document}
