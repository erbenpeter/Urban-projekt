\documentclass{article}
\usepackage[utf8]{inputenc}
\usepackage{t1enc}
\usepackage{geometry}
 \geometry{
 a4paper,
 total={210mm,297mm},
 left=20mm,
 right=20mm,
 top=20mm,
 bottom=20mm,
 }
\usepackage{amsmath}
\usepackage{amssymb}
\frenchspacing
\usepackage{fancyhdr}
\pagestyle{fancy}
\lhead{Urbán János tanár úr feladatsorai}
\chead{C08/09/6.}
\rhead{Összegek}
\lfoot{}
\cfoot{\thepage}
\rfoot{}

\usepackage{enumitem}
\usepackage{multicol}
\usepackage{calc}
\newenvironment{abc}{\begin{enumerate}[label=\textit{\alph*})]}{\end{enumerate}}
\newenvironment{abc2}{\begin{enumerate}[label=\textit{\alph*})]\begin{multicols}{2}}{\end{multicols}\end{enumerate}}
\newenvironment{abc3}{\begin{enumerate}[label=\textit{\alph*})]\begin{multicols}{3}}{\end{multicols}\end{enumerate}}
\newenvironment{abc4}{\begin{enumerate}[label=\textit{\alph*})]\begin{multicols}{4}}{\end{multicols}\end{enumerate}}
\newenvironment{abcn}[1]{\begin{enumerate}[label=\textit{\alph*})]\begin{multicols}{#1}}{\end{multicols}\end{enumerate}}
\setlist[enumerate,1]{listparindent=\labelwidth+\labelsep}

\newcommand{\degre}{\ensuremath{^\circ}}
\newcommand{\tg}{\mathop{\mathrm{tg}}\nolimits}
\newcommand{\ctg}{\mathop{\mathrm{ctg}}\nolimits}
\newcommand{\arc}{\mathop{\mathrm{arc}}\nolimits}
\renewcommand{\arcsin}{\arc\sin}
\renewcommand{\arccos}{\arc\cos}
\newcommand{\arctg}{\arc\tg}
\newcommand{\arcctg}{\arc\ctg}

\parskip 8pt
\begin{document}

\section*{Összegek}

\subsection*{2011.05.18.}
\begin{enumerate}
\item Hány olyan tízes számrendbeli négyjegyű szám van, amelyben legfeljebb két számjegy szerepel?
\item Hány olyan háromszög van, amelynek oldalhosszai $n$-nél nagyobb, de $2n$-nél nem nagyobb egész számok.
\item Hány különböző szigorúan növő függvény van, amely az $\{ 1, 2, \ldots, k\}$ halmazból képez az $\{ 1, 2, \ldots, n\}$ halmazba $(1\le k\le n)$?
\item Számítsuk ki a következő összeget! $1\cdot2\cdot3+2\cdot3\cdot4+3\cdot4\cdot5+\ldots +n(n+1)(n+2)$.
\item Számítsuk ki:
\begin{enumerate}
\item $1^3+2^3+3^3+\ldots+n^3$;
\item $1^4+2^4+3^4+\ldots+n^4$.
\end{enumerate}
\end{enumerate}

\subsection*{2011.05.19.}
\begin{enumerate}
\item Igazoljuk, hogy egy négyzet feldarabolható $n$ (nem feltétlenül egybevágó) négyzetre, ha $n>5$.
\item Igazoljuk, hogy egy szabályos háromszög feldarabolható n (nem feltétlenül egybevágó) szabályos háromszögre, ha $n>5$.
\item Minimálisan hány tetraéderre darabolható fel egy kocka?
\item Számítsuk ki a következő összegeket:
\begin{enumerate}
\item $1+11+111+\ldots+11\ldots1$;
\item $1^3+3^3+5^3+\ldots+(2n-1)^3$;
\item $-1^3+3^3-5^3+7^3-\ldots-(4n-3)^3+(4n-1)^3$;
\item $1^2+3^2+5^2+\ldots+(2n-1)^2$.
\end{enumerate}
\end{enumerate}

\subsection*{2011.05.24.}
\begin{enumerate}
\item Jelölje $f_n$ a Fibonacci sorozat $n$-edik tagját (az $n$-edik Fibonacci számot): $f_0=0$, $f_1=1$, $f_2=1$, $f_{n+2}=f_n+f_{n+1}$. Igazoljuk:
\begin{enumerate}
\item $n, k > 0$ egészekre $f_{n+k}=f_{n-1}f_k+f_nf_{k+1}$;
\item az $f_n$ sorozat bármely két szomszédos tagja relatív prím.
\end{enumerate}
\item Igazoljuk: ha $n > 0$, egész, akkor
\begin{enumerate}
\item $f_{n+1}^2-f_n\cdot f_{n+2}=(-1)^2$;
\item $f_1f_2+f_2f_3+f_3f_4+\ldots+f_{2n-1}f_{2n}=f_{2n}^2$;
\item $f_1f_2+f_2f_3+f_3f_4+\ldots+f_{2n}\cdot f_{2n+1}=f_{2n+1}^2-1$;
\item $f_3+f_6+\ldots+f_{3n}=\frac{f_{3n+2}-1}{2}$.
\end{enumerate}
\item Igazoljuk, hogy tetszőleges pozitív egész szám előállítható Fibonacci számok összegeként úgy, hogy az összegben minden Fibonacci szám legfeljebb egyszer fordul elő és nincs benne két szomszédos Fibonacci szám.
\end{enumerate}

\subsection*{2011.05.26.}
\begin{enumerate}
\item Igazoljuk a Fibonacci-számok következő tulajdonságait:
\begin{enumerate}
\item $f_{2n-1}={f_{n-1}}^2+{f_n}^2$;
\item $f_1+f_2+\ldots+f_n=f_{n+2}-1$;
\item ${f_1}^2+{f_2}^2+\ldots+{f_n}^2=f_nf_{n+1}$;
\end{enumerate}
\item Adjuk meg a következő, rekurzióval definiált sorozatok $n$-edik tagját $n$ függvényében.
\begin{enumerate}
\item $a_{n+2}=7a_{n+1}-12a_n$;
\item $a_{n+2}=a_{n+1}+a_n$;
\item $a_{n+2}=-3a_{n+1}+10a_n$;
\item $a_{n+3}=9a_{n+2}-26a_{n+1}+24a_n$.
\end{enumerate}
\item Adjuk meg a következő sorozatok n-edik tagját,
\begin{enumerate}
\item $a_{n+2}=5a_{n+1}-6a_n$, $a_1=1$, $a_2=-7$;
\item $a_{n+2}=4a_{n+1}-4a_n$, $a_1=2$, $a_2=4$.
\end{enumerate}
\end{enumerate}

\subsection*{2011.05.26.}
\begin{enumerate}
\item Hány olyan hosszúságú, csak 0-t és 1-et tartalmazó sorozat van, amelyben nincs szomszédos 1-es?
\item Igazoljuk a következő azonosságot:
$\binom {n+m}k=\binom n0\binom mk+\binom n1\binom m{k-1}+\ldots+\binom nk\binom m{k-k}+\ldots+\binom nn\binom m{k-n}$
$n, m, k > 0$ egészek és $m\ge k\ge n$.
\item Határozzuk meg az $a_n$ sorozatot, ha $a_{n+2}=2^n-2a_{n+1}+8a_n$.
\item* $\sum\limits_{i_n=1}^m \sum\limits_{i_{n-1}=1}^{i_n} \ldots \sum\limits_{i_1=1}^{i_2} \sum\limits_{i_0=1}^{i_1}$
\item Határozzuk meg az $a_n$ sorozatot, ha $a_1=1$, $a_2=-3$, $a_3=-29$ és $a_{n+3}=9a_{n+2}-26a_{n+1}+24a_n$.
\end{enumerate}

\subsection*{2011.06.01.}
\begin{enumerate}
\item Igazoljuk a Fibonacci sorozat következő tulajdonságait: $f_{3n}=f_{n+1}^3+f_n^3-f_{n-1}^3$.
\item Adjuk meg az $a_n$ sorozat $n$-edik tagját n függvényként, ha $a_{n+2}-2a_{n+1}+a_n=0$ és $a_1=1$, $a_2=3$.
\item Egy 500 forintos bankjegyet hányféleképpen lehet felváltani 10, 20 és 50 forintosokra?
\item Adjuk meg a következő rekurzív összefüggés általános megoldását: $a_{n+3}+3a_{n+2}+3a_{n+1}+a_n=0$.
\item Az 1 és 10 000 000 közti egészek között melyik fajta számból van több, amelyikben van 1 számjegy, vagy amelyikben nincs?
\end{enumerate}

\subsection*{2011.06.02.}
\underline{Kombinatorika}
\begin{enumerate}
\item Hányféleképpen válthatunk fel egy 2000 Ft-ost 100, 200 és 500 Ft-osra?
\item Igazoljuk, hogy $\binom n1+24\binom n2+36\binom n3+24\binom n4=n^4$.
\item Igazoljuk a Fibonacci-számok következő tulajdonságát: ha $n>1$, egész, akkor
\\$f_1-f_2+f_3-f_4+\ldots+f_{2n-1}=f_{2n-2}+1$
\item Számítsuk ki:
\begin{enumerate}
\item $\binom n2+2\binom n3+3\binom n4+\ldots+(n-1)\binom nn$;
\item $\binom n1-2\binom n3+3\binom n4-\ldots+(-1)^{n-1}n\binom nn$;
\end{enumerate}
 \item Adjuk meg a sorozat n-edik tagját n függvényében, ha $a_1=7$, $a_2=25$ és $a_{n+2}=7a_{n+1}-12a_n$.
\item Igazoljuk teljes indukcióval:
$\binom n1\frac{1}{1}-\binom n2\frac{1}{2}+\binom n3\frac{1}{3}-\ldots+(-1)^{n-1}\binom 1n\frac{1}{n}=1+\frac{1}{2}+\frac{1}{3}+\ldots+\frac{1}{n}$
\end{enumerate}

\end{document}
