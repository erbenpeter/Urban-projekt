\documentclass{article}
\usepackage[utf8]{inputenc}
\usepackage{t1enc}
\usepackage{geometry}
 \geometry{
 a4paper,
 total={210mm,297mm},
 left=20mm,
 right=20mm,
 top=20mm,
 bottom=20mm,
 }
\usepackage{amsmath}
\usepackage{amssymb}
\usepackage{pstricks-add}
\frenchspacing
\usepackage{fancyhdr}
\pagestyle{fancy}
\lhead{Urbán János tanár úr feladatsorai}
\chead{C08/8./1. csoport}
\rhead{Függvények}
\lfoot{}
\cfoot{\thepage}
\rfoot{}

\usepackage{enumitem}
\usepackage{multicol}
\usepackage{calc}
\newenvironment{abc}{\begin{enumerate}[label=\textit{\alph*})]}{\end{enumerate}}
\newenvironment{abc2}{\begin{enumerate}[label=\textit{\alph*})]\begin{multicols}{2}}{\end{multicols}\end{enumerate}}
\newenvironment{abc3}{\begin{enumerate}[label=\textit{\alph*})]\begin{multicols}{3}}{\end{multicols}\end{enumerate}}
\newenvironment{abc4}{\begin{enumerate}[label=\textit{\alph*})]\begin{multicols}{4}}{\end{multicols}\end{enumerate}}
\newenvironment{abcn}[1]{\begin{enumerate}[label=\textit{\alph*})]\begin{multicols}{#1}}{\end{multicols}\end{enumerate}}
\setlist[enumerate,1]{listparindent=\labelwidth+\labelsep}

\newcommand{\degre}{\ensuremath{^\circ}}
\newcommand{\tg}{\mathop{\mathrm{tg}}\nolimits}
\newcommand{\ctg}{\mathop{\mathrm{ctg}}\nolimits}
\newcommand{\arc}{\mathop{\mathrm{arc}}\nolimits}
\renewcommand{\arcsin}{\arc\sin}
\renewcommand{\arccos}{\arc\cos}
\newcommand{\arctg}{\arc\tg}
\newcommand{\arcctg}{\arc\ctg}
 \newcommand{\sgn}{\operatorname{sgn}}

\parskip 8pt
\begin{document}

\section*{Függvények}

\subsection*{2009.09.03.}
\begin{enumerate}
\item Ábrázoljuk az $x\mapsto x^2$ függvényt, ahol $x$ tetszőleges valós szám.
\item Ábrázoljuk az $f(x)=(x-1)^2+1$ függvényt, ahol $x$ tetszőleges valós szám.
\item Ábrázoljuk a következő függvényeket:
\begin{abc3}
\item $x\mapsto (x+2)^2$;
\item $x\mapsto x^2-2x+1$;
\item $x\mapsto x^2+2x$;
\item $x\mapsto x^2-x$;
\item $x\mapsto -x^2$;
\item $x\mapsto x-x^2$.
\end{abc3}
\item Határozzuk meg a következő függvények legnagyobb és legkisebb értékét:
\begin{abc3}
\item $x\mapsto x-x^2$, $0\le x \le 1$;
\item $x\mapsto 3x(1-x)$, $0\le x \le 1$;
\item $x\mapsto x^2-4x+5$, $0\le x \le 4$.
\end{abc3}
\end{enumerate}

\subsection*{2009.09.08.}
\begin{enumerate}
\item Ábrázoljuk a következő függvényeket:
\begin{abc3}
\item $x\mapsto 2x^2$;
\item $x\mapsto \frac 12x^2$;
\item $x\mapsto x^2-4x+2$;
\item $x\mapsto 6x-x^2-8$;
\item $x\mapsto 5x-x^2-2$.
\end{abc3}

\item Határozzuk meg a következő függvények legnagyobb és legkisebb értékét:
\begin{abc2}
\item $x\mapsto x^2-4x$, $0\le x \le 4$;
\item $x\mapsto 3x-x^2$, $0\le x \le 3$;
\item $x\mapsto 3x-3x^2$, $0\le x \le 1$.
\item $x\mapsto 2x-x^2$, $0\le x \le 1$.
\end{abc2}
\item Igazoljuk a következő egyenlőtlenségeket:
\begin{abc2}
\item $x^2-6x+10\ge 1$;
\item $5-x^2-4x\le 9$.
\end{abc2}
\end{enumerate}

\subsection*{2009.09.09./1.}
\begin{enumerate}
\item Ábrázoljuk a következő függvényeket:
\begin{abc4}
\item $x\mapsto |x^2-2x|$;
\item $x\mapsto x^2-2|x|$;
\item $x\mapsto x^2-4|x|+4$;
\item $x\mapsto |x^2-2x|+3$.
\end{abc4}
\item Oldjuk meg függvénygrafikonokkal a következő egyenleteket:
\begin{abc4}
\item $x^2=2-|x|$;
\item $x^2-2x=3$;
\item $|x|=2-x^2$;
\item $(x-1)^2=x+1$.
\end{abc4}

\item Oldjuk meg a következő egyenlőtlenségeket:
\begin{abc2}
\item $x^2-2x\ge x$;
\item $x^2<2-|x|$.
\end{abc2}
\end{enumerate}

\subsection*{2009.09.09./2.}
\begin{enumerate}
\item  Ábrázoljuk a következő függvényeket:
\begin{abc3}
\item $x\mapsto |x^2+x-6|$;
\item $x\mapsto |3x|-x^2$;
\item $x\mapsto x|x|$;
\item $x\mapsto |x^2-3|x|+2|$;
\item $x\mapsto x|x-1|$.
\end{abc3}
\item  Ábrázoljuk:
\begin{abc3}
\item $x\mapsto -2x^2$;
\item $x\mapsto -3x|x|$;
\item $x\mapsto -\frac 14(x+2)^2$.
\end{abc3}
\item Oldjuk meg grafikusan:
\begin{abc3}
\item $x^2-4x+3=0$;
\item $-x^2+6x=9$;
\item $x^2-4x+3<0$.
\end{abc3}

\end{enumerate}

\subsection*{2009.09.14.}
\begin{enumerate}
\item Határozzuk meg a következő függvények legnagyobb és legkisebb értékét:
\begin{abc2}
\item $x\mapsto x^2-2x+3$, $0\le x \le 4$;
\item $x\mapsto 4-x^2+2x$, $0\le x \le 3$;
\item $x\mapsto |3x^2-6x|$, $-1\le x \le 4$;
\item $x\mapsto |x^2-2|x|+3|$, $-2\le x \le 3$.
\end{abc2}
\item Oldjuk meg grafikonok segítségével:
\begin{abc4}
\item $||x|-2|>x^2$;
\item $|x-3|=x^2$;
\item $||x|+3|=x^2$;
\item $3|x|-2=x^2$.
\end{abc4}
\end{enumerate}

\subsection*{2009.09.15.}
\begin{enumerate}
\item Ábrázoljuk a következő függvényeket:
\begin{abc3}
\item $x\mapsto x^2-4x+4$;
\item $x\mapsto x^2-2x+3$;
\item $x\mapsto x^2-4|x|+3$;
\item $x\mapsto |x^2-2|x||$;
\item $x\mapsto x^2-5|x|+6$.
\end{abc3}
\item Oldjuk meg a következő egyenleteket, egyenlőtlenségeket grafikonokkal:
\begin{abc3}
\item $x^2-6x+8=0$;
\item $x^2+9x-10<0$;
\item $2x^2-5x+3>0$;
\item $-x^2+2x+5=0$;
\item $x^2<-2x$.
\end{abc3}

\item Oldjuk meg függvénygrafikonokkal:
\begin{abc3}
\item $|x^2-2x| < x$;
\item $|x^2-4x| < 3$;
\item $|x^2-3x| < 2-x$.
\end{abc3}

\end{enumerate}

\subsection*{2009.09.16.}
\begin{enumerate}
\item Oldjuk meg a következő egyenleteket függvénygrafikonokkal:
\begin{abc4}
\item $x^2-10x+24=0$;
\item $2x^2-7x+5=0$;
\item $6+3x-9x^2=0$;
\item $x^2=3x-2$;
\end{abc4}
\item Ábrázoljuk a következő függvényeket:
\begin{abc3}
\item $x\mapsto x[x]$, $-3\le x \le 4$;
\item $x\mapsto [x^2]$, $-4\le x \le 3$;
\item $x\mapsto [x]^2$, $-3\le x \le 4$.
\end{abc3}
\item Oldjuk meg a következő egyenlőtlenségeket függvénygrafikonok
segítségével:
\begin{abc4}
\item $x^2-2x<3$;
\item $|x^2-4x|<5$;
\item $|x^2-5x|<6$;
\item $|x^2-1|+|x^2-9|<8$.
\end{abc4}
\end{enumerate}

\subsection*{2009.09.17. -- Dolgozat}
\begin{enumerate}
\item Oldjuk meg függvénygrafikonokkal:
\begin{abc3}
\item $x^2-4x+3=0$;
\item $x^2-7x+12<0$;
\item $-3x^2+5x-2\ge 0$.
\end{abc3}
\item Ábrázoljuk a következő függvényeket:
\begin{abc2}
\item $x\mapsto |x^2+2x|$, $-3\le x \le 4$;
\item $x\mapsto x^2-2|x|+1$, $-2\le x \le 3$;
\item $x\mapsto x[x]$, $-2\le x \le 3$.
\end{abc2}
\item Oldjuk meg függvénygrafikonokkal:
\begin{abc2}
\item $|x^2-5x|<6$;
\item $|x^2-1|+|x^2-4|\le 3$;
\end{abc2}
\end{enumerate}

\subsection*{2009.09.23./1.}
\underline{Definíció}: ha $a\ge 0$, akkor $\sqrt a$ jelöli azt a 
\underline{\underline{nemnegatív}} számot, amelynek négyzete $a$.

Pl: $\sqrt 9 = 3$, $\sqrt{0,01}=0,1$, $\sqrt 0 = 0$.
\begin{enumerate}
\item Ábrázoljuk a következő függvényeket:
\begin{abc3}
\item $x\mapsto \sqrt x$, $x\ge 0$;
\item $x\mapsto \sqrt{x-2}$, $x\ge 2$;
\item $x\mapsto \sqrt{x+1}$, $x\ge -1$;
\item $x\mapsto \sqrt{x^2}$;
\item $x\mapsto \sqrt{x^2-2x+1}$;
\item $x\mapsto \sqrt{x^4}$.
\end{abc3}
\item Ábrázoljuk és jellemezzük a következő függvényt:
$$x\mapsto \sqrt{x-[x]}.$$
\end{enumerate}

\subsection*{2009.09.23./2.}
\begin{enumerate}
\item Függvénygrafikonok segítségével oldjuk meg a következő egyenleteket:
\begin{abc3}
\item $\sqrt x = x-2$;
\item $\sqrt{x-3}=x-9$;
\item $\sqrt{4-x}=3-\sqrt{5+x}$;
\item $\sqrt{3-x}=1+\sqrt{2-x}$;
\item $\sqrt{x}=2x^2-1$;
\item $\sqrt{-x}=x^2$.
\end{abc3}
\item Függvénygrafikonok segítségével oldjuk meg a következő egyenlőtlenségeket: 
\begin{abc2}
\item $\sqrt{x+3}\le 1+x$;
\item $\sqrt{x+2}> x$;
\item $\sqrt{2x-5}> 7$;
\item ($*$) $\sqrt{x^2+6x+9}-\sqrt{x^2-6x+9}> 1$.
\end{abc2}
\end{enumerate}

\subsection*{2009.09.24.}
\begin{enumerate}
\item Ábrázoljuk a következő függvényeket:
\begin{abc2}
\item $x\mapsto \sqrt{2-x}+1$, $x\le 2$;
\item $x\mapsto \sqrt{x^2-6x+9}-2$;
\item $x\mapsto \sqrt{-5-4x}$, $x\le -\dfrac{5}{4}$;
\item $x\mapsto \sqrt{5-x}+3$, $x\le 5$.
\end{abc2}
\item Oldjuk meg grafikonokkal:
\begin{abc3}
\item $\sqrt{15-x}=2-\sqrt{x-11}$;
\item $\sqrt{5-x}=x+1$;
\item $\sqrt{3x+3}\ge x+1$;
\item $\sqrt{x^2+10x+25}\ge x-2$;
\item $\sqrt{x^2+2x+1}\le x+3$;
\item $\left|\sqrt{x^2-8x+16}+x-1\right|=6$.
\end{abc3}
\end{enumerate}

\subsection*{2009.09.29. -- Dolgozat}
\begin{enumerate}
\item Ábrázoljuk az
$$ x\mapsto \sqrt{1-2x},\quad -2\le x \le \frac{1}{2}$$
függvényt.
\item Oldjuk meg a következő egyenlőtlenséget:
$$\sqrt{|x-2|}\le 1.$$
\end{enumerate}

\subsection*{2009.09.30.}
\begin{enumerate}
\item Ábrázoljuk az $x\mapsto \dfrac{1}{x}, x\ne 0$ függvényt.
\item Ábrázoljuk a következő függvényeket:
\begin{abc2}
\item $x\mapsto \dfrac{1}{x-1},\quad x\ne 1$;
\item $x\mapsto \dfrac{1}{x+2},\quad x\ne -2$;
\item $x\mapsto -\dfrac{1}{x},\quad x\ne 0$;
\item $x\mapsto \dfrac{1}{|x|},\quad x\ne 0$;
\item $x\mapsto \left|\dfrac{x}{x-1}\right|,\quad x\ne 1$;
\item $x\mapsto \dfrac{2x+1}{x+1},\quad x\ne -1$.
\end{abc2}
\item Ábrázoljuk a következő függvényeket:
\begin{abc2}
\item $x\mapsto \dfrac{1}{|x|-1},\quad x\ne 1, x\ne -1$;
\item $x\mapsto \left|\dfrac{|x|}{|x|-1}\right|,\quad x\ne 1, x\ne -1$.
\end{abc2}
\end{enumerate}

\subsection*{2009.09.30. -- Dolgozat}
\begin{enumerate}
\item Oldjuk meg függvénygrafikonokkal:
\begin{abc2}
\item $\sqrt{7-x}=x-1$;
\item $\sqrt{3-x}\le \sqrt{x+1}+2$.
\end{abc2}
\end{enumerate}

\subsection*{2009.10.01.}
\begin{enumerate}
\item Oldjuk meg függvénygrafikonokkal:
\begin{abc2}
\item $\dfrac{1}{x}=x$;
\item $\dfrac{1}{x-1}+1=x$;
\item $\left|\dfrac{1}{1-x}\right|=2$;
\item $\left|\dfrac{|x|}{|x|-1}\right|=|x|$.
\end{abc2}

\item Ábrázoljuk a következő függvényeket:
\begin{abc2}
\item $x\mapsto \dfrac{|x|-1}{|x|-2},\quad x\ne 2, x\ne -2$;
\item $x\mapsto \left|\dfrac{|x|-3}{|x|-2}\right|,\quad x\ne 2, x\ne -2$;
\item $x\mapsto \dfrac{x^2-6x+5}{x^2-2x+1},\quad x\ne 1$;
\item $x\mapsto \dfrac{3|x|-2}{|x|-1},\quad x\ne 1, x\ne -1$.
\end{abc2}

\end{enumerate}

\subsection*{2009.10.01. -- Dolgozat}
\begin{enumerate}
\item Ábrázoljuk a következő függvényeket:
\begin{abc2}
\item $x\mapsto \dfrac{1}{|x-1|},\quad x\ne 1$;
\item $x\mapsto \dfrac{x-1}{x-2},\quad x\ne 2$.
\end{abc2}
\end{enumerate}

\subsection*{2009.10.07.}
\begin{enumerate}
\item Oldjuk meg a következő egyenleteket, egyenlőtlenségeket függvénygrafikonok segítségével:
\begin{abc4}
\item $(x-1)^2=|x-1|$;
\item $\dfrac{|x|-1}{|x|-2}=2$;
\item ($*$) $\left|\dfrac{|x|-2}{|x|-1}\right|=|x|$;
\item $\sqrt{x-1}=x-1$;
\item $\sqrt{2-x}<x$;
\item $8-x^2<2x$;
\item ($*$) $x+\dfrac{1}{x}>2$;
\item $x+\dfrac{1}{x}<-2$.
\end{abc4}
\end{enumerate}

\subsection*{2009.10.08.}
\begin{enumerate}
\item Oldjuk meg függvénygrafikonokkal:
\begin{abc2}
\item $\sqrt{x}=\dfrac{1}{2}x$;
\item $\sqrt{x}=x-2$;
\item $\sqrt{x}=\dfrac{1}{x}+\dfrac{7}{4}$;
\item $\sqrt{x}=2x^2-1$;
\item $\sqrt{3x-5}=x^2-7$;
\item $\sqrt{2x-1}=5-\sqrt{x-1}$;
\item $\sqrt{x+2}=\dfrac{4}{x}$;
\item ($*$) $x^2-2=\sqrt{x+2}$;
\item $\sqrt{1+x}=1-\sqrt{1-x}$;
\item $\sqrt{\left|\dfrac{1}{4}-x\right|}\ge x+\dfrac{1}{2}x$.
\end{abc2}
\end{enumerate}

\subsection*{2009.10.13. -- Ismétlő feladatok}
\begin{enumerate}
\item Ábrázoljuk a következő függvényeket:
\begin{abc2}
\item $x\mapsto \sqrt{x-[x]}+x$;
\item $x\mapsto \sqrt{3-x}+1$;
\item $x\mapsto \left|\dfrac{|x|-2}{|x|-3}\right|,\quad x\ne 3,-3$; 
\item $x\mapsto \sqrt{x^2-2x+1}+\sqrt{x^2-4x+4}$.
\end{abc2}
\item Oldjuk meg függvénygrafikonokkal a következő egyenleteket, egyenlőtlenségeket:
\begin{abc2}
\item $x^2-9\le 0$;
\item $x(x-3)<0$;
\item $x^2-2x= 0$;
\item $\dfrac{x}{3}-\dfrac{4}{3}<\dfrac{4}{x}$;
\item $\dfrac{1}{x-1}-3<\dfrac{2}{x-3}$;
\item ($*$) $\dfrac{x^2-25}{x-4}< 0$;
\item ($*$) $\dfrac{x^2+2x-63}{x^2-8x+7}>3$.
\end{abc2}
\end{enumerate}

\subsection*{2009.10.14. -- Ismétlő feladatok}
\begin{enumerate}
\item Ábrázoljuk az $x\mapsto x^2-x+1,\quad x\ge \frac{1}{2}$ és 
$x\mapsto \frac{1}{2}+\sqrt{x-\frac{3}{4}},\quad x\ge \frac{3}{4}$ 
függvényeket egy koordináta-rendszerben. Oldjuk meg függvénygrafikonok 
segítségével az $x^2-x+1=\frac{1}{2}+\sqrt{x-\frac{3}{4}}$ egyenletet.
\item Ábrázoljuk vázlatosan a következő függvényeket:
\begin{abc3}
\item $x\mapsto x+\dfrac{1}{x},\quad x\ne 0$;
\item $x\mapsto \dfrac{2x}{1+x^2}$;
\item $x\mapsto \dfrac{|x|+2}{|x|+1}$.
\end{abc3}
\item A ,,szignum'' függvényt így értelmezzük:
$$\sgn x = \left\{
\begin{array}{rl}
1,~&\text{ha~} x>0\cr
0,~&\text{ha~} x=0\cr
-1,~&\text{ha~} x<0
\end{array}
\right. $$
Ábrázoljuk a következő függvényeket:
\begin{abc3}
\item $x\mapsto \sgn (x^2-1)$;
\item $x\mapsto \sgn \left(\dfrac{1}{x-1}+1\right)$;
\item $x\mapsto \sgn (1-\sqrt{x-1}),\quad x\ge 1$.
\end{abc3}
\end{enumerate}


\end{document}
