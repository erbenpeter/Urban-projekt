\documentclass{article}
\usepackage[magyar]{babel}
\usepackage[utf8]{inputenc}
\usepackage{t1enc}
\usepackage{graphicx}
\usepackage{geometry}
\usepackage{tikz,pgf}
\usetikzlibrary{arrows}
 \geometry{
 a4paper,
 total={210mm,297mm},
 left=20mm,
 right=20mm,
 top=20mm,
 bottom=20mm,
 }
\usepackage{amsmath}
\usepackage{amssymb}
\frenchspacing
\usepackage{fancyhdr}
\pagestyle{fancy}
\lhead{Urbán János tanár úr feladatsorai}
\chead{C08/08/1.}
\rhead{Valószínűségszámítás}
\lfoot{}
\cfoot{\thepage}
\rfoot{}

\usepackage{enumitem}
\usepackage{multicol}
\usepackage{calc}
\newenvironment{abc}{\begin{enumerate}[label=\textit{\alph*})]}{\end{enumerate}}
\newenvironment{abc2}{\begin{enumerate}[label=\textit{\alph*})]\begin{multicols}{2}}{\end{multicols}\end{enumerate}}
\newenvironment{abc3}{\begin{enumerate}[label=\textit{\alph*})]\begin{multicols}{3}}{\end{multicols}\end{enumerate}}
\newenvironment{abc4}{\begin{enumerate}[label=\textit{\alph*})]\begin{multicols}{4}}{\end{multicols}\end{enumerate}}
\newenvironment{abcn}[1]{\begin{enumerate}[label=\textit{\alph*})]\begin{multicols}{#1}}{\end{multicols}\end{enumerate}}
\setlist[enumerate,1]{listparindent=\labelwidth+\labelsep}

\newcommand{\degre}{\ensuremath{^\circ}}
\newcommand{\tg}{\mathop{\mathrm{tg}}\nolimits}
\newcommand{\ctg}{\mathop{\mathrm{ctg}}\nolimits}
\newcommand{\arc}{\mathop{\mathrm{arc}}\nolimits}
\renewcommand{\arcsin}{\arc\sin}
\renewcommand{\arccos}{\arc\cos}
\newcommand{\arctg}{\arc\tg}
\newcommand{\arcctg}{\arc\ctg}

\parskip 8pt
\title{c08-1-08-03valszam}
\begin{document}

\section*{Valószínűségszámítás}

\subsection*{2009.12.08.}
\begin{enumerate}
\item Két (különböző színű) kockával dobunk. Mennyi a valószínűsége, hogy a dobott számok összege prímszám?
\item Egy kockával 5-ször dobunk. Mennyi a valószínűsége, hogy a dobott számok összege osztható 6-tal? 
\item Egy dobozban van 6 piros és 4 fehér golyó. 4-szer húzunk ki egy-egy golyót visszatevés nélkül. Adjuk meg a kihúzott piros golyók számának eloszlását (annak valószínűségeit, hogy ekkor 0,1,2,3,4 piros golyót húzunk).
\item Egy játékban választhatunk, hogy egy szabályos kockával dobunk és akkor nyerünk, ha 6-ost dobunk, vagy két kockával dobunk és akkor nyerünk, ha a dobott számok összege 6. Melyiket válasszuk?
\item András és Botond egy-egy kockával dobnak. Az nyer, aki nagyobb számot dob. Ha egyformát dobnak, akkor döntetlen. Mennyi a valószínűsége, hogy Botond nyer?
\end{enumerate}

\subsection*{2009.12.09.}
\begin{enumerate}
\item Három különböző színű kockával dobunk. Számítsuk ki annak a valószínűségét, hogy legalább 15 a dobott pontok összege.
\item Egy kockával háromszor dobunk. Mennyi a valószínűsége, hogy legalább egy 6-ost dobunk?
\item Egy kockával addig dobunk ismételten, amíg 6-ost nem dobunk. Mennyi a várható dobásszám?
\item Egy pénzérmével addig dobunk ismételten, amíg kétszer egymás után ugyanazt nem dobjuk. Mennyi a várható dobásszám?
\item Két (különböző színű) kockával dobunk. Mennyi a dobott pontok összegének a várható értéke?
\end{enumerate}

\subsection*{2009.12.10.}
\begin{enumerate}
\item Két kockával (egy pirossal és egy kékkel) dobunk. Számítsuk ki a dobott pontok minimumának várható értékét.
\item Egy szabályos kockát n-szer feldobunk. Adjuk meg a dobott 6-osok számának eloszlását.
\item Három érmét $n$-szer feldobunk. A kísérlet kimenetele legyen az $a$ szám, ahányszor mind a három érme ugyanarra az oldalára esik. Adjuk meg a $P$(kimenetel=$k$) valószínűségeket.
\item A ,,90-ből 5'' lottón mennyi a találatok számának várható értéke, ha egy lottószelvényt véletlenszerűen töltünk ki?
\item Egy urnában 4 piros és 6 fehér golyó van. Kihúzunk 5 fehér golyót
\begin{abc}
\item visszatevéssel
\item visszatevés nélkül.
\end{abc}
Számítsuk ki mindkét esetben a kihúzott piros golyók számának várható értékét.
\end{enumerate}

\newpage

\subsection*{2009.12.10.}
\begin{enumerate}
\item A $[0;1]$ intervallumban véletlenszerűen kijelöltünk két pontot: $0<x<y<1$. Mennyi a valószínűsége, hogy az $x$, $y-x$ és $1-y$ hosszúságú szakaszokból háromszög szerkeszthető?
\item Négy kocka kiterített hálója látható az ábrán a lapokra írt számokkal.
Az a kocka nyer, amelyikben a nagyobb szám lesz felül. Mennyi eséllyel nyer $A$ a $B$ ellen, $B$ a $C$ ellen, $C$ a $D$ ellen és $D$ az $A$ ellen?
\begin{center}
\begin{tikzpicture}[line cap=round,line join=round,>=triangle 45,x=1.0cm,y=1.0cm]
\clip(-3.44,-0.78) rectangle (12.52,4.22);
\draw (-3.,3.)-- (-3.,2.);
\draw (-2.,2.)-- (-2.,0.);
\draw (-3.,2.)-- (-2.,2.);
\draw (-3.,3.)-- (-2.,3.);
\draw (-2.,3.)-- (-2.,4.);
\draw (-2.,4.)-- (-1.,4.);
\draw (-1.,4.)-- (-1.,0.);
\draw (-1.,0.)-- (-2.,0.);
\draw (-3.,2.)-- (0.,2.);
\draw (0.,2.)-- (0.,3.);
\draw (0.,3.)-- (-3.,3.);
\draw (-2.,3.)-- (-2.,2.);
\draw (-2.,1.)-- (-1.,1.);
\draw (1.,3.)-- (4.,3.);
\draw (4.,2.)-- (1.,2.);
\draw (1.,3.)-- (1.,2.);
\draw (4.,3.)-- (4.,2.);
\draw (2.,4.)-- (2.,0.);
\draw (3.,4.)-- (3.,0.);
\draw (3.,0.)-- (2.,0.);
\draw (2.,4.)-- (3.,4.);
\draw (3.,1.)-- (2.,1.);
\draw (5.,3.)-- (8.,3.);
\draw (8.,2.)-- (5.,2.);
\draw (6.,4.)-- (6.,0.);
\draw (7.,0.)-- (7.,4.);
\draw (7.,4.)-- (6.,4.);
\draw (5.,3.)-- (5.,2.);
\draw (6.,0.)-- (7.,0.);
\draw (8.,2.)-- (8.,3.);
\draw (7.,1.)-- (6.,1.);
\draw (9.,3.)-- (12.,3.);
\draw (12.,2.)-- (9.,2.);
\draw (10.,4.)-- (10.,0.);
\draw (11.,0.)-- (11.,4.);
\draw (11.,4.)-- (10.,4.);
\draw (9.,3.)-- (9.,2.);
\draw (12.,3.)-- (12.,2.);
\draw (11.,0.)-- (10.,0.);
\draw (11.,1.)-- (10.,1.);
\draw (-1.7,0) node[anchor=north west] {$A$};
\draw (2.3,0) node[anchor=north west] {$B$};
\draw (6.3,0) node[anchor=north west] {$C$};
\draw (10.3,0) node[anchor=north west] {$D$};
\draw (-1.7,3.8) node[anchor=north west] {$0$};
\draw (-1.7,2.8) node[anchor=north west] {$0$};
\draw (-2.7,2.8) node[anchor=north west] {$4$};
\draw (-0.7,2.8) node[anchor=north west] {$4$};
\draw (-1.7,1.8) node[anchor=north west] {$4$};
\draw (-1.7,0.8) node[anchor=north west] {$4$};
\draw (2.3,3.8) node[anchor=north west] {$3$};
\draw (1.3,2.8) node[anchor=north west] {$3$};
\draw (2.3,2.8) node[anchor=north west] {$3$};
\draw (3.3,2.8) node[anchor=north west] {$3$};
\draw (2.3,1.8) node[anchor=north west] {$3$};
\draw (2.3,0.8) node[anchor=north west] {$3$};
\draw (6.3,3.8) node[anchor=north west] {$6$};
\draw (6.3,2.8) node[anchor=north west] {$6$};
\draw (5.3,2.8) node[anchor=north west] {$2$};
\draw (7.3,2.8) node[anchor=north west] {$2$};
\draw (6.3,1.8) node[anchor=north west] {$2$};
\draw (6.3,0.8) node[anchor=north west] {$2$};
\draw (10.3,3.8) node[anchor=north west] {$1$};
\draw (9.3,2.8) node[anchor=north west] {$1$};
\draw (11.3,2.8) node[anchor=north west] {$1$};
\draw (10.3,2.8) node[anchor=north west] {$5$};
\draw (10.3,1.8) node[anchor=north west] {$5$};
\draw (10.3,0.8) node[anchor=north west] {$5$};
\end{tikzpicture}
\end{center}
%\includegraphics[width=0.3\textwidth]{Image.PNG}
\item Három szabályos kockát feldobunk. Mennyi a valószínűsége, hogy
\begin{abc}
\item mindegyik dobott szám különböző;
\item mindegyik dobott szám azonos?
\end{abc}
\end{enumerate}

\subsection*{2010.01.05.}
\begin{enumerate}
\item Ha 100 darab tízforintost feldobunk, mennyi a valószínűsége, hogy 50 esik fejre?
\item A következő események közül melyik a legvalószínűbb:
\begin{abc}
\item 6 dobásból legalább egyszer 6-ost dobunk a dobókockával;
\item 12 dobásból legalább kétszer 6-ost dobunk;
\item 18 dobásból legalább háromszor 6-ost dobunk?
\end{abc}
\item Hány ember esetén igaz az, hogy már $\frac{1}{2}$-nél nagyobb valószínűséggel van köztük kettő olyan akik az év azonos napján ünneplik születésnapjukat?
\item Két kockával (egy pirossal és egy kékkel) dobunk addig, amíg a dobott számok összege 12 szám lesz. Mennyi a várható dobásszám?
\item Egyszerre dobunk 6 szabályos dobókockával. Mennyi a valószínűsége, hogy legalább két dobókockán azonos pontszám lesz felül?
\end{enumerate}

\subsection*{2010.01.06.}
\begin{enumerate}
\item Egy érmével dobunk. Ha az eredmény fej, akkor még kétszer dobunk, ha írás, még egyszer. Mennyi az összes fej dobások számának várható értéke?
\item Számítsuk ki az ötös lottón kihúzott
\begin{abc}
\item legnagyobb;
\item legkisebb szám várható értékét.
\end{abc}
\item Egy érmével dobunk addig, amíg először fordul elő, hogy két egymást követő dobás azonos. Mennyi a szükséges dobások számának várható értéke?
\item Egy urnában van 4 fehér és 6 piros golyó. Ha egy golyót kihúzunk és az fehér, akkor visszatesszük, ha piros, akkor helyette fehér golyót teszünk vissza. Ezt a műveletet 3-szor megismételjük. Mennyi a valószínűsége, hogy ezután az urnából egy golyót kihúzva fehér lesz?
\item Egy kockát addig dobunk, amíg 6-ost nem kapunk. Mennyi annak a valószínűsége, hogy páros számú dobás kellett ahhoz, hogy először 6-ost dobjunk?
\end{enumerate}

\subsection*{2010.01.13.}
\begin{enumerate}
\item Egy dobozban öt papírdarab van, ezeken 1-től 5-ig az egész számok vannak felírva. Először András húz egy papírt véletlenszerűen, ezt nem teszi vissza, majd Béla is húz egy papírt. Mennyi a valószínűsége, hogy a két papírra írt szám összege páros?
\item Egy urnában 3 piros és 6 fehér golyó van. Addig húzunk visszatevés nélkül, amíg fehér golyót nem húzunk. Mennyi az addig kihúzott piros golyók várható száma?
\item Négy piros, három fehér és két kék golyót találomra egymásmellé teszünk. Mennyi a valószínűsége, hogy fehér golyók nem kerülnek egymás mellé?
\item Három házaspár színházba megy. A 6 jegy egy sorban egymás mellé szól. Ha véletlenszerűen osztják el egymás közt a jegyeket, mennyi a valószínűsége, hogy egyik házaspár sem ül egymás mellett?
\item Egy szabályos pénzérmével addig dobunk, amíg kétszer egymás után fejet nem dobunk. Mennyi a várható dobásszám?
\end{enumerate}

\end{document}