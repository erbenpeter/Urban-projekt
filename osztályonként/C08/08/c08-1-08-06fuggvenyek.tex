\documentclass{article}
\usepackage[utf8]{inputenc}
\usepackage{t1enc}
\usepackage{geometry}
 \geometry{
 a4paper,
 total={210mm,297mm},
 left=20mm,
 right=20mm,
 top=20mm,
 bottom=20mm,
 }
\usepackage{amsmath}
\usepackage{amssymb}
\frenchspacing
\usepackage{fancyhdr}
\pagestyle{fancy}
\lhead{Urbán János tanár úr feladatsorai}
\chead{C08/08/1. csoport}
\rhead{Függvények}
\lfoot{}
\cfoot{\thepage}
\rfoot{}

\usepackage{enumitem}
\usepackage{multicol}
\usepackage{calc}
\newenvironment{abc}{\begin{enumerate}[label=\textit{\alph*})]}{\end{enumerate}}
\newenvironment{abc2}{\begin{enumerate}[label=\textit{\alph*})]\begin{multicols}{2}}{\end{multicols}\end{enumerate}}
\newenvironment{abc3}{\begin{enumerate}[label=\textit{\alph*})]\begin{multicols}{3}}{\end{multicols}\end{enumerate}}
\newenvironment{abc4}{\begin{enumerate}[label=\textit{\alph*})]\begin{multicols}{4}}{\end{multicols}\end{enumerate}}
\newenvironment{abcn}[1]{\begin{enumerate}[label=\textit{\alph*})]\begin{multicols}{#1}}{\end{multicols}\end{enumerate}}
\setlist[enumerate,1]{listparindent=\labelwidth+\labelsep}

\newcommand{\degre}{\ensuremath{^\circ}}
\newcommand{\tg}{\mathop{\mathrm{tg}}\nolimits}
\newcommand{\ctg}{\mathop{\mathrm{ctg}}\nolimits}
\newcommand{\arc}{\mathop{\mathrm{arc}}\nolimits}
\renewcommand{\arcsin}{\arc\sin}
\renewcommand{\arccos}{\arc\cos}
\newcommand{\arctg}{\arc\tg}
\newcommand{\arcctg}{\arc\ctg}

\parskip 8pt
\begin{document}

\section*{Függvények}

\subsection*{2010.03.25.}
\begin{enumerate}
\item Ábrázoljuk a következő függvényeket:
\begin{abc2}
\item $f(x)=x^2-5x+6, |x|\le 4$;
\item $g(x)=\sqrt{x^2-4x+4}+\sqrt{x^2-6x+9}, |x|\le 4$;
\item $h(x)=x^2-6|x|+5, -2\le x\le 3$;
\item $k(x)=x\sqrt{x^2-2x+1}, -2\le x\le 4$.
\end{abc2}
\item Ábrázoljuk a következő függvényt:
$$f:[1;+\infty[\to \mathbb{R},\qquad f(x)=\sqrt{x+2\sqrt{x-1}}+\sqrt{x-2\sqrt{x-1}}.$$
\item ($*$) Hol veszi fel a következő függvény a legnagyobb értékét és mennyi ez az érték?
$$g:[-1;1]\to\mathbb{R},\qquad g(x)=3x+4\sqrt{1-x^2}.$$
\end{enumerate}

\subsection*{2010.03.30.}
\begin{enumerate}
\item Oldjuk meg függvények segítségével:
\begin{abc2}
\item $\dfrac{1}{x}<\dfrac{1}{2}$;
\item $(x-1)^2-5\le(x+4)^2$;
\item $\dfrac{3x}{2x+1}>1$;
\item $\dfrac{2x-7}{4-x}>1$;
\item $\dfrac{5x-5}{3x-2}>1$.
\end{abc2}
\item Ábrázoljuk a következő függvényeket:
\begin{abc3}
\item $x\mapsto x^2-2x-3$;
\item $x\mapsto x^2-4x+3$;
\item $x\mapsto x^2+x-30$;
\item $x\mapsto x^2+x-12$;
\item $x\mapsto 5x^2+7x+2$.
\end{abc3}
\end{enumerate}
\subsection*{2010.04.07.}
\begin{enumerate}
\item Oldjuk meg függvények segítségével:
\begin{abc3}
\item $\sqrt{x+5}=x^2-5$;
\item ($*$) $\sqrt{1+x}-\sqrt{1-x}=1$;
\item  $\sqrt{x-6}+\sqrt{7-x}=1$;
\item $\dfrac{x-1}{\sqrt{x}+1}=\dfrac{\sqrt{x}-1}{2}+4$;
\item $x+3=4 \sqrt{x}$.
\end{abc3}
\item Oldjuk meg függvényekkel:
\begin{abc2}
\item  $\sqrt{x}\le x-2$;
\item $x^2>2\sqrt{x^2-4x+4}$;
\item $\sqrt{x+2}>x$;
\item ($*$) $\dfrac{\sqrt{3-x}+1}{x}<1$.
\end{abc2}

\end{enumerate}
\subsection*{2010.04.08.}
\begin{enumerate}
\item Ábrázoljuk a következő függvényt:
$$x\mapsto |\sqrt{x-1}-2|+|\sqrt{x-1}-3|;\qquad x\ge 1.$$
\item Oldjuk meg a valós számok halmazán:
$$\sqrt{x+3-4\sqrt{x-1}}+\sqrt{x+8-6\sqrt{x-1}}=1.$$
\item Oldjuk meg a következő egyenletet:
$$4-x^2=\sqrt{4+x}.$$
\item Oldjuk meg a valós számok halmazán:
$$7-x^2=\sqrt{7+x}.$$
\item Oldjuk meg a valós számok halmazán a következő egyenlőtlenségeket:
\begin{abc2}
\item $x+\sqrt{2-x}>0$;
\item $x^2+x+1<3\sqrt{x^2-4x+4}$;
\item $\dfrac{3}{\sqrt{2-x}}-\sqrt{2-x}<2$;
\item $x^2>2\sqrt{x^2-4x+4}$.
\end{abc2}
\end{enumerate}

\subsection*{2010.04.13.}
\begin{enumerate}
\item Oldjuk meg függvénygrafikonok felhasználásával:
\begin{abc3}
\item $x^2-2\ge 2x-4$;
\item $4x-x^2\le 4x$;
\item $2x-x^2\ge x$.
\end{abc3}
\item ($*$) Oldjuk meg a valós számok halmazán:
$$\sqrt{10+x+6\sqrt{x+1}}+\sqrt{5-x+2\sqrt{4-x}}=7.$$
\item ($*$) Oldjuk meg a \underline{valós számok} halmazán:
$$x^4-2x^2=400x+9999.$$
\item ($*$) Oldjuk meg a valós számok halmazán:
$$\dfrac{\sqrt{x^2+8x}}{\sqrt{x+1}}+\sqrt{x+7}=\dfrac{7}{\sqrt{x+1}}.$$
\item Oldjuk meg a valós számok halmazán:
$$\dfrac{1-\sqrt{1-4x^2}}{x}<3.$$
\end{enumerate}
\subsection*{2010.04.14.}
\begin{enumerate}
\item Ábrázoljuk a következő függvényeket:
\begin{abc2}
\item $x\mapsto \sqrt{1-x^2},\quad |x|\le 1$;
\item $x\mapsto \dfrac{x-2}{x^2-5x+6},\quad x\ne 2, x\ne 3$;
\item $x\mapsto \dfrac{x-5}{x^2-7x+10},\quad x\ne 2, x\ne 5$;
\item ($*$) $y=\sqrt{4x-x^2},\quad 0\le x\le 4$.
\end{abc2}
\item Oldjuk meg a következő egyenleteket:
\begin{abc3}
\item $\sqrt{x}=\dfrac{2}{\sqrt{2+x}}-\sqrt{2+x}$;
\item $\sqrt{2x-1}=5$;
\item $\sqrt{7-\sqrt{7+x}}=x$.
\end{abc3}
\item Oldjuk meg a következő egyenlőtlenségeket:
\begin{abc2}
\item $\sqrt{x+1}-\sqrt{x}<\dfrac{1}{10}$;
\item $\dfrac{x-1}{x+1}>\dfrac{x}{x-1}$.
\end{abc2}
\end{enumerate}
\subsection*{2010.04.19.}
\begin{enumerate}
\item Oldjuk meg függvények segítségével a valós számok halmazán:
\begin{abc2}
\item $|x^2-6x+5|=x-3$;
\item $(x^2-7x+6)\sqrt{x^2-10x+21}\ge 0$;
\item $\dfrac{\sqrt{1+x}}{x-1}\ge \dfrac{5-x}{x-1}$;
\item $\sqrt{x^2-x-2}>x-1$;
\item $3|x-1|\le x+3$.
\end{abc2}
\item Határozzuk meg, hogy hol vesz fel pozitív értékeket a következő függvény:
$$x\mapsto \dfrac{1}{x-1}-\dfrac{1}{x+1},\qquad x\ne 1, x\ne -1, x\in \mathbb{R}.$$
\item Az $a\ge 0$ paraméter mely értékeire hány megoldása van az $|x^2-2x-3|=a$ egyenletnek?
\end{enumerate}
\subsection*{2010.04.21.}
\begin{enumerate}
\item Oldjuk meg függvénygrafikonokkal:
\begin{abc3}
\item $x^2+x-2=0$;
\item $x^2+2x-3=0$;
\item $2x^2-3x-2=0$;
\item $x^2+|x|-2=0$;
\item $2x^2-|x|-1=0$.
\end{abc3}

\item Vizsgáljuk meg grafikusan a következő paraméteres egyenleteket:
\begin{abc3}
\item $x^2-2x+a=0$;
\item $x^2-6x+a=0$;
\item $x^2+x+a=0$.
\end{abc3}
\item Oldjuk meg grafikonok segítségével a következő egyenletrendszert:
\begin{abc2}
\item $\left.
\begin{aligned}
x^2+y^2&=25\\
x^2+y&=13
\end{aligned}
\right\}$;

\item $\left.
\begin{aligned}
x+y&=3\\
xy&=2
\end{aligned}
\right\}$.
\end{abc2}
\end{enumerate}
\subsection*{2010.04.22.}
\begin{enumerate}
\item Ábrázoljuk a következő függvényeket:
\begin{abc3}
\item $x\mapsto \dfrac{1-x}{1+\sqrt{x}},\quad x\ge 0$;
\item $x\mapsto \sqrt[3]{x-1}$;
\item $x\mapsto |x^2-2x|-\dfrac{1}{2}$.
\end{abc3}
\item Oldjuk meg függvénygrafikonokkal:
\begin{abc2}
\item $\sqrt{x-7}-\dfrac{6}{\sqrt{x-7}}=1$;
\item $\sqrt{x}-\dfrac{2}{\sqrt{x}}=1$;
\item $\sqrt{x}\cdot\sqrt{1-x}=x$;
\item $\sqrt{2-x}>x$;
\item $\sqrt{x+2}+\sqrt{2x+4}<7$.
\end{abc2}
\item Grafikonok segítségével oldjuk meg:
\begin{abc2}
\item $\left.
\begin{aligned}
x^2+y^2&=25\\
y^2-x&=5
\end{aligned}
\right\}$;

\item $\left.
\begin{aligned}
xy-x+y&=5\\
2xy+x-y&=4
\end{aligned}
\right\}$.
\end{abc2}
\end{enumerate}
\subsection*{2010.04.26.}
\begin{enumerate}
\item Ábrázoljuk a következő függvényeket:
\begin{abc2}
\item $f(x)=x^2+5|x-1|+1$;
\item $g(x)=|x^2-3x+2|+|5-x|$;
\item $h(x)=1-\dfrac{1}{|x|},\quad x\ne 0$;
\item $k(x)=(x+1)(|x|-2)$.
\end{abc2}
\item Ábrázoljuk a derékszögű koordináta-rendszerben azokat a pontokat,
amelyeknek $(x;y)$ koordinátáira teljesül:
\begin{abc2}
\item $|x|+x=|y|+y$;
\item $|x-2|+|y+1|\le 1$;
\item $|2x+y|+|2x-y|\le 4$;
\item $|y-1|=x^2-4x+3$.

\end{abc2}
\item ($*$) Határozzuk meg a következő függvények legnagyobb és legkisebb értékét:
\begin{abc2}
\item $x\mapsto \dfrac{x^2}{x^4+1},\quad x\in\mathbb{R}$;
\item $x\mapsto \dfrac{2x}{x^2+1},\quad x\in\mathbb{R}$.
\end{abc2}
\end{enumerate}
\subsection*{2010.04.27.}
\begin{enumerate}
\item Oldjuk meg a valós számok halmazán: 
$\sqrt{x+5-4\sqrt{x+1}}+\sqrt{x+2-2\sqrt{x+1}}=1$.
\item Oldjuk meg függvénygrafikonokkal:
\begin{abc3}
\item $\sqrt{x+1}-\sqrt{x-1}=1$;
\item $\sqrt{25-x}-\sqrt{9-x}=2$;
\item $\sqrt{4x+2}+\sqrt{4x-2}=4$.
\end{abc3}
\item ($*$) Oldjuk meg a valós számok halmazán: $\sqrt{2x+1}+\sqrt{x+1}<1$.
\item Oldjuk meg a következő egyenleteket, egyenlőtlenségeket: 
\begin{abc3}
\item $\sqrt{2+x-x^2+1}>x-4$;
\item $\sqrt{x-5}-\sqrt{8-x}=2$;
\item $\sqrt{4-\sqrt{1-x}}>\sqrt{2-x}$.
\end{abc3}


\end{enumerate}
\subsection*{2010.04.28.}
\begin{enumerate}
\item Ábrázoljuk a következő függvényeket:
\begin{abc2}
\item $x\mapsto x^2-2|x|-3$;
\item $x\mapsto (|x+1|+1)(x-3)$.
\end{abc2}
\item Oldjuk meg függvénygrafikonok segítségével:
\begin{abc2}
\item $x^2-5x<6$;
\item $x^3>3x-2$.
\end{abc2}
\item Oldjuk meg a valós számok halmazán: $\sqrt{2-\sqrt{3+x}}<\sqrt{4+x}$.
\item Ábrázoljuk a derékszögű koordináta-rendszerben azoknak a $P(x;y)$ pontoknak a halmazát, amelyeknek koordinátáira teljesül:
$$(x-|x|)^2+(y-|y|)^2=4.$$
\end{enumerate}
\end{document}
