\documentclass{article}
\usepackage[magyar]{babel}
\usepackage[utf8]{inputenc}
\usepackage{t1enc}
\usepackage{geometry}
 \geometry{
 a4paper,
 total={210mm,297mm},
 left=20mm,
 right=20mm,
 top=20mm,
 bottom=20mm,
 }
\usepackage{amsmath}
\usepackage{amssymb}
\frenchspacing
\usepackage{fancyhdr}
\pagestyle{fancy}
\lhead{Urbán János tanár úr feladatsorai}
\chead{C08/08/1.}
\rhead{Kombinatorika}
\lfoot{}
\cfoot{\thepage}
\rfoot{}

\usepackage{enumitem}
\usepackage{multicol}
\usepackage{calc}
\newenvironment{abc}{\begin{enumerate}[label=\textit{\alph*})]}{\end{enumerate}}
\newenvironment{abc2}{\begin{enumerate}[label=\textit{\alph*})]\begin{multicols}{2}}{\end{multicols}\end{enumerate}}
\newenvironment{abc3}{\begin{enumerate}[label=\textit{\alph*})]\begin{multicols}{3}}{\end{multicols}\end{enumerate}}
\newenvironment{abc4}{\begin{enumerate}[label=\textit{\alph*})]\begin{multicols}{4}}{\end{multicols}\end{enumerate}}
\newenvironment{abcn}[1]{\begin{enumerate}[label=\textit{\alph*})]\begin{multicols}{#1}}{\end{multicols}\end{enumerate}}
\setlist[enumerate,1]{listparindent=\labelwidth+\labelsep}

\newcommand{\degre}{\ensuremath{^\circ}}
\newcommand{\tg}{\mathop{\mathrm{tg}}\nolimits}
\newcommand{\ctg}{\mathop{\mathrm{ctg}}\nolimits}
\newcommand{\arc}{\mathop{\mathrm{arc}}\nolimits}
\renewcommand{\arcsin}{\arc\sin}
\renewcommand{\arccos}{\arc\cos}
\newcommand{\arctg}{\arc\tg}
\newcommand{\arcctg}{\arc\ctg}

\parskip 8pt
\title{c08-1-08-02kombinatorika}
\begin{document}

\section*{Kombinatorika}

\subsection*{2009.10.20.}
\begin{enumerate}
\item Hányféleképpen oszthatunk 30 embert 3, egyenként 10 emberből álló csoportba?
\item Hányféleképpen bonthatjuk fel a 10-et 3 pozitív egész szám összegére, ha a sorrendben különböző felbontásokat is különbözőnek tekintjük?
\item Hányféleképpen választhatunk ki az első 3 ($n>0$, egész) egymást követő pozitív szám közül hármat úgy, hogy az összegük osztható legyen 3-mal?
\item $*$ Egy sorban 10 ember áll. Hányféleképpen választhatunk ki közülük hármat úgy, hogy ne legyenek köztük szomszédosak?
\end{enumerate}

\subsection*{2009.10.21.}
\begin{enumerate}
\item Egy sorban $n$ ember áll. Hányféleképpen választunk ki közülük hármat úgy, hogy ne legyen a kiválasztottak között két szomszédos?
\item Hány megoldása van a pozitív egészek körében az $x+y+z=n$ egyenletnek ($n\geq 3$, egész)?
\item $*$ Feldobunk $n$ ($\geq 1$) dobókockát. Hányféle eredményt kaphatunk?
\item Adott $n$ különböző könyv. Hányféleképpen választhatunk ki ezek közül páratlan számú könyvet?
\item Hány olyan négyjegyű szám van, amelynek a számjegyei pontosan kétfélék?
\item $*$ Hány olyan hatjegyű szám van, amelynek a számjegyei pontosan háromfélék?
\end{enumerate}

\subsection*{2009.10.21.}
\begin{enumerate}
\item Hányféleképpen lehet sorbarakni 10 golyót, amelyek közül 4 piros, 4 kék és 2 sárga, de a golyók mérete azonos?
\item Az 1 és $10^4-1$ közti egész számok tízes számrendszerbeli húzásához melyik számjegyből hány darabra van szükség?
\item Hány megoldása van e nemnegatív egész számok körében az $x+y+z=n$ egyenletnek ($n\geq 0$, egész)?
\item Legfeljebb hány részre bontja fel a síkot $n$ egyenes?
\item Legfeljebb hány részre bontja fel a teret $n$ sík?
\item Hány részre bontja fel a síkot $n$ egy ponton áthaladó egyenes?
\end{enumerate}

\subsection*{2009.11.04.}
\begin{enumerate}
\item Hány olyan háromszög van, amelyben valamennyi oldal hossza a $4,5,6,7$ értékek közül kerül ki?
\item Adott a síkon 4, egymást páronként metsző egyenes úgy, hogy nincs közöttük három egy ponton áthaladó. Hány háromszöget határoznak meg ezek az egyenesek?
\item Adott a síkon két párhuzamos egyenes, az egyiken $p$, a másikon $q$ pont. Hány olyan háromszög van, amleynek csúcsai az adott pontok közül kerülnek ki?
\item Adott a síkon $n$ pont, ezek közül $p$ egy egyenesen van, a többi $n-p$ pont között. Nincs 3 egy egyenesre illeszkedő. Hány olyan háromszög van, amelynek csúcsai az adott pontok közül kerülnek ki?
\item Hány olyan háromszög van, amelynek oldalhosszai $n$ -nél nagyobb, de $2n$ -nél nem nagyobb egész számok? Hány egyenlő szárú, illetve hány egyenlő oldalú van ezek között?
\item Hány olyan háromszög van, amelynek minden oldala egész hosszúságú, kerülete pedig 40?
\end{enumerate}

\subsection*{2009.11.10.}
\begin{enumerate}
\item Egy kör kerületén felveszünk $n$ pontot és az összes lehetséges húrt megrajzoljuk, ami ezeket a pontokat köti össze. Feltesszük, hogy nincs olyan pont a kör belsejében, amin három húr megy át. Hány részre osztják ezek a húrok a körlapot?
\item Van $n$ darab egyforma könyvünk. Hányféleképpen lehet ezeket három nem üres csoportra bontani?
\item Hányféleképpen választhatunk ki két 1 és 40 közötti egész számot úgy, hogy az összegük páros legyen?
\item Hányféleképpen lehet kiválasztani 20 érmét, ha 5,  10 és 20 forintosok közül választhatunk (mindegyikből korlátlan mennyiség van)?
\item Hányféleképpen választhatunk ki 3, 1 és 30 közötti egész számot úgy, hogy az összegük páros legyen?
\end{enumerate}

\subsection*{2009.11.11.}
\begin{enumerate}
\item Legfeljebb hány részre osztja a síkot $n$ darab körvonal?
\item Hányféleképpen válthatunk fel egy 200 forintost 10, 20 és 50 forintosokra?
\item Hány olyan tízjegyű szám van, amelynek minden jegye az 1, 2, 3 számjegyek közül kerül ki és pontosan kétszer fordul elő benne a 3-mas számjegy. Ezek közül hány szám osztható 9-cel?
\item Egy 3x3-as négyzet oldalú táblázat 9 kisebb négyzetből áll. Hányféleképpen színezhetjük ki a kis négyzeteket piros, kék, sárga színre úgy, hogy minden sorban és minden oszlopban mind a 3 szín előforduljon.
\item Helyezzünk el a síkon 6; 8 szakaszt úgy, hogy mindegyik szakasznak 3 másik szakasszal legyen metszéspontja.
\end{enumerate}

\subsection*{2009.11.18.}
\begin{enumerate}
\item Számítsuk ki: $$(a+b)^2=$$ $$(a+b)^3=$$ $$(a+b)^4=$$
\item Írjuk fel a binomiális tételt: $$(a+b)^n=\cdots$$
\item Hány egész megoldása van: 
\begin{abc}
\item $|x|+|y|\leq 10;$
\item $|x|+|y|\leq 100$.
\end{abc}
\item A Fibonacci sorozat definíciója: $$f_1=f_2=1$$ $$f_n+2=f_n+F_n+1$$ ha $n\geq 1$, egész.

Igazoljuk:
\begin{abc}
\item $f_1+f_3+\cdots +f_{2n-1}=f_{2n};$
\item $f_2+f_4+\cdots +f_{2n}=f_{2n+1}-1;$
\item $f_1^2+f_2^2+f_3^2+\cdots +f_n^2=f_n\cdot f_{n+1};$
\item $f_1\cdot f_2+f_2\cdot f_3+\cdots +f_{2n-1}\cdot f_{2n}=f_{2n}^2;$
\item $*$ $f_{n+1}^2-f_n\cdot f_{n+2}=(-1)^n$.
\end{abc}
\end{enumerate}

\subsection*{Kombinatorika}
\begin{enumerate}
\item Korlátlan mennyiségben állnak rendelkezésre 10, 20 és 50 forintos érmék. Hányféleképpen választhatunk ki ezek közül 20 érmét?
\item Hányféleképpen oszthatunk el $3n$ különböző könyvet 3 ember között, ha azt akarjuk, hogy mindenki $n$ könyvet kapjon?
\item Hányféleképpen oszthatunk szét 3 különböző dobozba 4 piros, 4 fehér és 4 kék golyót? (Egyes dobozok üresek is lehetnek.)
\item Egy konvex 8-szög átlóinak legfeljebb hány metszéspontja lehet a sokszögön belül és hány metszéspontja lehet a sokszögön kívül?
\item Van $3n+1$ tárgyunk, ezek közül $n$ egyforma, a többi ezektől és egymástól is különböző. Igazoljuk, hogy ezek közül $n$ tárgyat $2^{2n}$ féleképpen választhatunk ki.
\item Az $(1+x)^3+(1+x)^4+(1+x)^5+\cdots +(1+x)^{10}$ kifejezésben a műveletek után mi lesz $x^5$ együtthatója?
\end{enumerate}

\subsection*{2009.11.24.}
\begin{enumerate}
\item Feldobunk 6 játékkockát. Hányféle eredményt kaphatunk, ha a kockák sorrnedjét nem vesszük figyelembe és a kockák egyformák?
\item Feldobunk két játékkockát, mindegyiknek a lapjain a 0, 1, 3, 7, 15, 31, számok vannak. Hányféle  lehet a két kocka által mutatott számok összege?
\item Hányféleképpen tehetünk két zsebbe 9, páronként különböző pénzdarabot?
\item Adott $2n$ könyv. Hányféleképpen készíthetünk ezekből $n$ darab 2 könyvből álló csomagot, ha a csomagon belül a sorrend közömbös?
\item Hat golyónk van, 3 fekete, 1-1 páros, fehér, kék. Hányféleképpen lehet ezekből összeállítani, 4 golyóból álló sorozatokat?
\item $*$ Egy $n(>0)$ természetes szám 6-tal osztható. Hányféle módon bonthatjuk ezt fel 3 különböző pozitív egész összegére (a felbontás sorrendjével nem törődünk)?
\end{enumerate}

\subsection*{2009.11.25.}
\begin{enumerate}
\item Számítsuk ki az 1, 2, 3, 4 számjegyekből készíthető háromjegyű számok összegét.
\item Hány páros szám készíthető a 3, 6, 9, 4 számjegyekből, ha mindegyik számjegyet csak legfeljebb egyszer használhatjuk fel? És hány páratlan szám készíthatő ugyan ezekkel a feltételekkel?
\item Hány olyan tízes számrendszerbeli 6-jegyű szám van, amelyben a számjegyek összege páros?
\item Hány olyan tízjegyű szám van, amelyben a számjegyek összege 3?
\item Hányféleképpen oszthatunk el 6 különböző dobozban 5 piros és 5 kék golyót?
\item $*$ Két ember között 3 féle tárgyat osztanak el. Mindegyik fajtából $2n$ tárgy áll rendelkezésre és mindegyik ember összesen $3n$ tárgyat kap. Hányféleképpen történhet az elosztás?
\end{enumerate}
\subsection*{2009.12.01.}
\begin{enumerate}
\item Hányféleképpen képezhetünk $n$, $k$ különböző elemből $n$ számú, egyenként $k$ elemet tartalmazó csoportot?
\item Egy sorban van $n$ könyv. Hányféleképpen választhatunk ki közülük 3 könyvet úgy, hogy ezek között ne legyenek szomszédosak?
\item Adott $n$ tárgy. Hányféleképpen választhatunk ki ezek közül páratlan számút?
\item Adott $n$ egyforma és további $n$ az előzőktől és egymástól is különböző tárgy. Hányféleképpen választhatunk ki ezek közül $n$ tárgyat?
\item $*$ Hány olyan háromszög van, amelynek csúcsai egy konvex $n$-szög csúcsai közül kerülnek ki, de a két idomnak nincs közös oldala?
\item $*$ Hány részre bontják fel a konvex $n$ szöget az átlói, ha nincs köztük három egy ponton átmenő(és semelyik kettő nem párhuzamos)?
\end{enumerate}

\subsection*{2009.12.02.}
\begin{enumerate}
\item Hány olyan hatjegyű tízes számrendszerbeli szám van, amleynek a számjegyei pontosan kétfélék?
\item Hányféleképpen oszthatunk szét 4 különböző pénztárcába 3 kétszázforintost és 10 százforintost?
\item Hány olyan egymilliónál kisebb pozitív egész szám van, amelynek tízes számrendszerben felírt alakjában szerepelnek az 1, 2, 3, 4 számjegyek?
\item Hány olyan háromszög van, amelyben minden oldal hossza egész szám és kerülete 43 egység?
\item Adott $n$ különböző betű, közöttük $a$, $b$ és $c$ is szerepel. Hány olyan sorrendje van az $n$ szomszédnak? Hány olyan sorrend van, amelyben $a$, $b$ és $c$ közül semelyik kettő nem szomszédos?
\item Jelölje $f_n$ a Fibonacci-sorozat $n$-edik elemét $(f_1=f_2=1,~f_{n+2}=f_n+f_{n+1})$. Igazoljuk, hogy $$f_1+f_2+f_3+f_4+f_5+f_6+\cdots +f_{2n+1}+f_{2n}=f_{2n+2}-1.$$
\end{enumerate}

\end{document}