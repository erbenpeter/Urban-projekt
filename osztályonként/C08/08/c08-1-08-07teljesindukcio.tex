\documentclass{article}
\usepackage[utf8]{inputenc}
\usepackage{t1enc}
\usepackage{geometry}
\geometry{
	a4paper,
	total={210mm,297mm},
	left=20mm,
	right=20mm,
	top=20mm,
	bottom=20mm,
}
\usepackage{amsmath}
\usepackage{amssymb}
\frenchspacing
\usepackage{fancyhdr}
\pagestyle{fancy}
\lhead{Urbán János tanár úr feladatsorai}
\chead{C08/08/1. csoport}
\rhead{Teljes indukció}
\lfoot{}
\cfoot{\thepage}
\rfoot{}

\usepackage{enumitem}
\usepackage{multicol}
\usepackage{calc}
\newenvironment{abc}{\begin{enumerate}[label=\textit{\alph*})]}{\end{enumerate}}
\newenvironment{abc2}{\begin{enumerate}[label=\textit{\alph*})]\begin{multicols}{2}}{\end{multicols}\end{enumerate}}
\newenvironment{abc3}{\begin{enumerate}[label=\textit{\alph*})]\begin{multicols}{3}}{\end{multicols}\end{enumerate}}
\newenvironment{abc4}{\begin{enumerate}[label=\textit{\alph*})]\begin{multicols}{4}}{\end{multicols}\end{enumerate}}

\newcommand{\degre}{\ensuremath{^\circ}}
\newcommand{\tg}{\mathop{\mathrm{tg}}\nolimits}
\newcommand{\ctg}{\mathop{\mathrm{ctg}}\nolimits}
\newcommand{\arc}{\mathop{\mathrm{arc}}\nolimits}
\renewcommand{\arcsin}{\arc\sin}
\renewcommand{\arccos}{\arc\cos}
\newcommand{\arctg}{\arc\tg}
\newcommand{\arcctg}{\arc\ctg}
\newcommand{\sgn}{\operatorname{sgn}}



\parskip 8pt
\begin{document}

\section*{Teljes indukció}
	
\subsection*{2010. 05. 05}
\begin{enumerate}
\item Igazoljuk, hogy: $1+3+5+...+2n-1=n^{2}$\\
\underline{Teljes indukcióval:}\\
Az állítás $n=1$-re igaz: $1=1^{2}$\\
Tegyük fel, hogy az állítás $n$-re igaz, bizonyítsuk be, ebből következik, hogy $n+1$-re is igaz:\\ 
\underline{Feltétel:}
$1+3+5+...+2n-1=n^{2}$;\\
$1+3+5+...+2n-1+2n+1=n ^{2}+2n+1=(n+1)^{2}$,\\ tehát az állítás minden $n+1$-re is igaz
így igaz minden pozitív egészre.
\item Igazoljuk a következő alaptulajdonságokat:
\begin{abc}
\item $\dfrac{1}{1\cdot2}+\dfrac{1}{2\cdot3}+...+\dfrac{1}{n(n+1)}=\dfrac{n}{n+1}$;
\item $\dfrac{1}{1\cdot3}+\dfrac{1}{3\cdot5}+...+\dfrac{1}{(2n-1)(2n+1)}=\dfrac{1}{2n+1}$;
\item $n^{3}+5n$ osztható 6-al, ha $n\ge1$, egész szám;
\item $5^{n+2}\cdot3^{n-1}+1$ osztható 8-al, ha $n\ge1$, egész szám
\end{abc}
\end{enumerate}


\subsection*{2010. 05. 06}
\begin{enumerate}
\item Igazoljuk:
\begin{abc}
\item $1\cdot2\cdot3+2\cdot3\cdot4+...+n(n+1)(n+2)=\dfrac{n(n+1)(n+2)(n+3)}{4}$;
\item $1^{2}+2^{2}+3^{2}+...+m^{2}=\dfrac{n(n+1)(2n+1)}{6}$;
\item $(n+1)(n+2)(n+3)...(2n-1)\cdot2n=2^{n}\cdot1\cdot3\cdot5\cdot(2n-1)$;
\end{abc}
\item Igazoljuk, hogy ha $h>-1$ valós szám, akkor $n\ge1$ egész esetén $(1+h)^{n}\ge1+nh$.
\item a Fibonacci sorozat: $f_{1}=f_{2}=1,  f_{n+1}=f_{n}+f_{n-1}$\\
Igazoljuk teljes indukcióval:
\begin{abc}
\item $f_{1}^{2}+f_{2}^{2}+...+f_{n}^{2}=f_{n}\cdot f_{n+1}$;
\item $f_{1}\cdot f_{2}+f_{2}\cdot f_{3}+...+f_{2n-1}\cdot f_{2n}=f_{2n}^{2}$;
\item $f_{n+1}^{2}-f_{n}f_{n+2}=(-1)^{n}$.
\end{abc}
\end{enumerate}


\subsection*{2010. 05. 11}
\begin{enumerate}
\item Igazoljuk, hogy:
\begin{abc}
\item $8\mid 3^{2n}+7$, ha $n\ge0$, egész;
\item $3\mid 2\cdot7^{n}+1$, ha $n\ge0$, egész;
\item $15\mid 2^{4n}-1$, ha $n\ge0$, egész;
\item $9\mid 7^{n}+3n-1$, ha $n\ge0$, egész;
\end{abc}
\item Igazoljuk, hogy ha $n>1$,egész, akkor\\
$\dfrac{1}{n+1}+\dfrac{1}{n+2}+...+\dfrac{1}{2n}>\dfrac{1}{2}$.
\item Igazoljuk, hogy ha $h>1$ egész, akkor\\
$\dfrac{1}{2^{2}}+\dfrac{1}{3^{2}}+...+\dfrac{1}{n^{2}}<\dfrac{n-1}{n}$.
\end{enumerate}


\subsection*{2010. 05. 12}
\begin{enumerate}
\item Igazoljuk, teljes indukcióval, ha $n\ge1$, egész, akkor
\begin{abc}
\item $1^{2}-2^{2}+3^{2}-4^{2}+...+(2n-1)^{2}-(2n)^2=-n(2n+1)$;
\item $\dfrac{1^{2}}{1\cdot3}+\dfrac{2^{2}}{3\cdot5}+...+\dfrac{n^{2}}{(2n-1)(2n+1)}=\dfrac{n(n+1)}{2(2n+1)}$;
\item $\dfrac{1}{1\cdot5}+\dfrac{1}{5\cdot9}+\dfrac{1}{9\cdot13}+...+\dfrac{1}{(4n-3)(4n+1)}=\dfrac{n}{4n+1}$.
\end{abc}
\item Igazoljuk, hogy ha $n>2$, egész, akkor
$\dfrac{1}{n+1}+\dfrac{1}{n+2}+...+\dfrac{1}{2n}>\dfrac{7}{12}$.
\item (*) Igazoljuk, hogy ha $n\ge1$, egész, akkor
$\dfrac{1}{n+1}+\dfrac{1}{n+2}+...+\dfrac{1}{3n+1}>1$.
\item (*) Igazoljuk, hogy ha $n\ge1$, egész, akor $11\mid 30^{n}+12^{n}-8^{n}-1$.
\end{enumerate}


\subsection*{2010. 05. 18 -- Ismétlő feladatok:}
\begin{enumerate}
\item Igazoljuk, teljes indukcióval:
\begin{abc}
\item $\dfrac{3}{1\cdot2}+\dfrac{7}{2\cdot3}+\dfrac{13}{3\cdot4}+...+\dfrac{n^{2}+n+1}{n(n+1)}=\dfrac{n(n+2)}{n+1}$;
\item $1^{3}+3^{3}+5^{3}+...+(2n-1)^{3}=n^{2}(3n^{2}-1)$;
\item $1\cdot1!+2\cdot2!+3\cdot3!+...+n\cdot n!=(n+1)!-1$.
\end{abc}
\item Igazoljuk, hogy ha $n\ge1$, egész, akkor:
\begin{abc}
\item $19\mid 5^{2n-1}\cdot2^{n+1}+3^{n+1}\cdot2^{2n-1}$;
\item $25\mid 72^{2n+2}-47^{2n}+28^{2n-1}$.
\end{abc}
\item Igazoljuk teljes indukcióval, ha $n\ge1$, egész, akkor:
\begin{abc}
\item $f_{1}+f_{4}+f_{7}+...+f_{3n-2}=\dfrac{f_{3n}}{2}$;
\item $f_{3}+f_{6}+f_{9}+...+f_{3n}=\dfrac{f_{3n+2}-1}{2}$;
\end{abc}
ahol $f_n$ a Fibonacci sorozat $n$-edik eleme.
\end{enumerate}


\subsection*{2010. 05. 19 -- Dolgozat- Teljes indukció:}
\begin{enumerate}
\item Igazoljuk, teljes indukcióval:
\begin{abc}
\item $1\cdot4+2\cdot7+3\cdot10+...+n(3n+1)=n(n+1)^{2}$;
\item $\dfrac{1}{1\cdot4}+\dfrac{1}{4\cdot7}+\dfrac{1}{7\cdot10}+...+\dfrac{1}{(3n-2)(3n+1)}=\dfrac{n}{3n+1}$,
$n\ge1$ egész szám.
\end{abc}
\item Igazoljuk teljes indukcióval a következő oszthatóságokat:
\begin{abc}
\item $8\mid 5^{n}+2\cdot3^{n-1}+1$, $n\ge1$, egész,
\item $11\mid 36^{n}+10\cdot 3^{n}$, $n\ge1$, egész.
\end{abc}
\item Igazoljuk teljes indukcióval, hogy ha $n>1$, egész, akkor:
$\dfrac{1}{n+1}<\dfrac{1\cdot3\cdot5\cdot...\cdot(2n-1)}{2\cdot4\cdot6\cdot...\cdot2n}$.
\end{enumerate}


\end{document}