\documentclass{article}
\usepackage[utf8]{inputenc}
\usepackage{t1enc}
\usepackage{geometry}
 \geometry{
 a4paper,
 total={210mm,297mm},
 left=20mm,
 right=20mm,
 top=20mm,
 bottom=20mm,
 }
\usepackage{amsmath}
\usepackage{amssymb}
\frenchspacing
\usepackage{fancyhdr}
\pagestyle{fancy}
\lhead{Urbán János tanár úr feladatsorai}
\chead{C08/7./1.}
\rhead{Algebra}
\lfoot{}
\cfoot{\thepage}
\rfoot{}

\usepackage{enumitem}
\usepackage{multicol}
\usepackage{calc}
\newenvironment{abc}{\begin{enumerate}[label=\textit{\alph*})]}{\end{enumerate}}
\newenvironment{abc2}{\begin{enumerate}[label=\textit{\alph*})]\begin{multicols}{2}}{\end{multicols}\end{enumerate}}
\newenvironment{abc3}{\begin{enumerate}[label=\textit{\alph*})]\begin{multicols}{3}}{\end{multicols}\end{enumerate}}
\newenvironment{abc4}{\begin{enumerate}[label=\textit{\alph*})]\begin{multicols}{4}}{\end{multicols}\end{enumerate}}

\newcommand{\degre}{\ensuremath{^\circ}}
\newcommand{\tg}{\mathop{\mathrm{tg}}\nolimits}
\newcommand{\ctg}{\mathop{\mathrm{ctg}}\nolimits}
\newcommand{\arc}{\mathop{\mathrm{arc}}\nolimits}
\renewcommand{\arcsin}{\arc\sin}
\renewcommand{\arccos}{\arc\cos}
\newcommand{\arctg}{\arc\tg}
\newcommand{\arcctg}{\arc\ctg}
 \newcommand{\sgn}{\operatorname{sgn}}


\parskip 8pt
\begin{document}

\section*{Algebra}





\subsection*{2008. 11. 03.}
\begin{enumerate}
\item Egy osztályban kétszer annyi fiú van, mint lány. Ha a fiúk számából is, meg a lányok számából is elveszünk 5-öt, akkor háromszor annyi fiú lesz, mint lány. Hány fiú és hány lány van az osztályban?
\item Amikor Balázs annyi idős volt, mint Kati most, akkor Balázs éveinek száma kétszerese volt Katiénak. Hány éves most Kati és Balázs, ha éveik számát összeadva 35-öt kapunk?
\item Oldjuk meg az $|x|=\displaystyle{\frac{x}{2}}+2$ egyenletet.
\item Anna és Béla együtt 22 kg, Anna és Csaba együtt 28 kg. Tudjuk még, hogy Béla meg Csaba együtt 36 kg. Hány kg Anna, Béla és Csaba külön-külön?
\item Egy kétjegyű szám számjegyeinek összege 12. Ha számjegyeit felcseréljük, 18-cal nagyobb számot kapunk. Melyik ez a szám?
\item 525 Ft-ot egyenlő számú 5 és 10 Ft-os pénzérmékkel fizettünk ki. Hány 5 Ft-ossal és hány 10 Ft-ossal fizettünk?
\item Az óra mutatói déli 12 órakor fedik egymást. Mikor következik be legközelebb ugyanez a helyzet?
\end{enumerate}


\subsection*{2008. 11. 04.}
\begin{enumerate}
\item Oldjuk meg a következő egyenleteket:
\begin{abc4}
\item $2x+3=9$;
\item $2(x-1)=8$;
\item $3x+4=x+2$;
\item $3(x+2)-2(x-1)=15$.
\end{abc4}
\item Igazoljuk a következő azonosságokat:
\begin{abc3}
\item $(a+b)^2=a^2+2ab+b^2$;
\item $(a-b)^2=a^2-2ab+b^2$;
\item $(a+b)(a-b)=a^2-b^2$.
\end{abc3}
\item Oldjuk meg a következő egyenleteket:
\begin{abc2}
\item $\displaystyle{\frac{x}{2}+\frac{x}{9}=44}$;
\item $\displaystyle{\frac{x+1}{6}-\frac{x-1}{4}=0}$;
\item $\displaystyle{\frac{x-1}{2}+\frac{3x-1}{4}-\frac{5x-1}{6}=2}$;
\item $\displaystyle{1-\frac{6-2x}{3}=x-\frac{x+3}{2}}$;
\item $\displaystyle{x-\frac{6-2x}{3}=2x-4-\frac{x+3}{2}}$.
\end{abc2}
\end{enumerate}


\subsection*{2008. 11. 05.}
\begin{enumerate}
\item Egy focicsapat 11 játékosának átlagéletkora 22 év. Az egyik játékost kiállították, így a csapat átlagéletkora 21 év lett. Hány éves volt a kiállított játékos?
\item Egy apa most háromszor olyan idős, mint a fia. 14 év múlva már csak kétszer olyan idős lesz. Hány évesek most?
\item Melyik az a szám, amelynek $30\%$-a na $90$-nel nagyobb, mint a szám $\frac{1}{5}$-e?
\item Az A városból B városba 60 km/h átlagsebességgel indult el egy személyvonat. Ezután 1 órával egy gyorsvonat is indult A-ból B-be. Mekkora a két város távolsága, ha a gyorsvonat fél órával előbb ért B-be, mint a személyvonat?
\item Oldjuk meg a következő egyenleteket:
\begin{abc2}
\item $\displaystyle{\frac{x-4}{3}-\frac{1-x}{2}=\frac{1}{2}+\frac{x}{6}}$;
\item $\displaystyle{8-\frac{2x}{3}=3-\frac{x-1}{2}}$;
\item $\displaystyle{\frac{x}{2}-\left(\frac{x}{4}-\frac{x}{3}\right)=\left(\frac{x}{6}-\frac{x}{5}\right)+x+1}$;
\item $(3x+1)(x+1)=7(x+1)$.
\end{abc2}
\end{enumerate}


\subsection*{2008. 11. 06.}
\begin{enumerate}
\item Oldjuk meg:
\begin{abc2}
\item $(x+5)(x+2)-3(4x-3)=(x-5)^2$;
\item $x(x-2)=(x-3)(x-5)$;
\item $\displaystyle{\frac{2}{x-1}+1=\frac{3}{x-1}}$;
\item $\displaystyle{\left(\frac{2x-15}{6}\right)^2-\left(\frac{2x-3}{6}\right)^2=2}$;
\item $\displaystyle{\frac{x+1}{x-1}-\frac{2(x+1)}{2x-3}=\frac{1}{3}}$;
\item ($*$) $\displaystyle{\frac{6}{2+x}+\frac{x+2}{2-x}=\frac{x^2}{4-x^2}}$.
\end{abc2}
\item Egy 135 m hosszú vonat a vele egy irányban haladó gyalogos mellett 10 másodperc alatt robog el. Mekkora a vonat és a gyalogos sebessége, ha a vonaté 10-szer akkora, mint a gyalogosé?
\item Egy férfi azt mondta a másiknak, hogy 5 év múlva az 5 évvel ezelőtti életkora 3-szorosánál 5 évvel lesz idősebb, még 5 év múlva ez a férfi fele annyi idős lesz, mint a másik. Hány éves a férfi?
\end{enumerate}


\subsection*{2008. 11. 10.}
\begin{enumerate}
\item Egy hajó a folyón a két végállomása közötti utat 4 óra 40 perc alatt tette meg oda-vissza. A sebessége folyásirányban 16 km/h, ellenkező irányban 12 km/h volt. Milyen messze van egymástól a két végállomás?
\item Az A és B túristaházakból, amelyek egymástól 10 km távolságra vannak egyszerre indul el egy-egy túrista egymással szembe. Az egyik óránként 4 km-t, a másik 6 km-t halad. Mennyi idő múlva és hol találkoznak?
\item Oldjuk meg a következő egyenleteket:
\begin{abc3}
\item $\displaystyle{2x+\frac{x^2-4}{x+2}=1}$;
\item $\displaystyle{\frac{3}{x-2}+\frac{4}{x-2}=\frac{1}{3}}$;
\item $\displaystyle{\frac{7-x}{x-6}-5=\frac{1}{x-6}}$;
\item $\displaystyle{\frac{x+1}{x-3}-\frac{x-2}{x+3}=\frac{3(3x-1)}{x^2-9}}$;
\item $\displaystyle{\frac{x-4}{x+4}-\frac{x+4}{x-4}+\frac{16x}{x^2-16}=0}$;
\item $\displaystyle{\frac{x^2}{4-x^2}-\frac{x+2}{2-x}=\frac{6}{x+2}}$.
\end{abc3}
\end{enumerate}


\subsection*{2008. 11. 11.}
\begin{enumerate}
\item Egy motorcsónak 28km-t tett meg a folyón lefelé, ezután azonnal visszatért kiindulási helyére- Útja összesen 7 óra hosszat tartott. Mennyi a motorcsónak sebessége állóvízen, ha a folyó sebessége 3 km/h?
\item Két munkás együtt dolgozva 12 óra alatt fejezett be egy munkát. Ha először az első munkás csinálta volna meg a munka felét, azután a másik a másik felét, akkor a munka 25 óráig tartott volna. Hány óra alatt készülne el a munkákkal egyedül az egyik, és egyedül a másik munkás?
\item Az A város 78 km-re van B-től. A-ból egy kerékpáros indul B-be, 1 órával később egy másik kerékpáros B-ből A-ba. Ez utóbbi óránként 4 km-rel többet tesz meg, mint az első, így B-től 36 km-re találkoznak. Mennyi ideig kerékpároztak a találkozásig és mekkora volt a sebességük?
\item Egy futár A-ból B-be indult. Bizonyos idő múlva a már megtett út úgy aránylik a még hátralevőhöz, mint 2:3. Ha a futár még megtesz 60 km-t, akkor a már megtett út és a még hátralévő út arány 6:5. Mekkora az AB távolság?
\item Oldjuk meg:
\begin{abc}
\item $\displaystyle{\frac{3}{1-x^2}=\frac{2}{1+2x+x^2}-\frac{5}{1-2x+x^2}}$;
\item $\displaystyle{\frac{x}{x^2-4}-\frac{x}{x^2-2x}=\frac{4}{x^2+2x}}$;
\item $\displaystyle{\frac{2x+1}{x-3}-6=\frac{4x-5}{3-x}}$.
\end{abc}
\end{enumerate}





\end{document}