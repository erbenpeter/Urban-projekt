\documentclass{article}
\usepackage[utf8]{inputenc}
\usepackage{t1enc}
\usepackage{geometry}
 \geometry{
 a4paper,
 total={210mm,297mm},
 left=20mm,
 right=20mm,
 top=20mm,
 bottom=20mm,
 }
\usepackage{amsmath}
\usepackage{amssymb}
\frenchspacing
\usepackage{fancyhdr}
\pagestyle{fancy}
\lhead{Urbán János tanár úr feladatsorai}
\chead{C08/7./1.}
\rhead{Dolgozatok}
\lfoot{}
\cfoot{\thepage}
\rfoot{}

\usepackage{enumitem}
\usepackage{multicol}
\usepackage{calc}
\newenvironment{abc}{\begin{enumerate}[label=\textit{\alph*})]}{\end{enumerate}}
\newenvironment{abc2}{\begin{enumerate}[label=\textit{\alph*})]\begin{multicols}{2}}{\end{multicols}\end{enumerate}}
\newenvironment{abc3}{\begin{enumerate}[label=\textit{\alph*})]\begin{multicols}{3}}{\end{multicols}\end{enumerate}}
\newenvironment{abc4}{\begin{enumerate}[label=\textit{\alph*})]\begin{multicols}{4}}{\end{multicols}\end{enumerate}}
\newenvironment{abcn}[1]{\begin{enumerate}[label=\textit{\alph*})]\begin{multicols}{#1}}{\end{multicols}\end{enumerate}}
\setlist[enumerate,1]{listparindent=\labelwidth+\labelsep}

\newcommand{\degre}{\ensuremath{^\circ}}
\newcommand{\tg}{\mathop{\mathrm{tg}}\nolimits}
\newcommand{\ctg}{\mathop{\mathrm{ctg}}\nolimits}
\newcommand{\arc}{\mathop{\mathrm{arc}}\nolimits}
\renewcommand{\arcsin}{\arc\sin}
\renewcommand{\arccos}{\arc\cos}
\newcommand{\arctg}{\arc\tg}
\newcommand{\arcctg}{\arc\ctg}

\parskip 8pt
\begin{document}

\section*{Dolgozatok}

\subsection*{2008.09.23. - Röpdolgozat}
\begin{enumerate}
\item Az ötjegyű tízes számrendszerbeli pozitív egész számok között melyikből van több, amiben van 9-es számjegy, vagy amiben nincs?
\item Hányféleképpen mehetünk fel egy 4 lépcsőfokból álló lépcsőn, ha egyszerre 1 vagy 2 lépcsőfokot léphetünk? Oldjuk meg a feladatot $5,6,7$ és $8$ lépcsőfokból álló lépcsőre is!
\item Egy kocka lapjainak síkjai hány részre vágják szét a teret? 
\item Az $1000$-nél nem nagyobb pozitív egész számok között hány olyan van, ami nem osztható sem $3$-mal, sem $5$-tel, sem $7$-tel?


\end{enumerate}
\subsection*{2008.10.01. - Pótdolgozat}
\begin{enumerate}
\item Számítsuk ki az $1,2,3,4$ számjegyekből készíthető háromjegyű számok összegét.
\item Hány $10000$-nél kisebb pozitív egész szám készíthető az $1,2,3$ számjegyek felhasználásával?
\item Egy négyzet minden oldalát 6 részre osztjuk (egyenlő részekre). Hány olyan háromszög van, amelynek a csúcsai az osztópontok közül kerülnek ki?
\item Adottak a $0,1,2,3,4,5$ számjegyek. Számítsuk ki az ezek segítségével felírható összes négyjegyű páros szám összegét.
\item Egy konvex hétszög átlói hány részre bontják a sokszög belsejét, ha egyik belső ponton sem halad át kettőnél több átló?
\end{enumerate}
\subsection*{2008.10.15. - Dolgozat}
\begin{enumerate}
\item Számítsuk ki a következő összegeket:
 
a) $\displaystyle{3\choose 3}+{4\choose 3}+{5\choose 3}+{6\choose 3}+\ldots++{15\choose 3}=;$

b) $\displaystyle{10\choose 0}+2{10\choose 1}+3{10\choose 2}+\ldots+11{10\choose 10}=.$
\item Hány olyan 4 jegyű tízes számrendszerbeli szám van, amelyben a számjegyek növekvő sorrendben állnak (balról jobbra)? 
\item Hányféleképpen oszthatunk el 30 különböző könyvet 3 ember között úgy, hogy mindenkinek 10 könyv jusson?
\item A $3,4,5,6$ számjegyekből hány páros szám készíthető, ha minden számjegyet legfeljebb egyszer használhatjuk? 
\end{enumerate}
\subsection*{2008.11.12. - Egyenletek}
\begin{enumerate}
\item Egy apa most háromszor annyi idős, mint a fia. 10 év múlva már csak kétszer annyi idős lesz. Hány évesek most?
\item Az A városból B-be $60$ km/óra átlagsebességgel elindult egy személyvonat, majd $1$ óra múlva $90$ km/óra átlagsebességgel egy gyorsvonat. Mekkora a két város távolsága, ha a gyorsvonat $1/2$ órával előbb ért B-be, mint a személyvonat?
\item Oldjuk meg:

a) $\displaystyle\frac{x}{3}+\frac{x}{8}=44;$$\qquad\qquad\qquad\quad$$\space$\ b)  $\displaystyle6-\frac{6x-4}{5}=2x+\frac{2-5x}{3};$ 

c) $\displaystyle\frac{x^2}{4-x^2}=\frac{6}{2+x}+\frac{x+2}{2-x};$$\qquad$d) $\displaystyle\frac{3}{x(2x-1)}+\frac{5}{x}=\frac{1}{2x-1}.$
\end{enumerate}
\subsection*{2009.01.07. - Fügvények}
\begin{enumerate}
\item Ábrázoljuk a következő fügvényeket:

1) $x\mapsto |(x-2)-1|-1;$$\qquad\ \ \ \ \ \ $ 2) $x\mapsto |2x-1|-|2x+1|;$$\qquad$3) $x\mapsto ||x+2|-|x|-|x-2||;$

4) $x\mapsto |x+3|+|x|+|x-2|;$$\qquad\!$ 5) $x\mapsto |2x|-|x|+|x-1|.$ 
\end{enumerate}
\subsection*{2009. 01. 29.}
 
\begin{enumerate}
 
\item Ábrázoljuk: $x \mapsto [x]+[\displaystyle{\frac{x}{2}}]$
 
\item Szemléltessük a síkon azokat a P(x,y) pontokat, amelyeknek koordinátáira igaz, hogy: $[x]\cdot [y]=16$
 
\item Egy folyón átívelő híd hossza 500m. Egy 500m hosszú vasúti szerelvény 1 perc alatt halad át a hídon. Hány m/sec a vonat sebessége?
 
\item A tej tömegének 7,3\%-a tejszín. A tejszín tömegének 62\%-a vaj. Hány kg tejből készíthető 5kg vaj?
 
\item Egy fiú kerékpáron ment A-ból B-be, majd vissza. Vízszintes úton 16 km/h, lefelé 24 km/h sebességgel haladt. Oda-vissza összesen 3 órát tartott az útja. Mekkora az AB távolsága?
 
\end{enumerate}
 
\subsection*{2009. 02. 23. -- Egyenletek - egyenlőtlenségek}
 
\begin{enumerate}
 
\item Egy apa és fia életkorának összege 50 év. Öt év múlva az apa háromszor annyi idős lesz, mint a fia. Hány év múlva lesz a fiú feleannyi idős, mint az apa?
 
\item Egy tört értéke $\frac{5}{6}$. A számláló és a nevező összege háromjegyű szám. Ez a háromjegyű szám egy természetes szám négyzete. Melyik ez a tört?
 
\item Oldjuk meg:
 
a) $\displaystyle{\frac{x-4}{3}}-\displaystyle{\frac{1-x}{2}} = \displaystyle{\frac{1}{2}}+\displaystyle{\frac{x}{6}}$
 
b) $\displaystyle{\frac{5x-2}{x-1}}+\displaystyle{\frac{2}{3}}<\displaystyle{\frac{3x-4}{x-1}}$
 
c) $|x+2|+|x-2|>4$
 
d) $||x+2|-x-6|\le x+4$
 
\end{enumerate}
 
\subsection*{2009.03.02 -- Pótdolgozat}
 
\begin{enumerate}
 
\item Egy medencébe három csövön keresztül folyik a víz. Ha csak az első van nyitva 4 óra alatt, ha csak a második akkor 6 óra alatt, végül ha csak a harmadik, akkor 12 óra alatt telik meg a medence. Mennyi idő alatt telik meg a medence, ha mindhárom cső nyitva van?
 
\item Egy háromszögben a második szög 10$^{\circ}$-kal nagyobb az első kétszeresénél, a harmadik pedig 30$^{\circ}$-kal kisebb a másodiknál. Számítsuk ki a háromszög szögeit.
 
\item Oldjuk meg:
 
a) $\displaystyle{\frac{x+2}{x+6}}:\left(\displaystyle{\frac{1}{2}}\displaystyle{\frac{1}{x}}\right)=\displaystyle{\frac{2x}{3}}$
 
b) $\displaystyle{\frac{3-5x}{2+x}}>-4$
 
c) $|x+1|+|x-5|\le 7$
 
d) $||x-2|+x-4|\ge x$
 
\end{enumerate}
 
\subsection*{2009.03.30 -- Valószínűségszámítás}
 
\begin{enumerate}
 
\item Egy 52 lapos kártyacsomagból 13-at osztanak az egyik játékosnak. Mennyi a valószínűsége, hogy a pikk király ennél a játékosnál lesz?
 
\item Mennyi a valószínűsége, hogy két különböző színű kockával dobva először a hatodik dobásnál lesz a  dobott pontok összege 12?
 
\item Melyik a valószínűbb, az, hogy egy kockának 4 dobás közül legalább egyszer hatost dobunk, vagy két kockával 24 dobás közül legalább egyszer 12 lesz az összeg?
 
\item Egy urnában 8 piros és 12 fehér golyó van. 5-ször kihúzunk egy golyót
 
   a) visszatevéssel;
   
   b) visszatevés nélkül.
   
Mennyi a kihúzott piros golyók várható száma?
 
\end{enumerate}
 
\subsection*{2009.05.14}
 
\begin{enumerate}
 
\item Számítsuk ki minél egyszerűbben:
 
a) $(1+\displaystyle{\frac{1}{2}})\cdot(1+\displaystyle{\frac{1}{3}})\cdot(1+\displaystyle{\frac{1}{4}})\cdot(1+\displaystyle{\frac{1}{5}})\cdot...\cdot (1+\displaystyle{\frac{1}{101}})$
 
b) $1-\displaystyle{\frac{1}{3}}+\displaystyle{\frac{1}{3^2}}-\displaystyle{\frac{1}{3^3}}+\displaystyle{\frac{1}{3^4}}-...+\displaystyle{\frac{1}{3^{10}}}$
 
c) $1+\displaystyle{\frac{1}{1+2}}+\displaystyle{\frac{1}{1+2+3}}+\displaystyle{\frac{1}{1+2+3+4}}+...+\displaystyle{\frac{1}{1+2+3+...+100}}$
 
\item Igazoljuk a következő egyenlőtlenségeket:
 
a) $\displaystyle{\frac{1}{11}}+\displaystyle{\frac{1}{12}}+\displaystyle{\frac{1}{13}}+...+\displaystyle{\frac{1}{20}}>\displaystyle{\frac{3}{5}}$
 
b) $\displaystyle{\frac{1}{2}}<(1-\displaystyle{\frac{1}{2^2}})(1-\displaystyle{\frac{1}{4^2}})(1-\displaystyle{\frac{1}{6^2}})...(1-\displaystyle{\frac{1}{100^2}})<1$
 
 
\end{enumerate}
 
\end{document}
