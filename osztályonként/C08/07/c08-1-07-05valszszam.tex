\documentclass{article}
\usepackage[utf8]{inputenc}
\usepackage{t1enc}
\usepackage[magyar]{babel}
\usepackage{geometry}
 \geometry{
 a4paper,
 total={210mm,297mm},
 left=20mm,
 right=20mm,
 top=20mm,
 bottom=20mm,
 }
\usepackage{amsmath}
\usepackage{amssymb}
\frenchspacing
\usepackage{fancyhdr}
\pagestyle{fancy}
\lhead{Urbán János tanár úr feladatsorai}
\chead{C08/07/1.}
\rhead{Valószínűségszámítás}
\lfoot{}
\cfoot{\thepage}
\rfoot{}

\usepackage{enumitem}
\usepackage{multicol}
\usepackage{calc}
\newenvironment{abc}{\begin{enumerate}[label=\textit{\alph*})]}{\end{enumerate}}
\newenvironment{abc2}{\begin{enumerate}[label=\textit{\alph*})]\begin{multicols}{2}}{\end{multicols}\end{enumerate}}
\newenvironment{abc3}{\begin{enumerate}[label=\textit{\alph*})]\begin{multicols}{3}}{\end{multicols}\end{enumerate}}
\newenvironment{abc4}{\begin{enumerate}[label=\textit{\alph*})]\begin{multicols}{4}}{\end{multicols}\end{enumerate}}
\newenvironment{abcn}[1]{\begin{enumerate}[label=\textit{\alph*})]\begin{multicols}{#1}}{\end{multicols}\end{enumerate}}
\setlist[enumerate,1]{listparindent=\labelwidth+\labelsep}

\newcommand{\degre}{\ensuremath{^\circ}}
\newcommand{\tg}{\mathop{\mathrm{tg}}\nolimits}
\newcommand{\ctg}{\mathop{\mathrm{ctg}}\nolimits}
\newcommand{\arc}{\mathop{\mathrm{arc}}\nolimits}
\renewcommand{\arcsin}{\arc\sin}
\renewcommand{\arccos}{\arc\cos}
\newcommand{\arctg}{\arc\tg}
\newcommand{\arcctg}{\arc\ctg}

\parskip 8pt
\begin{document}

\section*{Valószínűségszámítás}

\subsection*{2009. 03. 05.}
\begin{enumerate}
\item Egy kockával 5-ször dobunk. Mennyi a valószínűsége, hogy az 5 dobásból 0, 1, 2, 3, 4, 5, 6-ost dobunk? Készíts táblázatot!

\item Egy pénzérmével 6-szor dobunk. Mennyi a valószínűsége, hogy 0, 1, 2, 3, 4, 5, 6 fejet dobunk? Készíts táblázatot!

\item Dobjunk fel egy érmét. Ha az eredmény fej, még egyszer dobunk, ha írás, még kétszer. Mennyi a valószínűsége, hogy összesen egy fejet dobunk?

\item 100 alma közül 10 kukacos. Véletlenszerűen kiveszünk az almák közül 5-öt. Mennyi a valószínűsége, hogy lesz a kiválasztott almák között kukacos?

\item Tíz kockával dobunk, mennyi annak a valószínűsége, hogy a dobott számok összege legalább 58?

\item Egy pénzdarabot 10-szer feldobunk és az eredményeket leírjuk, ha fej jön ki F-et, ha írás I-t írunk. Mennyi a valószínűsége, hogy az így kapott 10 elemű szorzat tartalmaz két azonos betűt egymás mellett?
\end{enumerate}
\subsection*{2009. 03. 10.}
\begin{enumerate}
\item Öt levelet megírt valaki 5 ismerősének, majd borítékokba rakta a leveleket. Ezután a lezárt borítékokat véletlenszerűen megcímzi. Mennyi a valószínűsége, hogy legalább 3 ismerőse a neki szánt levelet kapja meg? Mennyi a valószínűsége, hogy senki sem kapja meg a neki szánt levelet? 

\item Az 1, 2, 3, 4, 5 számjegyeket véletlenszerűen egymás mellé írjuk. Mennyi a valószínűsége, hogy a kapott ötjegyű szám osztható 8-cal?

\item Három kockával dobunk. Mennyi a valószínűsége, hogy a dobott számok négyzetösszege osztható 3-mal?

\item Mennyi a valószínűsége, hogy a lottón kihúzott öt számban mind a 10 számjegy előfordul (az egyjegyű számokat tekintsük olyan kétjegyűnek, amelynek az első jegye 0)?

\item Mennyi a valószínűsége, hogy véletlenül kiválasztott 12 ember születésnapja különböző hónapban legyen?

\item Jelölje $A_k$ annak a valószínűségét, hogy kockát $k$-szor feldobva először a $k$-adik dobásra kapunk 6-ost. Számoljuk ki $A_1, A_2, A_3, A_4, A_5$ és $A_6$ értékét!
\end{enumerate}
\subsection*{2009. 03. 17.}
\begin{enumerate}
\item A kockadobás várható értékét így számítjuk ki: 

$$
\begin{array}{c}
\text{ki: dobott szám}\cr
X:\text{valószínűség}
\end{array}
\left(
\begin{array}{cccccc}
1&2&3&4&54&6\cr
\frac{1}{6}&
\frac{1}{6}&
\frac{1}{6}&
\frac{1}{6}&
\frac{1}{6}&
\frac{1}{6}
\end{array}
\right)
$$

$$
E(X)=
1\cdot\frac{1}{6}+
2\cdot\frac{1}{6}+
3\cdot\frac{1}{6}+
4\cdot\frac{1}{6}+
5\cdot\frac{1}{6}+
6\cdot\frac{1}{6}=
\frac{21}{6}=3{,}5
$$








\item Számítsuk ki a két kockával dobott pontok összegének várható értékét.

\item Egy érmével dobunk. Ha az eredmény fej, még kétszer dobunk, ha írás, még egyszer. Mennyi az összes fej dobások várható értéke?

\item Péter feldob egy kockát. Ha páratlan számot dob, veszít 10 Ft-ot, ha 6-ost dob, nyer 40 Ft-ot, ha 2-est vagy 4-est dob, újra dobhat. A második dobásnál 10 Ft-ot nyer, ha párosat dob és 20 Ft-ot veszít, ha páratlant dob. Döntsük el, hogy a játék Péter számára előnyös, hátrányos vagy igazságos.

\item Egy érmével addig dobunk, amíg először fordul elő, hogy két egymás utáni dobás azonos. Mennyi a szükséges dobások várható száma?
\end{enumerate}
\subsection*{2009. 03. 18.}
\begin{enumerate}
\item Két kockával dobunk. Számítsuk ki a dobott számok 
\begin{abc}
\item maximumának
\item minimumának
\end{abc}
\quad várható értékét!

\item Mennyi a lottótalálatok számának várható értéke egy találomra kitöltött szelvény esetén? (ötös lottón)

\item Egy urnában van 5 piros, 3 fehér és 2 kék golyó. Háromszor húzunk:
\begin{abc2}
\item visszatevés nélkül
\item visszatevéssel.
\end{abc2}
Határozzuk meg mindkét esetben a kihúzott piros golyók várható számát!

\item Egy urnában 100 cédula van 1-től 100-ig számozva. Kihúzunk ezek közül 15 darabot visszatevéssel. Mennyi a kihúzott számok összegének várható értéke?

\item Mennyi az ötös lottón kihúzott számok összegének várható értéke?
\end{enumerate}
\subsection*{2009. 03. 19.}
\begin{enumerate}
\item Mennyi annak a valószínűsége, hogy egy tetszőlegesen választott pozitív egész szám négyzete 1-re végződik?

\item Mennyi annak a valószínűsége, hogy egy tetszőlegesen választott pozitív egész szám köbe 11-re végződik? 

\item Mennyi annak a valószínűsége, hogy $\binom{m}{3}$ osztható 3-mal, ha $n > 3$, egész szám?

\item Mennyi annak a valószínűsége, hogy egy 12 tagú társaságban mindenkinek más hónapban legyen a születésnapja?

\item Egy vonat 3 kocsijába 9 utas száll be. Minden utas véletlenszerűen választja meg, hogy melyik kocsiba száll. Mennyi annak a valószínűsége, hogy mindhárom kocsiba 3 ember száll? 
\end{enumerate}
\subsection*{2009. 03. 24.}
\begin{enumerate}
\item Az 52 lapos francia kártyából az egyik játékos 12 lapot kap. Mennyi a valószínűsége, hogy legfeljebb 3 ásza lesz?

\item Mennyi annak a valószínűsége, hogy egy 4 tagú társaságban van két ember, akinek az év azonos napján van a születésnapja?

\item Egy urnában van 5 piros és 10 fehér golyó.Visszatevés nélkül addig húzunk, amíg fehéret nem húzunk. Mennyi az eddig kihúzott piros golyók várható száma?

\item Egy társasjátéknál egy korong kerületén 0-tól 25-ig sorban számozott 26 mező van. Minden játékos a 0-val számozott mezőre állítja a bábuját és minden alkalommal, amikor sorra kerül, annyi mezővel lép előre, amennyit egy kockával dob. Ha a bábu a 13 számú mezőre ér, vissza kell mennie a 0-ra. Mennyi a valószínűsége, hogy egy játékos 5 dobásból a 25 számú mezőre ér?
\end{enumerate}
\subsection*{2009. 03. 25.}
\begin{enumerate}
\item Ötször dobunk egy kockával. Mennyi a valószínűsége, hogy a dobott számok összege 25-nél nagyobb szám lesz?

\item Egy urnában 4 piros és 6 fehér golyó van. Kihúzunk 3 golyót
\begin{abc}
\item visszatevéssel;
\item visszatevés nélkül.
\end{abc}
Számítsuk ki mindkét esetben a kihúzott piros golyók várható számát!

\item Egy urnában 4 piros, 4 fehér és 2 zöld golyó van. Az egyik játékos ebből addig húz ki visszatevés nélkül golyókat, amíg az első zöldet húzza. Ekkor a társától, a másik játékostól annyiszor 10 Ft-ot kap, ahány golyót kihúzott. Ezután a maradék golyók közül a másik játékos húzhat addig, amíg zöldet nem húz és ő is annyiszor 10 Ft-ot kap az elsőtől, ahány golyót kihúzott. Igazságos-e a játék?

\item Egy szabályos érmével addig dobunk, amíg először fordul elő, hogy két egymást követő dobás azonos. Mennyi a szükséges dobások várható száma?
\end{enumerate}
\subsection*{2009. 03. 26.}
\begin{enumerate}
\item Péter feldob egy kockát. Ha páratlan számot dob, veszít 10 Ft-ot, ha 6-ost dob nyer 40 Ft-ot, ha 2-est vagy 4-est, újra dobhat. A második dobásnál 10 Ft-ot nyer, ha párost dob és 20 Ft-ot veszít, ha páratlant dob. Döntsük el, hogy a játék Péter számára előnyös, igazságos vagy hátrányos.

\item Három különböző színű kockával dobunk. Mennyi a valószínűsége, hogy a dobott számok összege prímszám?

\item Egy kockával addig dobunk, amíg valamelyik, már korábban dobott szám újra előfordul. Mennyi a szükséges dobások várható száma?

\item Egy urnában van 4 piros és 6 fehér golyó. 5-ször kihúzunk egy golyót úgy, hogy feljegyezzük a számát, visszatesszük, összekeverjük és úgy húzunk újra. Mennyi a kihúzott piros golyók várható száma?
\end{enumerate}
\subsection*{2009. 04. 01.}
\begin{enumerate}
\item ($*$) 15 golyót véletlenszerűen helyezünk el 4 dobozba. Mennyi a valószínűsége, hogy mindegyik dobozba legalább 3 golyó kerül? 

\item Egy dobozban 8 piros és 10 fehér golyó van. Visszatevés nélkül  egymás után húzunk golyókat. Mennyi a valószínűsége, hogy először az 5. húzáskor kerül elő fehér golyó?

\item Egy dobozban 6 piros, 4 kék, 2 zöld golyó van. Visszatevéssel húzunk. 10 Ft-ért fogadhatunk a golyó színekre. Többlet esetén a piros golyónál 20, a kék golyónál 30, zöld golyónál 60 Ft a nyeremény. Melyik golyóra érdemes fogadni? 

\item Állítsuk növekvő sorrendbe a következő események valószínűségét:
\begin{abc}
\item Egy kockával 6-szor dobunk és és legalább 1 db 6-ost dobunk;
\item Egy kockával 12-szer dobunk és legalább 2 db 6-ost dobunk;
\item Egy kockával 18-szor dobunk és legalább 3 db 6-ost dobunk.
\end{abc}
\end{enumerate}
\subsection*{2009. 04. 06.}
\bf{Ismétlő feladatok} \rm
\begin{enumerate}
\item Egy urnában van 4 piros, 4 fehér és 4 zöld golyó. 4-szer húzunk visszatevéssel. Mennyi a kihúzott fehér golyók várható száma?

\item Elemezzük a ,,CRAPS'' játékot. Ebben két kockával dobunk, a dobott pontok összege számít. 
Ha az első dobás eredménye 7 vagy 11, a játékos nyer, ha 2, 3 vagy 12, akkor veszít. Minden más esetben az első dobás eredménye a ,,POINT'' és újra dobhat. 
A második dobástól kezdve a játékos nyer,  ha a ,,POINT''-ot dobja, veszít, ha 7-est dob minden más esetben újra dobhat. Mennyi a játékban a játékos nyerési esélye?
\end{enumerate}
\end{document}


